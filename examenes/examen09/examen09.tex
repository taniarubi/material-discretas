\documentclass[12pt, a4paper]{exam}

% Soporte para cambiar la fecha que sale en el examen
\usepackage{advdate}
% Soporte para escribir en varias columnas
\usepackage{multicol}
% Soporte para los acentos.
\usepackage[utf8]{inputenc} 
\usepackage[T1]{fontenc}    
% Idioma español.
\usepackage[spanish,mexico,es-tabla]{babel}
\usepackage{graphicx}
\usepackage{tikz}
\usepackage{amsmath,amssymb,amsthm}
\usepackage{verbatim} % comentarios
% Cambiamos los márgenes del documento. 
\usepackage[top=1.5cm,left=1.5cm,right=1.5cm]{geometry}

% Pie de página
\cfoot{Página \thepage\ de \numpages}

%%%%%%%%%%%%%%%%%%%%%%%%%%%%%%%%%%%%%%%%%%%%%%%%%%%%%%%%%%%%%%%%%%%%%%%%%%%%%%
\renewcommand{\thechoice}{\alph{choice}}

\makeatletter
\renewenvironment{checkboxes}%
   {\setcounter{choice}{0}\list{\checkbox@char}%
      {%
        \settowidth{\leftmargin}{W.\hskip\labelsep\hskip 2.5em}%
        \def\choice{%
          \if@correctchoice
            \color@endgroup \endgroup
          \fi
          \stepcounter{choice}
          \item[\checked@char]
          \do@choice@pageinfo
        } % choice
        \def\CorrectChoice{%
          \if@correctchoice
            \color@endgroup \endgroup
          \fi
          \ifprintanswers
            % We can't say \choice here, because that would
            % insert an \endgroup.
            % 2016/05/10: We say \color@begingroup in addition to
            % \begingroup in case \CorrectChoiceEmphasis involves color
            % and the text exactly fills the line (which would
            % otherwise create a blank line after this choice):
            % 2016/05/11: We leave hmode if we're in it,
            % i.e., if there's no blank line preceding this
            % \CorrectChoice command.  (Without this, the
            % \special created by a \color{whatever} command that might
            % be inserted by \CorrectChoice@Emphasis would be appended 
            % to the previous \choice, which could cause an extra
            % (blank) line to be inserted before this \CorrectChoice.)
            % Since \par and \endgraf seem to cancel \@totalleftmargin
            % (for reasons I don't understand), we'll do the following:
            % Motivated by  the def of \leavevmode, 
            %      \def\leavevmode{\unhbox\voidb@x}
            % we will now leave hmode (if we're in hmode):
            \ifhmode \unskip\unskip\unvbox\voidb@x \fi
            \begingroup \color@begingroup \@correctchoicetrue
            \CorrectChoice@Emphasis
            \stepcounter{choice}
            \item[\checked@char]
          \else
            \stepcounter{choice}
            \item[\checked@char]
          \fi
          \do@choice@pageinfo
        } % CorrectChoice
        \let\correctchoice\CorrectChoice
        \labelwidth\leftmargin\advance\labelwidth-\labelsep
        \topsep=0pt
        \partopsep=0pt
        \checkboxeshook
      }%
   }%
   {\if@correctchoice \color@endgroup \endgroup \fi \endlist}
 \makeatother

% Make checkbox character a circle with the letter
\checkboxchar{\tikz[baseline={([yshift=-.8ex]current bounding box.center)}]\node[shape=circle,minimum size=4mm,draw] at (0,0) {\thechoice};}
% Make checked box character bold WITH surd
%\checkedchar{\tikz[baseline={([yshift=-.8ex]current bounding box.center)}]\node[shape=circle,minimum size=8mm,draw] at (0,0) {} node at (0,0) {\thechoice\llap{$\surd$}};}
% Make checked box character bold
\checkedchar{\tikz[baseline={([yshift=-.8ex]current bounding box.center)}]\node[shape=circle,minimum size=4mm,draw] at (0,0) {} node at (0,0) {\thechoice};}
\printanswers
%%%%%%%%%%%%%%%%%%%%%%%%%%%%%%%%%%%%%%%%%%%%%%%%%%%%%%%%%%%%%%%%%%%%%%%%%%%%%%

\begin{document}
    %%%%%%%%%%%%%%%%%%%%%%%%%%%%%%%%%%%%%%%%%%%%%%%%%%%%%%%%%%%%%%%%%%%%%%%%%%%%%%%
    %%%%%%%%%%%%%%%%%%%%%%%%%%%%%%%% ENCABEZADO %%%%%%%%%%%%%%%%%%%%%%%%%%%%%%%%%%%
    \centering
    \hrule \hrule \hrule 
    \vspace{5mm}
    \begin{minipage}[c]{0.8\textwidth}
        \begin{center}
            {\large\textbf{Mission 09, Start!} \par
            \large \textbf{Estructuras Discretas} \par
            \large \textbf{Semestre 2023-1} \par
            \large \textbf{\today}	\par}
        \end{center}
    \end{minipage}

    \vspace{0.2in}
    \noindent
    \textbf{Tania Michelle Rubí Rojas}
    \vspace{2mm}
    \hrule \hrule \hrule 
    %%%%%%%%%%%%%%%%%%%%%%%%%%%%%%%%%%%%%%%%%%%%%%%%%%%%%%%%%%%%%%%%%%%%%%%%%%%%%%%
    %%%%%%%%%%%%%%%%%%%%%%%%%%%%%%%%%%%%%%%%%%%%%%%%%%%%%%%%%%%%%%%%%%%%%%%%%%%%%%%

    \vspace{5mm}
    \noindent
    Nombre y número de cuenta: \hrulefill\

    \textbf{Notación y convenciones para el examen:}
    {\tiny
    \begin{multicols}{2}
    \begin{itemize}\setlength\itemsep{0em}  
      \item En este examen, los antónimos serán complementarios.

      \item Los errores de escritura en las funciones son {\bf intencionales}, 
      por lo que cualquier afirmación que contenga una expresión mal escrita 
      es falsa.
    \end{itemize}
    \end{multicols}
    }

    \begin{questions}
        % Question 01
        \question{¿Cuál o cuáles de las siguientes expresiones son 
        \textbf{verdaderas}?}
        \begin{checkboxes}
            \choice La formalización de la proposición
            \begin{center}
                \textbf{''Si no ocurre que un objeto flota en el agua, 
                entonces es menos denso que el agua.''}
            \end{center}

            es $\neg q \rightarrow p$, donde 
            \begin{align*}
                p &= \text{Un objeto flota en el agua} \\ 
                q &= \text{Un objeto es menos denso que el agua} 
            \end{align*} %Correcta

            \choice La formalización de la proposición
            \begin{center}
                \textbf{''Mi mamá me dio permiso de ir a una fiesta, pero debo 
                bañar antes a mi perro llamado Hércules''}
            \end{center}

            es $p \land q$, donde 
            \begin{align*}
                p &: \text{Mi mamá me dio permiso de ir a una fiesta} \\ 
                q &: \text{Debo bañar antes a mi perro llamado Hércules}
            \end{align*}

            \choice La negación de la proposición
            \begin{center}
                \textbf{''Si el usuario no ha introducido una contraseña 
                válida pero ha pagado la cuota de suscripción, entonces se 
                le concede el acceso.''}
            \end{center}

            es \texttt{''El usuario ha introducido una contraseña válida o 
            no ha pagado la cuota de suscripción o se le concede el acceso''}.

            \choice La formalización de la proposición
            \begin{center}
                \textbf{''Si pudiera volver a empezar, cambiaría todo aquello 
                que olvidé por querer sobrevivir''}
            \end{center}

            \textbf{no} es $\neg q \rightarrow \neg p$, donde 
            \begin{align*}
                p &= \text{Yo puedo volver a empezar} \\ 
                q &= \text{Yo cambio todo aquello que olvidé por querer 
                sobrevivir}
            \end{align*}
            
            \choice Ninguna de las anteriores. 
        \end{checkboxes}

        \newpage
        % Question 02
        \question{¿Cuál o cuáles de las siguientes expresiones son 
        \textbf{verdaderas}?}
        \begin{checkboxes}
            \choice La formalización de la proposición
            \begin{center}
                \textbf{''Para pasar el examen es necesario que los alumnos 
                estudien, hagan la tarea y asistan a clase.''}
            \end{center}

            es $q \land r \land s \rightarrow p$, donde 
            \begin{align*}
                p &= \text{Los alumnos pasan el examen} \\
                q &= \text{Los alumnos estudian} \\ 
                r &= \text{Los alumnos hacen la tarea} \\ 
                s &= \text{Los alumnos asisten a clase}
            \end{align*}

            \choice La formalización del argumento 
            \begin{center}
                \textbf{''Si mi ascenso es un éxito, entonces podré comprar un 
                departamento. O bien, mi ascenso es un éxito o me despedirán.
                No podré comprar un departamento. Por lo tanto, me despedirán.''}
            \end{center}

            es $\Gamma = \{p \rightarrow q, p \lor r, \neg q\} \models r$, donde 
            \begin{align*}
                p &= \text{Mi ascenso es un éxito} \\ 
                q &= \text{Podré comprar un departamento} \\ 
                r &= \text{Me despedirán}
            \end{align*} % Correcta

            \choice La formalización de la proposición:
            \begin{center}
                \textbf{''Si lo que siento es amor, me siento feliz. Si no 
                lo siento, lo sentiré mañana''}
            \end{center}

            es $(p \rightarrow q) \land (\neg p \rightarrow r)$, donde 
            \begin{align*}
                p &= \text{Lo que siento es amor} \\
                q &= \text{Me siento feliz} \\ 
                r &= \text{Mañana sentiré amor}
            \end{align*} % Correcta

            \choice La formalización del argumento
            \begin{center}
                \textbf{''Si el amor duele o sana, entonces he aprendido. El 
                amor duele. Pero tener amor propio es necesario para aprender.
                Así, tengo amor propio o el amor sana. ''}
            \end{center}

            es $\Gamma = \{p \lor q \rightarrow s, p, s \rightarrow r\} 
            \models r \lor q$, donde 
            \begin{align*}
                p &= \text{El amor duele} \\
                q &= \text{El amor sana} \\ 
                r &= \text{Tengo amor propio} \\ 
                s &= \text{Aprendo}
            \end{align*} % Correcta 
            
            \choice Ninguna de las anteriores. 
        \end{checkboxes}

        \newpage
        % Question 03
        \question{¿Cuál o cuáles de las siguientes expresiones son 
        \textbf{verdaderas}?}
        \begin{checkboxes}
            \choice La formalización de la proposición
            \begin{center}
                \textbf{''Si yo hubiera estado allí, si lo hubiera encontrado 
                antes, no habría hecho alguna diferencia''}
            \end{center}

            es $p \land q \rightarrow \neg r$, donde 
            \begin{align*}
                p &= \text{Yo estuve allí} \\ 
                q &= \text{Yo lo encontré antes} \\ 
                r &= \text{Yo habría hecho alguna diferencia}
            \end{align*} % Correcta 

            \choice $s \rightarrow p \land t$ es consecuencia lógica de 
            $p \rightarrow q \land p \land (s \lor t) \land \neg (q \lor \neg p)$

            \choice $t \lor u$ es consecuencia lógica de 
            $(p \lor q) \land (r \land s) \land (r \rightarrow p \rightarrow t) \land 
            (s \rightarrow q \rightarrow u)$ % Correcta

            \choice $((p \rightarrow q) \land (q \rightarrow p) \land q)
            \rightarrow (\neg q \rightarrow \neg p)$ es una tautología. % Correcta 

            \choice Ninguna de las anteriores.
        \end{checkboxes}

        % Question 04
        \question{¿Cuál o cuáles de las siguientes expresiones son 
        \textbf{verdaderas}?}
        \begin{checkboxes}
            \choice $\neg (p \leftrightarrow (q \rightarrow (r \lor p))) 
            \equiv p \land q \land \neg r \land \neg p \lor (\neg p \land 
            (\neg q \lor \neg r \lor p))$

            \choice La negación de una proposición compuesta insatisfacible es 
            una tautología. % Correcta

            \choice Dos fórmulas proposicionales $p$ y $q$ son lógicamente 
            equivalentes si y sólo si $p \leftrightarrow q$ es satisfacible. 

            \choice La fórmula $p \land q \lor p \land \neg q \lor \neg p$ es 
            una tautología. % Correcta 
            
            \choice Ninguna de las anteriores. 
        \end{checkboxes}

        % Question 05
        \question
        {
            Tania tiene cuatro computadoras $A,B,C$ y $D$ en su casa 
            conectadas a una misma red. Ella estaba buscando la tarea de 
            Criptografía y Seguridad en internet en sitios de dudosa procedencia.
            La encontró, pero al momento de descargarla y abrir el documento, 
            se da cuenta de que éste traía premio (¡un malware!). Ahora ella 
            teme que su red haya sido infectada. Desesperada, Tania hace las 
            siguientes afirmaciones:
            \begin{itemize}
                \item Si $D$ está infectada, entonces $C$ también lo está. 
                \item Si $C$ está infectada, entonces también lo está $A$. 
                \item Si $D$ no está infectada, entonces $B$ no está infectada 
                pero $C$ está infectada. 
                \item Si $A$ está infectada, entonces $B$ está infectada o 
                $C$ no está infectada. 
            \end{itemize}

            De acuerdo a lo anterior, ¿cuál o cuáles de las siguientes 
            expresiones son \textbf{verdaderas}?
        }
        \begin{checkboxes}
            \choice \texttt{''A está infectada''} y \texttt{''C está infectada''}
            no pueden ser consecuencia lógica de las cuatro premisas anteriormente 
            mencionadas.

            \choice Solo algunas computadoras de Tania están infectadas. 

            \choice Todas las computadoras $A,B,C,D$ están infectadas. % Correcta

            \choice \texttt{''B está infectada''} y \texttt{''D está infectada''} 
            son consecuencia lógica de las cuatro premisas anteriormente 
            mencionadas. % Correcta
            
            \choice Ninguna de las anteriores. 
        \end{checkboxes}

        \newpage
        % Question 06
        \question{¿Cuál o cuáles de las siguientes expresiones son 
        \textbf{verdaderas}?}
        \begin{checkboxes}
            \choice La formalización de la proposición
            \begin{center}
                \textbf{''Si mi primo no regresó a casa temprano o si Sara no se 
                llevó su comida, entonces Laura lo hizo.''}
            \end{center}

            es $\neg p \lor (\neg q \rightarrow r)$, donde 
            \begin{align*}
                p &= \text{Mi primo regresó a casa temprano} \\ 
                q &= \text{Sara se llevó su comida} \\ 
                r &= \text{Laura se llevó su comida}
            \end{align*}

            \choice El argumento 
            \begin{center}
                \textbf{Si el programa es eficiente, entonces se ejecuta 
                rápidamente. O bien, el programa es eficiente o éste tiene 
                un bug. Sin embargo, el programa no se ejecuta rápidamente. 
                Por lo tanto, éste tiene un bug. }
            \end{center}

            es correcto. % Correcta  

            \choice La formalización de la proposición
            \begin{center}
                \textbf{''Si pudiera volver a empezar, cambiaría todo aquello 
                que olvidé por querer sobrevivir''}
            \end{center}

            \textbf{no} es $\neg q \rightarrow \neg p$, donde 
            \begin{align*}
                p &= \text{Yo puedo volver a empezar} \\ 
                q &= \text{Yo cambio todo aquello que olvidé por querer 
                sobrevivir}
            \end{align*} 

            \choice La formalización de la proposición
            \begin{center}
              \textbf{No es cierto que Nubecita no sea mayor que Chucho, tampoco
                que Lentejita no hable desde el corazón}
            \end{center}

            es $p \lor q$
            \begin{align*}
                p &= \text{Nubecita es mayor a Chucho} \\ 
                q &= \text{Lentejita habla desde el corazón} 
            \end{align*}
            
            \choice Ninguna de las anteriores. 
        \end{checkboxes}

        % Question 07
        \question
        {
            Supongamos que el siguiente enunciado es verdadero:
            \begin{center}
                \textbf{''Si mi agüita está hirviendo, entonces su temperatura
                debe ser de al menos 80°C''}
            \end{center}

            De acuerdo a lo anterior, ¿cuál o cuáles de los siguientes 
            enunciados deben ser también \textbf{verdaderos}?
        }

        \begin{checkboxes}
            \choice Si la temperatura de mi agüita es al menos de $80$°C, 
            mi agüita está hirviendo. 
            
            \choice Si la temperatura de mi agüita es menor que $80$°C, 
            entonces mi agüita no está hirviendo. % Correcta 

            \choice Mi agüita hierve sólo si su temperatura es de al menos 
            $80$°C. % Correcta

            \choice Si mi agüita no está hirviendo, entonces su temperatura 
            es inferior a $80$°C. 

            \choice Ninguna de las anteriores.
        \end{checkboxes}

        \newpage
        % Question 08
        \question{¿Cuál o cuáles de las siguientes expresiones son 
        \textbf{verdaderas}?}
        \begin{checkboxes}
            \choice La negación de la proposición
            \begin{equation*}
                \textbf{''Si está lloviendo, entonces no puedo ir a clase''}
            \end{equation*}

            es \texttt{''Está lloviendo y puedo ir a clase''} % Correcta

            \choice La formalización de la proposición
            \begin{align*}
                \textbf{Si el verano era un libro, entonces yo iba a escribir 
                algo hermoso en él}
            \end{align*}

            es $\neg q \rightarrow \neg p$, donde 
            \begin{align*}
                p &= \text{El verano es un libro} \\ 
                q &= \text{Voy a escribir algo hermoso en el libro}
            \end{align*} % Correcta

            \choice $((p \lor p) \lor (q \lor q)) \land (p \land q) \equiv \lnot(p \land q)$ 

            \choice $((p \lor p) \lor (q \lor q)) \land (p \land q) \equiv 
            p \land q$ % Correcta
            
            \choice Ninguna de las anteriores. 
        \end{checkboxes}

        % Question 09
        \question
        {
            Liliana está haciendo su práctica de sistemas operativos y su 
            programa no logra hacer lo que ella quiere. Cansada, decepcionada
            y triste de la situación (y de la vida), decide recurrir a los 
            aprendizajes que adquirió en el curso de Estructuras Discretas y 
            decide escribir las siguientes afirmaciones:
            \begin{itemize}
                \item El procesador C está funcionando y el procesador B no 
                está funcionando. 
                \item El procesador A está funcionando si y sólo si el 
                procesador B no está funcionando. 
                \item Al menos uno de los dos procesadores A y B no está 
                funcionando. 
            \end{itemize} 

            De acuerdo a lo anterior, ¿cuál o cuáles de las siguientes 
            expresiones son \textbf{verdaderas}?

        }
        \begin{checkboxes}
            \choice \texttt{''Los procesadores A y B están funcionando''} 
            \textbf{no} es consecuencia lógica de las tres hipótesis mencionadas 
            anteriormente. % Correcta

            \choice \texttt{''Los procesadores D y B están funcionando''} es 
            consecuencia lógica de las tres hipótesis mencionadas anteriormente.

            \choice \texttt{''Los procesadores A y C están funcionando''} es 
            consecuencia lógica de las tres hipótesis mencionadas anteriormente. % Correcta

            \choice \texttt{''Los procesadores A y B están funcionando''} 
            es consecuencia lógica de las tres hipótesis mencionadas 
            anteriormente. 
            
            \choice Ninguna de las anteriores. 
        \end{checkboxes}

        \newpage
        % Question 10
        \question{¿Cuál o cuáles de las siguientes expresiones son 
        \textbf{verdaderas}?}
        \begin{checkboxes}
            \choice Las proposiciones
            \begin{align*}
                p &= \text{Mi equipo favorito ganará si le grito a la TV} \\ 
                q &= \text{Mi equipo favorito ganará sólo si le grito a la TV} \\ 
            \end{align*}

            son lógicamente equivalentes. 

            \choice $(p \lor q) \lor (p \lor r) \equiv \neg r \rightarrow 
            (p \lor q)$ % Correcta 

            \choice La negación de la proposición
            \begin{align*}
                \textbf{''Si no te arrepientes tu alma se condenará''}
            \end{align*}

            es ''Te arrepientes y tu alma se condenará''. 

            \choice Si $\phi \equiv \varphi$ y unimos ambas fórmulas mediante un 
            conectivo condicional, entonces la fórmula resultante es una 
            contingencia. 

            \choice Ninguna de las anteriores. 
        \end{checkboxes}
        
        % Question 11
        \question{¿Cuál o cuáles de las siguientes expresiones son 
        \textbf{verdaderas}?}
        \begin{checkboxes}
            \choice El argumento 
            \begin{equation*}
                p \rightarrow q, p \lor r, \neg (r \land s) / \therefore
                (p \rightarrow q) \rightarrow q \lor \neg s
            \end{equation*}

            es correcto, y esto es gracias a la siguiente justificación: Para 
            que una interpretación $\mathcal{I}$ haga falsa a la conclusión, 
            debe cumplir con que $\mathcal{I}(p) = \mathcal{I}(q) = 0$ e 
            $\mathcal{I}(s) = 1$. Entonces, para que $\mathcal{I}$ haga 
            verdadera a $p \lor r$, se necesita que $\mathcal{I}(r) = 1$. Pero 
            entonces $\neg (r \land s)$ evalúa a falso. Así, toda interpretación
            que haga falsa a la conclusión debe de hacer falsa a almenos una 
            de las premisas. % Correcta

            \choice El argumento 
            \begin{equation*}
                p \lor q, \neg (p \land r), \neg q / \therefore 
                r \rightarrow s
            \end{equation*}

            es incorrecto, y esto es gracias a la siguiente justificación: 
            Cualquier interpretación $\mathcal{I}$ que haga falsa a la 
            conclusión debe de cumplir que $\mathcal{I}(r) = 1$ e 
            $\mathcal{I}(s) = 0$. En este caso, para que $\neg (p \land r)$
            evalúe a verdadero, se necesita que $\mathcal{I}(p) = 0$.
            Además, para que $p \lor q$ evalúe a verdadero, tiene que suceder 
            que $\mathcal{I}(q) = 1$. Sin embargo, esto último implica que 
            $\neg q$ evalúa a falso, y por lo tanto, este argumento es 
            incorrecto.

            \choice El enunciado 
            \begin{center}
                \textbf{''Dormir a las 10:30 es una condición necesaria para 
                que yo pueda levantarme temprano''}
            \end{center}

            es lógicamente equivalente al enunciado
            \begin{center}
                \textbf{''Si me duermo a las 10:30 entonces yo podré 
                levantarme temprano''}
            \end{center}

            \choice El conjunto 
            \begin{equation*}
                \Gamma = \{p \lor q \lor r, \neg (r \lor \neg s), 
                s \leftrightarrow t, p \rightarrow \neg t, q \rightarrow
                p \lor \neg t\}
            \end{equation*}

            es insatisfacible, y esto es gracias a la siguiente justificación: 
            Supongamos que existe una interpretación $\mathcal{I}$ que 
            satisface a $\Gamma$. Entonces se tiene que $\mathcal{I}(\neg (r 
            \lor \neg s)) = 1$, por lo que $\mathcal{I}(r) = 0$ e 
            $\mathcal{I}(s) = 1$. Así, $\mathcal{I}(t) = 1$ por 
            $\mathcal{I}(s \leftrightarrow t) = 1$. Además, $\mathcal{I}(p)
            = 0$ por $\mathcal{I}(p \rightarrow \neg t) = 1$. Luego, 
            $\mathcal{I}(q) = 0$ por $\mathcal{I}(q \rightarrow p \lor \neg t)
            = 1$. Pero con estos valores, $\mathcal{I}(p \lor q \lor r) = 0$. 
            Por lo tanto, nuestro conjunto no es satisfacible. % Correcta 
            
            \choice Ninguna de las anteriores. 
        \end{checkboxes}

        \newpage
        % Question 12
        \question{¿Cuál o cuáles de las siguientes expresiones son 
        \textbf{verdaderas}?}
        \begin{checkboxes}
            \choice La negación del enunciado
            \begin{center}
                \textbf{''Las flores florecerán sólo si llueve''}
            \end{center}

            es \texttt{''Las flores florecen y no llueve''} % Correcta

            \choice Si la negación de una fórmula $\varphi$ es una tautología, entonces
            $\varphi$ tiene que ser lógicamente equivalente con $p \land \neg p$. 
            % Correcta

            \choice La contrapositiva del enunciado
            \begin{center}
                \textbf{''La impresora está lenta solo si el archivo está 
                dañado''}
            \end{center}

            es \texttt{''Si la impresora está lenta, entonces el archivo 
            está dañado'}

            \choice La negación del enunciado
            \begin{center}
                \textbf{''Si aún te amo, volveré a extrañarte''}
            \end{center}

            es \texttt{''Si no vuelvo a extrañarte, entonces ya no te amo''} 

            \choice Ninguna de las anteriores.
        \end{checkboxes}

        % Question 13
        \question{¿Cuál o cuáles de las siguientes expresiones son 
        \textbf{verdaderas}?}
        \begin{checkboxes}
            \choice Nubecita le pregunta al profesor Odín si aprobó el examen 
            y el profesor le responde:
            \begin{center}
                \textbf{''Tanto Iván como Silvia son buenos o Carlos es bueno.
                Iván es malo o Silvia no es buena. Carlos es malo o Nubecita 
                aprobó el examen.''}
            \end{center} 

            Después de pensarlo por unos instantes, Nubecita deduce que 
            efectivamente aprobó el examen y este razonamiento es correcto. 
            % Correcta 

            \choice El argumento 
            \begin{equation*}
                p \rightarrow q, r \lor s, \neg s \rightarrow \neg t, \neg q 
                \lor s, \neg s, \neg p \land r \rightarrow u, w \lor t / \therefore
                u \land w
            \end{equation*}

            es correcto, y esto es gracias a la siguiente justificación (usando el método de refutación): 
            De las premisas abemos que $\mathcal{I}(\neg s) 
            = 1$, lo que implica que $\mathcal{I}(r) = \mathcal{I}(\neg q) 
            = 1$ por $\mathcal{I}(r \lor s) = 1$ y $\mathcal{I}(\neg q \lor 
            s) = 1$, respectivamente. Así, $\mathcal{I}(p) = 0$ por 
            $\mathcal{I}(p \rightarrow q) = 1$ y esto obliga a que 
            $\mathcal{I}(u) = 1$ por $\mathcal{I}(\neg p \land r \rightarrow
            u) = 1$. Luego, $\mathcal{I}(\neg t) = 1$ por $\mathcal{I}(\neg s 
            \rightarrow \neg t) = 1$, lo que hace que $\mathcal{I}(w) = 1$, 
            por $\mathcal{I}(w \lor t)$; pero esto obliga a que 
            $(\neg u \lor \neg w) = 0$, lo cual es una contradicción. Por 
            lo tanto, el argumento es correcto. % Correcta 

            \choice El argumento 
            \begin{equation*}
                \neg p \rightarrow r \land \neg s, t \rightarrow s, 
                u \rightarrow \neg p, \neg w, u \lor w / \therefore
                \neg t
            \end{equation*}

            es correcto, y esto es gracias a la siguiente justificación:
            Cualquier interpretación $\mathcal{I}$ que haga falsa a la 
            conclusión debe de cumplir que $\mathcal{I}(t) = 1$. En este
            caso, para que $u \lor w$ evalúe a verdadero se necesita que 
            $\mathcal{I}(u) = 1$, pues $\mathcal{I}(\neg w) = 1$. Luego, 
            $\mathcal{I}(s) = \mathcal{I}(\neg p) = 1$ pues 
            $\mathcal{I}(t \rightarrow s) = 1$ y $\mathcal{I}(u \rightarrow
            \neg p) = 1$. Esto implica que $\mathcal{I}(s) = 0$, por 
            $\mathcal{I}(\neg p \rightarrow r \land \neg s) = 1$, exhibiendo una
            contradicción. Así, toda interpretación que haga falsa a la
            conclusión debe de hacer 
            falsa a almenos una de las premisas. % Correcta 

            \choice Si $\phi \equiv \varphi$, entonces $\neg (\phi \leftrightarrow 
            \varphi)$ tiene por lo menos un modelo que la satisface. 

            \choice Ninguna de las anteriores.
        \end{checkboxes}

        \newpage
        % Question 14
        \question
        {
            Supongamos que el siguiente enunciado es verdadero:
            \begin{center}
                \textbf{''Si mi agüita está hirviendo, entonces su temperatura
                debe ser de al menos 80°C''}
            \end{center}

            De acuerdo a lo anterior, ¿cuál o cuáles de los siguientes 
            enunciados deben ser también \textbf{verdaderos}?
        }

        \begin{checkboxes}
            \choice Una condición necesaria para que mi agüita hierva es que 
            su temperatura sea por lo menos de $80$°C. % Correcta
            
            \choice Una condición suficiente para que mi agüita hierva es que 
            su temperatura sea por lo menos de $80$°C. 
            
            \choice La negación del enunciado es 
            \begin{center}
                ''Mi agüita no está hierviendo porque su temperatura no es 
                de $80$°C''
            \end{center}

            \choice La negación de la contrapositiva del enunciado es 
            \begin{center}
                ''Mi agüita está hirviendo y su temperatura es menor que 
                $80$°C''
            \end{center} % Correcta 

            \choice Ninguna de las anteriores.
        \end{checkboxes}

        % Question 15
        \question{¿Cuál o cuáles de las siguientes expresiones son 
        \textbf{verdaderas}?}
        \begin{checkboxes}
            \choice El enunciado 
            \begin{center}
                \textbf{''Ser divisible entre tres es una condición 
                necesaria para que un número sea divisible entre nueve''}
            \end{center}

            es lógicamente equivalente al enunciado 
            \begin{center}
                \textbf{''Si un número es divisible entre nueve, entonces 
                también es divisible entre tres.''}
            \end{center} % Correcta 

            \choice El enunciado 
            \begin{center}
                \textbf{''No estudiar con regularidad es una 
                condición suficiente para que Nubecita no apruebe el curso''}
            \end{center}

            es lógicamente equivalente al enunciado 
            \begin{center}
                \textbf{''Si Nubecita aprueba el curso, entonces estudió 
                con regularidad''}
            \end{center} % Correcta 

            \choice El enunciado 
            \begin{center}
                \textbf{''Una condición suficiente para que Laura tome el 
                curso de Análisis de Algoritmos es que apruebe el curso 
                de Estructuras Discretas''}
            \end{center}

            es lógicamente equivalente al enunciado 
            \begin{center}
                \textbf{''Si Laura no aprueba el curso de estructuras 
                discretas, entonces no puede tomar el curso de Análisis de 
                Algoritmos''}
            \end{center}

            \choice La formalización de la proposición
            \begin{center}
                \textbf{''Si hoy es lunes o no está lloviendo ni hace calor y 
                no es lunes, entonces hoy es lunes o no hace calor ni está 
                lloviendo''}
            \end{center}

            es $(p \lor (\neg p \land \neg (q \lor r))) \rightarrow (p \lor 
            \neg (r \lor q))$, donde 
            \begin{align*}
                p &= \text{Hoy es lunes} \\ 
                q &= \text{Está lloviendo} \\ 
                r &= \text{Hace calor}
            \end{align*} % Correcta

            \choice Ninguna de las anteriores.
        \end{checkboxes}
    \end{questions}
\end{document}

\begin{comment}
    %%%%%%%%%%%%%%%%%%%%%%%%%%%%%%%%%%%%%%%%%%%%%%%%%%%%%%%%%%%%%%%%%%%%%%%
    \question
    {
        Tenemos el siguiente argumento:
        \begin{center}
            \textbf{''Si hoy me siento sola y Pedro me terminó, entonces 
            Pedro es un imbécil; y si hoy no me siento sola entonces 
            me quiero mucho. Más aún, sabemos que no me quiero mucho 
            y que Pedro me terminó; luego entonces Pedro es un 
            imbécil.''}
        \end{center}

        y las siguientes proposiciones:
        \begin{align*}
            p &= \text{Hoy me siento sola} \\ 
            q &= \text{Pedro me terminó} \\ 
            r &= \text{Pedro es un imbécil} \\ 
            s &= \text{Me quiero mucho}
        \end{align*}

        De acuerdo a lo anterior, ¿cuál o cuáles de las siguientes 
        expresiones son \textbf{verdaderas}?
    }
    \begin{checkboxes}
        \choice La formalización del argumento es $\Gamma = \{(p \land q 
        \rightarrow r) \land (\neg p \rightarrow s), \neg s \land q, r\}$. 
        Y como no tenemos una conclusión explícita, entonces no es posible 
        determinar si el argumento es correcto o falso. 

        \choice La formalización del argumento es $\Gamma = \{p \land q 
        \rightarrow r, \neg p \rightarrow s, \neg s \land q\} \models r$.
        Usando el método de refutación, mostraremos que el argumento es 
        incorrecto. Como $\mathcal{I}(\neg s \land q) = 1$, entonces 
        $\mathcal{I}(s) = 0$ y $\mathcal{I}(q) = 1$. Así, $\mathcal{I}(p) 
        = 1$ ya que $\mathcal{I}(\neg p \rightarrow s) = 1$. Esto implica 
        que $\mathcal{I}(p \land q) = 1$ y $\mathcal{I}(r) = 0$, por 
        $\mathcal{I}(p \land q \rightarrow r) = 1$. Y como no hemos 
        logrado encontrar una contradicción (pues llegamos a que el 
        valor de verdad de $r$ es efectivamente falso), entonces el 
        argumento es incorrecto. 

        \choice La formalización del argumento es $\Gamma = \{p \land q 
        \rightarrow r, \neg p \rightarrow s, \neg s \land q\} \models r$. 
        Usando el método de refutación, mostraremos que el argumento es 
        correcto. Como $\mathcal{I}(\neg s \land q) = 1$, entonces 
        $\mathcal{I}(s) = 0$ y $\mathcal{I}(q) = 1$. Así, $\mathcal{I}(p) 
        = 1$ ya que $\mathcal{I}(\neg p \rightarrow s) = 1$. Luego, 
        $\mathcal{I}(p \land q \rightarrow r) = 1$ y como $\mathcal{I}(r)
        = 0$, entonces $p \land q$ debería de evaluar a falso, pero 
        sabemos que $\mathcal{I}(p) = \mathcal{I}(q) = 1$, lo cual nos 
        genera una contradicción. Por lo tanto, el argumento es 
        correcto. % Correcta 

        \choice La formalización del argumento es $\Gamma = \{(p \land q 
        \rightarrow r) \land (\neg p \rightarrow s), \neg s \land q\} 
        \models r$. Usando el método de refutación, mostraremos que el
        argumento es incorrecto. Como $\mathcal{I}((p \land q 
        \rightarrow r) \land (\neg p \rightarrow s)) = 1$, entonces 
        $\mathcal{I}(p \land q \rightarrow r) = \mathcal{I}(\neg p 
        \rightarrow s) = 1$. Como $\mathcal{I}(r) = 0$, entonces 
        $\mathcal{I}(p) = \mathcal{I}(q) = 0$ ya que $\mathcal{I}(p 
        \land q \rightarrow r) = 1$. Luego $\mathcal{I}(s) = 1$ por 
        $\mathcal{I}(\neg p \rightarrow s) = 1$ y aún así se sigue 
        cumpliendo que $\mathcal{I}(\neg s \land q) = 1$. Por lo 
        tanto, como no hemos podido encontrar alguna contradicción, 
        entonces el argumento es incorrecto.  
        
        \choice Ninguna de las anteriores. 
    \end{checkboxes}
    %%%%%%%%%%%%%%%%%%%%%%%%%%%%%%%%%%%%%%%%%%%%%%%%%%%%%%%%%%%%%%%%%%%%%%%
    \question{¿Cuál o cuáles de las siguientes expresiones son 
    \textbf{verdaderas}?}
    \begin{checkboxes}
        \choice El conjunto
        \begin{equation*}
            \Gamma = \{p \rightarrow q, p \lor r \land s, q \rightarrow t\}
        \end{equation*}

        es satisfacible. El modelo $\mathcal{I}(p) = \mathcal{I}(q) = 
        \mathcal{I}(r) = \mathcal{I}(s) = \mathcal{I}(t) = 1$ % Correcta

        \choice La fórmula $(p \lor q) \land \neg p \land \neg q$ es una 
        contradicción. % Correcta

        \choice La fórmula
        \begin{equation*}
            \neg p \lor q \rightarrow (p \land r \leftrightarrow s \land t 
            \rightarrow u \lor p)
        \end{equation*} 
        
        es satisfacible. El modelo $\mathcal{I}$ tal que $\mathcal{I}(p) = 1, 
        \mathcal{I}(q) = \mathcal{I}(r) = \mathcal{I}(s) = \mathcal{I}(t)
        = \mathcal{I}(u) = 0$ hace que la expresión evalúe a verdadero. 
        % Correcta

        \choice 
        
        \choice Ninguna de las anteriores. 
    \end{checkboxes}
    %%%%%%%%%%%%%%%%%%%%%%%%%%%%%%%%%%%%%%%%%%%%%%%%%%%%%%%%%%%%%%%%%%%%%%%
    \question{¿Cuál o cuáles de las siguientes expresiones son 
    \textbf{verdaderas}?}
    \begin{checkboxes}
        \choice $((p \land q) \lor (q \lor r)) \equiv (((p \land q) \lor q)
        \lor r)$ % Correcta

        \choice La fórmula $\neg p \lor q \rightarrow q$ es una 
        tautología.

        \choice Dos fórmulas proposicionales $p$ y $q$ son lógicamente 
        equivalentes si y sólo si $p \leftrightarrow q$ es satisfacible. 

        \choice El valor de verdad de una proposición compuesta depende 
        del valor de verdad de cada una de las proposiciones atómicas 
        que la componen. % Correcta
        
        \choice Ninguna de las anteriores. 
    \end{checkboxes}
    %%%%%%%%%%%%%%%%%%%%%%%%%%%%%%%%%%%%%%%%%%%%%%%%%%%%%%%%%%%%%%%%%%%%%%%
    \question{¿Cuál o cuáles de las siguientes expresiones son 
    \textbf{verdaderas}?}
    \begin{checkboxes}
        \choice Existe una fórmula proposicional que incluye las variables 
        proposicionales $p,q$ y $r$ que cumpla con la siguiente condición:
        \begin{center}
            <<La fórmula debe ser satisfacible cuando únicamente una de 
            $p,q$ y $r$ es verdadera. En caso contrario, la proposición
            no es satisfacible.>>
        \end{center} % Correcta

        \choice Existe una fórmula proposicional que incluye las variables 
        proposicionales $p,q$ y $r$ que cumpla con la siguiente condición:
        \begin{center}
            <<La fórmula debe ser satisfacible cuando $p,q$ y $r$ sean 
            falsas. En caso contrario, la proposición no es satisfacible.>>
        \end{center} % Correcta 

        \choice La fórmula $p \lor \neg p$ es una tautología. % Correcta

        \choice $p \rightarrow q \lor r \equiv p \land \neg q 
        \rightarrow r$ % Correcta 
        
        \choice Ninguna de las anteriores. 
    \end{checkboxes}
    %%%%%%%%%%%%%%%%%%%%%%%%%%%%%%%%%%%%%%%%%%%%%%%%%%%%%%%%%%%%%%%%%%%%%%%
    \question{¿Cuál o cuáles de las siguientes expresiones son 
    \textbf{verdaderas}?}
    \begin{checkboxes}
        \choice La fórmula $p \lor \neg r \rightarrow p \land q \lor r$
        es una contingencia. % Correcta

        \choice La fórmula $(p \lor q) \land \neg p \land \neg q$ es una 
        contradicción. % Correcta

        \choice La fórmula $p \rightarrow q \leftrightarrow \neg p 
        \lor q$ es una tautología. % Correcta

        \choice La fórmula $p \lor q \lor \neg p$ es una contradicción. 

        \choice Existe una fórmula proposicional que incluye las variables 
        proposicionales $p,q$ y $r$ que cumpla con la siguiente condición:
        \begin{center}
            <<La fórmula debe ser satisfacible cuando exactamente dos de 
            $p,q$ y $r$ sean verdaderas.>>
        \end{center} % Correcta 

        \choice Existe una fórmula proposicional que incluye las variables 
        proposicionales $p,q$ y $r$ que cumpla con la siguiente condición:
        \begin{center}
            <<La fórmula debe ser satisfacible cuando $p$ y $q$ sean 
            verdaderas y $r$ falsa.>>
        \end{center} % Correcta 
        
        \choice Ninguna de las anteriores. 
    \end{checkboxes}
    %%%%%%%%%%%%%%%%%%%%%%%%%%%%%%%%%%%%%%%%%%%%%%%%%%%%%%%%%%%%%%%%%%%%%%%
    \question{¿Cuál o cuáles de las siguientes expresiones son 
    \textbf{verdaderas}?}
    \begin{checkboxes}
        \choice La fórmula $p \leftrightarrow \neg p$ es una 
        contradicción. % Correcta 

        \choice $p \lor q \equiv \neg (\neg p \lor \neg q)$

        \choice La fórmula $p \land q \lor p \land \neg q \lor \neg p$ es 
        una tautología. % Correcta 
        
        \choice Ninguna de las anteriores. 
    \end{checkboxes}
    %%%%%%%%%%%%%%%%%%%%%%%%%%%%%%%%%%%%%%%%%%%%%%%%%%%%%%%%%%%%%%%%%%%%%%%
    \question{¿Cuál o cuáles de las siguientes expresiones son 
    \textbf{verdaderas}?}
    \begin{checkboxes}
        \choice Una fórmula proposicional $\varphi$ tiene $2^n$ valores 
        de verdad, donde $n$ es el número de variables proposicionales 
        que contenga $\varphi$. % Correcta  

        \choice $(p \rightarrow q) \rightarrow (r \rightarrow s) \equiv 
        (p \rightarrow r) \rightarrow (q \rightarrow s)$

        \choice $p \land q \rightarrow r \equiv (p \rightarrow r) \land 
        (q \rightarrow r)$

        \choice La negación de una proposición compuesta que es una 
        tautología es insatisfacible. % Correcta 
        
        \choice Ninguna de las anteriores. 
    \end{checkboxes}
    %%%%%%%%%%%%%%%%%%%%%%%%%%%%%%%%%%%%%%%%%%%%%%%%%%%%%%%%%%%%%%%%%%%%%%%
    \question{¿Cuál o cuáles de las siguientes expresiones son 
    \textbf{verdaderas}?}
    \begin{checkboxes}
        \choice $\neg (p \leftrightarrow q) \equiv \neg p \leftrightarrow q$
        % Correcta 

        \choice $p \land (q \lor r) \equiv \neg (p \land q) \rightarrow 
        (p \land r)$

        \choice La fórmula $(p \lor q) \land \neg p \rightarrow q$ es 
        una contradicción. 

        \choice La fórmula $((p \lor q) \land (p \rightarrow r) \land 
        (q \rightarrow r)) \rightarrow r$ es una tautología. % Correcta 
        
        \choice Ninguna de las anteriores. 
    \end{checkboxes}
    %%%%%%%%%%%%%%%%%%%%%%%%%%%%%%%%%%%%%%%%%%%%%%%%%%%%%%%%%%%%%%%%%%%%%%%
    \question
    {
        Pablo está preocupado porque perdió el dinero que su mamá le había 
        dado para comprar las tortillas. Él se puso a pensar en el recorrido 
        de su casa a la tortillería, y llegó a las siguientes conclusiones 
        que son verdaderas:
        \begin{itemize}
            \item[i.] Si yo estaba a un lado de la casa de mi vecina Adela, 
            entonces el dinero no está cerca de la tortillería. 

            \item[ii.] Si la fila de la tortillería era muy larga, entonces 
            el dinero está cerca de la tortillería.

            \item[iii.] Yo estaba a un lado de la casa de mi vecina Adela.

            \item[iv.] La fila de la tortillería era muy larga o el dinero 
            está cerca de mi casa. 

            \item[v.] Si en la fila de la tortillería estaba mi primo, 
            entonces el dinero está a un lado del árbol que está afuera 
            de mi casa. 
        \end{itemize}

        La formalización de estas proposiciones en forma de un conjunto de 
        premisas es
        \begin{equation*}
            \Gamma = \{p \rightarrow \neg q, r \rightarrow q, p, r \lor s, 
            t \rightarrow u\}
        \end{equation*}

        donde 
        \begin{align*}
            p &: \text{Yo estaba a un lado de la casa de mi vecina Adela} \\
            q &: \text{El dinero está cerca de la tortillería}\\ 
            r &: \text{La fila de la tortillería era muy larga} \\
            s &: \text{El dinero está cerca de mi casa} \\ 
            t &: \text{En la fila de la tortillería estaba mi primo} \\ 
            u &: \text{El dinero está a un lado del árbol que está afuera 
            de mi casa}
        \end{align*}

        De acuerdo a lo anterior, ¿cuál o cuáles de las siguientes 
        afirmaciones es \textbf{verdadera}?
    }
    \begin{checkboxes}
        \choice El dinero está cerca de la tortillería. 

        \choice El dinero está cerca de la casa de Pablo. % Correcta

        \choice El dinero no está cerca de la tortillería. % Correcta

        \choice El dinero está a un lado del árbol que está afuera de la 
        casa de Pablo. 
        
        \choice Ninguna de las anteriores.
    \end{checkboxes}
    %%%%%%%%%%%%%%%%%%%%%%%%%%%%%%%%%%%%%%%%%%%%%%%%%%%%%%%%%%%%%%%%%%%%%%%
    \end{comment}