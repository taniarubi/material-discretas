\documentclass[12pt, a4paper]{exam}

% Soporte para cambiar la fecha que sale en el examen
\usepackage{advdate}
% Soporte para escribir en varias columnas
\usepackage{multicol}
% Soporte para los acentos.
\usepackage[utf8]{inputenc} 
\usepackage[T1]{fontenc}    
% Idioma español.
\usepackage[spanish,mexico,es-tabla]{babel}
\usepackage{graphicx}
\usepackage{tikz}
\usepackage{amsmath,amssymb,amsthm}

% Cambiamos los márgenes del documento. 
\usepackage[top=1.5cm,left=1.5cm,right=1.5cm]{geometry}

% Pie de página
\cfoot{Página \thepage\ de \numpages}

%%%%%%%%%%%%%%%%%%%%%%%%%%%%%%%%%%%%%%%%%%%%%%%%%%%%%%%%%%%%%%%%%%%%%%%%%%%%%%
\renewcommand{\thechoice}{\alph{choice}}

\makeatletter
\renewenvironment{checkboxes}%
   {\setcounter{choice}{0}\list{\checkbox@char}%
      {%
        \settowidth{\leftmargin}{W.\hskip\labelsep\hskip 2.5em}%
        \def\choice{%
          \if@correctchoice
            \color@endgroup \endgroup
          \fi
          \stepcounter{choice}
          \item[\checked@char]
          \do@choice@pageinfo
        } % choice
        \def\CorrectChoice{%
          \if@correctchoice
            \color@endgroup \endgroup
          \fi
          \ifprintanswers
            % We can't say \choice here, because that would
            % insert an \endgroup.
            % 2016/05/10: We say \color@begingroup in addition to
            % \begingroup in case \CorrectChoiceEmphasis involves color
            % and the text exactly fills the line (which would
            % otherwise create a blank line after this choice):
            % 2016/05/11: We leave hmode if we're in it,
            % i.e., if there's no blank line preceding this
            % \CorrectChoice command.  (Without this, the
            % \special created by a \color{whatever} command that might
            % be inserted by \CorrectChoice@Emphasis would be appended 
            % to the previous \choice, which could cause an extra
            % (blank) line to be inserted before this \CorrectChoice.)
            % Since \par and \endgraf seem to cancel \@totalleftmargin
            % (for reasons I don't understand), we'll do the following:
            % Motivated by  the def of \leavevmode, 
            %      \def\leavevmode{\unhbox\voidb@x}
            % we will now leave hmode (if we're in hmode):
            \ifhmode \unskip\unskip\unvbox\voidb@x \fi
            \begingroup \color@begingroup \@correctchoicetrue
            \CorrectChoice@Emphasis
            \stepcounter{choice}
            \item[\checked@char]
          \else
            \stepcounter{choice}
            \item[\checked@char]
          \fi
          \do@choice@pageinfo
        } % CorrectChoice
        \let\correctchoice\CorrectChoice
        \labelwidth\leftmargin\advance\labelwidth-\labelsep
        \topsep=0pt
        \partopsep=0pt
        \checkboxeshook
      }%
   }%
   {\if@correctchoice \color@endgroup \endgroup \fi \endlist}
 \makeatother

% Make checkbox character a circle with the letter
\checkboxchar{\tikz[baseline={([yshift=-.8ex]current bounding box.center)}]\node[shape=circle,minimum size=4mm,draw] at (0,0) {\thechoice};}
% Make checked box character bold WITH surd
%\checkedchar{\tikz[baseline={([yshift=-.8ex]current bounding box.center)}]\node[shape=circle,minimum size=8mm,draw] at (0,0) {} node at (0,0) {\thechoice\llap{$\surd$}};}
% Make checked box character bold
\checkedchar{\tikz[baseline={([yshift=-.8ex]current bounding box.center)}]\node[shape=circle,minimum size=4mm,draw] at (0,0) {} node at (0,0) {\thechoice};}
\printanswers
%%%%%%%%%%%%%%%%%%%%%%%%%%%%%%%%%%%%%%%%%%%%%%%%%%%%%%%%%%%%%%%%%%%%%%%%%%%%%%

\begin{document}
    %%%%%%%%%%%%%%%%%%%%%%%%%%%%%%%%%%%%%%%%%%%%%%%%%%%%%%%%%%%%%%%%%%%%%%%%%%%%%%%
    %%%%%%%%%%%%%%%%%%%%%%%%%%%%%%%% ENCABEZADO %%%%%%%%%%%%%%%%%%%%%%%%%%%%%%%%%%%
    \centering
    \hrule \hrule \hrule 
    \vspace{5mm}
    \begin{minipage}[c]{0.8\textwidth}
        \begin{center}
            {\large\textbf{Mission 06, Start!} \par
            \large \textbf{Estructuras Discretas} \par
            \large \textbf{Semestre 2023-1} \par
            \large \textbf{\today}	\par}
        \end{center}
    \end{minipage}

    \vspace{0.2in}
    \noindent
    \textbf{Tania Michelle Rubí Rojas}
    \vspace{2mm}
    \hrule \hrule \hrule 
    %%%%%%%%%%%%%%%%%%%%%%%%%%%%%%%%%%%%%%%%%%%%%%%%%%%%%%%%%%%%%%%%%%%%%%%%%%%%%%%
    %%%%%%%%%%%%%%%%%%%%%%%%%%%%%%%%%%%%%%%%%%%%%%%%%%%%%%%%%%%%%%%%%%%%%%%%%%%%%%%

    \vspace{5mm}
    \noindent
    Nombre y número de cuenta: \hrulefill\

    \vspace{5mm}
    \noindent
    
    \textbf{Notación y convenciones para el examen:}
    {\tiny
    \begin{multicols}{2}
    \begin{itemize}\setlength\itemsep{0em}  
      \item $0\in\mathbb{N}$

      \item Dos números enteros son primos relativos  si no tienen ningún 
      divisor en común a excepción del $1$.

      \item El conjunto de los números primos es
      \begin{equation*}
        \{2,3,5,7,11,13,17,19,23,29, 31, 37, \ldots\}
      \end{equation*}

      \item Los errores de escritura en las funciones son {\bf intencionales}, 
      por lo que cualquier afirmación que contenga una expresión mal escrita 
      es falsa.
    \end{itemize}
    \end{multicols}
    }

    \begin{questions}
        % Question 01
        \question
        {
            Sea $A = \{a,b,c,d,e\}$. Definimos la relación $R$ sobre $A$ como 
            sigue:
            \begin{equation*}
                R = \{(a,b), (a,c), (a,e), (b,a), (b,c), (c,a), (c,b), (d,a), 
                (e,d)\}
            \end{equation*} 

            ¿Cuál o cuáles de las siguientes relaciones son la \textbf{cerradura
            transitiva} de $R$?
        }
        \begin{checkboxes}
            \choice $R \cup \{(a,a), (a,d), (b,b), (b,e), (b,d), (c,c), (c,e), 
            (c,d), (d,b), (d,c), (d,e), (d,d), (e,a),$ \\ $(e,c), (e,e)\}$

            \choice $R \cup \{(a,a), (a,d), (b,b), (b,d), (c,c), (c,e), (d,b), 
            (d,c), (d,e), (d,d), (e,a), (e,b), (e,c),$ \\ $(e,e)\}$

            \choice $R \cup \{(a,a), (a,d), (b,a), (b,b), (b,e), (b,d), (c,c), 
            (c,e), (c,d), (d,b), (d,c), (d,e), (d,d),$ \\ $(d,a), (e,a), (e,b), 
            (e,c), (e,e)\}$

            \choice $R \cup \{(a,a), (a,d), (b,b), (b,e), (b,d), (c,c), (c,e), 
            (c,d), (d,b), (d,c), (d,e), (d,d), (e,a),$ \\ $(e,b), (e,c), 
            (e,e)\}$ % Correcta
            
            \choice Ninguna de las anteriores. 
        \end{checkboxes}

        % Question 02
        \question
        {
            Definimos las relaciones $R$ y $S$ sobre el conjunto 
            $\mathbb{N}$, como sigue:
            \begin{align*}
                R &= \{(a,b) \in \mathbb{N} \times \mathbb{N} \; | \; 
                a+1 = b\} \\ 
                S &= \{(a,b) \in \mathbb{N} \times \mathbb{N} \; | \; 
                a = b+2\}
            \end{align*}

            ¿Cuál o cuáles de las siguientes expresiones son 
            \textbf{verdaderas}?
        }
        \begin{checkboxes}
            \choice $R \circ S = \{(a,b) \in \mathbb{N} \times \mathbb{N} 
            \; | \; a = b + 1\}$

            \choice $R \cap S = \varnothing$ % Correcta 

            \choice $(R \circ S) \subset (S \circ R)$ % Correcta

            \choice $S \circ R = \{(a,b) \in \mathbb{N} \times \mathbb{N} 
            \; | \; a = b + 1\}$ % Correcta
            
            \choice Ninguna de las anteriores. 
        \end{checkboxes}

        \newpage
        % Question 03
        \question
        {
            Definimos la relación $R \subseteq \{0,1\} \times \{0,1\}$ como 
            $R = \varnothing$. ¿Cuál o cuáles de las siguientes expresiones 
            son \textbf{verdaderas}?
        }
        \begin{checkboxes}
            \choice $R$ es antisimétrica. % Correcta

            \choice $R$ no es simétrica. 

            \choice La cerradura transitiva de $R$ es $R \cup \{(0,0), 
            (1,1)\}$.

            \choice $R$ es reflexiva. 
            
            \choice Ninguna de las anteriores. 
        \end{checkboxes}

        % Question 04
        \question{¿Cuál o cuáles de las siguientes expresiones son 
        \textbf{verdaderas}?}
        \begin{checkboxes}
            \choice Sea $R$ una relación binaria sobre un conjunto $A$. 
            Entonces $R$ es transitiva si y sólo si $R \circ R = R$. 

            \choice Sea $R$ una relación binaria sobre un conjunto $A$.
            Entonces $R \cup R^{-1}$ es la cerradura simétrica de $R$. 
            % Correcta

            \choice Si $R$ es antisimétrica, entonces $R^{-1}$ también lo es. 
            % Correcta

            \choice Sea $R$ una relación binaria sobre un conjunto $A$. 
            Entonces $R \cup \{(x,x) \; | \; x \in A\}$ es la cerradura 
            reflexiva de $R$. % Correcta
            
            \choice Ninguna de las anteriores. 
        \end{checkboxes}

        % Question 05
        \question
        {
            Sea $P$ el conjunto de todas las personas. Definimos la relación 
            $R$ sobre $P$ como sigue:
            \begin{equation*}
                aRb \Leftrightarrow a \text{ y } b \text{ tienen el mismo 
                apellido materno}
            \end{equation*}

            ¿Cuál o cuáles de las siguientes expresiones son 
            \textbf{verdaderas}?
        }
        \begin{checkboxes}
            \choice $R$ es antisimétrica. 

            \choice $R$ es reflexiva. % Correcta

            \choice No es posible obtener la cerradura transitiva de la 
            relación $R$. 

            \choice $R$ no es simétrica. 
            
            \choice Ninguna de las anteriores. 
        \end{checkboxes}

        % Question 06
        \question
        {
            Sea $P$ el conjunto de todas las personas. Definimos la relación 
            $R$ sobre $P$ como sigue:
            \begin{equation*}
                aRb \Leftrightarrow a \text{ es hermana biológica de } b
            \end{equation*}
            
            ¿Cuál o cuáles de las siguientes expresiones son 
            \textbf{verdaderas}?
        }
        \begin{checkboxes}
            \choice $R$ es antisimétrica. 

            \choice $R$ es simétrica. 

            \choice Si $P$ es el conjunto de todas las personas que son hombres,
            entonces $R = \varnothing$. % Correcta

            \choice $R$ es antirreflexiva. % Correcta
            
            \choice Ninguna de las anteriores. 
        \end{checkboxes}

        \newpage
        % Question 07
        \question{¿Cuál o cuáles de las siguientes expresiones son 
        \textbf{verdaderas}?}
        \begin{checkboxes}
            \choice Sea $R$ una relación binaria sobre un conjunto $A$. 
            Entonces $R$ es simétrica si y sólo si $R^{-1}$ también lo es. 
            % Correcta

            \choice Si $R$ y $S$ son relaciones binarias transitivas sobre un 
            conjunto $A$, entonces $R \cap S$ también lo es. % Correcta 

            \choice Si $R$ y $S$ son relaciones binarias reflexivas sobre un 
            conjunto $A$, entonces $R \cup S$ es reflexiva sobre $A$. % Correcta

            \choice Sea $R$ una relación binaria sobre un conjunto $A$. 
            Entonces $R$ es transitiva si y sólo si $R^{-1}$ también lo es.
            % Correcta
            
            \choice Ninguna de las anteriores. 
        \end{checkboxes}

        % Question 08
        \question{¿Cuál o cuáles de las siguientes expresiones son 
        \textbf{verdaderas}?}
        \begin{checkboxes}
            \choice Si $R$ y $S$ son relaciones binarias transitivas sobre un 
            conjunto $A$, entonces $R \cup S$ también lo es. 

            \choice Sea $R$ una relación binaria transitiva sobre un conjunto 
            $A$. Entonces $R$ es asimétrica si y sólo si $R$ es antirreflexiva.
            % Correcta 

            \choice Si $R$ y $S$ son relaciones binarias simétricas sobre un 
            conjunto $A$, entonces $R \cup S$ también lo es. % Correcta

            \choice Si $R$ es una relación reflexiva y transitiva sobre un 
            conjunto $A$, entonces $R^2 = R$. % Correcta 
            
            \choice Ninguna de las anteriores. 
        \end{checkboxes}

        % Question 09
        \question
        {
            Sea $A = \mathbb{Z}^+ \times \mathbb{Z}^+$. Definimos la relación 
            $R$ sobre $A$, como sigue:
            \begin{equation*}
                R = \{((a,b), (c,d)) \; | \; a+d = b+c\}
            \end{equation*}

            ¿Cuál o cuáles de las siguientes expresiones son 
            \textbf{verdaderas}?
        }
        \begin{checkboxes}
            \choice $R$ no es antisimétrica. % Correcta

            \choice $R$ es simétrica. % Correcta

            \choice $R$ no es reflexiva. 

            \choice La cerradura transitiva de $R$ es $R^2$. 
            
            \choice Ninguna de las anteriores. 
        \end{checkboxes}

        % Question 10
        \question
        {
            Definimos las relaciones $R$ y $S$ sobre $\mathbb{N}$ como sigue:
            \begin{align*}
                R &= \{(0,2), (0,5), (0,9), (1,9), (1,12), (1,15), (2,2)\} \\ 
                S &= \{(2,0), (2,6), (5,6), (9,8), (12,1), (12,7), (15,4)\}
            \end{align*}

            ¿Cuál o cuáles de las siguientes expresiones son \textbf{verdaderas}?
        }
        \begin{checkboxes}
            \choice $(S \circ R)^{-1} = \{(0,0), (6,0), (8,0), (8,1), (1,1),  
            (7,1), (0,2), (6,2), (4,2)\}$

            \choice $(S \circ R)^{-1} = \{(0,0), (6,0), (8,0), (8,1), (1,1),  
            (2,6), (7,1), (0,2), (6,2)\}$

            \choice $(S \circ R)^{-1} = \{(0,0), (0,2), (1,1), (4,1), (6,0), 
            (6,2), (7,1), (8,0), (8,1)\}$ % Correcta

            \choice $(S \circ R)^{-1} = \{(0,0), (0,2), (1,1), (4,1), (6,0), 
            (2,6), (7,1), (8,0), (8,1)\}$
            
            \choice Ninguna de las anteriores. 
        \end{checkboxes}

        \newpage
        % Question 11
        \question{¿Cuál o cuáles de las siguientes expresiones son 
        \textbf{verdaderas}?}
        \begin{checkboxes}
            \choice Es posible que una relación binaria, no vacía y definida 
            sobre un conjunto no vacío $A$, sea simétrica y antisimétrica al 
            mismo tiempo. % Correcta 

            \choice Si $R$ y $S$ son relaciones binarias simétricas sobre un 
            conjunto $A$, entonces $R \cap S$ también lo es. % Correcta

            \choice Toda relación binaria que no es reflexiva es antirreflexiva.
            
            \choice Sea $R$ una relación binaria simétrica y transitiva sobre 
            un conjunto $A$. Entonces $R$ también es reflexiva. 

            \choice Ninguna de las anteriores. 
        \end{checkboxes}

        % Question 12
        \question
        {
            Definimos la relación $R$ sobre $\mathbb{N}$ como sigue:
            \begin{equation*}
                R = \{(a,b) \; | \; a \text{ y } b \text{ son primos 
                relativos}\}
            \end{equation*}

            ¿Cuál o cuáles de las siguientes expresiones son 
            \textbf{verdaderas}?
        }
        \begin{checkboxes}
            \choice $R$ es transitiva. 

            \choice $R$ es antisimétrica. 

            \choice $R$ es reflexiva.

            \choice $R$ es simétrica. % Correcta
            
            \choice Ninguna de las anteriores. 
        \end{checkboxes}

        % Question 13
        \question{¿Cuál o cuáles de las siguientes expresiones 
        son \textbf{verdaderas}?}
        \begin{checkboxes}
            \choice Sea $R$ una relación binaria sobre un conjunto $A$. 
            Entonces existe un algoritmo para determinar la cerradura 
            antisimétrica de $R$. 

            \choice Sea $R$ una relación binaria sobre un conjunto $A$. 
            Si $R$ es antirreflexiva, entonces $R$ es asimétrica. 

            \choice Sea $R$ una relación binaria sobre un conjunto $A$. 
            Entonces $R$ es asimétrica si y sólo si $R \cap R^{-1} = 
            \varnothing$. % Correcta 

            \choice Sea $R$ una relación binaria sobre un conjunto $A$. 
            Si $R$ es simétrica y transitiva, entonces $R$ es 
            antirreflexiva. 
            
            \choice Ninguna de las anteriores. 
        \end{checkboxes}

        % Question 14
        \question
        {
            Definimos la relación $R$ sobre $A = \{1,2,3,4,6,12,13,17,19,23,24,
            26,29\}$ como sigue:
            \begin{equation*}
                R = \{(a, b) \; | \; \frac{b}{a} \text{ es un número primo}\}
            \end{equation*}
            
            ¿Cuál o cuáles de las siguientes relaciones son la 
            \textbf{cerradura transitiva} de $R$?
        }
        \begin{checkboxes}
            \choice $R \cup \{(4,1), (6,1), (12,3), (12,2), (24,6), (24,4), 
            (26,1), (24,3)\}^{-1}$

            \choice $R \cup \{(4,1), (6,1), (12,3), (12,2), (24,6), (24,4), 
            (26,1), (12,1), (24,3), (24,2), (24,1)\}^{-1}$ % Correcta

            \choice $R \cup \{(4,1), (6,1), (12,3), (12,2), (24,6), (24,4), 
            (26,1), (12,1), (24,3)\}^{-1}$

            \choice $R \cup \{(4,1), (6,1), (12,3), (12,2), (24,6), (24,4), 
            (26,1), (24,3), (24,2), (24,1)\}^{-1}$
            
            \choice Ninguna de las anteriores. 
        \end{checkboxes}

        \newpage
        % Question 15
        \question{¿Cuál o cuáles de las siguientes expresiones son 
        \textbf{verdaderas}?}
        \begin{checkboxes}
            \choice Sea $R$ una relación binaria sobre un conjunto $A$. Si $R$
            es simétrica, entonces $R \circ R$ también lo es. % Correcta 

            \choice Sea $R$ una relación binaria sobre un conjunto $A$. 
            Entonces $R$ es reflexiva si y sólo si $\{(x,x) \; | \; x \in A\} 
            \subseteq R$. % Correcta 

            \choice Sea $R$ una relación binaria sobre un conjunto $A$. Si $R$ 
            es relexiva y transitiva, entonces $R \cap R^{-1}$ es reflexiva, 
            simétrica y transitiva. % Correcta  

            \choice Sea $R$ una relación binaria sobre un conjunto $A$. 
            Entonces $R$ es reflexiva si y sólo si $R \subseteq \{(x,x) \; | \; 
            x \in A\}$. 
            
            \choice Ninguna de las anteriores. 
        \end{checkboxes}

        % Question 16
        \question{¿Cuál o cuáles de las siguientes expresiones son 
        \textbf{verdaderas}?}
        \begin{checkboxes}
            \choice Toda relación antisimétrica no es simétrica. 

            \choice Sea $R$ una relación binaria sobre un conjunto $A$. 
            Entonces $R$ es simétrica si y sólo si $R = R^{-1}$. % Correcta

            \choice Toda relación antirreflexiva sobre un conjunto no vacío no 
            es reflexiva. % Correcta

            \choice Una relación binaria $R$ sobre un conjunto $A$ puede ser 
            antisimétrica y asimétrica al mismo tiempo. % Correcta
            
            \choice Ninguna de las anteriores. 
        \end{checkboxes}

        % Question 17
        \question
        {
            Sea $A = \{a,b,c,d,e\}$. Definimos la relación $R$ sobre $A$ como 
            sigue:
            \begin{equation*}
                R = \{(a,e), (b,a), (b,d), (c,d), (d,a), (d,c), (e,a), (e,b), 
                (e,c), (e,e)\}
            \end{equation*} 

            ¿Cuál o cuáles de las siguientes relaciones son la \textbf{cerradura
            transitiva} de $R$?
        }
        \begin{checkboxes}
            \choice $R \cup \{(a,a), (a,b), (a,c), (a,e), (b,e), (b,b), (b,c), 
            (c,a), (c,c), (d,e), (d,b), (d,c), (d,d),$ \\ $(e,e),(e,b), (e,c), 
            (e,d)\}$

            \choice $R \cup \{(a,a), (a,b), (a,c), (b,e), (b,b), (b,c), (c,a), 
            (c,c), (d,e), (d,b), (d,c), (d,d), (e,e),$ \\ $(e,b), (e,c), 
            (e,d)\}$ 

            \choice $R \cup \{(a,a), (a,b), (a,c), (b,e), (b,b), (b,c), (c,a), 
            (c,c), (d,e), (d,b), (d,c), (d,d), (e,e),$ \\ $(e,b), (e,c)\}$

            \choice $R \cup \{(a,a), (a,b), (a,c), (b,a), (b,e), (b,b), (b,c), 
            (c,a), (c,c), (d,e), (d,b), (d,c), (d,d),$ \\ $(e,e),(e,b), (e,c), 
            (e,d)\}$
            
            \choice Ninguna de las anteriores. % Correcta
        \end{checkboxes}
    \end{questions}
\end{document}