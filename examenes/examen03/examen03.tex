\documentclass[12pt, a4paper]{exam}

% Soporte para cambiar la fecha que sale en el examen
\usepackage{advdate}
% Soporte para escribir en varias columnas
\usepackage{multicol}
% Soporte para los acentos.
\usepackage[utf8]{inputenc} 
\usepackage[T1]{fontenc}    
% Idioma español.
\usepackage[spanish,mexico,es-tabla]{babel}
\usepackage{graphicx}
\usepackage{tikz}
\usepackage{amsmath,amssymb,amsthm}

% Cambiamos los márgenes del documento. 
\usepackage[bottom=2.6cm,top=1.2cm,left=1.5cm,right=1.5cm]{geometry}

% Pie de página
\cfoot{Página \thepage\ de \numpages}

%%%%%%%%%%%%%%%%%%%%%%%%%%%%%%%%%%%%%%%%%%%%%%%%%%%%%%%%%%%%%%%%%%%%%%%%%%%%%%
\renewcommand{\thechoice}{\alph{choice}}

\makeatletter
\renewenvironment{checkboxes}%
   {\setcounter{choice}{0}\list{\checkbox@char}%
      {%
        \settowidth{\leftmargin}{W.\hskip\labelsep\hskip 2.5em}%
        \def\choice{%
          \if@correctchoice
            \color@endgroup \endgroup
          \fi
          \stepcounter{choice}
          \item[\checked@char]
          \do@choice@pageinfo
        } % choice
        \def\CorrectChoice{%
          \if@correctchoice
            \color@endgroup \endgroup
          \fi
          \ifprintanswers
            % We can't say \choice here, because that would
            % insert an \endgroup.
            % 2016/05/10: We say \color@begingroup in addition to
            % \begingroup in case \CorrectChoiceEmphasis involves color
            % and the text exactly fills the line (which would
            % otherwise create a blank line after this choice):
            % 2016/05/11: We leave hmode if we're in it,
            % i.e., if there's no blank line preceding this
            % \CorrectChoice command.  (Without this, the
            % \special created by a \color{whatever} command that might
            % be inserted by \CorrectChoice@Emphasis would be appended 
            % to the previous \choice, which could cause an extra
            % (blank) line to be inserted before this \CorrectChoice.)
            % Since \par and \endgraf seem to cancel \@totalleftmargin
            % (for reasons I don't understand), we'll do the following:
            % Motivated by  the def of \leavevmode, 
            %      \def\leavevmode{\unhbox\voidb@x}
            % we will now leave hmode (if we're in hmode):
            \ifhmode \unskip\unskip\unvbox\voidb@x \fi
            \begingroup \color@begingroup \@correctchoicetrue
            \CorrectChoice@Emphasis
            \stepcounter{choice}
            \item[\checked@char]
          \else
            \stepcounter{choice}
            \item[\checked@char]
          \fi
          \do@choice@pageinfo
        } % CorrectChoice
        \let\correctchoice\CorrectChoice
        \labelwidth\leftmargin\advance\labelwidth-\labelsep
        \topsep=0pt
        \partopsep=0pt
        \checkboxeshook
      }%
   }%
   {\if@correctchoice \color@endgroup \endgroup \fi \endlist}
 \makeatother

% Make checkbox character a circle with the letter
\checkboxchar{\tikz[baseline={([yshift=-.8ex]current bounding box.center)}]\node[shape=circle,minimum size=4mm,draw] at (0,0) {\thechoice};}
% Make checked box character bold WITH surd
%\checkedchar{\tikz[baseline={([yshift=-.8ex]current bounding box.center)}]\node[shape=circle,minimum size=8mm,draw] at (0,0) {} node at (0,0) {\thechoice\llap{$\surd$}};}
% Make checked box character bold
\checkedchar{\tikz[baseline={([yshift=-.8ex]current bounding box.center)}]\node[shape=circle,minimum size=4mm,draw] at (0,0) {} node at (0,0) {\thechoice};}
\printanswers
%%%%%%%%%%%%%%%%%%%%%%%%%%%%%%%%%%%%%%%%%%%%%%%%%%%%%%%%%%%%%%%%%%%%%%%%%%%%%%

\begin{document}
    %%%%%%%%%%%%%%%%%%%%%%%%%%%%%%%%%%%%%%%%%%%%%%%%%%%%%%%%%%%%%%%%%%%%%%%%%%%%%%%
    %%%%%%%%%%%%%%%%%%%%%%%%%%%%%%%% ENCABEZADO %%%%%%%%%%%%%%%%%%%%%%%%%%%%%%%%%%%
    \centering
    \hrule \hrule \hrule 
    \vspace{5mm}
    \begin{minipage}[c]{0.8\textwidth}
        \begin{center}
            {\large\textbf{Mission 03, Start!} \par
            \large \textbf{Estructuras Discretas} \par
            \large \textbf{Semestre 2023-1} \par
            \large \textbf{\today}	\par}
        \end{center}
    \end{minipage}

    \vspace{0.2in}
    \noindent
    \textbf{Tania Michelle Rubí Rojas}
    \vspace{2mm}
    \hrule \hrule \hrule 
    %%%%%%%%%%%%%%%%%%%%%%%%%%%%%%%%%%%%%%%%%%%%%%%%%%%%%%%%%%%%%%%%%%%%%%%%%%%%%%%
    %%%%%%%%%%%%%%%%%%%%%%%%%%%%%%%%%%%%%%%%%%%%%%%%%%%%%%%%%%%%%%%%%%%%%%%%%%%%%%%

    \vspace{5mm}
    \noindent
    Nombre y número de cuenta: \hrulefill\

    \vspace{5mm}
    \noindent
    
    \textbf{Notación y convenciones para el examen:}
    {\tiny
    \begin{multicols}{2}
    \begin{itemize}\setlength\itemsep{0em}  
      \item $0\in\mathbb{N}$
      \item $S=\{x | x\text{ es una cadena de 0's y 1's}\} = \{0,1,10,11,100,101,110,111,...\}\cup\{00,000,0000,...\}$
      \item Sea $x\in S$, diremos que $0x$ es el resultado de ``agregar un $0$'' al inicio\\ de $x$, por ejemplo: si $x=101$ entonces $0x=0101$
      \item Salvo la mencionada en la viñeta anterior, no hay ninguna operación definida para el conjunto $S$
      \item En la CdMx una persona menor de edad puede comprarse un automóvil.
      \item En la CdMx existe la copropiedad de automóviles.
    \end{itemize}
    \end{multicols}
    }
    \begin{questions}
        % Question 01
        \question{¿Cuál o cuáles de las siguientes expresiones son 
        \textbf{verdaderas}?}
        \begin{checkboxes}
            \choice Sea $R \subseteq \{1,2,3\} \times \{1,2,3\}$ donde 
            $R = \{(1,1), (2,3), (3,1), (2,1)\}$. Entonces $R$ es una 
            función. 

            \choice Toda función es una relación. % Correcta

            \choice Cuando una función es suprayectiva, su imagen es igual a 
            su codominio. % Correcta

            \choice Una función puede tener la misma salida para más de una 
            entrada. % Correcta 
            
            \choice Ninguna de las anteriores. 
        \end{checkboxes}

        % Question 02
        \question
        {
            Sea $S$ el conjunto de todas las cadenas de ceros y unos. 
            Definimos la función $f: S \rightarrow \mathbb{Z}$ como 
            sigue:
            \begin{equation*}
                f(s) = (\text{número de unos en } s) - (\text{ número de 
                ceros en } s)
            \end{equation*}

            De acuerdo a esta información, ¿cuál o cuáles de las siguientes 
            expresiones son \textbf{verdaderas}?
        }
        \begin{checkboxes}
            \choice El codominio de $f$ es el conjunto de números naturales. 

            \choice El dominio de $f$ es un subconjunto propio de $S$.

            \choice $f$ {\bf no es} inyectiva. % Correcta

            \choice $f$ {\bf no es} suprayectiva. 
            
            \choice Ninguna de las anteriores. 
        \end{checkboxes}

        % Question 03
        \question
        {
            Sea $P$ el conjunto de todas las personas en la CDMX y sea $A$ el 
            conjunto de todos los automóviles. Definimos la función 
            $f: P \rightarrow A$ como sigue:
            \begin{equation*}
                f(p) = \text{ el último coche que se ha comprado } p 
            \end{equation*}

            De acuerdo a esta información, ¿cuál o cuáles de las siguientes 
            expresiones son \textbf{verdaderas}?
        }
        \begin{checkboxes}
            \choice $f$ es suprayectiva. 

            \choice $f^{-1}(x) = \text{ la última persona que ha comprado } x$

            \choice $f$ es inyectiva. 

            \choice El dominio de $f$ es el conjunto de todas las personas 
            mayores de $18$ años.  
            
            \choice Ninguna de las anteriores. % Correcta
        \end{checkboxes}

        \newpage
        % Question 04
        \question
        {
            Cuando Odín le pide a Nubecita dar la definición de función 
            inyectiva, ella responde lo siguiente: <<Una función 
            $f: X \rightarrow Y$ es inyectiva si y sólo si cada elemento de 
            $X$ se envía mediante $f$ a exactamente un elemento de $Y$.>>
        }

        ¿La respuesta de Nubecita es correcta?
        \begin{checkboxes}
            \choice Claro que es correcta, pues esta definición nos garantiza 
            que cada elemento {\bf del dominio} está relacionado con sólo un 
            elemento {\bf del codominio}.

            \choice Qué oso, claro que {\bf no es correcta}. Esa definición permite  
            que algunos elementos {\bf del codominio} no puedan corresponderse con 
            algún elemento {\bf del dominio}. 

            \choice Claro que es correcta, pues esta definición nos garantiza 
            que no puede existir más de un elemento en el dominio que tenga 
            la misma imagen. 

            \choice Qué oso, claro que {\bf no es correcta}. Esa definición permite 
            que un elemento en el {\bf codominio} esté relacionado con dos o más 
            elementos en el {\bf dominio}. % Correcta 
            
            \choice Ninguna de las anteriores. 
        \end{checkboxes}

        % Question 05
        \question
        {
            Sea $S$ el conjunto de todas las cadenas de ceros y unos. 
            Definimos la función $f: S \rightarrow \mathbb{N}$ como 
            sigue:
            \begin{equation*}
                f(s) = \text{ la longitud de } s
            \end{equation*}

            De acuerdo a esta información, ¿cuál o cuáles de las siguientes 
            expresiones son \textbf{verdaderas}?
        }
        \begin{checkboxes}
            \choice $f$ {\bf no es} suprayectiva. % Correcta

            \choice La imagen de $f(0101)$ es $2$. 

            \choice El dominio de $f$ es el conjunto de todos los números 
            naturales. 

            \choice $f$ {\bf no es} inyectiva. % Correcta
            
            \choice Ninguna de las anteriores. 
        \end{checkboxes}
        
        % Question 06
        \question{¿Cuál o cuáles de las siguientes expresiones son 
        \textbf{verdaderas}?}
        \begin{checkboxes}
            \choice Dos funciones $f$ y $g$ pueden ser iguales si sus dominios
            y codominios no coinciden.  

            \choice No toda función es una relación.

            \choice Sean $A$ y $B$ dos conjuntos cualesquiera. Entonces para 
            toda $f: A \rightarrow B$ se tiene que $f \circ I_A = f$. % Correcta

            \choice Sean $A, B$ y $C$ conjuntos cualesquiera. Sean 
            $f: A \rightarrow B$ y $g: B \rightarrow C$ funciones. Si 
            $g \circ f$ es suprayectiva, entonces $g$ es suprayectiva. % Correcta
            
            \choice Ninguna de las anteriores. 
        \end{checkboxes}

        % Question 07
        \question
        {
            Sea $S$ el conjunto de todas las cadenas de ceros y unos. 
            Definimos la función $f: S \rightarrow \mathbb{N}$ como 
            sigue:
            \begin{equation*}
                f(s) = \text{ el número de ceros en } s
            \end{equation*}

            De acuerdo a esta información, ¿cuál o cuáles de las siguientes 
            expresiones son \textbf{verdaderas}?
        }
        \begin{checkboxes}
            \choice $f$ {\bf no es} suprayectiva. 

            \choice $f^{-1}(s) = \text{ el número de unos en } s$

            \choice $f$ es inyectiva. 

            \choice El codominio de $f$ son todos los números enteros. 
            
            \choice Ninguna de las anteriores. % Correcta
        \end{checkboxes}

        % Question 08
        \question
        {
            Sean $A$ y $B$ conjuntos finitos cuya cardinalidad es $n$ y $m$, 
            respectivamente. Supongamos que $f: A \rightarrow B$ es una 
            función. De acuerdo a esta información, ¿cuál o cuáles de las 
            siguientes expresiones son \textbf{verdaderas}?
        }
        \begin{checkboxes}
            \choice Si $n \leq m$, entonces $f$ debe ser inyectiva.  

            \choice Si $n > m$, entonces $f$ debe ser suprayectiva. 

            \choice Si $f$ es inyectiva, entonces $n < m$. 

            \choice Si $f$ es suprayectiva, entonces $n \geq m$. % Correcta
            
            \choice Ninguna de las anteriores. 
        \end{checkboxes}

        % Question 09
        \question
        {
            Sea $A = \{a,b,c,d\}$ donde el universo del discurso es 
            $\mathcal{U} = \{\varnothing, a, b, c, d, \{\varnothing\}\}$. 
            Definimos la función $f: \mathcal{P}(A) \times \mathcal{P}(A) 
            \rightarrow \mathcal{P}(A)$ como sigue:
            \begin{equation*}
                f(A,B) = A^c - (A \cap B)
            \end{equation*}

            De acuerdo a esta información, ¿cuál o cuáles de las siguientes 
            expresiones son \textbf{verdaderas}?
        }
        \begin{checkboxes}
            \choice El codominio de $f$ es $\mathcal{P}(A)$. % Correcta

            \choice $f$ es inyectiva.

            \choice La imagen de $f(\{a,b,d\}, \{a,c\})$ es $\{\varnothing, 
            \{\varnothing\}, c\}$. % Correcta

            \choice $f$ es suprayectiva. % Correcta
            
            \choice Ninguna de las anteriores. 
        \end{checkboxes}
        
        % Question 10
        \question{¿Cuál o cuáles de las siguientes expresiones son 
        \textbf{verdaderas}?}
        \begin{checkboxes}
            \choice No todas las relaciones son funciones. % Correcta

            \choice Sean $A = \{1,2\}$ y $B = \{1,3,5,7\}$. Si $R \subseteq
            A \times B$, entonces $R = \{(1,3), (1,5), (2,3), (2,5)\}$ es 
            una función. 

            \choice Dos funciones son iguales solo si ambas tienen la 
            misma regla de correspondencia. 

            \choice Sean $A = \{8,11,2,5\}$ y $B = \{1\}$. Si $S \subseteq
            A \times B$, entonces $S = \{(2,1), (5,1), (8,1), (11,1)\}$ no 
            es una función. 
            
            \choice Ninguna de las anteriores. 
        \end{checkboxes}

        % Question 11
        \question
        {
            Sean $P$ el conjunto de todas las personas y $T$ el conjunto de 
            {\underline todos} los zapatos. Definimos la relación $R \subseteq P \times T$ 
            como sigue:
            \begin{equation*}
                R = \{(x,y) \; | \; \text{ y es el zapato izquierdo } 
                \text{que lleva puesto } x\}
            \end{equation*}

            De acuerdo a esta información, ¿cuál o cuáles de las siguientes 
            expresiones son \textbf{verdaderas}?
        }
        \begin{checkboxes}
            \choice $R$ {\bf no es} una función. % Correcta

            \choice $R^{-1} = \{(y,x) \; | \; x \text{ es el zapato izquierdo 
            que lleva puesto } y\}$ 

            \choice El codominio de $R$ es el conjunto de todos los zapatos 
            izquierdos. 

            \choice $R$ es inyectiva. 
            
            \choice Ninguna de las anteriores. 
        \end{checkboxes}

        % Question 12
        \question
        {
            Sea $S$ el conjunto de todos los subconjuntos finitos de enteros 
            positivos. Definimos la función $f: \mathbb{N} \rightarrow S$ como 
            sigue:
            \begin{equation*}
                f(n) = \text{ el conjunto de todos los divisores positivos de } n
            \end{equation*}

            De acuerdo a esta información, ¿cuál o cuáles de las siguientes 
            expresiones son \textbf{verdaderas}?
        }
        \begin{checkboxes}
            \choice $f$ es suprayectiva.

            \choice $f$ es inyectiva. 

            \choice El codominio de $f$ es $\mathbb{N}$. 

            \choice $f$ no es función. % Correcta
            
            \choice Ninguna de las anteriores.
        \end{checkboxes} 

        % Question 13
        \question
        {
            Definimos la función $f: \mathbb{R} \times \mathbb{R} \rightarrow 
            \mathbb{R} \times \mathbb{R}$ como sigue:
            \begin{equation*}
                f(x,y) = (2y, -x)
            \end{equation*}

            De acuerdo a esta información, ¿cuál o cuáles de las siguientes 
            expresiones son \textbf{verdaderas}?
        }
        \begin{checkboxes}
            \choice $f$ no tiene inversa. 

            \choice $f$ {\bf no es} inyectiva. 

            \choice El dominio de $f$ es $\mathbb{R}^2$. % Correcta

            \choice $f$ es suprayectiva. % Correcta
            
            \choice Ninguna de las anteriores. 
        \end{checkboxes}
        
        % Question 14
        \question{¿Cuál o cuáles de las siguientes expresiones son 
        \textbf{verdaderas}?}
        \begin{checkboxes}
            \choice Sean $A$ y $B$ conjuntos. Si $f: A \rightarrow B$ es una 
            función inyectiva y suprayectiva con función inversa 
            $f^{-1}: B \rightarrow A$, entonces $f^{-1} \circ f = I_A$. % Correcta

            \choice {\bf no es} posible realizar composición de funciones definidas 
            sobre conjuntos infinitos. 

            \choice Sean $A, B$ y $C$ conjuntos. Si $f: A \rightarrow B$ y 
            $g: B \rightarrow C$ son ambas funciones inyectivas, entonces 
            $g \circ f$ es inyectiva. % Correcta

            \choice Sean $A$ y $B$ conjuntos. Si $f: A \rightarrow B$ es 
            suprayectiva, entonces cada elemento en $B$ es la imagen de un 
            elemento de $A$. % Correcta
            
            \choice Ninguna de las anteriores. 
        \end{checkboxes}

        % Question 15 
        \question{¿Cuál o cuáles de las siguientes expresiones son 
        \textbf{verdaderas}?}
        \begin{checkboxes}
            \choice Sea $f$ una función. Si cada elemento {\bf del dominio} de $f$ 
            tiene una imagen, entonces $f$ debe de ser suprayectiva.  

            \choice Sea $f$ una función. Si cada elemento {\bf del codominio} de 
            $f$ tiene una imagen , entonces $f$ debe de ser suprayectiva.

            \choice Toda relación es una función. 

            \choice Sean $A$ y $B$ conjuntos. Sea $f: A \rightarrow B$ una 
            función. Si $|A| > |B|$ entonces $f$ no puede ser inyectiva. 
            % Correcta
            
            \choice Ninguna de las anteriores. 
        \end{checkboxes}

        % Question 16
        \question
        {
            Definimos la función $f: \mathcal{P}(\{1,2,3\}) \rightarrow 
            \mathbb{N}$ como sigue:
            \begin{equation*}
                f(A) = |A|
            \end{equation*}

            De acuerdo a esta información, ¿cuál o cuáles de las siguientes 
            expresiones son \textbf{verdaderas}?
        }
        \begin{checkboxes} 
            \choice $f$ es suprayectiva. 

            \choice La inversa de $f$ es una función. 

            \choice $f \circ f = \{(\{1\}, 1), (\{2\}, 1), (\{3\}, 1), 
            (\{1,2\}, 2), (\{1,3\}, 2), (\{2,3\}, 2)\}$

            \choice $f$ {\bf no es} inyectiva. % Correcta
            
            \choice Ninguna de las anteriores. 
        \end{checkboxes}

        % Question 17
        \question
        {
            Sea $S$ el conjunto de todas las cadenas de ceros y unos. 
            Definimos la función $f: S \rightarrow S$ como sigue:
            \begin{equation*}
                f(s) = 0s
            \end{equation*}

            De acuerdo a esta información, ¿cuál o cuáles de las siguientes 
            expresiones son \textbf{verdaderas}?
        }
        \begin{checkboxes}
            \choice La función inversa de $f$ existe. 

            \choice $f$ es inyectiva. % Correcta

            \choice La imagen de $f$ es $S$. 

            \choice $f$ {\bf no es} suprayectiva. % Correcta
            
            \choice Ninguna de las anteriores. 
        \end{checkboxes}
        
        % Question 18
        \question
        {
            Definimos la función $f: \mathbb{N} \rightarrow \mathbb{N}$ como 
            sigue:
            \begin{equation*}
                f(n) = \text{ la suma de {\underline todos} los divisores positivos de } n
            \end{equation*}

            De acuerdo a esta información, ¿cuál o cuáles de las siguientes 
            expresiones son \textbf{verdaderas}?
        }
        \begin{checkboxes}
            \choice $f$ {\bf no es} suprayectiva. % Correcta. 

            \choice $(f \circ f)(n) = \text{ el número de divisores positivos 
            de } n$

            \choice $f^{-1}$ es una función. 

            \choice $f$ {\bf no es} inyectiva. % Correcta 
            
            \choice Ninguna de las anteriores. 
        \end{checkboxes}

        % Question 19
        \question
        {
            Sea $S$ el conjunto de todas las cadenas de ceros y unos. Definimos 
            la función $f: S \rightarrow S$ como sigue:
            \begin{equation*}
                f(s) = \text{ la cadena } s \text{ escrita al revés}
            \end{equation*}

            De acuerdo a esta información, ¿cuál o cuáles de las siguientes 
            expresiones son \textbf{verdaderas}?
        }
        \begin{checkboxes}
            \choice El dominio de $f$ es $\{0,1\}$. 

            \choice $f$ {\bf no es} inyectiva. 

            \choice $f$ {\bf no es} suprayectiva.

            \choice El dominio y el codominio de $f$ son diferentes al dominio 
            y codominio de $f^{-1}$. 
            
            \choice Ninguna de las anteriores. % Correcta
        \end{checkboxes}

        % Question 20
        \question
        {
            Sea $P$ el conjunto de todas las personas. Definimos la 
            función $f: P \rightarrow \{x \; | \; 0 < x < 32\}$ como sigue:
            \begin{equation*}
                f(p) = \text{ el día de nacimiento de } p
            \end{equation*}

            De acuerdo a esta información, ¿cuál o cuáles de las siguientes 
            expresiones son \textbf{verdaderas}?
        }
        \begin{checkboxes}
            \choice $f$ es suprayectiva. % Correcta

            \choice $f$ es inyectiva. 

            \choice $f^{-1}$ es una función. 

            \choice La imagen de $f$ es $\{1, 2, 3, \ldots, 30, 31\}$ % Correcta
            
            \choice Ninguna de las anteriores. 
        \end{checkboxes}

        % Question 21
        \question
        {
            Sea $S$ el conjunto de todas las cadenas de ceros y unos. Definimos  
            la función $f: S \rightarrow S$ como sigue:
            \begin{equation*}
                f(s) = s+1
            \end{equation*}

            De acuerdo a esta información, ¿cuál o cuáles de las siguientes 
            expresiones son \textbf{verdaderas}?
        }
        \begin{checkboxes}
            \choice $f$ es inyectiva.

            \choice $f^{-1}$ es una función.

            \choice $f$ {\bf no es} suprayectiva.

            \choice El codominio de $f$ es $\mathbb{N}$.
            
            \choice Ninguna de las anteriores. % Correcta
        \end{checkboxes}
        
        % Question 22
        \question{¿Cuál o cuáles de las siguientes expresiones son 
        \textbf{verdaderas}?}
        \begin{checkboxes}
            \choice Sean $A, B$ y $C$ conjuntos cualesquiera. Sean 
            $f: A \rightarrow B$ y $g: B \rightarrow C$ funciones. Si 
            $g \circ f$ es inyectiva, entonces $f$ es inyectiva. % Correcta

            \choice Si $f$ es una función suprayectiva, entonces el imagen de 
            $f$ y el codominio de $f$ son disjuntos. 

            \choice $f$ es biyectiva si y sólo si $f^{-1}$ existe. % Correcta

            \choice Si $f$ es una función inyectiva, entonces {\bf no es posible} 
            que dos elementos en el dominio de $f$ correspondan con un mismo 
            elemento en el codominio de $f$. % Correcta
            
            \choice Ninguna de las anteriores. 
        \end{checkboxes}

        % Question 23
        \question{¿Cuál o cuáles de las siguientes expresiones son 
        \textbf{verdaderas}?}
        \begin{checkboxes}
            \choice Dos funciones $f$ y $g$ con el mismo dominio y codominio
            son iguales si la imagen de cada elemento {\bf del dominio} es igual bajo 
            $f$ que bajo $g$. % Correcta

            \choice Sean $A$ y $B$ dos conjuntos cualesquiera. Entonces para 
            toda $f: A \rightarrow B$ se tiene que $I_B \circ f = f$. % Correcta

            \choice La composición de funciones cualesquiera es conmutativa. 

            \choice Siempre es posible obtener la función inversa de una función 
            $f$. 
            
            \choice Ninguna de las anteriores. 
        \end{checkboxes}
    \end{questions}
\end{document}