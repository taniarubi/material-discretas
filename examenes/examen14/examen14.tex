\documentclass[oneside]{style}

\title{Versión 01}
\principal{Examen 14}
\author{Tania Michelle Rubí Rojas}
\semester{Semestre 2023-1}

\begin{document}
\maketitle

Para cada uno de los siguientes ejercicios, \textbf{justifica ampliamente} tu 
respuesta:

\begin{questions}[label=\protect\circled{\bfseries\arabic*}]
    % Ejercicio 01
    \question
    {
        Definimos el conjunto de árboles binarios no vacíos $\mathcal{T}$ 
        cuyos nodos están etiquetados por elementos de un conjunto $A$
        como sigue:
        \begin{itemize}
            \item Si $r \in A$, entonces \texttt{tree(void,r,void)} $\in 
            \mathcal{T}$.

            \item Si $T_1, T_2 \in \mathcal{T}$ y $e \in A$, entonces 
            \texttt{tree($T_1$,e,$T_2$)} $\in \mathcal{T}$.

            \item Estos y sólo estos elementos pertenecen a $\mathcal{T}$. 
        \end{itemize}

        Dada esta definición, \textbf{demuestra usando inducción estructural} 
        que el número de vértices ($|V|$) de un árbol binario no vacío $T$ es 
        igual al número de aristas ($|E|$) de $T$ más una unidad. 
    }

    % Ejercicio 02
    \question
    {
        Definimos el conjunto de cadenas $\mathcal{L}$ como sigue:
        \begin{itemize}
            \item $\epsilon \in \mathcal{L}$, es decir, la cadena vacía 
            pertenece al conjunto $\mathcal{L}$. 
            
            \item Si $w \in \mathcal{L}$, entonces $0w0, 1w1 \in \mathcal{L}$.
            
            \item Estos y sólo estos elementos pertenecen a $\mathcal{L}$. 
        \end{itemize}

        Dada esta definición, \textbf{demuestra usando inducción estructural}
        que $\forall \sigma \in \mathcal{L}, \; |\sigma|$ es par. 
    }text

    % Ejercicio 03
    \question
    {
        Definimos el conjunto $S \subseteq \mathbb{Z}^2$ como sigue:
        \begin{itemize}
            \item $(0,0) \in S$ 

            \item Si $(x,y) \in S$, entonces $(x,y+1), (x+1,y+1), (x+2,y+1) 
            \in S$

            \item Estos y sólo estos elementos pertenecen a $S$. 
        \end{itemize}

        Dada esta definición, \textbf{demuestra usando inducción estructural}
        que $\forall(a,b) \in S, \; a \leq 2b$. 
    }
\end{questions}
\end{document}
