\documentclass[12pt, a4paper]{exam}

% Soporte para cambiar la fecha que sale en el examen
\usepackage{advdate}
% Soporte para escribir en varias columnas
\usepackage{multicol}
% Soporte para los acentos.
\usepackage[utf8]{inputenc} 
\usepackage[T1]{fontenc}    
% Idioma español.
\usepackage[spanish,mexico,es-tabla]{babel}
\usepackage{graphicx}
\usepackage{tikz}
\usepackage{amsmath,amssymb,amsthm}

% Cambiamos los márgenes del documento. 
\usepackage[top=1.5cm,left=1.5cm,right=1.5cm]{geometry}

% Pie de página
\cfoot{Página \thepage\ de \numpages}

%%%%%%%%%%%%%%%%%%%%%%%%%%%%%%%%%%%%%%%%%%%%%%%%%%%%%%%%%%%%%%%%%%%%%%%%%%%%%%
\renewcommand{\thechoice}{\alph{choice}}

\makeatletter
\renewenvironment{checkboxes}%
   {\setcounter{choice}{0}\list{\checkbox@char}%
      {%
        \settowidth{\leftmargin}{W.\hskip\labelsep\hskip 2.5em}%
        \def\choice{%
          \if@correctchoice
            \color@endgroup \endgroup
          \fi
          \stepcounter{choice}
          \item[\checked@char]
          \do@choice@pageinfo
        } % choice
        \def\CorrectChoice{%
          \if@correctchoice
            \color@endgroup \endgroup
          \fi
          \ifprintanswers
            % We can't say \choice here, because that would
            % insert an \endgroup.
            % 2016/05/10: We say \color@begingroup in addition to
            % \begingroup in case \CorrectChoiceEmphasis involves color
            % and the text exactly fills the line (which would
            % otherwise create a blank line after this choice):
            % 2016/05/11: We leave hmode if we're in it,
            % i.e., if there's no blank line preceding this
            % \CorrectChoice command.  (Without this, the
            % \special created by a \color{whatever} command that might
            % be inserted by \CorrectChoice@Emphasis would be appended 
            % to the previous \choice, which could cause an extra
            % (blank) line to be inserted before this \CorrectChoice.)
            % Since \par and \endgraf seem to cancel \@totalleftmargin
            % (for reasons I don't understand), we'll do the following:
            % Motivated by  the def of \leavevmode, 
            %      \def\leavevmode{\unhbox\voidb@x}
            % we will now leave hmode (if we're in hmode):
            \ifhmode \unskip\unskip\unvbox\voidb@x \fi
            \begingroup \color@begingroup \@correctchoicetrue
            \CorrectChoice@Emphasis
            \stepcounter{choice}
            \item[\checked@char]
          \else
            \stepcounter{choice}
            \item[\checked@char]
          \fi
          \do@choice@pageinfo
        } % CorrectChoice
        \let\correctchoice\CorrectChoice
        \labelwidth\leftmargin\advance\labelwidth-\labelsep
        \topsep=0pt
        \partopsep=0pt
        \checkboxeshook
      }%
   }%
   {\if@correctchoice \color@endgroup \endgroup \fi \endlist}
 \makeatother

% Make checkbox character a circle with the letter
\checkboxchar{\tikz[baseline={([yshift=-.8ex]current bounding box.center)}]\node[shape=circle,minimum size=4mm,draw] at (0,0) {\thechoice};}
% Make checked box character bold WITH surd
%\checkedchar{\tikz[baseline={([yshift=-.8ex]current bounding box.center)}]\node[shape=circle,minimum size=8mm,draw] at (0,0) {} node at (0,0) {\thechoice\llap{$\surd$}};}
% Make checked box character bold
\checkedchar{\tikz[baseline={([yshift=-.8ex]current bounding box.center)}]\node[shape=circle,minimum size=4mm,draw] at (0,0) {} node at (0,0) {\thechoice};}
\printanswers
%%%%%%%%%%%%%%%%%%%%%%%%%%%%%%%%%%%%%%%%%%%%%%%%%%%%%%%%%%%%%%%%%%%%%%%%%%%%%%

\begin{document}
    %%%%%%%%%%%%%%%%%%%%%%%%%%%%%%%%%%%%%%%%%%%%%%%%%%%%%%%%%%%%%%%%%%%%%%%%%%%%%%%
    %%%%%%%%%%%%%%%%%%%%%%%%%%%%%%%% ENCABEZADO %%%%%%%%%%%%%%%%%%%%%%%%%%%%%%%%%%%
    \centering
    \hrule \hrule \hrule 
    \vspace{5mm}
    \begin{minipage}[c]{0.8\textwidth}
        \begin{center}
            {\large\textbf{Mission 05, Start!} \par
            \large \textbf{Estructuras Discretas} \par
            \large \textbf{Semestre 2023-1} \par
            \large \textbf{\today}	\par}
        \end{center}
    \end{minipage}

    \vspace{0.2in}
    \noindent
    \textbf{Tania Michelle Rubí Rojas}
    \vspace{2mm}
    \hrule \hrule \hrule 
    %%%%%%%%%%%%%%%%%%%%%%%%%%%%%%%%%%%%%%%%%%%%%%%%%%%%%%%%%%%%%%%%%%%%%%%%%%%%%%%
    %%%%%%%%%%%%%%%%%%%%%%%%%%%%%%%%%%%%%%%%%%%%%%%%%%%%%%%%%%%%%%%%%%%%%%%%%%%%%%%

    \vspace{5mm}
    \noindent
    Nombre y número de cuenta: \hrulefill\

    \vspace{5mm}
    \noindent
    
    \textbf{Notación y convenciones para el examen:}
    {\tiny
    \begin{multicols}{2}
    \begin{itemize}\setlength\itemsep{0em}  
      \item $0\in\mathbb{N}$
      \item El operador $\texttt{++}$ concatena dos listas. Ejemplo: 
      \begin{equation*}
        [a_1, \ldots, a_n] \texttt{ ++ } [b_1, \ldots, b_n] = 
        [a_1, \ldots, a_n, b_1, \ldots, b_n]
      \end{equation*}

      \item La longitud de una lista $l$ es el número de elementos que tiene 
      $l$. 
      
      \item Las variables atómicas de una fórmula proposicional $F \in 
      \texttt{LPROP}$ son \texttt{false, true} y cualquier variable 
      proposicional. 

      \item En un árbol binario, una hoja es un nodo cuyo subárbol derecho e 
      izquierdo son vacíos (\texttt{void}). 

      \item En un árbol binario, un nodo interno es cualquier nodo en el árbol 
      que no sea una hoja. 

      \item Los errores de escritura en las funciones son {\bf intencionales}, 
      por lo que cualquier afirmación que contenga una expresión mal escrita 
      es falsa.
    \end{itemize}

    Sea $\mathcal{L}_A$ el conjunto de listas con elementos en el conjunto $A$, 
    definido de la siguiente manera: 
    \begin{itemize}
        \item $[] \in \mathcal{L}_A$
        \item Si $a \in A$ y $l \in \mathcal{L}(A)$, entonces $(a:l) \in 
        \mathcal{L}_A$. 
        \item Estos y sólo estos elementos pertenecen a $\mathcal{L}_A$.  
    \end{itemize}

    \columnbreak

    Sea $\mathcal{A}_S$ el conjunto de árboles binarios con todos los nodos 
    etiquetados por elementos de un conjunto $S$, definido de la siguiente 
    manera:
    \begin{itemize}
        \item $\texttt{void} \in \mathcal{A}_S$, es decir, el árbol vacío 
        es un árbol binario. 
        \item Si $T_1$ y $T_2$ son árboles binarios y $r$ es un elemento 
        de $S$, entonces \texttt{tree}$(T_1, r, T_2) \in \mathcal{A}_S$, 
        donde $T_1$ es el subárbol izquierdo y $T_2$ es el subárbol derecho.
        Al nodo etiquetado con $r$ se le llama raíz del árbol. 
        \item Estos y sólo estos elementos pertenecen a $\mathcal{A}_S$. 
    \end{itemize}

    Sea $\mathcal{LPROP}$ el conjunto de fórmulas bien construidas de la 
    lógica proposicional, definido como sigue: 
    \begin{itemize}
        \item Una variable proposicional pertenece a $\mathcal{LPROP}$. 
        \item Las constantes lógicas \texttt{true} y \texttt{false} 
        pertenecen a $\mathcal{LPROP}$. 
        \item Si $A, B \in \mathcal{LPROP}$, entonces $(\lnot A), (A \land B), 
          (A \lor B), (A \rightarrow B) \in \mathcal{LPROP}$
        \item Estos y sólo estos elementos pertenecen a $\mathcal{LPROP}$.
    \end{itemize}
    \end{multicols}
    }

    \begin{questions}
        % Question 01
        \question
        {
            Sea \texttt{mist}$: \texttt{LPROP} \rightarrow \mathbb{Z}^+$ una 
            función recursiva que recibe una fórmula proposicional $F$, definida 
            de la siguiente manera:
            \begin{align*}
                \texttt{mist}(\texttt{true}) &= 1 \\ 
                \texttt{mist}(\texttt{false}) &= 1 \\ 
                \texttt{mist}(p) &= 1
                && \text{con $p$ una variable proposicional} \\ 
                \texttt{mist}((\neg A)) &= 1 + \texttt{mist}(A) \\ 
                \texttt{mist}((A \land B)) &= 1 + \texttt{mist}(A) +
                \texttt{mist}(B) \\ 
                \texttt{mist}((A \lor B)) &= 1 + \texttt{mist}(A) +
                \texttt{mist}(B) \\ 
                \texttt{mist}((A \rightarrow B)) &= 1 + \texttt{mist}(A) +
                \texttt{mist}(B) 
            \end{align*}

            ¿Cuál o cuáles de las siguientes expresiones son \textbf{verdaderas}?
        }
        \begin{checkboxes}
            \choice \texttt{mist} no es una función inyectiva. % Correcta

            \choice \texttt{mist} regresa el número de conectivos lógicos que 
            tiene la fórmula $F$.

            \choice \texttt{mist} no es realmente una función recursiva. 

            \choice \texttt{mist} regresa el número de subfórmulas proposicionales
            que tiene la fórmula $F$. % Correcta
            
            \choice Ninguna de las anteriores. 
        \end{checkboxes}

        % Question 02
        \question
        {
            Sea \texttt{mist}$: \mathbb{N} \times \mathbb{N} \rightarrow 
            \mathcal{L}_\mathbb{N}$ una función recursiva que recibe dos 
            números naturales $n$ y $m$, definida de la siguiente manera:
            \begin{align*}
                \texttt{mist}(0,m) &= [] \\ 
                \texttt{mist}(n, m) &= (a:\texttt{mist}(n-1,m))
            \end{align*}

            ¿Cuál o cuáles de las siguientes expresiones son \textbf{verdaderas}?
        }
        \begin{checkboxes}
            \choice \texttt{mist}$(2,3) = (a : \texttt{mist}(1, 3)) = 
            (a : (a : \texttt{mist}(0,3))) = (a : (a : [])) = [a]$

            \choice La imagen de \texttt{mist} es el conjunto de listas de 
            números enteros. 

            \choice \texttt{mist}$(3,2) = (a: \texttt{mist}(2,2)) = 
            (a : \texttt{mist}(1,2)) = (a : (a : \texttt{mist}(0,2))) = 
            (a : (a : [])) = [a,a]$

            \choice \texttt{mist} es una función que está mal definida. % Correcta
            
            \choice Ninguna de las anteriores. 
        \end{checkboxes}

        % Question 03
        \question
        {
            Sea \texttt{mist}$: \mathbb{Z} \times \mathcal{L}_\mathbb{Z} 
            \rightarrow \mathbb{N}$ una función recursiva que recibe un 
            número entero $n$ y una lista de números enteros $l$, definida de 
            la siguiente manera:
            \[
                \texttt{mist}(n, l) =  
                \begin{cases} 
                0 & \text{si } l = [] \\
                1 + \texttt{mist}(n,xs) & \text{si } l = (x:xs) \text{ y } n = x \\
                \texttt{mist}(n,xs) & \text{si } l = (x:xs) \text{ y } n \neq x \\
                \end{cases}
            \]

            ¿Qué hace la función \texttt{mist}?
        }
        \begin{checkboxes}
            \choice \texttt{mist} regresa la longitud de la lista $l$.

            \choice \texttt{mist} regresa el número de elementos repetidos que 
            existen en la lista $l$. 

            \choice \texttt{mist} regresa el número de apariciones del elemento 
            $n$ en la lista $l$. % Correcta 

            \choice \texttt{mist} regresa el número de elementos que son 
            diferentes al elemento $n$ en la lista $l$.  
            
            \choice Ninguna de las anteriores. 
        \end{checkboxes}

        % Question 04
        \question
        {
            Sea \texttt{mist}$:\mathbb{N} \times \mathbb{N} \times 
            \mathcal{L}_\mathbb{Z} \rightarrow \mathcal{L}_\mathbb{Z}$ una 
            función que recibe un número natural $n$, un número entero $m$ y 
            una lista de números enteros $l$, definida de la siguiente manera:
            \begin{align*}
                \texttt{mist}(n, m, []) &= [] \\ 
                \texttt{mist}(0, m, (x:xs)) &= (m:xs) \\ 
                \texttt{mist}(n, m, (x:xs)) &= (x:\texttt{mist}(n-1, m, xs))
            \end{align*}

            ¿Cuál o cuáles de las siguientes expresiones son \textbf{verdaderas}?
        }
        \begin{checkboxes}
            \choice \texttt{mist} regresa la lista resultante de eliminar el 
            elemento $m$ de la lista $l$. 

            \choice Si la longitud de la lista $l$ es $k$, entonces \texttt{mist}
            regresa una lista cuya longitud es $k+1$. 

            \choice \texttt{mist} regresa la lista resultante de agregar al 
            elemento $n$ en la $m-$ésima posición de la lista $l$. 

            \choice \texttt{mist} es una función que realmente no es recursiva. 
            
            \choice Ninguna de las anteriores. % Correcta
        \end{checkboxes}

        % Question 05
        \question
        {
            Sea \texttt{mist}$: \mathbb{N} \times \mathcal{L}_\mathbb{Z} 
            \rightarrow \mathcal{L}_\mathbb{Z}$ una función recursiva que 
            recibe un número natural $n$ y una lista de números enteros 
            $l$, definida de la siguiente manera:
            \begin{align*}
                \texttt{mist}(n, []) &= [] \\
                \texttt{mist}(0, (x:xs)) &= (x:xs) \\  
                \texttt{mist}(n, (x:xs)) &= \texttt{mist}(n-1, xs)
            \end{align*}

            ¿Qué hace la función \texttt{mist}?
        }
        \begin{checkboxes}
            \choice \texttt{mist} regresa la lista resultante de restar una 
            unidad a cada uno de los elementos de $l$.

            \choice \texttt{mist} es una función identidad. 

            \choice \texttt{mist} regresa la lista resultante de eliminar 
            los primeros $n$ elementos de la lista $l$. % Correcta

            \choice \texttt{mist} regresa la lista resultante de eliminar 
            los últimos $n$ elementos de la lista $l$. 
            
            \choice Ninguna de las anteriores. 
        \end{checkboxes}

        % Question 06
        \question
        {
            Sea \texttt{mist}$: \mathcal{L}_\mathbb{Z} \rightarrow 
            \mathcal{L}_\mathbb{Z}$ una función que recibe una lista de números 
            enteros $l$, definida de la siguiente manera:
            \begin{align*}
                \texttt{mist}([]) &= [[]] \\ 
                \texttt{mist}((x:xs)) &= (x:xs)\texttt{ ++ }\texttt{mist}(xs)
            \end{align*}

            ¿Cuál o cuáles de las siguientes expresiones son \textbf{verdaderas}?
        }
        \begin{checkboxes}
            \choice Si la longitud de la lista $l$ es $k$, entonces \texttt{mist} 
            regresa la lista resultante de concatenar la lista $l$ un número de 
            $k$ veces. 

            \choice \texttt{mist} es una función que está mal definida. % Correcta

            \choice \texttt{mist} es una función suprayectiva. 

            \choice \texttt{mist}$([1]) = [1] \texttt{ ++ mist}([]) = [1] 
            \texttt{ ++ } [] = [1]$
            
            \choice Ninguna de las anteriores. 
        \end{checkboxes}

        % Question 07
        \question
        {
            Sea \texttt{mist}$: \mathcal{L}_{\mathbb{Z}} \times 
            \mathcal{L}_{\mathbb{Z}} \rightarrow \mathcal{L}_{\mathbb{Z}}$ 
            una función que recibe dos listas de números enteros $l_1$ y 
            $l_2$, definida de la siguiente manera:
            \begin{align*}
                \texttt{mist}([], []) &= [] \\ 
                \texttt{mist}([], ys) &= ys \\ 
                \texttt{mist}(xs, []) &= xs \\ 
                \texttt{mist}((x:xs), (y:ys)) &= (y : \texttt{mist}((x:xs), ys))
            \end{align*}

            ¿Cuál o cuáles de las siguientes expresiones son \textbf{verdaderas}?
        }
        \begin{checkboxes}
            \choice \texttt{mist} regresa la lista resultante de concatenar el 
            primer elemento de la lista $l_2$ con $l_1$.

            \choice \texttt{mist} regresa la lista resultante de concatenar la 
            lista $l_2$ con la lista $l_1$. % Correcta

            \choice \texttt{mist} es una función inyectiva. 

            \choice El dominio de la función \texttt{mist} es $
            \mathcal{L}_{\mathbb{Z}}$
            
            \choice Ninguna de las anteriores. 
        \end{checkboxes}

        % Question 08
        \question
        {
            Sea \texttt{mist}$: \mathbb{Z} \times \mathcal{L}_{\mathbb{Z}} 
            \rightarrow \mathcal{L}_{\mathbb{Z}}$ una función que recibe un 
            número entero $n$ y una lista de números enteros $l$, definida 
            de la siguiente manera:
            \[
                \texttt{mist}(n, l) =  
                \begin{cases} 
                [n] & \text{si } l = [] \\
                n : (x : xs) & \text{si } l = (x:xs) \text{ y } n \leq x \\
                (x : \texttt{mist}(n,xs)) & \text{si } l = (x:xs) \text{ y } 
                n > x 
                \end{cases}
            \]

            ¿Cuál o cuáles de las siguientes expresiones son \textbf{verdaderas}?
        }
        \begin{checkboxes}
            \choice \texttt{mist} no es realmente una función recursiva. 

            \choice \texttt{mist} regresa la lista resultante de sustituir 
            al $n$-ésimo elemento de la lista $l$ por el elemento $n$. 

            \choice \texttt{mist} regresa la lista resultante de agregar al 
            elemento $n$ una posición atrás del primer elemento en $l$ que sea 
            mayor o igual a $n$. % Correcta

            \choice \texttt{mist} regresa la lista resultante de ordenar de 
            manera ascendente todos los elementos de la lista $l$. 
            
            \choice Ninguna de las anteriores. 
        \end{checkboxes}

        % Question 09
        \question
        {
            ¿Cuál o cuáles de las siguientes funciones recursivas reciben un 
            árbol binario $T$ y regresan una lista de números naturales $l$ 
            correspondiente al recorrido del árbol $T$ donde primero visitamos 
            el subárbol izquierdo, luego el subárbol derecho y finalmente la 
            raíz de $T$? 
        }
        \begin{checkboxes}
            \choice 
            \begin{align*}
                \texttt{mist}: \mathcal{A}_\mathbb{N} &\rightarrow 
                \mathcal{L}_\mathbb{N} \\ 
                \texttt{mist}(\texttt{void}) &= [] \\ 
                \texttt{mist}(\texttt{tree}(T_1,c,T_2)) &= 
                [c]\texttt{ ++ }\texttt{mist}(T_1)\texttt{ ++ }\texttt{mist}(T_2)
            \end{align*}

            \choice
            \begin{align*}
                \texttt{mist}: \mathcal{A}_\mathbb{N} &\rightarrow 
                \mathcal{L}_\mathbb{N} \\ 
                \texttt{mist}(\texttt{void}) &= [] \\ 
                \texttt{mist}(\texttt{tree}(T_1,c,T_2)) &= 
                \texttt{mist}(T_2)\texttt{ ++ }\texttt{mist}(T_1)\texttt{ ++ }[c]
            \end{align*} 

            \choice
            \begin{align*}
                \texttt{mist}: \mathcal{A}_\mathbb{N} &\rightarrow 
                \mathcal{L}_\mathbb{N} \\ 
                \texttt{mist}(\texttt{void}) &= [] \\ 
                \texttt{mist}(\texttt{tree}(T_1,c,T_2)) &= 
                \texttt{mist}(T_1)\texttt{ ++ }[c]\texttt{ ++ }\texttt{mist}(T_2)
            \end{align*}

            \choice
            \begin{align*}
                \texttt{mist}: \mathcal{A}_\mathbb{N} &\rightarrow 
                \mathcal{L}_\mathbb{N} \\ 
                \texttt{mist}(\texttt{void}) &= [] \\ 
                \texttt{mist}(\texttt{tree}(T_1,c,T_2)) &= 
                \texttt{mist}(T_1)\texttt{ ++ }\texttt{mist}(T_2)\texttt{ ++ }[c]
            \end{align*} % Correcta
            
            \choice Ninguna de las anteriores. 
        \end{checkboxes}

        % Question 10
        \question
        {
            Sea \texttt{mist}$: \texttt{LPROP} \rightarrow 
            \mathcal{L}_{\texttt{LPROP}}$ una función recursiva que 
            recibe una fórmula proposicional $F$, definida de la siguiente 
            manera:
            \begin{align*}
                \texttt{mist}(\texttt{true}) &= [] \\ 
                \texttt{mist}(\texttt{false}) &= [] \\ 
                \texttt{mist}(p) &= [p]
                && \text{con $p$ una variable proposicional} \\ 
                \texttt{mist}((\neg A)) &= \texttt{mist}(A) \texttt{ ++ } [] \\ 
                \texttt{mist}((A \land B)) &= \texttt{mist}(A) \texttt{ ++ } 
                \texttt{mist}(B) \\ 
                \texttt{mist}((A \lor B)) &= \texttt{mist}(A) \texttt{ ++ } 
                \texttt{mist}(B) \\ 
                \texttt{mist}((A \rightarrow B)) &= \texttt{mist}(A) 
                \texttt{ ++ } \texttt{mist}(B) 
            \end{align*}

            ¿Cuál o cuáles de las siguientes expresiones son \textbf{verdaderas}?
        }
        \begin{checkboxes}
            \choice \texttt{mist} es una función inyectiva. 

            \choice \texttt{mist} regresa una lista con todas las variables 
            atómicas que están presenten en $F$. 

            \choice \texttt{mist}$(\texttt{mist}((p \land q) \lor r)) = 3$

            \choice \texttt{mist}$((p \rightarrow q) \land (p \rightarrow r)) 
            = \texttt{mist}((p \rightarrow q)) \texttt{ ++ } 
            \texttt{mist}((p \rightarrow r)) $ \\ 
            $= \texttt{mist}(p) \texttt{ ++ }
            \texttt{mist}(q) \texttt{ ++ } \texttt{mist}(p) \texttt{ ++ }
            \texttt{mist}(r) = [p] \texttt{ ++ } [q] \texttt{ ++ } [p] 
            \texttt{ ++ } [r] = [p,q,p,r]$ % Correcta
            
            \choice Ninguna de las anteriores. 
        \end{checkboxes}

        % Question 11
        \question
        {
            Sea \texttt{mist}$: \mathcal{L}_\mathbb{Z} \times 
            \mathcal{L}_\mathbb{Z} \rightarrow \mathcal{L}_\mathbb{Z}$ una 
            función recursiva que recibe dos listas de números enteros $l_1$ 
            y $l_2$, definida de la siguiente manera:
            \begin{align*}
                \texttt{mist}([], (y:ys)) &= (y:ys) \\ 
                \texttt{mist}((x:xs), ys) &= \texttt{mist}(xs, (x:ys))
            \end{align*}

            ¿Qué hace la función \texttt{mist}?
        }
        \begin{checkboxes}
            \choice \texttt{mist} regresa una lista cuyo único elemento es el 
            último elemento de $l_2$.

            \choice \texttt{mist} regresa la lista resultante de concatenar la 
            reversa de la lista $l_1$ con la lista $l_2$. % Correcta

            \choice \texttt{mist} regresa la lista resultante de concatenar la 
            lista $l_1$ con la lista $l_2$. 

            \choice \texttt{mist} regresa la lista resultante de reemplazar 
            todos los elementos de la lista $l_2$ por todos los elementos de la 
            lista $l_1$. 
            
            \choice Ninguna de las anteriores. 
        \end{checkboxes}

        % Question 12
        \question
        {
            Sea \texttt{mist}$: \mathcal{L}_{\mathcal{L}_\mathbb{Z}} 
            \rightarrow \mathcal{L}_\mathbb{Z}$ una función recursiva que 
            recibe una lista de listas de números enteros $l$, definida de la 
            siguiente manera:
            \begin{align*}
                \texttt{mist}([]) &= [] \\ 
                \texttt{mist}((xs:xss)) &= xs \texttt{ ++ mist}(xss)
            \end{align*}

            ¿Cuál o cuáles de las siguientes expresiones son \textbf{verdaderas}?
        }
        \begin{checkboxes}
            \choice \texttt{mist} regresa exactamente la misma lista que le 
            pasamos como parámetro.  

            \choice \texttt{mist} regresa la lista resultante de sumar todos 
            los elementos de la lista $l$.

            \choice \texttt{mist} regresa la lista resultante de concatenar las 
            listas que son elementos de $l$. % Correcta

            \choice El dominio de la función \texttt{mist} es el conjunto de
            todas las listas de números enteros. 
            
            \choice Ninguna de las anteriores. 
        \end{checkboxes}

        \newpage
        % Question 13
        \question
        {
            Sea $T = \{(a, b) \; | \; a, b \in \mathbb{Z}\}$. Sea, además, 
            \texttt{mist}$: \mathcal{L}_{\mathbb{Z}} \times 
            \mathcal{L}_{\mathbb{Z}} \rightarrow \mathcal{L}_T$ una función 
            recursiva que recibe dos listas de números enteros $l_1$ y 
            $l_2$, definida de la siguiente manera:
            \begin{align*}
                \texttt{mist}([], []) &= [] \\
                \texttt{mist}([], ys) &= [] \\
                \texttt{mist}(xs, []) &= [] \\  
                \texttt{mist}((x:xs), (y:ys)) &= ((x,y) : \texttt{mist}(xs, ys))
            \end{align*}

            ¿Cuál o cuáles de las siguientes expresiones son \textbf{verdaderas}?
        }
        \begin{checkboxes}
            \choice \texttt{mist}$([1,2],[3,4]) = ((1,3) : \texttt{mist}([2], [4]))
            = ((1, 3) : ((2,4) : \texttt{mist}([],[]))) $ \\ 
            $= ((1, 3) : ((2,4) : [])) = [(1,3), (2,4)]$ % Correcta

            \choice Si $n$ y $m$ son las longitudes de las listas $l_1$ y $l_2$, 
            respectivamente, de tal forma que $n < m$, entonces \texttt{mist} 
            regresa una lista cuya longitud es $n$. % Correcta

            \choice Si $n$ y $m$ son las longitudes de las listas $l_1$ y $l_2$, 
            respectivamente, de tal forma que $n < m$, entonces \texttt{mist} 
            regresa una lista cuya longitud es $m$.

            \choice \texttt{mist} no es una función inyectiva. % Correcta
            
            \choice Ninguna de las anteriores. 
        \end{checkboxes}

        % Question 14
        \question
        {
            Sea \texttt{mist}$: \mathbb{N} \times \mathcal{L}_\mathbb{Z} 
            \rightarrow \mathcal{L}_\mathbb{Z}$ una función recursiva que 
            recibe un número natural $n$ y una lista de números enteros $l$, 
            definida de la siguiente manera:
            \begin{align*}
                \texttt{mist}(n, []) &= [] \\  
                \texttt{mist}(0, xs) &= xs \\
                \texttt{mist}(n, (x:xs)) &= x : \texttt{mist}(n-1, xs)
            \end{align*}

            ¿Qué hace la función \texttt{mist}?
        }
        \begin{checkboxes}
            \choice \texttt{mist} regresa la lista resultante de eliminar los 
            primeros $n$ elementos de la lista $l$. 

            \choice \texttt{mist} es una función que está mal definida. 

            \choice \texttt{mist} regresa la lista resultante de concatenar 
            todos los elementos de la lista $l$. 

            \choice \texttt{mist} regresa la lista que tiene como elementos 
            a los primeros $n$ elementos de la lista $l$. 
            
            \choice Ninguna de las anteriores. % Correcta
        \end{checkboxes}

        % Question 15
        \question
        {
            Sea \texttt{mist}$: \mathbb{Z} \times \mathcal{L}_\mathbb{Z} 
            \rightarrow \mathcal{L}_\mathbb{Z}$ una función recursiva que 
            recibe un número entero $n$ y una lista de números enteros $l$, 
            definida de la siguiente manera:
            \begin{align*}
                \text{mist}(n, []) &= [n] \\ 
                \text{mist}(n, (x:xs)) &= (x:\text{mist}(n,xs))
            \end{align*}

            ¿Qué hace la función \texttt{mist}?
        }
        \begin{checkboxes}
            \choice \texttt{mist} regresa la lista resultante de multiplicar 
            todos los elementos de $l$ por el número $n$. 

            \choice \texttt{mist} regresa la lista $l \texttt{++} [n]$. 
            % Correcta

            \choice \texttt{mist} regresa la lista $(l:n)$

            \choice \texttt{mist} regresa la misma lista que le pasamos como 
            parámetro. 
            
            \choice Ninguna de las anteriores. 
        \end{checkboxes}
        
        \newpage
        % Question 16
        \question
        {
            Sea \texttt{mist}$: \mathbb{N} \rightarrow \mathbb{N}$ una función 
            recursiva que recibe un número natural $n$, definida de la siguiente 
            manera:
            \begin{align*}
                \texttt{mist}(0) &= 0 \\ 
                \texttt{mist}(n) &= \frac{n(n+1)(2n+1)}{6} 
            \end{align*}

            ¿Cuál o cuáles de las siguientes expresiones son \textbf{verdaderas}?
        }
        \begin{checkboxes}
            \choice \texttt{mist} es una función recursiva que regresa la suma 
            de los primeros $n$ números naturales elevados al cuadrado. 

            \choice \texttt{mist} regresa la suma de los primeros $n$ números 
            naturales. 

            \choice \texttt{mist} regresa la multiplicación de los primeros 
            $n$ números naturales. 

            \choice \texttt{mist} regresa la suma de los primeros $n$ números 
            naturales que son impares. 
            
            \choice Ninguna de las anteriores. % Correcta
        \end{checkboxes}

        % Question 17
        \question
        {
            Sea \texttt{mist}$: \mathcal{A}_\mathbb{Z} \rightarrow 
            \mathcal{A}_\mathbb{Z}$ una función recursiva que recibe un árbol 
            binario $T$, definida de la siguiente manera:
            \begin{align*}
                \texttt{mist}(\texttt{void}) &= \texttt{void} \\
                \texttt{mist}(\texttt{tree}(\texttt{void}, c, 
                \texttt{void})) &= c \\ 
                \texttt{mist}(\texttt{tree}(T_1,c,T_2)) &= 
                \texttt{mist}(T_1) \\ 
                \texttt{mist}(\texttt{tree}(T_1,c,\texttt{void})) &= 
                \texttt{mist}(T_1) \\ 
                \texttt{mist}(\texttt{tree}(\texttt{void},c,T_2)) &= 
                \texttt{mist}(T_2)
            \end{align*}

            ¿Cuál o cuáles de las siguientes expresiones son \textbf{verdaderas}?
        }
        \begin{checkboxes}
            \choice \texttt{mist} regresa la hoja más a la izquierda en un 
            árbol binario. 

            \choice \texttt{mist} regresa el nodo interno más a la izquierda 
            en un árbol binario. 

            \choice La función \texttt{mist} está mal definida. % Correcta

            \choice \texttt{mist} regresa la hoja más a la derecha en un 
            árbol binario. 
            
            \choice Ninguna de las anteriores. 
        \end{checkboxes}

        % Question 18
        \question
        {
            Sea \texttt{mist}$: \mathcal{A}_\mathbb{N} \rightarrow \mathbb{N}$ 
            una función recursiva que recibe un árbol binario $T$, definida de 
            la siguiente manera:
            \begin{align*}
                \texttt{mist}(\texttt{void}) &= 0 \\ 
                \texttt{mist}(\texttt{tree}(T_1,c,T_2)) &= 
                \texttt{mist}(T_1) + c + \texttt{mist}(T_2)
            \end{align*}

            ¿Cuál o cuáles de las siguientes expresiones son \textbf{verdaderas}?
        }
        \begin{checkboxes}
            \choice \texttt{mist} es una función inyectiva.

            \choice $\texttt{mist}(\texttt{tree}(\texttt{tree}(\texttt{void},1,
            \texttt{void}),5,\texttt{tree}(\texttt{void},0,\texttt{void})))$ \\ 
            $= \texttt{mist}(\texttt{tree}(\texttt{void}, 1, \texttt{void}) + 5 + 
            \texttt{mist}(\texttt{tree}(\texttt{void}, 0, \texttt{void})))$ \\ 
            $= 1 + 5 + 0 = 6$ 

            \choice \texttt{mist} es una función suprayectiva. 

            \choice La función \texttt{mist} está mal definida. % Correcta 
            
            \choice Ninguna de las anteriores. 
        \end{checkboxes}

        \newpage
        % Question 19
        \question
        {
            Sea \texttt{mist}$: \texttt{LPROP} \rightarrow 
            \mathcal{L}_{\texttt{LPROP}}$ una función recursiva que 
            recibe una fórmula proposicional $F$, definida de la siguiente 
            manera:
            \begin{align*}
                \texttt{mist}(\texttt{true}) &= [\texttt{true}] \\ 
                \texttt{mist}(\texttt{false}) &= [\texttt{false}] \\ 
                \texttt{mist}(p) &= [p]
                && \text{con $p$ una variable proposicional} \\ 
                \texttt{mist}((\neg A)) &= ((\neg A) : \texttt{mist}(A)) \\ 
                \texttt{mist}((A \land B)) &= ((A \land B) : (\texttt{mist}(A)
                \texttt{ ++ } \texttt{mist}(B))) \\ 
                \texttt{mist}((A \lor B)) &= ((A \lor B) : (\texttt{mist}(A)
                \texttt{ ++ } \texttt{mist}(B))) \\
                \texttt{mist}((A \rightarrow B)) &= ((A \rightarrow B) : 
                (\texttt{mist}(A) \texttt{ ++ } \texttt{mist}(B))) \\  
            \end{align*}

            ¿Cuál o cuáles de las siguientes expresiones son \textbf{verdaderas}?
        }
        \begin{checkboxes}
            \choice La función \texttt{mist} está mal definida. 

            \choice \texttt{mist} regresa la lista resultante de concatenar 
            la fórmula $F$ con todas sus variables atómicas. 

            \choice \texttt{mist}$(p \land q) = (p \land q) : \texttt{mist}
            (p) \texttt{ ++ mist}(q) = (p \land q) : [p] \texttt{ ++ } [q]
            = (p \land q) : [p,q] $ \\ $= [(p \land q), p, q]$ % Correcta

            \choice La función \texttt{mist} es suprayectiva. 
            
            \choice Ninguna de las anteriores. 
        \end{checkboxes}

        % Question 20
        \question
        {
            Sea \texttt{mist}$: \mathbb{Z} \times \mathcal{L}_\mathbb{Z} 
            \rightarrow \{\texttt{true, false}\}$ una función recursiva que 
            recibe un número entero $n$ y una lista de números enteros $l$, 
            definida de la siguiente manera:
            \[
                \texttt{mist}(n, l) =  
                \begin{cases} 
                \texttt{false} & \text{si } l = [] \\
                \texttt{true} & \text{ si } l = (n:xs) \\ 
                \texttt{mist}(n,xs) & \text{si } l = (x:xs) \text{ y } 
                n \neq x \\
                \end{cases}
            \]

            ¿Cuál o cuáles de las siguientes expresiones son \textbf{verdaderas}?
        }
        \begin{checkboxes}
            \choice \texttt{mist} no es una función suprayectiva. 

            \choice Si el elemento $n$ pertenece a la lista $l$, entonces 
            \texttt{mist} regresa \texttt{true}. En caso contrario, regresa 
            \texttt{false}. % Correcta

            \choice Si el elemento $n$ se encuentra repetido en la lista $l$, 
            entonces \texttt{mist} regresa \texttt{true}. En caso contrario, 
            regresa \texttt{false}. 

            \choice \texttt{mist} es una función inyectiva. 
            
            \choice Ninguna de las anteriores. 
        \end{checkboxes}

        \newpage
        % Question 21
        \question
        {
            Sea \texttt{mist}$: \mathcal{L}_\mathbb{Z} \times 
            \mathcal{L}_\mathbb{Z} \rightarrow \{\texttt{true, false}\}$ una 
            función recursiva que recibe dos listas de números enteros $l_1$ 
            y $l_2$, definida de la siguiente manera:
            \begin{align*}
                \texttt{mist}([], []) &= \texttt{true} \\ 
                \texttt{mist}([], (y:ys)) &= \texttt{false} \\ 
                \texttt{mist}((x:xs), []) &= \texttt{false} \\
                \texttt{mist}((x:xs), (y:ys)) &= \texttt{mist}(xs, ys)
            \end{align*}

            ¿Cuál o cuáles de las siguientes expresiones son \textbf{verdaderas}?
        }
        \begin{checkboxes}
            \choice La función \texttt{mist} está mal definida.   

            \choice Si las listas $l_1$ y $l_2$ tienen los mismos elementos (no 
            importa el orden), entonces \texttt{mist} regresa \texttt{true}. 
            En caso contrario, regresa \texttt{false}.  

            \choice Si los elementos y la longitud de las listas $l_1$ y $l_2$
            son iguales, entonces \texttt{mist} regresa \texttt{true}. En caso 
            contrario, regresa \texttt{false}. 

            \choice Si la longitud de las listas $l_1$ y $l_2$ es igual, 
            entonces \texttt{mist} regresa \texttt{true}. En caso contrario, 
            regresa \texttt{false}. % Correcta
            
            \choice Ninguna de las anteriores. 
        \end{checkboxes}

        % Question 22
        \question
        {
            ¿Cuál es de las siguientes funciones recursivas \texttt{mist} 
            reciben un número entero $n$ y regresan el número $n$ elevado 
            al cuadrado?
        }
        \begin{checkboxes}
            \choice 
            \begin{align*}
                \texttt{mist}: \mathbb{Z} &\rightarrow \mathbb{N} \\ 
                \texttt{mist}(0) &= 0 \\ 
                \texttt{mist}(n) &= n \cdot n
            \end{align*}

            \choice
            \begin{align*}
                \texttt{mist}: \mathbb{Z} &\rightarrow \mathbb{N} \\ 
                \texttt{mist}(0) &= 1 \\ 
                \texttt{mist}(n) &= n \cdot \texttt{mist}(n-1)
            \end{align*}

            \choice
            \begin{align*}
                \texttt{mist}: \mathbb{Z} &\rightarrow \mathbb{N} \\ 
                \texttt{mist}(0) &= 0 \\ 
                \texttt{mist}(n) &= n \cdot \texttt{mist}(n-1)
            \end{align*}

            \choice
            \begin{align*}
                \texttt{mist}: \mathbb{Z} &\rightarrow \mathbb{N} \\ 
                \texttt{mist}(0) &= 0 \\ 
                \texttt{mist}(n+1) &= (n+1) \cdot (n+1)
            \end{align*}
            
            \choice Ninguna de las anteriores. % Correcta
        \end{checkboxes}

        % Question 23
        \question
        {
            Sea \texttt{mist}$:\mathcal{A}_\mathbb{Z} \rightarrow 
            \mathcal{A}_\mathbb{Z}$ una función que recibe un árbol binario 
            $T$, definida de la siguiente manera:
            \begin{align*}
                \texttt{mist}(\texttt{void}) &= \texttt{void} \\
                \texttt{mist}(\texttt{tree}(T_1,c,T_2)) &= 
                \texttt{tree}(c, \texttt{mist}(T_2), \texttt{mist}(T_1))
            \end{align*}

            ¿Cuál o cuáles de las siguientes expresiones son \textbf{verdaderas}?
        }
        \begin{checkboxes}
            \choice \texttt{mist}$(\texttt{tree}(\texttt{void}, 5, 
            \texttt{void})) = \texttt{tree}(5, \texttt{mist}(\texttt{void}), 
            \texttt{mist}(\texttt{void})) = \texttt{mist}(\texttt{void}, 5, 
            \texttt{void})$

            \choice \texttt{mist} regresa el árbol $T$ que le pasamos como 
            parámetro. 

            \choice La función \texttt{mist} nunca llega a ninguno de sus 
            casos base. 

            \choice \texttt{mist} regresa el árbol $T$ pero reemplazando cada 
            subárbol derecho por el subárbol izquierdo y viceversa.
            
            \choice Ninguna de las anteriores. % Correcta
        \end{checkboxes}

        % Question 24
        \question
        {
            ¿Cuál o cuáles de las siguientes funciones recursivas reciben dos 
            listas de números enteros $l_1$ y $l_2$ y regresan la concatenación 
            de $l_1$ con $l_2$?
        }
        \begin{checkboxes}
            \choice 
            \begin{align*}
                \texttt{mist}: \mathcal{L}_\mathbb{Z} &\rightarrow 
                \mathcal{L}_\mathbb{Z} \\ 
                \texttt{mist}([], ys) &= ys \\ 
                \texttt{mist}((x:xs), ys) &= x : \texttt{mist}(xs, ys)
            \end{align*}

            \choice $\texttt{mist}: \mathcal{L}_\mathbb{Z} \times 
            \mathcal{L}_\mathbb{Z} \rightarrow \mathcal{L}_\mathbb{Z}$
            \[
                \texttt{mist}(l_1, l_2) =  
                \begin{cases} 
                l_2 & \text{si } l_1 = [] \\
                [x] \texttt{ ++ mist}(xs, ys) & \text{si } l_1 = (x:xs), 
                l_2 = ys \text{ y } x \leq x \\
                \texttt{mist}(xs, ys) & \text{en otro caso}
                \end{cases} % Correcta
            \]

            \choice $\texttt{mist}: \mathcal{L}_\mathbb{Z} \times 
            \mathcal{L}_\mathbb{Z} \rightarrow \mathcal{L}_\mathbb{Z}$
            \[
                \texttt{mist}(l_1, l_2) =  
                \begin{cases} 
                l_2 & \text{si } l_1 = [] \\
                x : \texttt{mist}(xs, ys) & \text{si } l_1 = (x:xs), 
                l_2 = ys \text{ y } x > x \\
                \texttt{mist}(xs, ys) & \text{en otro caso}
                \end{cases}
            \]

            \choice
            \begin{align*}
                \texttt{mist}: \mathcal{L}_\mathbb{Z} \times 
                \mathcal{L}_\mathbb{Z} &\rightarrow \mathcal{L}_\mathbb{Z} \\ 
                \texttt{mist}([], xs) &= xs \\ 
                \texttt{mist}((y:ys), xs) &= [y] \texttt{ ++ mist}(ys, xs)
            \end{align*} % Correcta
            
            \choice Ninguna de las anteriores. 
        \end{checkboxes}
    \end{questions}
\end{document}