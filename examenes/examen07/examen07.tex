\documentclass[12pt, a4paper]{exam}

% Soporte para cambiar la fecha que sale en el examen
\usepackage{advdate}
% Soporte para escribir en varias columnas
\usepackage{multicol}
% Soporte para los acentos.
\usepackage[utf8]{inputenc} 
\usepackage[T1]{fontenc}    
% Idioma español.
\usepackage[spanish,mexico,es-tabla]{babel}
\usepackage{graphicx}
\usepackage{tikz}
\usepackage{tikz-cd}
\usepackage{amsmath,amssymb,amsthm}

% Cambiamos los márgenes del documento. 
\usepackage[top=1.5cm,left=1.5cm,right=1.5cm]{geometry}

% Pie de página
\cfoot{Página \thepage\ de \numpages}

%%%%%%%%%%%%%%%%%%%%%%%%%%%%%%%%%%%%%%%%%%%%%%%%%%%%%%%%%%%%%%%%%%%%%%%%%%%%%%
\renewcommand{\thechoice}{\alph{choice}}

\makeatletter
\renewenvironment{checkboxes}%
   {\setcounter{choice}{0}\list{\checkbox@char}%
      {%
        \settowidth{\leftmargin}{W.\hskip\labelsep\hskip 2.5em}%
        \def\choice{%
          \if@correctchoice
            \color@endgroup \endgroup
          \fi
          \stepcounter{choice}
          \item[\checked@char]
          \do@choice@pageinfo
        } % choice
        \def\CorrectChoice{%
          \if@correctchoice
            \color@endgroup \endgroup
          \fi
          \ifprintanswers
            % We can't say \choice here, because that would
            % insert an \endgroup.
            % 2016/05/10: We say \color@begingroup in addition to
            % \begingroup in case \CorrectChoiceEmphasis involves color
            % and the text exactly fills the line (which would
            % otherwise create a blank line after this choice):
            % 2016/05/11: We leave hmode if we're in it,
            % i.e., if there's no blank line preceding this
            % \CorrectChoice command.  (Without this, the
            % \special created by a \color{whatever} command that might
            % be inserted by \CorrectChoice@Emphasis would be appended 
            % to the previous \choice, which could cause an extra
            % (blank) line to be inserted before this \CorrectChoice.)
            % Since \par and \endgraf seem to cancel \@totalleftmargin
            % (for reasons I don't understand), we'll do the following:
            % Motivated by  the def of \leavevmode, 
            %      \def\leavevmode{\unhbox\voidb@x}
            % we will now leave hmode (if we're in hmode):
            \ifhmode \unskip\unskip\unvbox\voidb@x \fi
            \begingroup \color@begingroup \@correctchoicetrue
            \CorrectChoice@Emphasis
            \stepcounter{choice}
            \item[\checked@char]
          \else
            \stepcounter{choice}
            \item[\checked@char]
          \fi
          \do@choice@pageinfo
        } % CorrectChoice
        \let\correctchoice\CorrectChoice
        \labelwidth\leftmargin\advance\labelwidth-\labelsep
        \topsep=0pt
        \partopsep=0pt
        \checkboxeshook
      }%
   }%
   {\if@correctchoice \color@endgroup \endgroup \fi \endlist}
 \makeatother

% Make checkbox character a circle with the letter
\checkboxchar{\tikz[baseline={([yshift=-.8ex]current bounding box.center)}]\node[shape=circle,minimum size=4mm,draw] at (0,0) {\thechoice};}
% Make checked box character bold WITH surd
%\checkedchar{\tikz[baseline={([yshift=-.8ex]current bounding box.center)}]\node[shape=circle,minimum size=8mm,draw] at (0,0) {} node at (0,0) {\thechoice\llap{$\surd$}};}
% Make checked box character bold
\checkedchar{\tikz[baseline={([yshift=-.8ex]current bounding box.center)}]\node[shape=circle,minimum size=4mm,draw] at (0,0) {} node at (0,0) {\thechoice};}
\printanswers
%%%%%%%%%%%%%%%%%%%%%%%%%%%%%%%%%%%%%%%%%%%%%%%%%%%%%%%%%%%%%%%%%%%%%%%%%%%%%%

\begin{document}
    %%%%%%%%%%%%%%%%%%%%%%%%%%%%%%%%%%%%%%%%%%%%%%%%%%%%%%%%%%%%%%%%%%%%%%%%%%%%%%%
    %%%%%%%%%%%%%%%%%%%%%%%%%%%%%%%% ENCABEZADO %%%%%%%%%%%%%%%%%%%%%%%%%%%%%%%%%%%
    \centering
    \hrule \hrule \hrule 
    \vspace{5mm}
    \begin{minipage}[c]{0.8\textwidth}
        \begin{center}
            {\large\textbf{Mission 07, Start!} \par
            \large \textbf{Estructuras Discretas} \par
            \large \textbf{Semestre 2023-1} \par
            \large \textbf{\today}	\par}
        \end{center}
    \end{minipage}

    \vspace{0.2in}
    \noindent
    \textbf{Tania Michelle Rubí Rojas}
    \vspace{2mm}
    \hrule \hrule \hrule 
    %%%%%%%%%%%%%%%%%%%%%%%%%%%%%%%%%%%%%%%%%%%%%%%%%%%%%%%%%%%%%%%%%%%%%%%%%%%%%%%
    %%%%%%%%%%%%%%%%%%%%%%%%%%%%%%%%%%%%%%%%%%%%%%%%%%%%%%%%%%%%%%%%%%%%%%%%%%%%%%%

    \vspace{5mm}
    \noindent
    Nombre y número de cuenta: \hrulefill\

    \vspace{5mm}
    \noindent
    
    \textbf{Notación y convenciones para el examen:}
    {\tiny
    \begin{multicols}{2}
    \begin{itemize}\setlength\itemsep{0em}  
      \item $0\in\mathbb{N}$

      \item \texttt{gcd}$(a,b)$ hace referencia al máximo común divisor de dos 
      números enteros $a$ y $b$. 

      \item $[x]$ representa la clase de equivalencia de $a$. 
      
      \item $A/R$ representa el conjunto cociente de $A$ bajo $R$.
      
      \item $\{0,1,00,11,101,000,010,111, \ldots\}$ es el conjunto de todas 
      las cadenas de ceros y unos. 
    \end{itemize}
    \end{multicols}
    }

    \begin{questions}
        % Question 01
        \question{¿Cuál o cuáles de las siguientes expresiones son 
        \textbf{verdaderas}?}
        \begin{checkboxes}
            \choice Definimos la relación $R$ sobre $\mathbb{Z}$ como sigue:
            \begin{equation*}
                xRy \Leftrightarrow \texttt{gcd}(x,y) = 1
            \end{equation*}

            Entonces $R$ es una relación de equivalencia. 

            \choice Sea $(A, \preceq)$ un conjunto parcialmente ordenado. Sean 
            $X$ un conjunto diferente del vacío y $h: X \rightarrow A$ una 
            función inyectiva. Definimos la relación $R$ sobre $X$ como sigue:
            \begin{equation*}
                xRy \Leftrightarrow h(x) \preceq h(y)
            \end{equation*}

            Entonces $(X, R)$ es un conjunto parcialmente ordenado. % Correcta

            \choice Sea $A = \{1,2,3,4,5,6\}$. Definimos la relación $R$ sobre  
            $A$ como sigue:
            \begin{align*}
                \{&(1,1), (6,5), (2,2), (5,6), (3,3), (6,4), (4,4), (4,6), \\ 
                &(5,5), (5,4), (6,6), (4,5), (2,3), (3,2)\}
            \end{align*}

            Entonces las clases de equivalencia de $R$ son:
            \begin{equation*}
                [1] = \{1\} \quad \quad \quad \quad 
                [2] = [3] = \{2,3\} \quad \quad \quad \quad   
                [4] = [5] = [6] = \{4,5,6\} 
            \end{equation*} % Correcta

            \choice Sea $W$ el conjunto de todas las palabras en español. 
            Definimos la relación $R$ sobre $W$ como sigue:
            \begin{equation*}
                xRy \Leftrightarrow x \text{ tiene el mismo número de letras 
                que } y
            \end{equation*}

            Entonces $R$ es una relación de equivalencia cuyo conjunto cociente 
            es 
            \begin{align*}
                A/R = \{
                        &\{\text{las palabras que tienen una letra}\}, \\ 
                        &\{\text{las palabras que tienen dos letras}\}, \\ 
                        &\{\text{las palabras que tienen tres letras}\}, \\  
                        &\ldots
                      \}
            \end{align*} % Correcta 
            
            \choice Ninguna de las anteriores. 
        \end{checkboxes}

        \newpage
        % Question 02
        \question
        {
            Sea $R$ una relación reflexiva y transitiva sobre un
            conjunto no vacío $A$. Definimos la relación $S$ como sigue:
            \begin{equation*}
                xSy \Leftrightarrow xRy \text{ y } yRx
            \end{equation*} 

            ¿Cuál o cuáles de las siguientes expresiones son 
            \textbf{verdaderas}?
        }
        \begin{checkboxes}
            \choice Si $wRz, wSa$ y $zSb$ entonces $aRb$. % Correcta 

            \choice $S$ es un orden total. 

            \choice Sea el conjunto cociente $A/S$. Definimos la relación 
            binaria $T$ sobre $A/S$ de la siguiente manera:
            \begin{equation*}
                [x]T[y] \Leftrightarrow xRy
            \end{equation*}

            Entonces $(A/S, T)$ es un conjunto parcialmente ordenado. 
            % Correcta

            \choice $S$ es una relación de equivalencia. % Correcta
            
            \choice Ninguna de las anteriores. 
        \end{checkboxes}

        % Question 03
        \question{¿Cuál o cuáles de las siguientes expresiones son 
        \textbf{verdaderas}?}
        \begin{checkboxes}
            \choice Sea $S$ el conjunto de todas las cadenas de ceros y unos 
            cuya longitud es mayor o igual a tres. Definimos la relación $R$ 
            sobre $S$ como sigue:
            \begin{equation*}
                xRy \Leftrightarrow x \text{ tiene los primeros tres caracteres 
                que } y
            \end{equation*}

            Entonces la clase de equivalencia de $01010101$ y $11111$ es el 
            conjunto de todas las cadenas que empiezan con $010$ y $111$, 
            respectivamente.  % Correcta 

            \choice Definimos la relación $R$ sobre $\mathbb{R}$ como sigue:
            \begin{equation*}
                R = \{(x,y) \; | \; \lfloor 2x \rfloor = \lfloor 2y \rfloor\}
            \end{equation*}

            donde $\lfloor 2x \rfloor$ se define como el mayor entero 
            $i \in \mathbb{Z}$ tal que $i \leq x$. Entonces $R$ es una relación 
            de equivalencia y las clases de equivalencia de $\frac{1}{4}$ y 
            $\frac{1}{2}$ son $\{x \in \mathbb{R} \; | \; 0 \leq x < \frac{1}{2}\}$
            y $\{x \in \mathbb{R} \; | \; \frac{1}{2} \leq x < 1\}$, 
            respectivamente. % Correcta 

            \choice Definimos la relación binaria $R$ sobre $\mathbb{Z}$ como 
            sigue:
            \begin{equation*}
                xRy \Leftrightarrow x \text{ y } y \text{ tienen la misma 
                paridad (par o impar)}
            \end{equation*} 
            
            Entonces $R$ \textbf{no} es una relación de equivalencia.

            \choice Sea $A = \{1,2,3,4,5\}$. Definimos la relación $R$ sobre 
            $A$ como sigue:
            \begin{align*}
                R = \{&(1,1), (1,3), (1,4), (2,2), (2,5), (3,1), (3,3), (3,4) \\ 
                &(4,1), (4,3), (4,4), (5,2), (5,5)\}
            \end{align*}

            Entonces $\{1,3,4\}$ y $\{2,3,5\}$ son dos clases de equivalencia 
            de $R$. 
            
            \choice Ninguna de las anteriores. 
        \end{checkboxes}

        \newpage
        % Question 04
        \question{¿Cuál o cuáles de las siguientes expresiones son 
        \textbf{verdaderas}?}
        \begin{checkboxes}
            \choice Si $R$ es una relación de equivalencia sobre un conjunto 
            finito no vacío $A$, entonces todas las clases de equivalencia de 
            $R$ tienen el mismo número de elementos. 

            \choice Sea $A$ un conjunto. Definimos la relación binaria $R$ sobre 
            $\mathcal{P}(A)$ como sigue:
            \begin{equation*}
                R = \{(X,Y) \; | \; X \text{ tiene la misma cardinalidad que } 
                Y\}
            \end{equation*} 
            
            Entonces $R$ es de equivalencia % Correcta

            \choice Sea $P$ el conjunto de todas las personas vivas. Definimos
            la relación $R$ sobre $P$ como sigue:
            \begin{equation*}
                R = \{(p,q) \; | \; \text{ la edad de } p \text{ es menor que 
                la edad de } q\}
            \end{equation*}

            Entonces $R$ es un orden parcial. 
            
            \choice En un conjunto parcialmente ordenado, un elemento maximal 
            siempre es un elemento máximo. 
            
            \choice Ninguna de las anteriores.
        \end{checkboxes}

        % Question 05
        \question{¿Cuál o cuáles de las siguientes expresiones son 
        \textbf{verdaderas}?}
        \begin{checkboxes}
            \choice Sea $R$ una relación de equivalencia sobre un conjunto $A$. 
            Sean $x,y \in A$. Entonces 
            \begin{equation*}
                [x] \neq [y] \Leftrightarrow (x,y) \notin R
            \end{equation*} % Correcta

            \choice Sean $R$ y $S$ relaciones sobre un conjunto $A$. Si $R$ y 
            $S$ son relaciones de equivalencia, entonces $R \cap S$ también 
            lo es. % Correcta

            \choice Sea $(A, \preceq)$ un orden total. Si $A$ tiene un elemento 
            maximal $m$, entonces $m$ es un elemento máximo. % Correcta
            

            \choice Sea $R$ una relación binaria sobre un conjunto $A$. Si $R$ 
            es simétrica y antisimétrica, entonces $R$ también es reflexiva. 

            \choice Ninguna de las anteriores. 
        \end{checkboxes}

        % Question 06
        \question{¿Cuál o cuáles de las siguientes expresiones son 
        \textbf{verdaderas}?}
        \begin{checkboxes}
            \choice Definimos la relación $R$ sobre $\mathbb{N}^2$ como sigue:
            \begin{equation*}
                ((a,b), (c,d)) \in R \Leftrightarrow a+b = c+d
            \end{equation*}

            Entonces $R$ \textbf{no} es una relación de equivalencia. 

            \choice Definimos la relación $R$ sobre $\mathbb{N}^2$ como sigue:
            \begin{equation*}
                (a,b)R(c,d) \Leftrightarrow b = d
            \end{equation*}

            Entonces $R$ \textbf{no} es una relación de equivalencia.

            \choice Sean $(A, \preceq)$ un orden parcial y $X \subseteq A$. Si 
            $X$ tiene dos elementos maximales distintos, entonces $X$ no tiene 
            máximo. % Correcta

            \choice La relación vacía sobre $\varnothing$ es una relación de 
            equivalencia y un orden parcial. 

            \choice Ninguna de las anteriores. 
        \end{checkboxes}

        \newpage
        % Question 07
        \question{¿Cuál o cuáles de las siguientes expresiones son 
        \textbf{verdaderas}?}
        \begin{checkboxes}
            \choice Un conjunto parcialmente ordenado puede tener más de un 
            elemento mínimo. 

            \choice Sea $A = \{1,2,3\}$. La partición $P = \{\{1,2\}, \{3\}\}$
            induce la siguiente relación de equivalencia sobre $A$:
            \begin{equation*}
                R = \{(1,1), (1,2), (2,2), (2,1), (3,3)\}
            \end{equation*} % Correcta 

            \choice La relación vacía sobre un conjunto no vacío $A$ es siempre 
            una relación de equivalencia. 

            \choice Si $R$ es una relación de equivalencia sobre un conjunto 
            $A$, entonces $xRy \Leftrightarrow [x] = [y]$. % Correcta
            
            \choice Ninguna de las anteriores. 
        \end{checkboxes}

        % Question 08
        \question
        {
            Sea $(A, \preceq)$ un conjunto parcialmente ordenado cuyo diagrama
            de Hasse es el siguiente:
            \begin{center}
                \begin{tikzpicture}
                    \matrix (A) [matrix of nodes, row sep=0.5cm, 
                    column sep=0.5cm]
                    { 
                        \textcolor{green}{$l$} & & \textcolor{green}{$m$} \\  
                        $j$ & \textcolor{red}{$k$} & \\
                        $i$ & $h$ & $g$ \\
                        \textcolor{red}{$d$} & $e$ & \textcolor{red}{$f$} \\ 
                        $a$ & $b$ & $c$ \\
                    };
                    \draw (A-5-1)--(A-4-1);
                    \draw (A-4-1)--(A-5-2);
                    \draw (A-4-1)--(A-3-2);
                    \draw (A-4-1)--(A-3-1);
                    \draw (A-5-2)--(A-4-2);
                    \draw (A-5-3)--(A-4-3);
                    \draw (A-4-3)--(A-3-3);
                    \draw (A-4-2)--(A-3-2);
                    \draw (A-3-1)--(A-2-1);
                    \draw (A-3-2)--(A-2-1);
                    \draw (A-3-2)--(A-2-2);
                    \draw (A-3-3)--(A-2-2);
                    \draw (A-1-1)--(A-2-1);
                    \draw (A-1-1)--(A-2-2);
                    \draw (A-1-3)--(A-2-2);
                \end{tikzpicture}
            \end{center}

            ¿Cuál o cuáles de las siguientes expresiones son 
            \textbf{verdaderas}?
        }
        \begin{checkboxes}
            \choice Si $B = \{l,m\}$ entonces las cotas inferiores de $B$ en $A$ son 
            $\{a,b,c,d,e,f,g,h,k\}$ y el ínfimo de $B$ no existe. 

            \choice Si $B = \{d\}$ entonces entonces las cotas superiores de $B$ 
            en $A$ son $\{h,i,j,k,l,m\}$ y el supremo de $B$ no existe. 

            \choice Si $B = \{d,f,k\}$ entonces el supremo de $B$ es 
            $k$ y las cotas superiores de $B$ en $A$ son $\{k,l,m\}$. % Correcta

            \choice  $(A, \preceq)$ no tiene elemento mínimo y, si $B = \{d,f,k\}$, entonces supremo de $B$ es $k$. % Correcta
            
            \choice Ninguna de las anteriores. 
        \end{checkboxes}

        % Question 09
        \question{¿Cuál o cuáles de las siguientes expresiones son 
        \textbf{verdaderas}?}
        \begin{checkboxes}
            \choice Un elemento máximo nunca puede ser un elemento maximal. 

            \choice Todos los elementos de un conjunto parcialmente ordenado 
            son comparables. 

            \choice Todo orden total es una relación de equivalencia. 

            \choice Definimos la relación $R$ sobre $\mathbb{Z}$ como sigue:
            \begin{equation*}
                xRy \Leftrightarrow 2 \text{ divide a } x+y 
            \end{equation*} 

            Entonces $R$ es una relación de equivalencia cuyo conjunto 
            cociente es $\mathbb{Z}/R = \{[0], [1]\}$. % Correcta 
            
            \choice Ninguna de las anteriores. 
        \end{checkboxes}

        \newpage
        % Question 10
        \question{¿Cuál o cuáles de las siguientes expresiones son 
        \textbf{verdaderas}?}
        \begin{checkboxes}
            \choice Sea $A \subseteq \mathbb{R}$. Definimos la relación $R$ 
            sobre $A^2$ como sigue:
            \begin{equation*}
                (a,b)R(c,d) \Leftrightarrow a \leq c \text{ y } b \geq d
            \end{equation*}

            Entonces $R$ \textbf{no} es un orden parcial. 

            \choice Sea $A$ un conjunto no vacío. Sean $R$ y $S$ relaciones de 
            equivalencia sobre $A$ tales que inducen la misma partición en $A$. 
            Entonces $R = S$. % Correcta

            \choice Sean $A$ un conjunto no vacío y $f: A \rightarrow A$ una 
            función. Definimos la relación binaria $R$ sobre $A$ como sigue:
            \begin{equation*}
                R = \{(x,y) \; | \; f(x) = f(y)\}
            \end{equation*} 
            
            Entonces $R$ es una relación de equivalencia. % Correcta 

            \choice Sea $P$ el conjunto de todos los planetas. Definimos la 
            relación $R$ sobre $P \times \mathbb{Z}^+$ como sigue:
            \begin{equation*}
                (x,y) \in R \Leftrightarrow x \text{ está a la posición } y 
                \text{ respecto al Sol}
            \end{equation*}

            Entonces $R$ es un orden parcial. 
            
            \choice Ninguna de las anteriores. 
        \end{checkboxes}

        % Question 11
        \question{¿Cuál o cuáles de las siguientes expresiones son 
        \textbf{verdaderas}?}
        \begin{checkboxes}
            \choice En un conjunto parcialmente ordenado pueden existir 
            elementos maximales sin que necesariamente exista un elemento 
            máximo. % Correcta

            \choice Sea $A \subseteq \mathbb{N}$. Definimos la relación $R$ 
            sobre $A$ como sigue:
            \begin{equation*}
                xRy \Leftrightarrow y = x^k \quad \quad 
                \text{para alguna } k \in \mathbb{N}
            \end{equation*}

            Entonces $(A, R)$ es un conjunto totalmente ordenado. 

            \choice Sea $R$ una relación de equivalencia sobre un conjunto $A$. 
            Sean $x,y \in A$. Entonces
            \begin{equation*}
                x \in [y] \Leftrightarrow [x] \cap [y] \neq \varnothing
            \end{equation*} % Correcta 

            \choice Sea $(A, \preceq)$ un conjunto parcialmente ordenado. Si 
            $m$ es cota superior de $S \subseteq A$, entonces $m$ es un 
            elemento maximal de $A$.
            
            \choice Ninguna de las anteriores. 
        \end{checkboxes}

        % Question 12
        \question{¿Cuál o cuáles de las siguientes expresiones son 
        \textbf{verdaderas}?}
        \begin{checkboxes}
            \choice Un conjunto parcialmente ordenado puede tener más de un 
            elemento máximo. 

            \choice Sean $A$ un conjunto finito no vacío y
              $(A, \preceq)$ un COPO. Entonces $(A,\preceq)$ tiene al menos 
              un elemento maximal y otro minimal. % Correcta

            \choice Sea $R$ una relación reflexiva sobre un conjunto 
            $A$. Entonces $R$ es una relación de equivalencia si y sólo si 
            el hecho de que $(x,y), (x,z) \in R$ implica que $(y,z) \in R$. 
            % Correcta 

            \choice Sea $R$ una relación binaria sobre un conjunto infinito 
            $A$. Si $R$ es una relación de equivalencia, entonces $R$ tiene una 
            infinita cantidad de clases de equivalencia. 
            
            \choice Ninguna de las anteriores. 
        \end{checkboxes}

        % Question 13
        \question
        {
            Sea $(A, \preceq)$ un conjunto parcialmente ordenado cuyo diagrama
            de Hasse es el siguiente:
            \begin{center}
                \begin{tikzpicture}
                    \matrix (A) [matrix of nodes, row sep=0.5cm, 
                    column sep=0.5cm]
                    { 
                        & $h$ & \\  
                        $f$ & & $g$ \\
                        $d$ & & $e$\\   
                        & $c$ & \\ 
                        \textcolor{red}{$a$} &  & \textcolor{red}{$b$} \\    
                    };
                    \draw (A-1-2)--(A-2-1);
                    \draw (A-1-2)--(A-2-3);
                    \draw (A-2-1)--(A-3-1);
                    \draw (A-2-1)--(A-3-3);
                    \draw (A-2-3)--(A-3-1);
                    \draw (A-2-3)--(A-3-3);
                    \draw (A-3-1)--(A-4-2);
                    \draw (A-3-3)--(A-4-2);
                    \draw (A-4-2)--(A-5-1);
                    \draw (A-4-2)--(A-5-3);
                \end{tikzpicture}
            \end{center}

            ¿Cuál o cuáles de las siguientes expresiones son 
            \textbf{verdaderas}?
        }
        \begin{checkboxes}
            \choice El elemento maximal de $A$ es $h$ y $A$ no tiene 
            elemento minimal. 

            \choice Si $B = \{a,b\}$, entonces $B$ no tiene cota inferior en 
            $A$ y la cota superior de $B$ en $A$ es $\{h,f,d,g,e,c\}$. 
            % Correcta

            \choice Si $B = \{a,b\}$, entonces el supremo de $B$ es $c$ y el 
            ínfimo de $B$ no existe. % Correcta

            \choice El elemento máximo de $A$ es $h$ y $A$ no tiene 
            elemento mínimo. % Correcta 
            
            \choice Ninguna de las anteriores. 
        \end{checkboxes}

        % Question 14
        \question{¿Cuál o cuáles de las siguientes expresiones son 
        \textbf{verdaderas}?}
        \begin{checkboxes}
            \choice Definimos la relación $R$ sobre $\mathbb{Z}^+$ como sigue:
            \begin{equation*}
                xRy \Leftrightarrow \frac{y}{x} = 2^k \quad \quad \text{con }
                k \in \{0,1,2,\ldots\}
            \end{equation*}

            Entonces $R$ es una relación de equivalencia. 

            \choice Sea $S$ el conjunto de todas las cadenas de ceros y unos. 
            Definimos la relación $R$ sobre $S$ como sigue:
            \begin{equation*}
                xRy \Leftrightarrow x \text{ tiene la misma cantidad de ceros 
                que } y
            \end{equation*}

            Entonces $R$ es una relación de equivalencia. % Correcta

            \choice Sea $A = \{a,b\}$. Entonces todas las posibles relaciones 
            de equivalencia sobre $A$ son 
            \begin{align*}
                R_1 &= \{(a,a), (b,b)\} \\ 
                R_2 &= \{(a,a), (a,b), (b,b), (b,a)\}
            \end{align*} 

            Además, las particiones que inducen son $\{\{a\}, \{b\}\}$ y 
            $\{\{a,b\}\}$, respectivamente. % Correcta

            \choice Sea $R$ una relación binaria sobre un conjunto $A$. Si $R$ 
            es simétrica y transitiva, entonces $R$ debe ser reflexiva.
            
            \choice Ninguna de las anteriores. 
        \end{checkboxes}

        \newpage
        % Question 15
        \question{¿Cuál o cuáles de las siguientes expresiones son 
        \textbf{verdaderas}?}
        \begin{checkboxes}
            \choice Sea $R$ una relación binaria sobre un conjunto $A$. Si 
            $R$ es un orden parcial y $B \subseteq A$, entonces $R \cap B^2$ 
            es un orden parcial sobre $B$. % Correcta 

            \choice Sea $R$ una relación de equivalencia definida sobre un 
            conjunto $A$. Entonces
            \begin{center}
                $R = \{(x,x) \; | \; x \in A\}$ si y sólo 
                si para toda $x \in A$ sucede que $[x] = \{x\}$
            \end{center} % Correcta

            \choice Sean $R$ y $S$ dos relaciones binarias sobre un conjunto 
            $A$ tal que $S$ está definida como sigue:
            \begin{equation*}
                S = \{(x,y) \; | \; (x,z) \in R \text{ y } (z,y) \in R 
                \text{ para alguna } z \in A\}
            \end{equation*} 

            Si $R$ es una relación de equivalencia, entonces $S$ también lo es. 
            % Correcta 

            \choice Sea $R$ una relación reflexiva y transitiva sobre un 
            conjunto $A$. Entonces $R \cap R^{-1}$ es una relación de 
            equivalencia sobre $A$. % Correcta
            
            \choice Ninguna de las anteriores. 
        \end{checkboxes}

        % Question 16
        \question{¿Cuál o cuáles de las siguientes expresiones son 
        \textbf{verdaderas}?}
        \begin{checkboxes}
            \choice No existe una relación binaria $R$ sobre un conjunto $A$ tal 
            que $R$ sea reflexiva, simétrica, transitiva y antisimétrica al mismo 
            tiempo. 

            \choice Supongamos que $R$ es una relación binaria de orden parcial 
            sobre un conjunto $A$ y que $B \subseteq A$. Entonces $R$ también 
            es un orden parcial sobre $B$. % Correcta

            \choice Si $R$ y $S$ son relaciones de equivalencia sobre un conjunto
            $A$, entonces $R \cup S$ también es una relación de equivalencia. 

            \choice Sea $P$ el conjunto de todas las personas vivas. Definimos 
            la relación $R$ sobre $P$ como sigue:
            \begin{equation*}
                (x,y) \in R \Leftrightarrow x \text{ es al menos tan alto como }
                y
            \end{equation*}

            Entonces $R$ es una relación de equivalencia. 
            
            \choice Ninguna de las anteriores. 
        \end{checkboxes}

        \newpage
        % Question 17
        \question
        {
            Sea $(A, \preceq)$ un conjunto parcialmente ordenado cuyo diagrama
            de Hasse es el siguiente:
            \begin{center}
                \begin{tikzpicture}
                    \matrix (A) [matrix of nodes, row sep=0.95cm, 
                    column sep=0.5cm]
                    { 
                        $8$ & & & &\\  
                        $4$ & \textcolor{red}{$6$} & & \textcolor{red}{$9$} 
                        & $10$ \\
                        \textcolor{red}{$2$} & & \textcolor{red}{$3$} & & 
                        $5$ & $7$  \\   
                        & & & $1$ & & \\ 
                    };
                    \draw (A-1-1)--(A-2-1);
                    \draw (A-2-1)--(A-3-1);
                    \draw (A-3-1)--(A-2-2);
                    \draw (A-3-1)--(A-4-4);
                    \draw (A-3-1)--(A-2-5);
                    \draw (A-3-1)--(A-2-2);
                    \draw (A-2-2)--(A-3-3);
                    \draw (A-2-4)--(A-3-3);
                    \draw (A-3-3)--(A-4-4);
                    \draw (A-2-5)--(A-3-5);
                    \draw (A-3-5)--(A-4-4);
                    \draw (A-3-6)--(A-4-4);
                \end{tikzpicture}
            \end{center}

            ¿Cuál o cuáles de las siguientes expresiones son 
            \textbf{verdaderas}?
        }
        \begin{checkboxes}
            \choice Si $B = \{2,3,6,9\}$ entonces los elementos maximales de $(A,\preceq)$
            son $\{6,9\}$ y $B$ no tiene supremo.

            \choice Si $B = \{2,3,6,9\}$ entonces los elementos minimales de $(A,\preceq)$ 
            son $\{2,3\}$ y $B$ no tiene cota inferior en $A$. 

            \choice Si $B = \{2,3,6,9\}$ entonces el ínfimo de $B$ es $1$. 
            % Correcta 

            \choice Si $B = \{2,3,6,9\}$ entonces $B$ no tiene elemento máximo 
            ni mínimo y la cota superior de $B$ en $A$ es $8$.

            \choice Ninguna de las anteriores. 
        \end{checkboxes}

        % Question 18
        \question
        {
            Sea $(A, \preceq)$ un conjunto parcialmente ordenado cuyo diagrama
            de Hasse es el siguiente:
            \begin{center}
                \begin{tikzpicture}
                    \matrix (A) [matrix of nodes, row sep=0.5cm, 
                    column sep=0.5cm]
                    { 
                        $1$ & & & & $5$ & & & $8$ & & \\  
                        & $2$ & & $4$ & & $6$ & & & $9$ &\\
                        & & $3$ & & & & $7$ & & & $10$ \\   
                        & & & & & & & & $11$ & \\ 
                        & & & & & & & & & \\    
                        & & & & & & & & & \\ 
                        & & & & & & $12$ & & & \\ 
                    };
                    \draw (A-1-1)--(A-2-2);
                    \draw (A-2-2)--(A-3-3);
                    \draw (A-3-3)--(A-2-4);
                    \draw (A-2-4)--(A-1-5);
                    \draw (A-1-5)--(A-2-6);
                    \draw (A-2-6)--(A-3-7);
                    \draw (A-3-7)--(A-1-8);
                    \draw (A-1-8)--(A-2-9);
                    \draw (A-2-9)--(A-3-10);
                    \draw (A-3-10)--(A-4-9);
                    \draw (A-4-9)--(A-7-7);
                \end{tikzpicture}
            \end{center}

            ¿Cuál o cuáles de las siguientes expresiones son 
            \textbf{verdaderas}?
        }
        \begin{checkboxes}
            \choice El único elemento minimal de $A$ es $12$, lo que implica 
            que éste es el elemento mínimo de $A$. Por otro lado, los 
            elementos maximales de $A$ son $\{1,5,8\}$.

            \choice Es cierto que $7 \preceq 12$ y $12 \preceq 7$. 

            \choice El conjunto $A$ no tiene supremo, pero su ínfimo es el 
            elemento $12$. 

            \choice $A$ no tiene elemento máximo ni mínimo. % Correcta
            
            \choice Ninguna de las anteriores. 
        \end{checkboxes}
    \end{questions}
\end{document}