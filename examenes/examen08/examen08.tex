\documentclass[12pt, a4paper]{exam}

% Soporte para cambiar la fecha que sale en el examen
\usepackage{advdate}
% Soporte para escribir en varias columnas
\usepackage{multicol}
% Soporte para los acentos.
\usepackage[utf8]{inputenc} 
\usepackage[T1]{fontenc}    
% Idioma español.
\usepackage[spanish,mexico,es-tabla]{babel}
\usepackage{graphicx}
\usepackage{tikz}
\usepackage{amsmath,amssymb,amsthm}
\usepackage[linguistics]{forest}
\usepackage{verbatim} % comentarios

% Cambiamos los márgenes del documento. 
\usepackage[top=1.5cm,left=1.5cm,right=1.5cm]{geometry}

% Pie de página
\cfoot{Página \thepage\ de \numpages}

%%%%%%%%%%%%%%%%%%%%%%%%%%%%%%%%%%%%%%%%%%%%%%%%%%%%%%%%%%%%%%%%%%%%%%%%%%%%%%
\renewcommand{\thechoice}{\alph{choice}}

\makeatletter
\renewenvironment{checkboxes}%
   {\setcounter{choice}{0}\list{\checkbox@char}%
      {%
        \settowidth{\leftmargin}{W.\hskip\labelsep\hskip 2.5em}%
        \def\choice{%
          \if@correctchoice
            \color@endgroup \endgroup
          \fi
          \stepcounter{choice}
          \item[\checked@char]
          \do@choice@pageinfo
        } % choice
        \def\CorrectChoice{%
          \if@correctchoice
            \color@endgroup \endgroup
          \fi
          \ifprintanswers
            % We can't say \choice here, because that would
            % insert an \endgroup.
            % 2016/05/10: We say \color@begingroup in addition to
            % \begingroup in case \CorrectChoiceEmphasis involves color
            % and the text exactly fills the line (which would
            % otherwise create a blank line after this choice):
            % 2016/05/11: We leave hmode if we're in it,
            % i.e., if there's no blank line preceding this
            % \CorrectChoice command.  (Without this, the
            % \special created by a \color{whatever} command that might
            % be inserted by \CorrectChoice@Emphasis would be appended 
            % to the previous \choice, which could cause an extra
            % (blank) line to be inserted before this \CorrectChoice.)
            % Since \par and \endgraf seem to cancel \@totalleftmargin
            % (for reasons I don't understand), we'll do the following:
            % Motivated by  the def of \leavevmode, 
            %      \def\leavevmode{\unhbox\voidb@x}
            % we will now leave hmode (if we're in hmode):
            \ifhmode \unskip\unskip\unvbox\voidb@x \fi
            \begingroup \color@begingroup \@correctchoicetrue
            \CorrectChoice@Emphasis
            \stepcounter{choice}
            \item[\checked@char]
          \else
            \stepcounter{choice}
            \item[\checked@char]
          \fi
          \do@choice@pageinfo
        } % CorrectChoice
        \let\correctchoice\CorrectChoice
        \labelwidth\leftmargin\advance\labelwidth-\labelsep
        \topsep=0pt
        \partopsep=0pt
        \checkboxeshook
      }%
   }%
   {\if@correctchoice \color@endgroup \endgroup \fi \endlist}
 \makeatother

% Make checkbox character a circle with the letter
\checkboxchar{\tikz[baseline={([yshift=-.8ex]current bounding box.center)}]\node[shape=circle,minimum size=4mm,draw] at (0,0) {\thechoice};}
% Make checked box character bold WITH surd
%\checkedchar{\tikz[baseline={([yshift=-.8ex]current bounding box.center)}]\node[shape=circle,minimum size=8mm,draw] at (0,0) {} node at (0,0) {\thechoice\llap{$\surd$}};}
% Make checked box character bold
\checkedchar{\tikz[baseline={([yshift=-.8ex]current bounding box.center)}]\node[shape=circle,minimum size=4mm,draw] at (0,0) {} node at (0,0) {\thechoice};}
\printanswers
%%%%%%%%%%%%%%%%%%%%%%%%%%%%%%%%%%%%%%%%%%%%%%%%%%%%%%%%%%%%%%%%%%%%%%%%%%%%%%

\begin{document}
    %%%%%%%%%%%%%%%%%%%%%%%%%%%%%%%%%%%%%%%%%%%%%%%%%%%%%%%%%%%%%%%%%%%%%%%%%%%%%%%
    %%%%%%%%%%%%%%%%%%%%%%%%%%%%%%%% ENCABEZADO %%%%%%%%%%%%%%%%%%%%%%%%%%%%%%%%%%%
    \centering
    \hrule \hrule \hrule 
    \vspace{5mm}
    \begin{minipage}[c]{0.8\textwidth}
        \begin{center}
            {\large\textbf{Mission 08, Start!} \par
            \large \textbf{Estructuras Discretas} \par
            \large \textbf{Semestre 2023-1} \par
            \large \textbf{\today}	\par}
        \end{center}
    \end{minipage}

    \vspace{0.2in}
    \noindent
    \textbf{Tania Michelle Rubí Rojas}
    \vspace{2mm}
    \hrule \hrule \hrule 
    %%%%%%%%%%%%%%%%%%%%%%%%%%%%%%%%%%%%%%%%%%%%%%%%%%%%%%%%%%%%%%%%%%%%%%%%%%%%%%%
    %%%%%%%%%%%%%%%%%%%%%%%%%%%%%%%%%%%%%%%%%%%%%%%%%%%%%%%%%%%%%%%%%%%%%%%%%%%%%%%

    \vspace{5mm}
    \noindent
    Nombre y número de cuenta: \hrulefill\

    \textbf{Notación y convenciones para el examen:}
    {\tiny
    \begin{multicols}{2}
    \begin{itemize}\setlength\itemsep{0em}  
      \item En este examen, los antónimos serán complementarios.

      \item Los errores de escritura en las funciones son {\bf intencionales}, 
      por lo que cualquier afirmación que contenga una expresión mal escrita 
      es falsa.
    \end{itemize}
    \end{multicols}
    }

    \begin{questions}
        % Question 01
        \question{¿Cuál o cuáles de las siguientes expresiones son 
        \textbf{verdaderas}?}
        \begin{checkboxes}
            \choice La formalización de la proposición
            \begin{center}
                \textbf{''Las oportunidades se acaban y las ganas también.''}
            \end{center} 

            es $p \land q$, donde 
            \begin{align*}
                p &= \text{Las oportunidades se acaban} \\ 
                q &= \text{Las ganas se acaban}
            \end{align*} % Correcta 

            \choice La formalización de la proposición
            \begin{center}
                \textbf{''El backend radica en realizar diversas tareas, no 
                sólo retornar Json cuando los piden.''}
            \end{center}
            
            es $p \land \neg q$, donde 
            \begin{align*}
                p &= \text{El backend radica en realizar diversas tareas} \\ 
                q &= \text{El backend radica solamente en retornar Json cuando 
                los piden}
            \end{align*} % Correcta 

            \choice El enunciado 
            \begin{center}
                \textbf{''Las primeras críticas de la serie 1899 son alucinantes, 
                catalogándola como una especie de Titanic meets Black Mirror con 
                dosis de terror, ciencia ficción, misterio y puzzles.''}
            \end{center}

            es una proposición compuesta por $5$ proposiciones atómicas 
            diferentes.
 
            \choice La formalización de la proposición
            \begin{center}
                \textbf{''Todos en algún momento tendremos una etapa de 
                Britney Pelona. No tengo pruebas, pero tampoco dudas. ''}
            \end{center}

            es $p \land \neg q \land \neg r$
            \begin{align*}
                p &= \text{Todos en algún momento tendremos una etapa de 
                Britney Pelona} \\ 
                q &= \text{Tengo pruebas} \\ 
                r &= \text{Tengo dudas}
            \end{align*} % Correcta

            \choice Ninguna de las anteriores.
        \end{checkboxes}

        % Question 02
        \question{¿Cuál es el resultado de eliminar correctamente todos los 
        paréntesis superfluos en las siguientes expresiones de acuerdo a su 
        precedencia y asociatividad de operadores?}
        \begin{align*}
            e_1 &= \neg ((p \rightarrow (\neg q)) \lor \neg (r \land (\neg r))) 
            \land ((p \lor q) \rightarrow (p \land q))\\ 
            %e_ &= ((\neg p) \lor (\neg q)) \rightarrow \neg ((\neg q) \land r) 
            %\land \neg (p \rightarrow (\neg q)) \\ 
            e_2 &= ((\neg p) \land (q \rightarrow p)) \lor ((\neg p) \rightarrow 
            (q \land r) \lor (p \land (\neg q))) \\ 
        \end{align*}
        \begin{checkboxes}
            \choice Mucho ojo cuate, la respuesta correcta es:
            \begin{align*}
                e_1 &= \neg ((p \rightarrow \neg q) \lor \neg (r \land \neg r)) 
                \land (p \lor q \rightarrow p \land q )\\ 
                %e_2 &= (\neg p \lor \neg q) \rightarrow \neg (\neg q \land r) 
                %\land \neg (p \rightarrow \neg q) \\ 
                e_2 &= \neg p \land (q \rightarrow p) \lor \neg p \rightarrow 
                q \land r \lor p \land \neg q \\ 
            \end{align*}

            \choice Mucho ojo cuate, la respuesta correcta es:
            \begin{align*}
                e_1 &= \neg ((p \rightarrow \neg q) \lor \neg (r \land \neg r)) 
                \land (p \lor q) \rightarrow p \land q \\ 
                %e_2 &= \neg p \lor \neg q \rightarrow \neg (\neg q \land r) 
                %\land \neg (p \rightarrow \neg q)  \\ 
                e_2 &= \neg p \land (q \rightarrow p) \lor \neg p \rightarrow 
                (q \land r) \lor (p \land \neg q)
            \end{align*}

            \choice Mucho ojo cuate, la respuesta correcta es:
            \begin{align*}
                e_1 &= \neg (p \rightarrow \neg q \lor \neg (r \land \neg r)) 
                \land (p \lor q) \rightarrow p \land q \\ 
                %e_2 &= \neg p \lor \neg q \rightarrow \neg \neg q \land r 
                %\land \neg (p \rightarrow \neg q)  \\ 
                e_2 &= \neg p \land (q \rightarrow p) \lor (\neg p \rightarrow 
                q \land r \lor p \land \neg q)
            \end{align*}

            \choice Mucho ojo cuate, la respuesta correcta es:
            \begin{align*}
                e_1 &= \neg (p \rightarrow \neg q \lor \neg (r \land \neg r)) 
                \land (p \lor q \rightarrow p \land q) \\ 
                %e_2 &= \neg p \lor \neg q \rightarrow \neg (\neg q \land r) 
                %\land \neg (p \rightarrow \neg q) \\ 
                e_2 &= \neg p \land (q \rightarrow p) \lor \neg p \rightarrow 
                (q \land r) \lor (p \land \neg q)
            \end{align*}
            
            \choice Ninguna de las anteriores. % Correcta
        \end{checkboxes}

        % Question 03
        \question{¿Cuál o cuáles de las siguientes expresiones son 
        \textbf{verdaderas}?}
        \begin{checkboxes}
            \choice El enunciado 
            \begin{center}
                \textbf{''Netflix estrena en 4 días la nueva serie de los 
                creadores de DARK: 1899''}
            \end{center}

            es una proposición atómica. % Correcta

            \choice El enunciado
            \begin{center}
                \textbf{''Todos luchamos nuestras propias guerras privadas.''}
            \end{center}

            \textbf{no} es una proposición porque contiene la palabra 
            \texttt{Todos} y la lógica proposicional no alcanza para
            darle un valor de verdad binario. 

            \choice Una proposición es compuesta si se puede descomponer 
            en dos o más proposiciones atómicas.  % Correcta

            \choice El enunciado 
            \begin{center}
                \textbf{''La falta de comunicación da demasiado espacio para 
                la imaginación''}
            \end{center}

            es una proposición atómica. % Correcta
            
            \choice Ninguna de las anteriores. 
        \end{checkboxes}

        \newpage
        % Question 04
        \question{¿Cuál o cuáles de las siguientes expresiones son 
        \textbf{verdaderas}?}
        \begin{checkboxes}
            \choice La formalización de la proposición
            \begin{center}
                \textbf{''Nubecita sabe programar en Python o ella ha 
                aprendido a programar en C y Java''}
            \end{center}

            es $p \lor (q \land r)$, donde 
            \begin{align*}
                p &: \text{Nubecita sabe programar en Python} \\ 
                q &: \text{Nubecita ha aprendido a programar en C} \\ 
                r &: \text{Nubecita ha aprendido a programar en Java}
            \end{align*} % Correcta

            \choice La formalización de la proposición
            \begin{center}
                \textbf{''Marco es pobre o es tanto rico como infeliz''}
            \end{center}

            es $\neg p \lor p \land \neg q$, donde 
            \begin{align*}
                p &: \text{Marco es rico} \\ 
                q &: \text{Marco es feliz}
            \end{align*} % Correcta 

            \choice La formalización de la proposición
            \begin{center}
                \textbf{''Les compartiré un Roadmap de SQL que pude armar con 
                el contenido que he creado o mi lenguaje del amor no es arruinar 
                mi horario del sueño por hablar un ratito más con Circe''}
            \end{center}

            es $p \lor \neg q$, donde 
            \begin{align*}
                p &: \text{Voy a compartir un Roadmap de SQL que pude armar con 
                el contenido que he} \\ &\text{creado} \\ 
                q &: \text{Mi lenguaje del amor es arruinar mi horario del 
                sueño por hablar un ratito} \\ &\text{más con Circe}
            \end{align*} % Correcta

            \choice La formalización de la proposición
            \begin{center}
                \textbf{''Hoy recibí siete cartas tuyas al mismo tiempo y 
                fue un buen día. Cada una de ellas representaba el inmenso 
                amor que siento por tí y al mismo tiempo me recorbaban lo 
                mucho que te extraño.''}
            \end{center}

            es $(p \land q) \land (r \land s)$, donde 
            \begin{align*}
                p &= \text{Hoy recibí siete cartas tuyas al mismo tiempo} \\ 
                q &= \text{Hoy fue un buen día} \\ 
                r &= \text{Cada una de tus siete cartas representaba el inmenso 
                amor que siento por tí} \\ 
                s &= \text{Cada una de tus siete cartas me recorbaban lo mucho 
                que te extraño}
            \end{align*} % Correcta
            
            \choice Ninguna de las anteriores. 
        \end{checkboxes}

        \newpage
        % Question 05
        \question
        {
            ¿Cuál es el resultado de colocar correctamente todos los paréntesis 
            en las siguientes expresiones de acuerdo a su precedencia y 
            asociatividad de operadores?
            \begin{align*}
                %e_ &= p \land q \lor r \rightarrow s \rightarrow r \\ 
                e_1 &= p \lor q \rightarrow r \rightarrow s \leftrightarrow t \\ 
                e_2 &= p \rightarrow q \rightarrow p \lor q \rightarrow q
            \end{align*}
        }
        \begin{checkboxes}
            \choice Mucho ojo cuate, la respuesta correcta es:
            \begin{align*}
                %e_1 &= (((p \land q) \lor r) \rightarrow (s \rightarrow r)) \\ 
                e_1 &= (((p \lor q) \rightarrow (r \rightarrow s)) \leftrightarrow t) \\ 
                e_2 &= (p \rightarrow (q \rightarrow ((p \lor q) \rightarrow q)))
            \end{align*} % Correcta 

            \choice Mucho ojo cuate, la respuesta correcta es:
            \begin{align*}
                %e_1 &= ((((p \land q) \lor r) \rightarrow s) \rightarrow r) \\ 
                e_1 &= ((((p \lor q) \rightarrow r) \rightarrow s) \leftrightarrow t) \\ 
                e_2 &= (((p \rightarrow q) \rightarrow (p \lor q)) \rightarrow q)
            \end{align*}

            \choice Mucho ojo cuate, la respuesta correcta es:
            \begin{align*}
                %e_1 &= ((((p \land q) \lor r) \rightarrow s) \rightarrow r) \\ 
                e_1 &= (((p \lor q) \rightarrow (r \rightarrow s)) \leftrightarrow t) \\ 
                e_2 &= ((p \rightarrow q) \rightarrow ((p \lor q) \rightarrow q))
            \end{align*}

            \choice Mucho ojo cuate, la respuesta correcta es:
            \begin{align*}
                %e_1 &= (((p \land q) \lor r) \rightarrow (s \rightarrow r)) \\ 
                e_1 &= ((((p \lor q) \rightarrow r) \rightarrow s) \leftrightarrow t)\\ 
                e_2 &= (p \rightarrow (q \rightarrow ((p \lor q) \rightarrow q)))
            \end{align*}
            
            \choice Ninguna de las anteriores. 
        \end{checkboxes}

        % Question 06
        \question{¿Cuál o cuáles de las siguientes expresiones son 
        \textbf{verdaderas}?}
        \begin{checkboxes}
            \choice El enunciado
            \begin{center}
                \textbf{''Cuando era niño despertaba todos los días esperando 
                que ese fuera el día en que volviera mi mamá, esperé por 
                quince años.''}
            \end{center}

            es una proposición compuesta. % Correcta 

            \choice El enunciado
            \begin{center}
                \textbf{''Me dijeron <<Arriba las estúpidas>> y volé''}
            \end{center}

            \textbf{no} es una proposición, pues como no se indica quién 
            me dijo esa oración, entonces no puedo determinar su valor de 
            verdad. 

            \choice Una proposición obtenida a partir de otras proposiciones
            mediante el uso de conectivos lógicos se llama compuesta. % Correcta

            \choice El enunciado
            \begin{center}
                \textbf{''Hay veces que el dolor duerme en una canción''}
            \end{center}

            es una proposición atómica. % Correcta 
            
            \choice Ninguna de las anteriores.
        \end{checkboxes}

        \newpage
        % Question 07
        \question
        {
            ¿Cuál es el árbol de sintaxis correcto de la siguiente expresión?
            \begin{equation*}
                (p \rightarrow q) \land (q \rightarrow r) \rightarrow \neg r 
                \rightarrow \neg p
            \end{equation*}
        }
        \begin{multicols}{2}
            \begin{checkboxes}
                \forestset{
                    default preamble={
                    for tree={circle,draw}
                    }
                }
                \choice 
                \begin{forest}
                    for tree={dotted}
                    [$\rightarrow$
                        [$\land$
                            [$\rightarrow$ [$p$] [$q$]]
                            [$\rightarrow$ 
                                [$\rightarrow$ [$q$] [$r$]]
                                [$\neg$ [$r$]]
                            ]
                        ]
                        [$\neg$ [$p$]]
                    ]
                \end{forest} 

                \vspace*{0.5cm}

                \choice 
                \begin{forest}
                    for tree={dotted}
                    [$\rightarrow$
                        [$\land$
                            [$\rightarrow$ 
                                [$\rightarrow$ [$q$] [$r$]]
                                [$\neg$ [$r$]]
                            ]
                            [$\rightarrow$ [$p$] [$q$]]
                        ]
                        [$\neg$ [$p$]]
                    ]
                \end{forest} 

                \choice 
                \begin{forest}
                    for tree={dotted}
                    [$\land$
                        [$\rightarrow$ [$p$] [$q$]]
                        [$\rightarrow$ 
                            [$\rightarrow$ [$q$] [$r$]] 
                            [$\rightarrow$ [$\neg$ [$r$] [$p$]]]
                        ]
                    ]
                \end{forest}

                \vspace*{0.5cm}
    
                \choice 
                \begin{forest}
                    for tree={dotted}
                    [$\rightarrow$
                        [$\land$
                            [$\rightarrow$ [$p$] [$q$]]
                            [$\rightarrow$ [$q$] [$r$]]
                        ]
                        [$\rightarrow$ [$\neg$ [$r$]] [$\neg$ [$p$]]]
                    ]
                \end{forest} % Correcta

                \vspace*{0.2cm}
                
                \choice Ninguna de las anteriores. 
            \end{checkboxes}
        \end{multicols}

        % Question 08
        \question{¿Cuál o cuáles de las siguientes expresiones son 
        \textbf{verdaderas}?}
        \begin{checkboxes}
            \choice Los enunciados 
            \begin{align*}
                &a. \quad  x < 2 \text{ o no es cierto que } 1 < x < 3 \\ 
                &b. \quad x \leq 1 \text{ o bien } x < 2 \text{ o } x \geq 3
            \end{align*}

            tienen el mismo valor de verdad. % Correcta

            \choice El enunciado ''$2 + 2 = 5$'' \textbf{no} es una 
            proposición. 

            \choice El enunciado
            \begin{center}
                \textbf{''En las hojas de otoño esconderé mis temores''}
            \end{center}

            \textbf{no} es una proposición. 

            \choice El enunciado
            \begin{center}
                \textbf{''Al contrario del INE, a mí sí tóquenme''}
            \end{center}

            es una proposición compuesta. 
            
            \choice Ninguna de las anteriores. 
        \end{checkboxes}

        \newpage
        % Question 09
        \question{¿Cuál o cuáles de las siguientes expresiones son 
        \textbf{verdaderas}?}
        \begin{checkboxes}
            \choice La formalización de la proposición
            \begin{center}
                \textbf{''No estoy preparado para afrontar otro examen''}
            \end{center}

            es $\neg p$, donde $p:$ Estoy preparado para afrontar otro examen.
            % Correcta 

            \choice El enunciado
            \begin{center}
                \textbf{''En Corea del Sur, el agua de lluvia se almacena para 
                luego limpiar las calles a través de un sistema hidráulico.''}
            \end{center}

            es una proposición atómica. % Correcta 

            \choice La formalización del enunciado (tóxico)
            \begin{center}
                \textbf{''Respóndeme o me entierro un cuchillo en el 
                estómago''}
            \end{center}

            es $p \lor q$, donde 
            \begin{align*}
                p &: \text{Respóndeme} \\ 
                q &: \text{Me entierro un cuchillo en el estómago}
            \end{align*}

            \choice La formalización de la proposición 
            \begin{center}
                \textbf{''La amistad y el amor siempre brillaron y 
                hoy también lo harán''}
            \end{center}

            es $(p \land q) \land (r \land s)$, donde 
            \begin{align*}
                p &: \text{ La amistad siempre brilló} \\ 
                q &: \text{ El amor siempre brilló} \\ 
                r &: \text{ La amistad brillará hoy} \\ 
                s &: \text{ El amor brillará hoy}
            \end{align*} % Correcta
            
            \choice Ninguna de las anteriores. 
        \end{checkboxes}

        % Question 10
        \question{¿Cuál o cuáles de las siguientes expresiones son 
        \textbf{verdaderas}?}
        \begin{checkboxes}
            \choice El enunciado
            \begin{center}
                \textbf{''Todo arde si le aplicas la chispa adecuada''}
            \end{center}

            es una proposición atómica. % Correcta

            \choice El enunciado 
            \begin{center}
                \textbf{''Gracias a Ivanna este año fue un poquito mejor''}
            \end{center}

            es una proposición atómica. % Correcta

            \choice El enunciado 
            \begin{center}
                \textbf{''La película más bonita del mundo es también la 
                más triste''}
            \end{center}

            es una proposición compuesta. 
            
            \choice El enunciado
            \begin{center}
                \textbf{''Luciérnaga azul ven y dime qué quieres de mí''}
            \end{center}

            \textbf{no} es una proposición. % Correcta

            \choice Ninguna de las anteriores. 
        \end{checkboxes}

        \newpage
        % Question 11
        \question{¿Cuál o cuáles de las siguientes expresiones son 
        \textbf{verdaderas}?}
        \begin{checkboxes}
            \choice El enunciado 
            \begin{center}
                \textbf{''Cada problema tiene solución. Pero si no hablo, 
                entonces nadie puede ayudarme''}
            \end{center}

            \textbf{no} es una proposición ya que a la primera oración no es 
            posible asignarle un valor de verdad. 

            \choice La formalización de la proposición
            \begin{center}
                \textbf{''Daniel se fracturó la mano y no pudo presentar su 
                examen de discretas''}
            \end{center}

            es $p \land \neg q$, donde 
            \begin{align*}
                p &= \text{Daniel se fracturó la mano} \\ 
                q &= \text{Daniel pudo presentar su examen de discretas}
            \end{align*} % Correcta

            \choice La formalización de la proposición
            \begin{center}
                \textbf{''Las cicatrices están sanando y todo está volviendo 
                a ser normal''}
            \end{center}

            es $q \land p$, donde 
            \begin{align*}
                p &: \text{Las cicatrices están sanando} \\ 
                q &: \text{Todo está volviendo a ser normal}
            \end{align*} % Correcta

            \choice El enunciado 
            \begin{center}
                \textbf{''El Museo de la Ciudad de Mérida exhibe pinturas de 
                autores yucatecos, entre los cuales incluyó a Fernando 
                Castro Pacheco y Rosa María García Ruíz''}
            \end{center}

            es una proposición compuesta por $3$ proposiciones atómicas
            diferentes. % Correcta
            
            \choice Ninguna de las anteriores. 
        \end{checkboxes}

        % Question 12
        \question{¿Cuál o cuáles de las siguientes expresiones son 
        \textbf{verdaderas}?}
        \begin{checkboxes}
            \choice El enunciado 
            \begin{center}
                \textbf{''El olvido recordó y la obscuridad se iluminó. La 
                risa rompió a llorar''}
            \end{center}

            es una proposición compuesta. % Correcta

            \choice El enunciado
            \begin{center}
                \textbf{''Te quiero Sebastián, te lo he dicho frente al mar. 
                El cariño no se oxida ante la sal.''}
            \end{center}

            es una proposición compuesta. % Correcta

            \choice El enunciado 
            \begin{center}
                \textbf{''Mi cuerpo está en calma y mi mente en paz''}
            \end{center}

            es una proposición compuesta. % Correcta

            \choice La formalización de la proposición
            \begin{center}
                \textbf{''Hoy no he venido a disculparme ni tampoco a 
                suplicar''}
            \end{center}
    
            no existe pues esta expresión no es una proposición, es más bien 
            una oración imperativa. 
            
            \choice Ninguna de las anteriores. 
        \end{checkboxes}

        \newpage
        % Question 13
        \question{¿Cuál es el resultado de eliminar correctamente todos los 
        paréntesis superfluos en las siguientes expresiones de acuerdo a su 
        precedencia y asociatividad de operadores?}
        \begin{align*}
            %e_1 &= ((p \land q) \lor ((\neg p) \lor (p \land \neg q))) 
            %\rightarrow p \lor (\neg p) \\ 
            e_1 &= ((((\neg p) \lor q) \lor (p \land (\neg q))) \land (\neg 
            (\neg p) \leftrightarrow (s \lor p) \land s)) \\ 
            e_2 &= (((\neg p) \lor q) \rightarrow s) \land ((p \land (\neg q)) 
            \lor p) \lor (p \rightarrow (q \lor r))
        \end{align*}
        \begin{checkboxes}
            \choice Mucho ojo cuate, la respuesta correcta es:
            \begin{align*}
                %e_1 &= p \land q \lor \neg p \lor p \land \neg q
                %\rightarrow p \lor \neg p  \\ 
                e_1 &= (\neg p \lor q \lor p \land \neg q) \land (\neg 
                \neg p \leftrightarrow (s \lor p) \land s) \\ 
                e_2 &= \neg p \lor q \rightarrow s \land p \land \neg q 
                \lor p \lor p \rightarrow q \lor r
            \end{align*}

            \choice Mucho ojo cuate, la respuesta correcta es:
            \begin{align*}
                %e_1 &= ((p \land q) \lor ((\neg p) \lor (p \land \neg q))) 
                %\rightarrow p \lor (\neg p) \\ 
                e_1 &= (\neg p \lor q \lor p \land \neg q) \land \neg 
                \neg p \leftrightarrow (s \lor p) \land s \\ 
                e_2 &= (\neg p \lor q \rightarrow s) \land p \land \neg q 
                \lor p \lor (p \rightarrow q \lor r)
            \end{align*}

            \choice Mucho ojo cuate, la respuesta correcta es:
            \begin{align*}
                %e_1 &= (p \land q \lor (\neg p \lor p \land \neg q)) 
                %\rightarrow p \lor \neg p \\ 
                e_1 &= (\neg p \lor q \lor p \land \neg q) \land (\neg 
                \neg p \leftrightarrow (s \lor p) \land s) \\ 
                e_2 &= (\neg p \lor q \rightarrow s) \land p \land \neg q 
                \lor p \lor (p \rightarrow q \lor r)
            \end{align*}

            \choice Mucho ojo cuate, la respuesta correcta es:
            \begin{align*}
                %e_1 &= p \land q \lor (\neg p \lor p \land \neg q)
                %\rightarrow p \lor \neg p \\ 
                e_1 &= (\neg p \lor q \lor p \land \neg q) \land \neg 
                \neg p \leftrightarrow (s \lor p) \land s \\ 
                e_2 &= (\neg p \lor q \rightarrow s) \land (p \land \neg q 
                \lor p) \lor (p \rightarrow q \lor r)
            \end{align*} 
            
            \choice Ninguna de las anteriores. % Correcta 
        \end{checkboxes}
    
        % Question 14
        \question{¿Cuál o cuáles de las siguientes expresiones son 
        \textbf{verdaderas}?}
        \begin{checkboxes}    
            \choice El enunciado
            \begin{center}
                \textbf{''Sacrificar la siesta por ver a alguien me parece 
                el acto de amor más grande que puedo hacer por alguien''}
            \end{center}
    
            es una proposición atómica. % Correcta
    
            \choice El enunciado 
            \begin{center}
                \textbf{¿Te queda claro lo que hay que hacer?}
            \end{center}
    
            \textbf{no} es una proposición. % Correcta
    
            \choice El enunciado 
            \begin{center}
                \textbf{''Se descubrió una población de osos polares que no 
                depende del todo del hielo antártico para cazar, comer y 
                reproducirse. Este hallazgo fue en el sureste de Groenlandia. 
                Esto representa una alternativa para la conservación de la 
                especie, por el deshielo de los polos''}
            \end{center}
    
            es una proposición compuesta de $5$ proposiciones atómicas 
            diferentes.

            \choice El enunciado
            \begin{center}
                \textbf{''Morirás cada mañana y renacerás al anochecer'}
            \end{center}
    
            es una proposición compuesta. % Correcta
            
            \choice Ninguna de las anteriores. 
        \end{checkboxes}

        \newpage
        % Question 15
        \question{¿Cuál o cuáles de las siguientes expresiones son 
        \textbf{verdaderas}?}
        \begin{checkboxes}
            \choice La formalización de la proposición
            \begin{center}
                \textbf{''x es un número primo y divisible entre 9''}
            \end{center}
    
            es $p \land q$, donde 
            \begin{align*}
                p &: \text{ $x$ es un número primo} \\ 
                q &: \text{ $x$ es divisible entre $9$}
            \end{align*} % Correcta

            \choice La formalización de la proposición
            \begin{center}
                \textbf{''Las últimas semanas han sido duras para el mercado 
                laboral en el mundo Tech.''}
            \end{center}

            es una proposición atómica. % Correcta

            \choice El enunciado 
            \begin{center}
                \textbf{''En el cuarto hay una mesa pequeña y una silla. 
                Fuera de la ventana, hay una planicie sin nada. La construcción 
                es vieja, nadie sabe cuándo fue hecha. Vivo aquí, sola''}
            \end{center}

            es una proposición compuesta de cinco proposiciones atómicas 
            diferentes. 

            \choice El enunciado
            \begin{center}
                \textbf{''El mundo es hermoso. Aunque estés llena de 
                tristezas, abre tus ojos. Haz lo que quieras, se quién eres, 
                y nunca te apures en crecer.''}
            \end{center}

            es una proposición compuesta. 

            \choice Ninguna de las anteriores. 
        \end{checkboxes}

        % Question 16
        \question{¿Cuál o cuáles de las siguientes expresiones son 
        \textbf{verdaderas}?}
        \begin{checkboxes}
            \choice El enunciado 
            \begin{center}
                \textbf{''El infierno se enfrío y el cielo se cubrió de una 
                telaraña gris, de nubes y dolor''}
            \end{center}

            es una proposición compuesta que tiene cuatro proposiciones atómicas
            diferentes. % Correcta  

            \choice El enunciado
            \begin{center}
                \textbf{''Si las puertas de la percepción fueran 
                limpiadas, todo parecería como lo es, infinito''}
            \end{center}

            es una proposición compuesta. % Correcta 

            \choice El enunciado 
            \begin{center}
                \textbf{''El universo es más grande que cualquier cosa que 
                pueda caber en tu mente.''}
            \end{center}

            es una proposición atómica. % Correcta

            \choice El enunciado 
            \begin{center}
                \textbf{''Existen muchas experiencias humanas que desafían los 
                límites de nuestro lenguaje. Esa es una de las razones por las 
                cuales tenemos a la poesía''}
            \end{center}

            \textbf{no} es una proposición. 

            \choice Ninguna de las anteriores. 
        \end{checkboxes}

        \newpage
        % Question 17
        \question{¿Cuál o cuáles de las siguientes expresiones son 
        \textbf{verdaderas}?}
        \begin{checkboxes}
            \choice La formalización de la proposición
            \begin{center}
                \textbf{''Es bonito resignificar en algo nuevo las cosas que 
                por mucho tiempo me dolieron y me despertaban por las noches.
                Es bonito que la vida siga''}
            \end{center}

            es $p \land q \land r$, donde 
            \begin{align*}
                p &= \text{Es bonito resignificar en algo nuevo las cosas que 
                por mucho tiempo} \\ &\text{me dolieron} \\ 
                q &= \text{Es bonito resignificar en algo nuevo las cosas que 
                por mucho tiempo} \\ &\text{me despertaban por las noches} \\ 
                r &= \text{Es bonito que la vida siga}  
            \end{align*} % Correcta

            \choice El enunciado
            \begin{center}
                \textbf{''Hace un año comencé a aprender JavaScript y 
                sigo sin saber JavaScript''}
            \end{center}

            es una proposición compuesta. % Correcta

            \choice El enunciado
            \begin{center}
                \textbf{''Elon Musk despidió a un programador de Twitter que 
                lo contradijo en público respecto a por qué la app funciona 
                lenta en algunos países''}
            \end{center}

            es una proposición atómica. % Correcta 

            \choice El enunciado
            \begin{center}
                \textbf{''Divulgar sobre Open Source, también es hacer Open 
                Source''}
            \end{center}

            es una proposición compuesta, pues tiene el conectivo 
            \texttt{también}. 

            \choice Ninguna de las anteriores. 
        \end{checkboxes}

        % Question 19
        \question{¿Cuál o cuáles de las siguientes expresiones son 
        \textbf{verdaderas}?}
        \begin{checkboxes}
            \choice La formalización del enunciado
            \begin{center}
                \textbf{''Las habilidades necesarias en Data Science no siempre 
                son fáciles de demostrar en una entrevista''}
            \end{center}

            es $\neg p$, donde 
            \begin{align*}
                p &: \text{Las habilidades necesarias en Data Science siempre 
                son fáciles de} \\ &\text{demostrar en una entrevista}
            \end{align*} % Correcta 

            \choice El enunciado
            \begin{center}
                \textbf{''Quiero tener un carro para llevar a mis amigas a 
                toda velocidad y decirles <<ya mi vida no tiene sentido>>''}
            \end{center}

            \textbf{no} es una proposición.

            \choice El enunciado
            \begin{center}
                \textbf{''Si nos pusieramos de acuerdo, hoy mismo podría ser 
                viernes. Sólo nos falta organización.''}
            \end{center}

            \textbf{no} es una proposición. 

            \choice El enunciado
            \begin{center}
                \textbf{''Hay pequeños detalles y momentos que arreglan grandes 
                días de mierda. Hoy es un día de esos.''}
            \end{center}

            es una proposición compuesta por tres proposiciones atómicas 
            diferentes. % Correcta 

            \choice Ninguna de las anteriores. 
        \end{checkboxes}

        \newpage
        % Question 20
        \question
        {
            ¿Cuál es el árbol de sintaxis correcto de la siguiente expresión?
            \begin{equation*}
                \neg p \land q \rightarrow p \land (q \lor \neg r)
            \end{equation*}
        }
        \begin{multicols}{2}
            \begin{checkboxes}
                \forestset{
                    default preamble={
                    for tree={circle,draw}
                    }
                }
                \choice 
                \begin{forest}
                    for tree={dotted}
                    [$\rightarrow$
                        [$\land$ 
                            [$\neg$ [$p$]] 
                            [$q$]
                        ]
                        [$\land$ 
                            [$p$] 
                            [$\lor$ [$q$] [$\neg$ [$r$]]]
                        ]
                    ]
                \end{forest} % Correcta

                \choice 
                \begin{forest}
                    for tree={dotted}
                    [$\land$
                        [$\neg$ [$p$]]
                        [$\rightarrow$ 
                            [$q$]
                            [$\land$ [$p$] [$\lor$ [$q$] [$\neg$ [$r$]]]]]
                    ]
                \end{forest}
    
                \choice 
                \begin{forest}
                    for tree={dotted}
                    [$\land$
                        [$\rightarrow$ 
                            [$\land$ [$\neg$ [$p$]] [$q$]]
                            [$p$]
                        ]
                        [$\lor$ [$q$] [$\neg$ [$r$]]] 
                    ]
                \end{forest}

                \choice 
                \begin{forest}
                    for tree={dotted}
                    [$\land$
                        [$\land$ 
                            [$\neg$ [$p$]]
                            [$\rightarrow$ [$q$] [$p$]]
                        ]
                        [$\lor$ [$q$] [$\neg$ [$r$]]] 
                    ]
                \end{forest}

                \vspace*{1.5cm}
                
                \choice Ninguna de las anteriores. 
            \end{checkboxes}
        \end{multicols}
\end{questions}
\end{document}

    \begin{comment}
    \question
    {
        ¿Cuál es el resultado de colocar correctamente todos los paréntesis 
        en las siguientes expresiones de acuerdo a su precedencia y 
        asociatividad de operadores?
        \begin{align*}
            e_1 &= \neg p \lor q \rightarrow p \land \neg q \rightarrow 
            \neg p \lor \neg q \rightarrow \neg p \land \neg q \\ 
            e_2 &= \neg \neg p \land \neg q \rightarrow s \leftrightarrow 
            \neg s \rightarrow \neg p \lor q\\ 
            e_3 &= \neg p \land (\neg p \land q) \lor p \land p \land \neg q
        \end{align*}
    }
    \begin{checkboxes}
        \choice Mucho ojo cuate, la respuesta correcta es:
        \begin{align*}
            e_1 &= (((\neg p) \lor q) \rightarrow (((p \land (\neg q)) \rightarrow 
            ((\neg p) \lor (\neg q))) \rightarrow ((\neg p) \land (\neg q)))) \\ 
            e_2 &= (((((\neg \neg p) \land (\neg q)) \rightarrow s) \leftrightarrow 
            (\neg s)) \rightarrow ((\neg p) \lor q)) \\ 
            e_3 &= (((\neg p) \land ((\neg p)) \land q) \lor ((p \land p) \land (\neg q)))
        \end{align*}

        \choice Mucho ojo cuate, la respuesta correcta es:
        \begin{align*}
            e_1 &= (((\neg p) \lor q) \rightarrow ((p \land (\neg q)) \rightarrow 
            (((\neg p) \lor (\neg q)) \rightarrow ((\neg p) \land (\neg q))))) \\ 
            e_2 &= ((((\neg (\neg p)) \land (\neg q)) \rightarrow s) \leftrightarrow 
           ( (\neg s) \rightarrow ((\neg p) \lor q))) \\ 
            e_3 &= (((\neg p) \land ((\neg p) \land q)) \lor ((p \land p) \land (\neg q)))
        \end{align*} % Correcta

        \choice Mucho ojo cuate, la respuesta correcta es:
        \begin{align*}
            e_1 &= (((((\neg p) \lor q) \rightarrow (p \land (\neg q))) \rightarrow 
            ((\neg p) \lor (\neg q))) \rightarrow ((\neg p) \land (\neg q))) \\ 
            e_2 &= (((\neg \neg p) \land (\neg q)) \rightarrow (s \leftrightarrow 
            ((\neg s) \rightarrow ((\neg p) \lor q))))  \\ 
            e_3 &= (((\neg p) \land (\neg p \land q)) \lor ((p \land p) \land (\neg q)))
        \end{align*}

        \choice Mucho ojo cuate, la respuesta correcta es:
        \begin{align*}
            e_1 &= (((\neg p) \lor q) \rightarrow (p \land (\neg q))) \rightarrow 
            (((\neg p) \lor (\neg q)) \rightarrow ((\neg p) \land (\neg q)))\\ 
            e_2 &= (((\neg (\neg p)) \land (\neg q)) \rightarrow s) \leftrightarrow 
            ((\neg s) \rightarrow ((\neg p) \lor q)) \\ 
            e_3 &= (((((\neg p) \land ((\neg p) \land q)) \lor p) \land p) \land (\neg q))
        \end{align*}
        
        \choice Ninguna de las anteriores. 
    \end{checkboxes}
    %%%%%%%%%%%%%%%%%%%%%%%%%%%%%%%%%%%%%%%%%%%%%%%%%%%%%%%%%%%%%%%%%%%%%%%%%%%
    \question
    {
        ¿Cuál es el resultado de colocar correctamente los paréntesis en 
        las siguientes expresiones de acuerdo a su precedencia y 
        asociatividad de los operadores?
        \begin{align*}
            e_1 &= p \leftrightarrow q \land r \leftrightarrow p \leftrightarrow 
            p \leftrightarrow q \land r \leftrightarrow q \land p \rightarrow 
            q \land r \land \neg p \rightarrow s\\ 
            e_2 &= p \lor q \land \neg p \land \neg q \land \neg p \lor q \rightarrow 
            p \land r \leftrightarrow s \land t \rightarrow u \lor p\\ 
            e_3 &= p \lor q \lor r \land \neg (r \lor \neg s) \land s \leftrightarrow t 
            \land p \rightarrow \neg t \land q \rightarrow p \lor \neg t
        \end{align*}
    }
    \begin{checkboxes}
        \choice Mucho ojo cuate, la respuesta correcta es:
        \begin{align*}
            e_1 &= p \leftrightarrow (q \land r) \leftrightarrow p \leftrightarrow 
            p \leftrightarrow (q \land r) \leftrightarrow (q \land p) \rightarrow 
           ( (q \land r) \land \neg p) \rightarrow s \\ 
            e_2 &= p \lor q \land \neg p \land \neg q \land \neg p \lor q \rightarrow 
            p \land r \leftrightarrow s \land t \rightarrow u \lor p\\ 
            e_3 &= p \lor q \lor r \land \neg (r \lor \neg s) \land s \leftrightarrow t 
            \land p \rightarrow \neg t \land q \rightarrow p \lor \neg t
        \end{align*}

        \choice Mucho ojo cuate, la respuesta correcta es:
        \begin{align*}
            e_1 &= p \leftrightarrow (q \land r) \leftrightarrow p \leftrightarrow 
            p \leftrightarrow (q \land r) \leftrightarrow (q \land p) \rightarrow 
           ( (q \land r) \land \neg p) \rightarrow s  \\ 
            e_2 &= p \lor q \land \neg p \land \neg q \land \neg p \lor q \rightarrow 
            p \land r \leftrightarrow s \land t \rightarrow u \lor p\\ 
            e_3 &= p \lor q \lor r \land \neg (r \lor \neg s) \land s \leftrightarrow t 
            \land p \rightarrow \neg t \land q \rightarrow p \lor \neg t
        \end{align*} % Correcta

        \choice Mucho ojo cuate, la respuesta correcta es:
        \begin{align*}
            e_1 &= (((((p \leftrightarrow (q \land r)) \leftrightarrow p) \leftrightarrow 
            p) \leftrightarrow (q \land r)) \leftrightarrow (q \land p)) \rightarrow 
            (((q \land r) \land \neg p) \rightarrow s)\\ 
            e_2 &= ((((((((p \lor q) \land \neg p) \land \neg q) \land \neg p) \lor q) 
            \rightarrow (p \land r)) \leftrightarrow (s \land t)) \rightarrow (u \lor p)) \\ 
            e_3 &= (((((p \lor q) \lor r) \land \neg (r \lor \neg s)) \land s) 
            \leftrightarrow ((t \land p) \rightarrow ((\neg t \land q) \rightarrow (p \lor \neg t))))
        \end{align*}

        \choice Mucho ojo cuate, la respuesta correcta es:
        \begin{align*}
            e_1 &= (p \leftrightarrow ((q \land r) \leftrightarrow (p \leftrightarrow 
            (p \leftrightarrow ((q \land r) \leftrightarrow ((q \land p) \rightarrow 
            ((q \land (r \land \neg p)) \rightarrow s))))))) \\ 
            e_2 &= ((((p \lor (((q \land \neg p) \land \neg q) \land \neg p)) \lor q) 
            \rightarrow (p \land r)) \leftrightarrow ((s \land t) \rightarrow (u \lor p)))\\ 
            e_3 &= (((p \lor q) \lor ((r \land \neg (r \lor \neg s)) \land s)) \leftrightarrow ((t 
            \land p) \rightarrow ((\neg t \land q) \rightarrow (p \lor \neg t))))
        \end{align*}
        
        \choice Ninguna de las anteriores. 
    \end{checkboxes}
    %%%%%%%%%%%%%%%%%%%%%%%%%%%%%%%%%%%%%%%%%%%%%%%%%%%%%%%%%%%%%%%%%%%%%%%%%%%
    \question
    {
        ¿Cuál es el resultado de colocar correctamente todos los paréntesis 
        en las siguientes expresiones de acuerdo a su precedencia y 
        asociatividad de operadores?
        \begin{align*}
            e_1 &= p \rightarrow q \rightarrow p \lor q \rightarrow q \lor \neg p 
            \lor q \rightarrow r \\ 
            e_2 &= p \rightarrow q \leftrightarrow \neg q \rightarrow \neg p 
            \lor p \land q \land r \rightarrow p \\ 
            e_3 &= p \rightarrow q \land r \rightarrow s \rightarrow p 
            \rightarrow s \lor p \rightarrow q \land q \rightarrow p 
        \end{align*}
    }
    \begin{checkboxes}
        \choice Mucho ojo cuate, la respuesta correcta es:
        \begin{align*}
            e_1 &= (p \rightarrow (q \rightarrow ((p \lor q) \rightarrow 
            (((q \lor (\neg p)) \lor q) \rightarrow r)))) \\ 
            e_2 &= ((p \rightarrow q) \leftrightarrow ((\neg q) \rightarrow (((\neg p) 
            \lor ((p \land q) \land r)) \rightarrow p))) \\ 
            e_3 &= (p \rightarrow ((q \land r) \rightarrow (s \rightarrow (p 
            \rightarrow ((s \lor p) \rightarrow ((q \land q) \rightarrow p)))))) 
        \end{align*} % Correcta

        \choice Mucho ojo cuate, la respuesta correcta es:
        \begin{align*}
            e_1 &= ((((p \rightarrow q) \rightarrow (p \lor q)) \rightarrow 
            ((q \lor (\neg p)) \lor q)) \rightarrow r) \\ 
            e_2 &= ((p \rightarrow q) \leftrightarrow (((\neg q) \rightarrow 
            ((((\neg p) \lor p) \land q) \land r)) \rightarrow p)) \\
            e_3 &= ((((((p \rightarrow (q \land r)) \rightarrow s) \rightarrow p) 
            \rightarrow (s \lor p)) \rightarrow (q \land q)) \rightarrow p)
        \end{align*}

        \choice Mucho ojo cuate, la respuesta correcta es:
        \begin{align*}
            e_1 &= ((((p \rightarrow q) \rightarrow (p \lor q)) \rightarrow ((q \lor \neg p) 
            \lor q)) \rightarrow r) \\ 
            e_2 &= ((p \rightarrow q) \leftrightarrow (\neg q \rightarrow ((\neg p 
            \lor ((p \land q) \land r)) \rightarrow p)))\\ 
            e_3 &= (p \rightarrow ((q \land r) \rightarrow (s \rightarrow (p 
            \rightarrow ((s \lor p) \rightarrow ((q \land q) \rightarrow p)))))) 
        \end{align*}

        \choice Mucho ojo cuate, la respuesta correcta es:
        \begin{align*}
            e_1 &= ((((p \rightarrow q) \rightarrow (p \lor q)) \rightarrow ((q \lor \neg p) 
            \lor q)) \rightarrow r) \\ 
            e_2 &= (p \rightarrow (q \leftrightarrow ((\neg q) \rightarrow (((((\neg p) 
            \lor p) \land q) \land r) \rightarrow p)))) \\ 
            e_3 &= ((((((p \rightarrow (q \land r)) \rightarrow s) \rightarrow p) 
            \rightarrow (s \lor p)) \rightarrow (q \land q)) \rightarrow p)
        \end{align*}
        
        \choice Ninguna de las anteriores. 
    \end{checkboxes}
    %%%%%%%%%%%%%%%%%%%%%%%%%%%%%%%%%%%%%%%%%%%%%%%%%%%%%%%%%%%%%%%%%%%%%%%%%%%
    \question
        {
            ¿Cuál es el árbol de sintaxis correcto de la siguiente expresión?
            \begin{equation*}
                \neg p \land q \rightarrow p \land (q \lor \neg r)
            \end{equation*}
        }
        \begin{multicols}{2}
            \begin{checkboxes}
                \forestset{
                    default preamble={
                    for tree={circle,draw}
                    }
                }
                \choice 
                \begin{forest}
                    for tree={dotted}
                    [$\rightarrow$
                        [$\land$ 
                            [$\neg$ [$p$]] 
                            [$q$]
                        ]
                        [$\land$ 
                            [$p$] 
                            [$\lor$ [$q$] [$\neg$ [$r$]]]
                        ]
                    ]
                \end{forest} % Correcta

                \choice 
                \begin{forest}
                    for tree={dotted}
                    [$\land$
                        [$\neg$ [$p$]]
                        [$\rightarrow$ 
                            [$q$]
                            [$\land$ [$p$] [$\lor$ [$q$] [$\neg$ [$r$]]]]]
                    ]
                \end{forest}
    
                \choice 
                \begin{forest}
                    for tree={dotted}
                    [$\land$
                        [$\rightarrow$ 
                            [$\land$ [$\neg$ [$p$]] [$q$]]
                            [$p$]
                        ]
                        [$\lor$ [$q$] [$\neg$ [$r$]]] 
                    ]
                \end{forest}

                \choice 
                \begin{forest}
                    for tree={dotted}
                    [$\land$
                        [$\land$ 
                            [$\neg$ [$p$]]
                            [$\rightarrow$ [$q$] [$p$]]
                        ]
                        [$\lor$ [$q$] [$\neg$ [$r$]]] 
                    ]
                \end{forest}

                \vspace*{1.5cm}
                
                \choice Ninguna de las anteriores. 
            \end{checkboxes}
        \end{multicols}
    %%%%%%%%%%%%%%%%%%%%%%%%%%%%%%%%%%%%%%%%%%%%%%%%%%%%%%%%%%%%%%%%%%%%%%%%%%%
            \question
        {
            ¿Cuál es el resultado de colocar correctamente todos los paréntesis 
            en las siguientes expresiones de acuerdo a su precedencia y 
            asociatividad de operadores?
            \begin{align*}
                &p \rightarrow q \land q \rightarrow p \lor \neg p \land q\\ 
                &p \land q \rightarrow q \lor q \rightarrow r \land q
            \end{align*}
        }
        \begin{checkboxes}
            \choice Mucho ojo cuate, la respuesta correcta es:
            \begin{align*}
                &(p \rightarrow ((q \land q) \rightarrow ((p \lor (\neg p)) \land q)))\\ 
                &(((p \land q) \rightarrow (q \lor q)) \rightarrow (r \land q))
            \end{align*}

            \choice Mucho ojo cuate, la respuesta correcta es:
            \begin{align*}
                &((p \rightarrow (q \land q)) \rightarrow ((p \lor (\neg p)) \land q))\\ 
                &((p \land q) \rightarrow ((q \lor q) \rightarrow (r \land q)))
            \end{align*}

            \choice Mucho ojo cuate, la respuesta correcta es:
            \begin{align*}
                &((p \rightarrow (q \land q)) \rightarrow (p \lor ((\neg p) \land q)))\\ 
                &(((p \land q) \rightarrow (q \lor q)) \rightarrow (r \land q))
            \end{align*}

            \choice Mucho ojo cuate, la respuesta correcta es:
            \begin{align*}
                &(p \rightarrow ((q \land q) \rightarrow (p \lor ((\neg p) \land q))))\\ 
                &((p \land q) \rightarrow ((q \lor q) \rightarrow (r \land q)))
            \end{align*} % Correcta
            
            \choice Ninguna de las anteriores. 
        \end{checkboxes}
    %%%%%%%%%%%%%%%%%%%%%%%%%%%%%%%%%%%%%%%%%%%%%%%%%%%%%%%%%%%%%%%%%%%%%%%%%%%
\end{comment}
