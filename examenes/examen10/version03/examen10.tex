\documentclass[oneside]{style}

\title{Versión 03}
\principal{Examen 10}
\author{Tania Michelle Rubí Rojas}
\semester{Semestre 2023-1}

\begin{document}
\maketitle

\vspace{5mm}
\noindent
Nombre y número de cuenta: \hrulefill\

\vspace*{5mm}
Para cada uno de los siguientes ejercicios, \textbf{justifica ampliamente} tu 
respuesta:

\begin{questions}[label=\protect\circled{\bfseries\arabic*}]

    % Ejercicio 01
    \question
    {
        Utiliza \textbf{Funciones de Interpretación} para determinar la 
        correctud del siguiente argumento:
        \begin{equation*}
            \{p \lor q, q \rightarrow r, p \land s \rightarrow t, \neg r, 
            \neg q \rightarrow u \land s\} \models t
        \end{equation*} 

    }  
    
    % Ejercicio 02
    \question
    {
        \textbf{Demuestra} la siguiente equivalencia lógica usando la 
        \textbf{regla de Leibniz}.
        \begin{equation*}
            p \rightarrow q \rightarrow p \equiv \neg p \rightarrow 
            p \rightarrow q
        \end{equation*}

        \textbf{Nota:} Debes mostrar claramente quiénes son $E[z \; := \; X]$, 
        $E[z \; := \; Y]$, además de decir quiénes son $X$ y $Y$. 
    }

    % Ejercicio 03
    \question
    {
        \begin{verbatim}
            Para que estemos en una película romántica es suficiente que el amor esté 
            basado en pensamiento mágico. El amor es sano sólo si el amor se construye 
            o el amor esta basado en pensamiento mágico. Afirmamos que el amor es sano.
            Luego, el hecho de que el amor no se construya es una condición necesaria
            para que no estemos en una película romántica. Para que el amor no sea sano 
            o el amor se encuentre es necesario que el amor se construya o el amor esté 
            basado en pensamiento mágico o el amor se encuentre. Por lo tanto, estamos en 
            una película romántica y el amor se encuentra.
        \end{verbatim}

        Para el texto anterior, \textbf{realiza} lo siguiente:
        \begin{itemize}
            \item \textbf{Traduce} el argumento al lenguaje de la Lógica 
            Proposicional. \textbf{Indica} claramente cuáles son las premisas 
            y cuál es la conclusión. 
            
            \item Utiliza \textbf{deducción natural} y \textbf{tu 
            traducción del inciso anterior} para indicar si el argumento es 
            correcto o no. 
        \end{itemize}
    }   

    % Ejercicio 04
    \question
    {
        \textit{Proposición:} Para cualesquiera dos conjuntos $A$ y $B$ se tiene 
        que 
        \begin{equation*}
            A - B = A \Rightarrow A \cap B = \varnothing
        \end{equation*}

        \textsc{Demostración}: Sean $A$ y $B$ conjuntos cualesquiera. Supongamos 
        que $A-B = A$. Queremos llegar a que $A \cap B = \varnothing$, es decir, 
        a que $A \cap B$ no tiene elementos. Llegaremos a esto haciendo un tipo 
        de demostración que se denomina \textit{por contradicción}. La idea es 
        suponer que $A \cap B$ no es vacío y llegar a un absurdo, es decir, a algo que simplemente no puede ser cierto. 
        Entonces, para llegar a una contradicción supongamos que existe 
        $x \in A \cap B$. Como $x \in A \cap B$, tenemos que $x \in A$ y 
        $x \in B$. Por hipótesis, $A - B = A$ y como $x \in A$, entonces 
        $x \in A - B$. Así, $x \in A$ y $x \notin B$, pero esto contradice que 
        $x \in A \cap B$. Por lo tanto, dicha $x$ no puede existir y $A \cap B$
        no tiene elementos. 

        Para el texto anterior, \textbf{realiza} lo siguiente:
        \begin{itemize}
            \item \textbf{Traduce} el argumento al lenguaje de la Lógica 
            Proposicional (puedes parafresear algunas frases en caso de que sea 
            necesario). \textbf{Indica} claramente cuáles son las premisas y
            cuál es la conclusión.
            
            \item Utiliza \textbf{Tableaux} y \textbf{tu traducción del inciso 
            anterior} para indicar si el argumento es correcto o no. 
        \end{itemize}
    
    }
\end{questions}
\end{document}
