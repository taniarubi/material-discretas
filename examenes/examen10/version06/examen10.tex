\documentclass[oneside]{style}

\title{Versión 06}
\principal{Examen 10}
\author{Tania Michelle Rubí Rojas}
\semester{Semestre 2023-1}

\begin{document}
\maketitle

\vspace{2.5mm}
\noindent
Nombre y número de cuenta: \hrulefill\

\vspace{5mm}
\noindent
\textbf{Indicaciones especiales:}
{\small
\begin{multicols}{2}
\begin{itemize}
  \item No se pueden utilizar resultados que resuelvan directamente los 
  ejercicios. 

  \item Para cada ejercicio, se debe indicar claramente cuál es la conclusión 
  obtenida de la demostración. 

  \item Para cada demostración, se debe indicar si se está utilizando 
  el método directo o indirecto.  

  \item Se deben justificar cada uno de los pasos que se realicen. 
  
  \item Para el ejercicio $3$, absolutamente todo debe demostrarse mediante la 
  regla de Leibniz. No se permiten pasos mágicos o demostraciones al lector. 

  \item La letra debe ser lo más clara posible. En caso de que sea ilegible, 
  la calificación automática será de cero. 
\end{itemize}
\end{multicols}
}

\begin{questions}[label=\protect\circled{\bfseries\arabic*}]

    % Ejercicio 01
    \question
    {
        Utiliza \textbf{funciones de interpretación} para determinar la 
        correctud del siguiente argumento:
        \begin{equation*}
            \{\neg p \rightarrow r \land \neg s, t \rightarrow s, 
            u \rightarrow \neg p, \neg w, u \lor w\} \models \neg t
        \end{equation*}
    }  

    % Ejercicio 02
    \question
    {
        Utiliza \textbf{tableaux} para determinar la correctud del siguiente 
        argumento:
        \begin{equation*}
            \{p \rightarrow \neg q, r \lor s \rightarrow t, 
            t \rightarrow q\} \models p \rightarrow \neg r \land \neg s
        \end{equation*}
    }

    % Ejercicio 03
    \question
    {
        \textbf{Demuestra} la siguiente equivalencia lógica usando la 
        \textbf{regla de Leibniz}.
        \begin{equation*}
            \neg (p \land q) \land (p \lor \neg q) \equiv \neg q
        \end{equation*}

        \textbf{Nota:} Debes mostrar claramente quiénes son $E[z \; := \; X]$, 
        $E[z \; := \; Y]$, además de decir quiénes son $X$ y $Y$. 
    }

    % Ejercicio 04
    \question
    {
        Para el siguiente texto:
        \begin{verbatim}
            Para que estemos en una película romántica es suficiente que el amor esté 
            basado en pensamiento mágico. El amor es sano sólo si el amor se construye 
            o el amor esta basado en pensamiento mágico. Afirmamos que el amor es sano.
            Luego, el hecho de que el amor no se construya es una condición necesaria
            para que no estemos en una película romántica. El hecho de que el amor no 
            sea sano o el amor se encuentre es una condición necesaria para que el amor 
            se construya o el amor esté basado en pensamiento mágico o el amor se 
            encuentre. Por lo tanto, estamos en una película romántica y el amor se 
            encuentra.
        \end{verbatim}

        \textbf{realiza} lo siguiente:
        \begin{itemize}
            \item \textbf{Traduce} el argumento al lenguaje de la Lógica 
            Proposicional. \textbf{Indica} claramente cuáles son las premisas 
            y cuál es la conclusión. 
            
            \item Utiliza \textbf{deducción natural} y \textbf{tu 
            traducción del inciso anterior} para indicar si el argumento es 
            correcto o no. 
        \end{itemize}
    }   
\end{questions}
\end{document}
