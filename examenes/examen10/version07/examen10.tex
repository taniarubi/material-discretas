\documentclass[oneside]{style}

\title{Versión 07}
\principal{Examen 10}
\author{Tania Michelle Rubí Rojas}
\semester{Semestre 2023-1}

\begin{document}
\maketitle

\vspace{2.5mm}
\noindent
Nombre y número de cuenta: \hrulefill\

\vspace{5mm}
\noindent
\textbf{Indicaciones especiales:}
{\small
\begin{multicols}{2}
\begin{itemize}
  \item No se pueden utilizar resultados que resuelvan directamente los 
  ejercicios. 

  \item Para cada ejercicio, se debe indicar claramente cuál es la conclusión 
  obtenida de la demostración. 

  \item Para cada demostración, se debe indicar si se está utilizando 
  el método directo o indirecto (contradicción).  

  \item Se deben justificar cada uno de los pasos que se realicen. 
  
  \item Para el ejercicio $2$, absolutamente todo debe demostrarse mediante la 
  regla de Leibniz. No se permiten pasos mágicos o demostraciones al lector. 

  \item Para el ejercicio $3$, únicamente se pueden utilizar las reglas que 
  se les sean dadas (entre ellas no están Modus Tollens o Silogismo 
  Disyuntivo). En caso de que quieran utilizar alguna regla que no se les haya 
  brindado, deberán demostrarla usando deducción natural para poder usarla. 
  
  \item Debe existir orden y limpieza en la resolución de cada uno de los 
  ejercicios. 

  \item La letra debe ser lo más clara posible. En caso de que sea ilegible, 
  la calificación automática será de cero. 
\end{itemize}
\end{multicols}
}

\begin{questions}[label=\protect\circled{\bfseries\arabic*}]

    % Ejercicio 01
    \question
    {
        Utiliza \textbf{tableaux} para determinar la correctud del siguiente 
        argumento:
        \begin{equation*}
            \{\neg p \lor \neg q \rightarrow r \land s, \neg s\} 
            \models q
        \end{equation*}
    } 

    % Ejercicio 02
    \question
    {
        \textbf{Demuestra} la siguiente equivalencia lógica usando la 
        \textbf{regla de Leibniz}.
        \begin{equation*}
            (\neg p \land (p \rightarrow q)) \rightarrow \neg q 
            \equiv p \lor \neg q
        \end{equation*}

        \textbf{Nota:} Debes mostrar claramente quiénes son $E[z \; := \; X]$, 
        $E[z \; := \; Y]$, además de decir quiénes son $X$ y $Y$. 
    }

    % Ejercicio 03
    \question
    {
        Para el siguiente texto:
        \begin{verbatim}
            Netflix canceló la serie 1899 o la canción más escuchada de Lady 
            Gaga en 2022 fue Bloody Mary. Netflix no canceló la serie 1899 o 
            mi ex novio me pidió permiso para seguirme en Instagram. Por lo 
            tanto, la canción más escuchada de Lady Gaga en 2022 fue Bloody
            Mary o mi ex novio me pidió permiso para seguirme en Instagram. 
        \end{verbatim}

        \textbf{realiza} lo siguiente:
        \begin{itemize}
            \item \textbf{Traduce} el argumento al lenguaje de la Lógica 
            Proposicional. \textbf{Indica} claramente cuáles son las premisas 
            y cuál es la conclusión. 
            
            \item Utiliza \textbf{deducción natural} y \textbf{tu 
            traducción del inciso anterior} para indicar si el argumento es 
            correcto o no. 
        \end{itemize}
    }   

    % Ejercicio 04
    \question
    {
        Utiliza \textbf{funciones de interpretación} para determinar la 
        correctud del siguiente argumento:
        \begin{equation*}
            \{p \lor (q \land r), \neg r \lor s, s \rightarrow \neg t\} \models 
            t \rightarrow p
        \end{equation*}
    } 
\end{questions}
\end{document}
