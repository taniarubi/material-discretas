\documentclass[oneside]{style}

\title{Versión 01}
\principal{Examen 10}
\author{Tania Michelle Rubí Rojas}
\semester{Semestre 2023-1}

\begin{document}
\maketitle

\vspace{5mm}
\noindent
Nombre y número de cuenta: \hrulefill\

\vspace*{5mm}
Para cada uno de los siguientes ejercicios, \textbf{justifica ampliamente} tu 
respuesta:

\begin{questions}[label=\protect\circled{\bfseries\arabic*}]

    % Ejercicio 01
    \question
    {
        \textbf{Analiza} el siguiente argumento:
        \begin{verbatim}
            Si Akise no es capaz de asesinar a Yukkiteru o Yuno sí es capaz 
            de hacerlo, entonces el juego de supervivencia se está llevando a 
            cabo en el segundo mundo. Yuno fue la ganadora del juego de 
            supervivencia del primer mundo o ella no es capaz de asesinar a 
            Yukkiteru. Akise no es un simple humano. Así, Akise es capaz de 
            asesinar a Yukkiteru sólo si Akise es un simple ser humano. Luego, 
            Yuno no fue la ganadora del juego de supervivencia del primer mundo 
            si Akise no es capaz de asesinar a Yukkiteru y el juego de 
            supervivencia se está llevando a cabo en el segundo mundo. Entonces 
            el hecho de que Deus Ex Machina salvara a Minene de la explosión de 
            una bomba es una condición necesaria para que Minene usara su 
            muerte para intentar abrir la bóveda donde se escondía el undécimo 
            usuario de diario. Así, Muru muru amañó el juego de supervivencia 
            clandestinamente o la sexta usuaria de diario es la líder del culto 
            Omekata. Como los padres del quinto usuario de diario están muertos, 
            esto implica que la sexta usuaria de diario es la líder del culto 
            Omekata. Entonces, Deus Ex Machina no salvó a Minene de la explosión
            de una bomba o la sexta usuaria de diario es la líder del culto 
            Omekata. Afirmamos que la sexta usuaria de diario es la líder del 
            culto Omekata. Así, los hechos de que Minene no usara su muerte para 
            intentar abrir la bóveda donde se escondía el undécimo usuario de 
            diario y que Muru muru amañara clandestinamente el juego de 
            supervivencia son condiciones suficientes para que Deus Ex Machina 
            le concediera a Minene la mitad de su poder. Luego, Akise muere a 
            manos de Yuno o los padres del quint usuario del diario están 
            muertos. Por lo tanto, Yuno no es capaz de asesinar a Yukkiteru y 
            Deus Ex Machina le concedió a Minene la mitad de su poder, además 
            de que Akise muere a manos de Yuno.  
        \end{verbatim}

        Para el texto anterior, \textbf{realiza} lo siguiente:
        \begin{itemize}
            \item \textbf{Traduce} el argumento al lenguaje de la Lógica 
            Proposicional. \textbf{Indica} claramente cuáles son las premisas y
            cuál es la conclusión. 
            
            \item Utiliza \textbf{Tableaux} y \textbf{tu traducción del inciso 
            anterior} para indicar si el argumento es correcto o no. 
        \end{itemize}

    }    
    
    % Ejercicio 02
    \question
    {
        \textbf{Demuestra} la siguiente equivalencia lógica usando la regla de 
        Leibniz.
        \begin{equation*}
            \neg p \lor s \rightarrow q \land r \equiv s \lor \neg p 
            \rightarrow \neg (q \rightarrow \neg r)
        \end{equation*}

        \textbf{Nota}: Para este ejercicio, sólo puedes usar las equivalencias 
        \begin{equation*}
            p\lor q \equiv q \lor p \quad \quad \quad \quad 
            p \rightarrow q \equiv \neg p \lor q \quad \quad \quad \quad 
            \neg (p \land q) \equiv \neg p \lor \neg q \quad \quad \quad \quad 
            \neg (p \lor q) \equiv \neg p \land \neg q \quad \quad \quad \quad
            \neg \neg p \equiv p 
        \end{equation*}

        \textbf{Hint:} Recuerda que es más fácil desaparecer cosas que hacerlas aparecer. 
    }

    \newpage
    % Ejercicio 03
    \question
    {

        Para cualesquiera dos conjuntos $A$ y $B$ se cumple que $\mathcal{P}(A 
        \cap B) = \mathcal{P}(A) \cap \mathcal{P}(B)$.

        \textit{Demostración:} Tenemos que 
        \begin{align*}
            \mathcal{P}(A \cap B) &\Leftrightarrow X \subseteq A \cap B \\ 
            &\Leftrightarrow X \subseteq A \text{ y } X \subseteq B \\ 
            &\Leftrightarrow X \in \mathcal{P}(A) \text{ y } X \in \mathcal{P}(B) \\
            &\Leftrightarrow X \in \mathcal{P}(A) \cap \mathcal{P}(B)  
        \end{align*}

        Concluimos entonces que $\mathcal{P}(A \cap B) = \mathcal{P}(A) \cap 
        \mathcal{P}(B)$. 

        \vspace{5mm}

        Para el texto anterior, \textbf{realiza} lo siguiente:
        \begin{itemize}
            \item \textbf{Traduce} el argumento principal al lenguaje de la Lógica 
            Proposicional. \textbf{Indica} claramente cuáles son las premisas y
            cuál es la conclusión. 
            
            \item Utiliza \textbf{funciones de interpretación} y \textbf{tu 
            traducción del inciso anterior} para indicar si el argumento es 
            correcto o no. 
        \end{itemize}
    
    }

    % Ejercicio 04
    \question
    {
        \textbf{Analiza} el siguiente argumento:
        \begin{verbatim}
            El hecho de que David le escribiera una canción a la chica 
            misteriosa es una condición necesaria para que ella se enamorara 
            de él y ella no eligiera salir con otro muchacho. Si la 
            chica misteriosa tiene un amor platónico o David quiere una 
            relación seria, entonces la chica misteriosa se enamoró de 
            David. El hecho de que la chica misteriosa eligiera salir con 
            otro muchacho es una condición suficiente para que David se 
            pusiera triste. Además, la chica misteriosa tiene un amor platónico
            sólo si David no está triste. Pero la chica misteriosa tiene 
            un amor platónico. Por lo tanto, David le escribió una canción 
            a la chica misteriosa. 
        \end{verbatim}

        Para el texto anterior, \textbf{realiza} lo siguiente:
        \begin{itemize}
            \item \textbf{Traduce} el argumento al lenguaje de la Lógica 
            Proposicional. \textbf{Indica} claramente cuáles son las premisas y
            cuál es la conclusión. 
            
            \item Utiliza \textbf{deducción natural} y \textbf{tu 
            traducción del inciso anterior} para indicar si el argumento es 
            correcto o no. 
        \end{itemize}

    }

    % Ejercicio 05
    \question
    {
        \textbf{Demuestra} la correctud del argumento utilizando los hints
        brindados en este ejercicio:
        \begin{equation*}
            \models ((((\neg b \lor \neg c) \rightarrow (f \land m) \land 
            (f \rightarrow d) \land \neg d) \rightarrow b) \land \neg b) 
            \rightarrow \neg (((\neg b \lor \neg c) \rightarrow (f \land m) \land 
            (f \rightarrow d) \land \neg d) \rightarrow b)
        \end{equation*}

        \textbf{Hints:}
        \begin{itemize}
            \item Debes asumir que $\{p \rightarrow q, \neg q\} 
            \models \neg p$ es correcto. 

            \item Puedes utilizar la propiedad $\Gamma \models p \rightarrow q
            \Leftrightarrow \Gamma \cup \{p\} \models q$
        \end{itemize}
    }

    % Ejercicio 06
    \question
    {
        \textbf{Ejercicio Opcional (Hasta 2.5 puntos extras).} 
        
        \textbf{Analiza} el siguiente argumento:
        \begin{verbatim}
            Cuchis es el mejor detective del condado y fue llamado para resolver 
            un misterio muy misterioso relacionado con un asesinato. Después de 
            un par de entrevistas con los sospechosos, Cuchis determinó los 
            siguientes hechos:
                1. Frijolito fue asesinado por un golpe en la cabeza con una 
                guitarra. 
                2. Juan Carlos Bodoque o Nubecita estaban en el comedor principal 
                en el momento del asesinato. 
                3. Si Hanna estaba en el jardín en el momento del asesinato, 
                entonces Chucho, el dragoncito, asesinó a Frijolito con una 
                dosis letal de amor. 
                4. Si Juan Carlos Botoque estaba en el comedor principal en el 
                momento del asesinato, entonces Kike asesinó a Frijolito. 
                5. Si Hanna no estaba en el jardín en el momento del asesinato, 
                entonces Nubecita no estaba en el comedor principal cuando 
                se cometió el homicidio. 
                6. Si Nubecita estaba en el comedor principal en el momento 
                del asesinato, entonces Jessy, La Diabla, asesinó a Frijolito. 
        \end{verbatim}

        Para el texto anterior, \textbf{realiza} lo siguiente:
        \begin{itemize} 
            \item \textbf{Traduce} el argumento al lenguaje de la Lógica 
            Proposicional. \textbf{Indica} claramente cuáles son las premisas y
            cuál es la conclusión. 

            \item ¿Es posible que Cuchis pueda deducir quién es el asesino de 
            Frijolito? En caso afirmativo, \textbf{escribe} el nombre del 
            asesino y utiliza cualquier método de demostración visto en clase y 
            \textbf{tu traducción del inciso anterior} para determinar si tu 
            conclusión es correcta o no. En caso contrario, explica por qué no 
            es posible determinar quién es la persona que nos privó de la 
            existencia de Frijolito :(
        \end{itemize}

        \textbf{Nota}: Puedes suponer que sólo hay una causa de muerte. 
    }
\end{questions}
\end{document}
