\documentclass[oneside]{style}

\title{Versión 04}
\principal{Examen 10}
\author{Tania Michelle Rubí Rojas}
\semester{Semestre 2023-1}

\begin{document}
\maketitle

\vspace{5mm}
\noindent
Nombre y número de cuenta: \hrulefill\

\vspace*{5mm}
Para cada uno de los siguientes ejercicios, \textbf{justifica ampliamente} tu 
respuesta:

\begin{questions}[label=\protect\circled{\bfseries\arabic*}]

    % Ejercicio 01
    \question
    {
        \textbf{Analiza} el siguiente argumento:
        \begin{verbatim}
            Si A ganó la carrera, entonces B fue el segundo lugar o C fue el 
            segundo lugar. Si B fue el segundo lugar, entonces A no ganó la 
            carrera. Si D fue el segundo, entonces C no fue el segundo lugar. 
            Afirmamos que A ganó la carrera. Por lo tanto, D no fue el 
            segundo lugar. 
        \end{verbatim}

        Para el texto anterior, \textbf{realiza} lo siguiente:
        \begin{itemize}
            \item \textbf{Traduce} el argumento al lenguaje de la Lógica 
            Proposicional. \textbf{Indica} claramente cuáles son las premisas y
            cuál es la conclusión. 
            
            \item Utiliza \textbf{Funciones de Interpretación} y \textbf{tu 
            traducción del inciso anterior} para indicar si el argumento es 
            correcto o no. 
        \end{itemize}
    }  

    % Ejercicio 02
    \question
    {
        \textit{Proposición:} Para cualesquiera dos conjuntos $A$ y $B$ se tiene 
        que 
        \begin{equation*}
            A - B = A \Leftarrow A \cap B = \varnothing
        \end{equation*}

        \textsc{Demostración}: Sean $A$ y $B$ conjuntos cualesquiera. 
        Supongamos que $A \cap B = \varnothing$ y veamos que entonces 
        $A - B = A$. Mostraremos que se cumplen ambas contenciones. Claramente, 
        $A - B \subseteq A$, pues siempre que $x \in A$ y $x \in B$, se tiene 
        particularmente que $x \in A$. Para la contención recíproca, supongamos 
        que $x \in A$. Por hipótesis, $A \cap B$ no tiene elementos y, como $x \in A$, entonces $x$ no 
        puede pertenecer a $B$. Así, $x \in A$ y $x \notin B$; por lo que 
        $x \in A - B$. Por lo tanto, $A - B = A$. Concluimos entonces que 
        $A - B = A \Leftarrow A \cap B = \varnothing$.

        \vspace*{5mm}

        Para el texto anterior, \textbf{realiza} lo siguiente:
        \begin{itemize}
            \item \textbf{Traduce} el argumento al lenguaje de la Lógica 
            Proposicional (puedes parafresear algunas frases en caso de que sea 
            necesario). \textbf{Indica} claramente cuáles son las premisas y
            cuál es la conclusión.
            
            \item Utiliza \textbf{Tableaux} y \textbf{tu traducción del inciso 
            anterior} para indicar si el argumento es correcto o no. 
        \end{itemize}
    }

    % Ejercicio 03
    \question
    {
        Para el siguiente texto:
        \begin{verbatim}
            Para que estemos en una película romántica es suficiente que el amor esté 
            basado en pensamiento mágico. El amor es sano sólo si el amor se construye 
            o el amor esta basado en pensamiento mágico. Afirmamos que el amor es sano.
            Luego, el hecho de que el amor no se construya es una condición necesaria
            para que no estemos en una película romántica. El hecho de que el amor no 
            sea sano o el amor se encuentre es una condición necesaria para que el amor 
            se construya o el amor esté basado en pensamiento mágico o el amor se 
            encuentre. Por lo tanto, estamos en una película romántica y el amor se 
            encuentra.
        \end{verbatim}

        \textbf{realiza} lo siguiente:
        \begin{itemize}
            \item \textbf{Traduce} el argumento al lenguaje de la Lógica 
            Proposicional. \textbf{Indica} claramente cuáles son las premisas 
            y cuál es la conclusión. 
            
            \item Utiliza \textbf{deducción natural} y \textbf{tu 
            traducción del inciso anterior} para indicar si el argumento es 
            correcto o no. 
        \end{itemize}
    }   
    
    \newpage
    % Ejercicio 04
    \question
    {
        \textbf{Demuestra} la siguiente equivalencia lógica usando la 
        \textbf{regla de Leibniz}.
        \begin{equation*}
            (p \lor q) \land (\neg p \land (\neg p \land q)) \equiv 
            (\neg p \land q)
        \end{equation*}

        \textbf{Nota:} Debes mostrar claramente quiénes son $E[z \; := \; X]$, 
        $E[z \; := \; Y]$, además de decir quiénes son $X$ y $Y$. 
    }
\end{questions}
\end{document}
