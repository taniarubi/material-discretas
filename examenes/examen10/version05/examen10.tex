\documentclass[oneside]{style}

\title{Versión 05}
\principal{Examen 10}
\author{Tania Michelle Rubí Rojas}
\semester{Semestre 2023-1}

\begin{document}
\maketitle

\vspace{5mm}
\noindent
Nombre y número de cuenta: \hrulefill\

\vspace*{5mm}
Para cada uno de los siguientes ejercicios, \textbf{justifica ampliamente} tu 
respuesta:

\begin{questions}[label=\protect\circled{\bfseries\arabic*}]

    % Ejercicio 01
    \question
    {
        \textit{Proposición:} Para cualesquiera dos conjuntos $A$ y $B$ se tiene 
        que 
        \begin{equation*}
            A - B = A \Rightarrow A \cap B = \varnothing
        \end{equation*}

        \textsc{Demostración}: Sean $A$ y $B$ conjuntos cualesquiera. Supongamos 
        que $A-B = A$. Queremos llegar a que $A \cap B = \varnothing$, es decir, 
        a que $A \cap B$ no tiene elementos. Llegaremos a esto haciendo un tipo 
        de demostración que se denomina \textit{por contradicción}. La idea es 
        suponer que $A \cap B$ no es vacío y llegar a un absurdo, es decir, a algo 
        que simplemente no puede ser cierto. Entonces, para llegar a una 
        contradicción supongamos que existe $x \in A \cap B$. Como $x \in A \cap B$, 
        tenemos que $x \in A$ y $x \in B$. Por hipótesis, $A - B = A$ y como 
        $x \in A$, entonces $x \in A - B$. Así, $x \in A$ y $x \notin B$, pero esto 
        contradice que $x \in A \cap B$. Por lo tanto, dicha $x$ no puede existir y 
        $A \cap B$ no tiene elementos. 

        \vspace*{5mm}

        Para el texto anterior, \textbf{realiza} lo siguiente:
        \begin{itemize}
            \item \textbf{Traduce} el argumento al lenguaje de la Lógica 
            Proposicional. \textbf{Indica} claramente cuáles son las premisas y
            cuál es la conclusión.
            
            \item Utiliza \textbf{Tableaux} y \textbf{tu traducción del inciso 
            anterior} para indicar si el argumento es correcto o no. 
        \end{itemize}
    }

    % Ejercicio 02
    \question
    {
        \textbf{Demuestra} la siguiente equivalencia lógica usando la 
        \textbf{regla de Leibniz}.
        \begin{equation*}
            p \land (p \lor q) \equiv p
        \end{equation*}

        \textbf{Nota:} Debes mostrar claramente quiénes son $E[z \; := \; X]$, 
        $E[z \; := \; Y]$, además de decir quiénes son $X$ y $Y$. 
    }

    % Ejercicio 03
    \question
    {
        Para el siguiente texto:
        \begin{verbatim}
            Maura no logró despertar sólo si las personas en el barco Kerberos
            cometieron el mismo error que las del barco Prometheus. Elliot es 
            el responsable de los asesinatos en el barco Kerberos o Daniel 
            invirtió los códigos de la simulación o Maura logró recordar.  
            Además, el hecho de que las personas en el barco Kerberos no 
            cometieran el mismo error que las del barco Prometheus es una 
            condición necesaria para que Daniel haya invertido los códigos de 
            la simulación. Luego, el hecho de que las personas en el barco 
            Kerberos cometieran el mismo error que las del barco Prometheus es 
            una condición suficiente para que Maura no lograra recordar. Por 
            lo tanto, Maura logró despertar o Elliot es el responsable de los 
            asesinatos en el barco Kerberos. 
        \end{verbatim}

        \textbf{realiza} lo siguiente:
        \begin{itemize}
            \item \textbf{Traduce} el argumento al lenguaje de la Lógica 
            Proposicional. \textbf{Indica} claramente cuáles son las premisas 
            y cuál es la conclusión. 
            
            \item Utiliza \textbf{deducción natural} y \textbf{tu 
            traducción del inciso anterior} para indicar si el argumento es 
            correcto o no. 
        \end{itemize}
    }   

    \newpage
    % Ejercicio 04
    \question
    {
        \textbf{Analiza} el siguiente argumento:
        \begin{verbatim}
            Si Sebastián me extraña, entonces él fue mi novio hace siete 
            meses o él vive lejos de mí. Si Sebastián fue mi novio hace siete
            meses, me extraña o me ama. Afirmamos que Sebastián no me ama. 
            Por otro lado, Sebastián me extraña o no fue mi novio hace siete 
            meses. Por consiguiente, Sebastián me extraña o él vive lejos de 
            mí. 
        \end{verbatim}

        Para el texto anterior, \textbf{realiza} lo siguiente:
        \begin{itemize}
            \item \textbf{Traduce} el argumento al lenguaje de la Lógica 
            Proposicional. \textbf{Indica} claramente cuáles son las premisas y
            cuál es la conclusión. 
            
            \item Utiliza \textbf{Funciones de Interpretación} y \textbf{tu 
            traducción del inciso anterior} para indicar si el argumento es 
            correcto o no. 
        \end{itemize}
    }  
\end{questions}
\end{document}
