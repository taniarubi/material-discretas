\documentclass[12pt, a4paper]{exam}

% Soporte para cambiar la fecha que sale en el examen
\usepackage{advdate}
% Soporte para escribir en varias columnas
\usepackage{multicol}
% Soporte para los acentos.
\usepackage[utf8]{inputenc} 
\usepackage[T1]{fontenc}    
% Idioma español.
\usepackage[spanish,mexico,es-tabla]{babel}
\usepackage{graphicx}
\usepackage{tikz}
\usepackage{amsmath,amssymb,amsthm}
\usepackage{verbatim} % comentarios
\usepackage{tasks}
% Cambiamos los márgenes del documento. 
\usepackage[top=1.5cm,left=1.5cm,right=1.5cm]{geometry}

% Pie de página
\cfoot{Página \thepage\ de \numpages}

%%%%%%%%%%%%%%%%%%%%%%%%%%%%%%%%%%%%%%%%%%%%%%%%%%%%%%%%%%%%%%%%%%%%%%%%%%%%%%
\renewcommand{\thechoice}{\alph{choice}}

\makeatletter
\renewenvironment{checkboxes}%
   {\setcounter{choice}{0}\list{\checkbox@char}%
      {%
        \settowidth{\leftmargin}{W.\hskip\labelsep\hskip 2.5em}%
        \def\choice{%
          \if@correctchoice
            \color@endgroup \endgroup
          \fi
          \stepcounter{choice}
          \item[\checked@char]
          \do@choice@pageinfo
        } % choice
        \def\CorrectChoice{%
          \if@correctchoice
            \color@endgroup \endgroup
          \fi
          \ifprintanswers
            % We can't say \choice here, because that would
            % insert an \endgroup.
            % 2016/05/10: We say \color@begingroup in addition to
            % \begingroup in case \CorrectChoiceEmphasis involves color
            % and the text exactly fills the line (which would
            % otherwise create a blank line after this choice):
            % 2016/05/11: We leave hmode if we're in it,
            % i.e., if there's no blank line preceding this
            % \CorrectChoice command.  (Without this, the
            % \special created by a \color{whatever} command that might
            % be inserted by \CorrectChoice@Emphasis would be appended 
            % to the previous \choice, which could cause an extra
            % (blank) line to be inserted before this \CorrectChoice.)
            % Since \par and \endgraf seem to cancel \@totalleftmargin
            % (for reasons I don't understand), we'll do the following:
            % Motivated by  the def of \leavevmode, 
            %      \def\leavevmode{\unhbox\voidb@x}
            % we will now leave hmode (if we're in hmode):
            \ifhmode \unskip\unskip\unvbox\voidb@x \fi
            \begingroup \color@begingroup \@correctchoicetrue
            \CorrectChoice@Emphasis
            \stepcounter{choice}
            \item[\checked@char]
          \else
            \stepcounter{choice}
            \item[\checked@char]
          \fi
          \do@choice@pageinfo
        } % CorrectChoice
        \let\correctchoice\CorrectChoice
        \labelwidth\leftmargin\advance\labelwidth-\labelsep
        \topsep=0pt
        \partopsep=0pt
        \checkboxeshook
      }%
   }%
   {\if@correctchoice \color@endgroup \endgroup \fi \endlist}
 \makeatother

% Make checkbox character a circle with the letter
\checkboxchar{\tikz[baseline={([yshift=-.8ex]current bounding box.center)}]\node[shape=circle,minimum size=4mm,draw] at (0,0) {\thechoice};}
% Make checked box character bold WITH surd
%\checkedchar{\tikz[baseline={([yshift=-.8ex]current bounding box.center)}]\node[shape=circle,minimum size=8mm,draw] at (0,0) {} node at (0,0) {\thechoice\llap{$\surd$}};}
% Make checked box character bold
\checkedchar{\tikz[baseline={([yshift=-.8ex]current bounding box.center)}]\node[shape=circle,minimum size=4mm,draw] at (0,0) {} node at (0,0) {\thechoice};}
\printanswers
%%%%%%%%%%%%%%%%%%%%%%%%%%%%%%%%%%%%%%%%%%%%%%%%%%%%%%%%%%%%%%%%%%%%%%%%%%%%%%

\begin{document}
    %%%%%%%%%%%%%%%%%%%%%%%%%%%%%%%%%%%%%%%%%%%%%%%%%%%%%%%%%%%%%%%%%%%%%%%%%%%%%%%
    %%%%%%%%%%%%%%%%%%%%%%%%%%%%%%%% ENCABEZADO %%%%%%%%%%%%%%%%%%%%%%%%%%%%%%%%%%%
    \centering
    \hrule \hrule \hrule 
    \vspace{5mm}
    \begin{minipage}[c]{0.8\textwidth}
        \begin{center}
            {\large\textbf{Mission 12, Start!} \par
            \large \textbf{Estructuras Discretas} \par
            \large \textbf{Semestre 2023-1} \par
            \large \textbf{\today}	\par}
        \end{center}
    \end{minipage}

    \vspace{0.2in}
    \noindent
    \textbf{Tania Michelle Rubí Rojas}
    \vspace{2mm}
    \hrule \hrule \hrule 
    %%%%%%%%%%%%%%%%%%%%%%%%%%%%%%%%%%%%%%%%%%%%%%%%%%%%%%%%%%%%%%%%%%%%%%%%%%%%%%%
    %%%%%%%%%%%%%%%%%%%%%%%%%%%%%%%%%%%%%%%%%%%%%%%%%%%%%%%%%%%%%%%%%%%%%%%%%%%%%%%

    \vspace{5mm}
    \noindent
    Nombre y número de cuenta: \hrulefill\

    \begin{questions}
        % Question 01
        \question
        {
            Sea el conjunto de todas las personas nuestro universo del discurso.
            Si tenemos los siguientes predicados:
            \begin{tasks}(2)
                \task $H(x,y): x$ es hermana de $y$
                \task $A(x,y): x$ es amigo de $y$
                \task $C(x,y): x$ conoce a $y$
            \end{tasks}

            ¿Cuál o cuáles son las representaciones correctas para la siguiente 
            oración?
            \begin{center}
                Mariana es hermana de Carlos, pero no conoce a todos los 
                amigos de él.
            \end{center}
        }
        \begin{checkboxes}
            \choice $H(\texttt{Mariana},\texttt{Carlos}) \land \exists x 
            (A(x,\texttt{Carlos}) \land C(\texttt{Mariana},x))$
            
            \choice $H(\texttt{Mariana},\texttt{Carlos}) \land \neg \forall x 
            (A(x,\texttt{Carlos}) \rightarrow C(\texttt{Mariana},x))$ 
            % Correcta 

            \choice $H(\texttt{Mariana}, \texttt{Carlos}) \rightarrow \neg 
            \forall x (A(x,\texttt{Carlos}) \rightarrow C(\texttt{Mariana},x))$

            \choice $\exists x (H(\texttt{Mariana},\texttt{Carlos}) \land 
            (A(\texttt{Carlos},x) \land C(\texttt{Mariana},x)))$
             
            \choice Ninguna de las anteriores. 
        \end{checkboxes}

        % Question 02
        \question
        {
            Sea el conjunto de todas las personas nuestro universo del discurso.
            Si tenemos el siguiente predicado: 
            \begin{equation*}
                T(x,y): x \text{ puede tomarle el pelo a } y 
            \end{equation*}

            De acuerdo a esto, ¿cuál o cuáles de las siguientes expresiones son 
            \textbf{verdaderas}?
        }
        \begin{checkboxes}
            \choice La traducción del enunciado 
            \begin{equation*}
                \forall x T(x, \texttt{Juan})
            \end{equation*}

            es \textit{¨Todo el mundo puede tomarle el pelo a Juan¨} % Correcta 

            \choice La traducción del enunciado 
            \begin{equation*}
                \forall x \exists y T(x,y)
            \end{equation*}

            es \textit{¨Cualquiera puede tomarle el pelo a alguien¨} % Correcta 

            \choice La traducción del enunciado 
            \begin{equation*}
                \forall x \exists y \exists z (T(x,y) \land T(x,z) \rightarrow y = z)
            \end{equation*}

            es \textit{¨Hay exactamente una persona a quien cualquiera puede 
            tomarle el pelo¨}. % Correcta

            \choice La traducción del enunciado 
            \begin{equation*}
                \exists x \exists y (T(x,y) \land x \neq y)
            \end{equation*}

            es \textit{¨Hay alguien que puede tomarle el pelo a exactamente una 
            persona distinta de sí mismo¨}.

            \choice Ninguna de las anteriores.
        \end{checkboxes}

        \newpage
        % Question 03
        \question{¿Cuál o cuáles de las siguientes expresiones son 
        \textbf{verdaderas}?}
        \begin{checkboxes}
            \choice $\neg (\forall x \exists y P(x,y)) \equiv 
            \exists x \forall y \neg P(y,x)$

            \choice $\forall x (P(x) \rightarrow A(x)) \equiv 
            \exists x P(x) \rightarrow A(x)$

            \choice $\neg \exists x (\neg P(x)) \land \forall y 
            (Q(y) \rightarrow R(y)) \equiv \forall x (P(x)) \lor 
            \exists y (Q(y) \land \neg R(y))$

            \choice $\exists x (P(x) \land A(x)) \equiv 
            \forall x P(x) \land A(x)$

            \choice Ninguna de las anteriores. % Correcta 
        \end{checkboxes}

        % Question 04
        \question
        {
            Sea el conjunto de todos los perritos y todos los carteros nuestro 
            universo del discurso. Si tenemos los siguientes predicados:
            \begin{tasks}(2)
                \task $P(x): x$ es un perro 
                \task $C(x): x$ es un cartero 
                \task $M(x,y): x$ es mordido por $y$
            \end{tasks}

            De acuerdo a esto, ¿cuál o cuáles de las siguientes expresiones son 
            \textbf{verdaderas}?
        }
        \begin{checkboxes}
            \choice La traducción del enunciado
            \begin{equation*}
                \forall x (P(x) \rightarrow \forall y (C(y) \land M(y,x)))
            \end{equation*}

            es \textit{¨Los perros muerden a los carteros¨}. 

            \choice La traducción del enunciado 
            \begin{equation*}
                \exists x (P(x) \land \exists y (C(y) \land \neg M(y,x)))
            \end{equation*}

            es \textit{¨Existe un perro que no muerde carteros¨}

            \choice La traducción del enunciado 
            \begin{equation*}
                \exists x (P(x) \land C(x) \land  M(x,x) )
            \end{equation*}

            es \textit{¨Hay un perro que es cartero y se muerde a sí mismo¨}. 
            % Correcta 

            \choice La traducción del enunciado 
            \begin{equation*}
                \exists x (C(x) \land \forall y (P(y) \rightarrow \neg M(x, y)))
            \end{equation*}

            es \textit{¨Hay un cartero que no es mordido por perros¨}. % Correcta 

            \choice Ninguna de las anteriores.
        \end{checkboxes}

        % Question 05
        \question{¿Cuál o cuáles de las siguientes expresiones son 
        \textbf{verdaderas}?}
        \begin{checkboxes}
            \choice $\forall x \forall y P(x,y) \equiv \forall y 
            \forall x P(x,y)$ % Correcta 

            \choice $\forall x P(x) \equiv \forall y (P(y))$ % Correcta

            \choice $\exists x Q(x) \equiv \neg \forall x \neg Q(x)$ % Correcta

            \choice $\exists x (P(x) \lor Q(x)) \equiv \exists x (P(x)) 
            \lor \exists xQ(x)$ % Correcta 

            \choice Ninguna de las anteriores.
        \end{checkboxes}

        \newpage
        % Question 06
        \question
        {
            Sea el conjunto de todos los países que tienen un equipo de futbol 
            nuestro universo del discurso. Si tenemos el siguiente predicado:
            \begin{equation*}
                G(x,y): x \text{ le gana a } y 
            \end{equation*}

            ¿Cuál o cuáles son las representaciones correctas para la siguiente 
            oración?
            \begin{center}
                Si Alemania gana contra Italia, entonces Alemania no pierde 
                todos sus partidos. 
            \end{center}
        }
        \begin{checkboxes}
            \choice $G(\texttt{Alemania, Italia}) \rightarrow \neg \forall x 
            G(\texttt{Alemania}, x)$

            \choice $G(\texttt{Alemania, Italia}) \rightarrow \exists x \neg
            G(\texttt{Alemania}, x)$

            \choice $G(\texttt{Alemania, Italia}) \rightarrow \forall x 
            G(\texttt{Alemania}, x)$

            \choice $G(\texttt{Alemania, Italia}) \rightarrow \neg \exists x 
            G(x, \texttt{Alemania})$ 

            \choice Ninguna de las anteriores. % Correcta
        \end{checkboxes}

        % Question 07
        \question
        {
            Sea el conjunto de todas las computadoras y todos los sistemas 
            operativos nuestro universo del discurso. Si tenemos los 
            predicados:
            \begin{tasks}(2)
                \task $C(x): x$ es una computadora
                \task $H(x,y): x$ es hackeada por $y$
                \task $F(x,y): x$ funciona con el sistema operativo $y$
            \end{tasks}

            De acuerdo a esto, ¿cuál o cuáles de las siguientes expresiones son 
            \textbf{verdaderas}?
        }
        \begin{checkboxes}
            \choice La traducción del enunciado
            \begin{equation*}
                \exists x \exists y (C(x) \land C(y) \land x \neq y \land H(x,y))
            \end{equation*}

            es \textit{¨Hay una computadora que ha sido hackeada desde otra 
            computadora diferente de sí misma¨} % Correcta 

            \choice La traducción del enunciado
            \begin{equation*}
                \forall x (C(x) \rightarrow F(x, \texttt{Linux}))
            \end{equation*}

            es \textit{¨Una computadora funciona con el sistema operativo Linux¨}

            \choice La traducción del enunciado
            \begin{equation*}
                C(x) \land F(x, \texttt{Linux}) \rightarrow \neg \forall y H(x,y)
            \end{equation*}

            es \textit{¨Si una computadora tiene el sistema operativo Linux, 
            entonces no puede ser hackeada¨}

            \choice La traducción del enunciado
            \begin{equation*}
                \forall x \forall y \forall z(C(x) \land C(z) \land x \neq z 
                \land F(x,y) \rightarrow H(x,z))
            \end{equation*}

            es \textit{¨Todas las computadoras que tienen cualquier sistema 
            operativo pueden ser hackeadas¨}

            \choice Ninguna de las anteriores.
        \end{checkboxes}

        % Question 08
        \question{¿Cuál o cuáles de las siguientes expresiones son 
        \textbf{verdaderas}?}
        \begin{checkboxes}
            \choice $\neg (\forall x (P(x) \rightarrow \exists y (H(y) \land 
            B(x,y)))) \equiv \exists x \forall y (P(x) \land H(y) \rightarrow 
            \neg B(x,y))$ % Correcta 

            \choice $\exists x P(x) \equiv \exists y P(y)$ % Correcta

            \choice $\neg (\exists x (P(x) \land Q(x))) \equiv 
            \forall x (P(x) \rightarrow \neg Q(x))$ % Correcta 

            \choice $\forall x P(x) \equiv \neg \exists x \neg P(x)$
            % Correcta

            \choice Ninguna de las anteriores.
        \end{checkboxes}

        % Question 09
        \question
        {
            Sea $\mathbb{N}$ nuestro universo del discurso. Si tenemos los 
            siguientes predicados:
            \begin{tasks}(2)
                \task $P(x): x$ es un número par 
                \task $E(x): x$ es múltiplo de $4$
            \end{tasks}

            ¿Cuál o cuáles son las representaciones correctas para la siguiente 
            oración?
            \begin{center}
                Existe un número que es múltiplo de $4$ y que no es un 
                número par 
            \end{center}
        }
        \begin{checkboxes}
            \choice $\neg \exists x (E(x) \land P(x))$

            \choice $\neg \forall x (P(x) \rightarrow E(x))$

            \choice $\exists x (E(x) \land \neg P(x))$ % Correcta 

            \choice $\neg \forall x (E(x) \rightarrow P(x))$ % Correcta
            
            \choice Ninguna de las anteriores. 
        \end{checkboxes}

        % Question 10
        \question
        {
            Sea el conjunto de los días de la semana nuestro universo del 
            discurso. Si tenemos los predicados:
            \begin{tasks}(2)
                \task $S(x):$ el día $x$ está soleado 
                \task $N(x):$ el día $x$ está nublado 
            \end{tasks}

            De acuerdo a esto, ¿cuál o cuáles de las siguientes expresiones son 
            \textbf{verdaderas}?
        }
        \begin{checkboxes}
            \choice La traducción del enunciado 
            \begin{equation*}
                \exists x N(x) \rightarrow \forall y S(y)
            \end{equation*}

            es \textit{¨Si algún día está nublado, entonces todos los días 
            estarán soleados¨}. % Correcta 

            \choice La traducción del enunciado 
            \begin{equation*}
                \exists x S(\texttt{Lunes}) \rightarrow \forall y S(y)
            \end{equation*}

            es \textit{¨El lunes estuvo soleado, por lo que todos los días 
            estarán soleados¨}. % Correcta 

            \choice La traducción del enunciado 
            \begin{equation*}
                \forall x (N(x) \rightarrow S(x))
            \end{equation*}

            es \textit{¨Siempre estará soleado sólo si está nublado¨}. 

            \choice La traducción del enunciado 
            \begin{equation*}
                \exists x \exists y (S(x) \land N(y) \land x \neq y)
            \end{equation*}

            es \textit{¨Ningún día es al mismo tiempo soleado y nublado¨}. 

            \choice Ninguna de las anteriores.
        \end{checkboxes}
        
        \newpage
        % Question 11
        \question
        {
            Sea el conjunto de todos los países que tienen un equipo de futbol 
            nuestro universo del discurso. Si tenemos el siguiente predicado:
            \begin{equation*}
                G(x,y): x \text{ le gana a } y 
            \end{equation*}

            ¿Cuál o cuáles son las representaciones correctas para la siguiente 
            oración?
            \begin{center}
                Brasil vence a cada equipo contra el que Alemania pierde, 
                excepto a sí mismo. 
            \end{center}
        }
        \begin{checkboxes}
            \choice $\neg \exists x (G(x, \texttt{Alemania}) \land 
            x \neq \texttt{Brasil} \land G(\texttt{Brasil}, x))$

            \choice $\forall x (G(x, \texttt{Alemania}) \land x \neq 
            \texttt{Brasil} \rightarrow G(\texttt{Brasil}, x))$ % Correcta 

            \choice $\neg \forall x (G(x, \texttt{Alemania}) 
            \rightarrow x \neq \texttt{Brasil} \land G(\texttt{Brasil}, x))$

            \choice $\forall x (G(x, \texttt{Alemania}) \rightarrow 
            G(\texttt{Brasil}, x))$

            \choice Ninguna de las anteriores.
        \end{checkboxes}

        % Question 12
        \question
        {
            Sea el conjunto de personas de toda la comunidad académica 
            nuestro universo del discurso. Si tenemos los predicados:
            \begin{tasks}(2)
                \task $E(x): x$ es un estudiante  
                \task $M(x): x$ es un maestro 
                \task $P(x,y): x$ le hace una pregunta a $y$ 
            \end{tasks}

            De acuerdo a esto, ¿cuál o cuáles de las siguientes expresiones son 
            \textbf{verdaderas}?
        }
        \begin{checkboxes}
            \choice La traducción del enunciado
            \begin{equation*}
                \exists x (E(x) \land \forall y (M(y) \rightarrow 
                \neg P(x,y)))
            \end{equation*}

            es \textit{¨Hay un estudiante al que ningún profesor le ha hecho 
            preguntas¨}.

            \choice La traducción del enunciado
            \begin{equation*}
                \exists x \forall y (E(x) \land M(y) \rightarrow P(x,y))
            \end{equation*}

            es \textit{¨Un estudiante le ha hecho preguntas a todos los 
            profesores¨}. % Correcta 

            \choice La traducción del enunciado
            \begin{equation*}
                \forall x (E(x) \rightarrow P(x, \texttt{Profesor García}))
            \end{equation*} 

            es \textit{¨Cada estudiante le ha hecho una pregunta al profesor 
            García¨}. % Correcta

            \choice La traducción del enunciado
            \begin{equation*}
                \forall x ((M(x) \rightarrow P(x, \texttt{Profesor López})) 
                \lor P(\texttt{Profesor Pérez}, x))
            \end{equation*}

            es \textit{¨Todo profesor ha hecho una pregunta al profesor López o 
            bien el profesor Pérez les ha hecho una pregunta¨}. 

            \choice Ninguna de las anteriores.
        \end{checkboxes}

        % Question 13
        \question{¿Cuál o cuáles de las siguientes expresiones son 
        \textbf{verdaderas}?}
        \begin{checkboxes}
            \choice En posible traducir cualquier tipo de expresión a Lógica de 
            Primer Orden, de ahí su inmenso poder. 

            \choice $\forall x S(x) \rightarrow \neg \exists y G(y) \equiv 
            \exists y G(y) \rightarrow \neg \forall x S(x)$ % Correcta 

            \choice En Lógica de Primer Orden, no es posible que el universo 
            del discurso sea un conjunto infinito. 

            \choice $\forall x (T(x) \lor M(x)) \equiv \forall x (\neg T(x) 
            \rightarrow M(x))$ % Correcta 

            \choice Ninguna de las anteriores.
        \end{checkboxes}

        \newpage
        % Question 14
        \question{¿Cuál o cuáles de las siguientes expresiones son 
        \textbf{verdaderas}?}
        \begin{checkboxes}
            \choice En Lógica de Primer Orden, siempre sucede que la negación 
            de una implicación da como resultado otra implicación. 

            \choice $\neg (\exists x T(x) \rightarrow \forall y M(y)) \equiv 
            \exists x T(x) \land \forall y \neg M(y)$

            \choice $\neg \forall x (F(x) \rightarrow I(x)) \equiv \exists x 
            (F(x) \land \neg I(x))$ % Correcta 

            \choice En Lógica de Primer Orden, no es posible aplicar leyes 
            distributivas entre cuantificadores y conectivos. 

            \choice Ninguna de las anteriores.
        \end{checkboxes}

        % Question 15
        \question{¿Cuál o cuáles de las siguientes expresiones son 
        \textbf{verdaderas}?}
        \begin{checkboxes}
            \choice El argumento 
            \begin{verbatim}
            Todos los hombres son mortales.
            Sócrates es un hombre. 
            Por lo tanto, Sócrates es mortal. 
            \end{verbatim}

            es correcto, y esto lo podemos justificar usando la instanciación 
            universal. % Correcta 

            \choice El argumento 
            \begin{verbatim}
            Si un número entero es impar, entonces su cuadrado es par. 
            x es un número entero impar 
            Por lo tanto, x*x es par 
            \end{verbatim}

            es correcto, y esto lo podemos justificar usando el modus ponens 
            universal. % Correcto 

            \choice El argumento 
            \begin{verbatim}
            Todas las personas sanas comen una manzana verde al día. 
            Erick no es una persona sana. 
            Por lo tanto, Erick no come una manzana verde al día. 
            \end{verbatim}

            es correcto, y esto lo podemos justificar usando el modus tollens 
            universal. % Correcto 

            \choice El argumento 
            \begin{verbatim}
            Johan es un artista que tiene el cabello chino. 
            Por lo tanto, existe un artista que tiene el cabello chino. 
            \end{verbatim}

            es correcto, y esto lo podemos justificar usando la generalización 
            existencial. % Correcta 

            \choice Ninguna de las anteriores.
        \end{checkboxes}

        % Question 16
        \question{¿Cuál o cuáles de las siguientes expresiones son 
        \textbf{verdaderas}?}
        \begin{checkboxes}
            \choice Todos los predicados son enunciados. 
            
            \choice Los predicados pueden ser elementos del universo 
            de discurso. 

            \choice El universo de discurso se supone no vacío y debe ser 
            claro y estar bien definido. % Correcta 

            \choice Una constante es la representación de un elemento 
            en particular dentro del universo de discurso. % Correcta 

            \choice Ninguna de las anteriores.
        \end{checkboxes}

        % Question 17
        \question{¿Cuál o cuáles de las siguientes expresiones son 
        \textbf{verdaderas}?}
        \begin{checkboxes}
            \choice El argumento 
            \begin{verbatim}
            Todo el mundo habla consigo mismo. 
            Por lo tanto, todo el mundo habla con todo el mundo. 
            \end{verbatim}

            es correcto, y esto lo podemos justificar usando la generalización
            universal. 

            \choice El argumento 
            \begin{verbatim}
            Cualquier suma de dos números enteros es un número entero. 
            La suma de a+b es un número entero. 
            Por lo tanto, los números a y b son números enteros. 
            \end{verbatim}

            es correcto, y esto lo podemos justificar usando la instanciación
            universal. 

            \choice El argumento 
            \begin{verbatim}
            Todos los perritos felices son sacados a pasear al menos 
            una vez al día.
            Blacky es sacada a pasear al menos una vez al día. 
            Por lo tanto, Blacky es una perrita feliz.  
            \end{verbatim}

            es correcto, y esto lo podemos justificar usando la instanciación
            universal

            \choice El argumento 
            \begin{verbatim}
            Todo el mundo es amigo de todo el mundo. 
            Por lo tanto, todo el mundo es amigo de sí mismo. 
            \end{verbatim}

            es correcto, y esto lo podemos justificar usando la instanciación
            universal. % Correcta 

            \choice Ninguna de las anteriores.
        \end{checkboxes}

        % Question 18
        \question
        {
            Sea el conjunto de todas las personas nuestro universo de discurso. 
            Si tenemos el predicado:
            \begin{center}
                $Q(x,y): x $ quiere a $y$
            \end{center}

            ¿Cuál o cuáles de las siguientes expresiones son 
            \textbf{verdaderas}?}
            \begin{checkboxes}
                \choice La traducción del enunciado
                \begin{equation*}
                    \neg (\exists x \exists y Q(x,y))
                \end{equation*}
                
                es \textit{Nadie quiere a nadie}. % Correcta 

                \choice La traducción del enunciado 
                \begin{equation*}
                    \neg \exists x \forall y Q(x,y)
                \end{equation*}

                es \textit{Nadie quiere a todos}. % Correcta 

                \choice La traducción del enunciado 
                \begin{equation*}
                    \exists x \forall y Q(x,y)
                \end{equation*}
                
                es \textit{Alguien quiere a todos}. % Correcta 

                \choice La traducción del enunciado 
                \begin{equation*}
                    \forall y \exists x \neg Q(x,y)
                \end{equation*}
                
                es \textit{Alguien quiere a nadie}. 

                \choice Ninguna de las anteriores.
            \end{checkboxes}

            \newpage
    \end{questions}
\end{document}
