\documentclass[oneside]{style}

\title{Versión 01}
\principal{Examen 12}
\author{Tania Michelle Rubí Rojas}
\semester{Semestre 2023-1}

\begin{document}
\maketitle

\vspace{5mm}
\noindent
Nombre y número de cuenta: \hrulefill\

\vspace*{5mm}
Para cada uno de los siguientes ejercicios, \textbf{justifica ampliamente} tu 
respuesta:

\begin{questions}[label=\protect\circled{\bfseries\arabic*}]

    % Ejercicio 01
    \question
    {
        \textbf{Traduce} los siguientes enunciados a Lógica de Primer Orden, 
        \textbf{indicando} de manera clara el universo del discurso y 
        utilizando únicamente el predicado \texttt{G(x,y):x le gana a y } 
        (puedes usar las constantes que requieras): 
        \begin{itemize}
            \item Si Argentina gana un partido de fútbol contra Croacia, 
            entonces no pierde todos sus partidos. 

            \item Francia vence a cada equipo contra el que Inglaterra 
            pierde, excepto a él mismo. 
        \end{itemize}
    }    
    
    % Ejercicio 02
    \question
    {
        \textbf{Transforma} las siguientes fórmulas mediante equivalencias 
        lógicas, de manera que las negaciones sólo figuren frente a predicados.
        \begin{itemize}
            \item $\neg \exists x \forall y \neg \exists w \exists z \neg 
            (\neg P(x,y) \land Q(x) \rightarrow \forall w T(x,w))$

            \item $\neg (\forall x \exists w \neg (\neg P(a,x) \lor 
            R(c,w)) \land \exists z \neg \forall y (T(b,z) \land 
            \neg Q(y,a)))$
        \end{itemize}
    }

     % Ejercicio 03
     \question
     {
         \textbf{Traduce} los siguientes enunciados a Lógica de Primer Orden, 
         \textbf{indicando} de manera clara la traducción de los predicados 
         que utilizarás y el universo de discurso:
         \begin{itemize}
            \item Ningún directorio puede abrirse ni ningún archivo puede 
            cerrarse si se han detectado errores en el sistema. 

            \item Pueden reescribirse archívos de vídeo cuando hay al menos 
            $8$ megabytes de memoria disponible y la velocidad de conexión 
            es al menos de $56$ kilobits por segundo. 
         \end{itemize}
     }

    % Ejercicio 04
    \question
    {
        Para cada uno de los siguientes argumentos lógicos, \textbf{determina} su 
        correctud utilizando reglas de inferencia 
        \begin{itemize}
            \item Todos los perritos felices son sacados a pasear al menos 
            una vez al día. Blacky es sacada a pasear al menos una vez el día. 
            Por consiguiente, Blacky es una perrita feliz. 

            \item Cualquier suma de dos números enteros es un número 
            entero. La suma $a+b$ es un número natural. Por lo tanto, 
            los números $a$ y $b$ son números enteros. 
        \end{itemize}
    }

    % Ejercicio 05
    \question
     {
         \textbf{Traduce} los siguientes enunciados a Lógica de Primer Orden, 
         \textbf{indicando} de manera clara la traducción de los predicados 
         que utilizarás y el universo de discurso:
         \begin{itemize}
            \item Ningún directorio puede abrirse ni ningún archivo puede 
            cerrarse si se han detectado errores en el sistema. 

            \item Pueden reescribirse archívos de vídeo cuando hay al menos 
            $8$ megabytes de memoria disponible y la velocidad de conexión 
            es al menos de $56$ kilobits por segundo. 
         \end{itemize}
     }
\end{questions}
\end{document}
