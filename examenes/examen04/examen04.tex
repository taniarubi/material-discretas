\documentclass[12pt, a4paper]{exam}

% Soporte para cambiar la fecha que sale en el examen
\usepackage{advdate}
% Soporte para escribir en varias columnas
\usepackage{multicol}
% Soporte para los acentos.
\usepackage[utf8]{inputenc} 
\usepackage[T1]{fontenc}    
% Idioma español.
\usepackage[spanish,mexico,es-tabla]{babel}
\usepackage{graphicx}
\usepackage{tikz}
\usepackage{amsmath,amssymb,amsthm}

% Cambiamos los márgenes del documento. 
\usepackage[top=1.5cm,left=1.5cm,right=1.5cm]{geometry}

% Pie de página
\cfoot{Página \thepage\ de \numpages}

%%%%%%%%%%%%%%%%%%%%%%%%%%%%%%%%%%%%%%%%%%%%%%%%%%%%%%%%%%%%%%%%%%%%%%%%%%%%%%
\renewcommand{\thechoice}{\alph{choice}}

\makeatletter
\renewenvironment{checkboxes}%
   {\setcounter{choice}{0}\list{\checkbox@char}%
      {%
        \settowidth{\leftmargin}{W.\hskip\labelsep\hskip 2.5em}%
        \def\choice{%
          \if@correctchoice
            \color@endgroup \endgroup
          \fi
          \stepcounter{choice}
          \item[\checked@char]
          \do@choice@pageinfo
        } % choice
        \def\CorrectChoice{%
          \if@correctchoice
            \color@endgroup \endgroup
          \fi
          \ifprintanswers
            % We can't say \choice here, because that would
            % insert an \endgroup.
            % 2016/05/10: We say \color@begingroup in addition to
            % \begingroup in case \CorrectChoiceEmphasis involves color
            % and the text exactly fills the line (which would
            % otherwise create a blank line after this choice):
            % 2016/05/11: We leave hmode if we're in it,
            % i.e., if there's no blank line preceding this
            % \CorrectChoice command.  (Without this, the
            % \special created by a \color{whatever} command that might
            % be inserted by \CorrectChoice@Emphasis would be appended 
            % to the previous \choice, which could cause an extra
            % (blank) line to be inserted before this \CorrectChoice.)
            % Since \par and \endgraf seem to cancel \@totalleftmargin
            % (for reasons I don't understand), we'll do the following:
            % Motivated by  the def of \leavevmode, 
            %      \def\leavevmode{\unhbox\voidb@x}
            % we will now leave hmode (if we're in hmode):
            \ifhmode \unskip\unskip\unvbox\voidb@x \fi
            \begingroup \color@begingroup \@correctchoicetrue
            \CorrectChoice@Emphasis
            \stepcounter{choice}
            \item[\checked@char]
          \else
            \stepcounter{choice}
            \item[\checked@char]
          \fi
          \do@choice@pageinfo
        } % CorrectChoice
        \let\correctchoice\CorrectChoice
        \labelwidth\leftmargin\advance\labelwidth-\labelsep
        \topsep=0pt
        \partopsep=0pt
        \checkboxeshook
      }%
   }%
   {\if@correctchoice \color@endgroup \endgroup \fi \endlist}
 \makeatother

% Make checkbox character a circle with the letter
\checkboxchar{\tikz[baseline={([yshift=-.8ex]current bounding box.center)}]\node[shape=circle,minimum size=4mm,draw] at (0,0) {\thechoice};}
% Make checked box character bold WITH surd
%\checkedchar{\tikz[baseline={([yshift=-.8ex]current bounding box.center)}]\node[shape=circle,minimum size=8mm,draw] at (0,0) {} node at (0,0) {\thechoice\llap{$\surd$}};}
% Make checked box character bold
\checkedchar{\tikz[baseline={([yshift=-.8ex]current bounding box.center)}]\node[shape=circle,minimum size=4mm,draw] at (0,0) {} node at (0,0) {\thechoice};}
\printanswers
%%%%%%%%%%%%%%%%%%%%%%%%%%%%%%%%%%%%%%%%%%%%%%%%%%%%%%%%%%%%%%%%%%%%%%%%%%%%%%

\begin{document}
    %%%%%%%%%%%%%%%%%%%%%%%%%%%%%%%%%%%%%%%%%%%%%%%%%%%%%%%%%%%%%%%%%%%%%%%%%%%%%%%
    %%%%%%%%%%%%%%%%%%%%%%%%%%%%%%%% ENCABEZADO %%%%%%%%%%%%%%%%%%%%%%%%%%%%%%%%%%%
    \centering
    \hrule \hrule \hrule 
    \vspace{5mm}
    \begin{minipage}[c]{0.8\textwidth}
        \begin{center}
            {\large\textbf{Mission 04, Start!} \par
            \large \textbf{Estructuras Discretas} \par
            \large \textbf{Semestre 2023-1} \par
            \large \textbf{\today}	\par}
        \end{center}
    \end{minipage}

    \vspace{0.2in}
    \noindent
    \textbf{Tania Michelle Rubí Rojas}
    \vspace{2mm}
    \hrule \hrule \hrule 
    %%%%%%%%%%%%%%%%%%%%%%%%%%%%%%%%%%%%%%%%%%%%%%%%%%%%%%%%%%%%%%%%%%%%%%%%%%%%%%%
    %%%%%%%%%%%%%%%%%%%%%%%%%%%%%%%%%%%%%%%%%%%%%%%%%%%%%%%%%%%%%%%%%%%%%%%%%%%%%%%

    \vspace{5mm}
    \noindent
    Nombre y número de cuenta: \hrulefill\

    \begin{questions}
        % Question 01
        \question
        {
            Definimos un conjunto $A$ de manera recursiva como sigue:
            \begin{itemize}
                \item $6 \in A$
                \item Si $a,b \in A$, entonces $a+b \in A$
                \item Nada pertenece a $A$ a menos que se obtenga del caso base 
                y la regla recursiva. 
            \end{itemize}

            ¿Cuál o cuáles de las siguientes expresiones son 
            \textbf{verdaderas}?
        }
        \begin{checkboxes}
            \choice $A$ es el conjunto de todos los múltiplos enteros de $6$, 
            exceptuando al cero. 

            \choice La regla recursiva carece del componente \textsc{conexión}.

            \choice La regla de extremo permite que $0 \in A$.

            \choice En la regla recursiva, los valores de $a$ y $b$ tienen que 
            ser diferentes. 
            
            \choice Ninguna de las anteriores. % Correcta
        \end{checkboxes}

        % Question 02
        \question
        {
            ¿Cuál de las siguientes definiciones recursivas construyen al 
            conjunto $A = \{2n+1 \; | \; n \in \mathbb{N}\}$?
        }
        \begin{checkboxes}
            \choice Definición:
            \begin{itemize}
                \item $3 \in A$
                \item Si $n \in A$, entonces $n+2 \in A$
                \item Estos y solo estos son elementos de $A$.
            \end{itemize}

            \choice Definición:
            \begin{itemize}
                \item $1 \in A$
                \item Si $n \in A$, entonces $n+2 \in A$
                \item Estos y solo estos son elementos de $A$.
            \end{itemize} % Correcta

            \choice Definición:
            \begin{itemize}
                \item $1,3 \in A$
                \item Si $n \in A$, entonces $n+4 \in A$
                \item Estos y solo estos son elementos de $A$.
            \end{itemize} % Correcta

            \choice Definición:
            \begin{itemize}
                \item $1 \in A$
                \item Si $n \in A$, entonces $n+3 \in A$
                \item Estos y solo estos son elementos de $A$.
            \end{itemize}
            
            \choice Ninguna de las anteriores. 
        \end{checkboxes}

        % Question 03
        \question
        {
            Definimos un conjunto $A$ de manera recursiva como sigue:
            \begin{itemize}
                \item Si $x \in \mathbb{N}$ entonces $x \in A$
                \item Si $u,v \in \mathbb{N}$ entonces $u \cdot v \in A$
                \item Nada pertenece a $A$ a menos que se obtenga del caso base
                y la regla recursiva. 
            \end{itemize}

            ¿Cuál o cuáles de las siguientes expresiones son 
            \textbf{verdaderas}?
        }
        \begin{checkboxes}
            \choice La definición no incluye ninguna regla realmente recursiva.
            % Correcta 

            \choice El caso base no está definido.

            \choice Una de las reglas recursivas se puede omitir sin alterar
            la definición del conjunto. % Correcta

            \choice $A \subset \mathbb{N}$
            
            \choice Ninguna de las anteriores. 
        \end{checkboxes}

        % Question 04
        \question{¿Cuál o cuáles de las siguientes expresiones son 
        \textbf{verdaderas}?}
        \begin{checkboxes}
            \choice Únicamente podemos definir conjuntos de números de manera 
            recursiva. 

            \choice Una definición recursiva consta de tres partes: caso base, 
            caso recursivo y conexión.

            \choice No es posible definir de manera recursiva el conjunto de 
            números reales que son divisibles entre dos. % Correcta

            \choice En un definición recursiva, el caso base construye al 
            conjunto a partir de dos elementos o más. 
            
            \choice Ninguna de las anteriores.
        \end{checkboxes}

        % Question 05
        \question
        {
            Sea $A$ el conjunto de todas las cadenas de a's y b's. Definimos un 
            conjunto $B$ de manera recursiva como sigue:
            \begin{itemize}
                \item $a,b \in B$
                \item Si $x \in B$, entonces $ax, bx \in B$ 
                \item Nada pertenece a $B$ a menos que se obtenga de los casos 
                base y las reglas recursivas. 
            \end{itemize}

            ¿Cuál o cuáles de las siguientes expresiones son 
            \textbf{verdaderas}?
        }
        \begin{checkboxes}
            \choice La definición no incluye ninguna regla realmente 
            recursiva. 

            \choice Sea $f(w) = aw$ tal que $w \in A$. Entonces $B = f$.

            \choice $babaaba \in B$ % Correcta

            \choice $A = B$ % Correcta
            
            \choice Ninguna de las anteriores. 
        \end{checkboxes}

        % Question 06
        \question{¿Cuál o cuáles de las siguientes expresiones son 
        \textbf{verdaderas}?}
        \begin{checkboxes}
            \choice Una definición recursiva tiene únicamente un caso base. 

            \choice Únicamente podemos definir recursivamente conjuntos 
            finitos y conjuntos infinitos más pequeños que $\mathbb{N}$. 

            \choice En una definición recursiva, el propósito de la regla 
            recursiva es construir elementos del conjunto a partir de los 
            elementos que ya se encuentran en él. % Correcta

            \choice Si una definición recursiva es válida, entonces podemos 
            omitir la regla de exclusión, pues se sobreentiende que existe. 
            
            \choice Ninguna de las anteriores. 
        \end{checkboxes}

        \newpage
        % Question 07
        \question
        {
            Sea $A$ el conjunto de todas las cadenas de ceros y unos. 
            Definimos un conjunto $B$ de manera recursiva como sigue:
            \begin{itemize}
                \item Para cada $a \in A$, $aa^R \in B$ (donde $a^R$ es $a$ 
                escrita de atrás hacia adelante).
                \item Si $v,w$ son cadenas tales que $vv^R, ww^R \in B$, 
                entonces $wvv^Rw^R \in B$
                \item Nada pertenece a $A$ a menos que se obtenga de los casos 
                base y las reglas recursivas. 
            \end{itemize}

            ¿Cuál o cuáles de las siguientes expresiones son 
            \textbf{verdaderas}?
        }
        \begin{checkboxes}
            \choice $B$ es el conjunto de todas las cadenas palíndromas en 
            $A$. 

            \choice La regla recursiva construye los mismos elementos que 
            el caso base. % Correcta

            \choice Por el caso base, $00,11 \in B$. Si $v = 00 \in B$ y 
            $w = 11 \in B$, por la regla recursiva, $1001 \in B$. 

            \choice El componente \textsc{conexión} de la regla recursiva está
            mal escrito. 
            
            \choice Ninguna de las anteriores. 
        \end{checkboxes}

        % Question 08
        \question
        {
            Definimos un conjunto $A$ de manera recursiva como sigue:
            \begin{itemize}
                \item $2 \in A$
                \item Si $s \in A$, entonces $w-2 \in A$
                \item Si $w \in A$, entonces $s+6 \in A$
            \end{itemize}

            ¿Cuál o cuáles de las siguientes expresiones son 
            \textbf{verdaderas}?
        }
        \begin{checkboxes}
            \choice $A$ es el conjunto de todos los múltiplos de $2$. 

            \choice El componente \textsc{conexión} de una de las reglas 
            recursivas está mal escrito. % Correcta

            \choice El elemento $3$ puede ser un elemento de $A$. % Correcta

            \choice La definición recursiva está incompleta. % Correcta
            
            \choice Ninguna de las anteriores. 
        \end{checkboxes}

        % Question 09
        \question
        {
            Sea $A$ el conjunto de todas las cadenas de ceros y unos. Definimos 
            un conjunto $B$ de manera recursiva como sigue:
            \begin{itemize}
                \item $01,10 \in B$
                \item Si $w \in B$, entonces $10w,1w0,w10,01w,0w1,w01 \in B$
                \item Nada pertenece a $B$ a menos que se obtenga del caso base
                y la regla recursiva. 
            \end{itemize}

            ¿Cuál o cuáles de las siguientes expresiones son 
            \textbf{verdaderas}?
        }
        \begin{checkboxes}
            \choice $B$ es el conjunto de todas las cadenas de ceros y unos cuya 
            longitud es par. 

            \choice $B$ es el conjunto de todas las cadenas de ceros y unos que 
            tienen la misma cantidad de ceros que de unos. % Correcta 

            \choice $B$ es el conjunto de todas las cadenas de ceros y unos que 
            son palíndromas. 

            \choice $B$ es el conjunto de todas las cadenas de ceros y unos que 
            terminan en al menos un cero. 
            
            \choice Ninguna de las anteriores. 
        \end{checkboxes}

        % Question 10
        \question
        {
            Sea $A$ el conjunto de todas las cadenas de a's y b's. Definimos un 
            conjunto $B$ de manera recursiva como sigue:
            \begin{itemize}
                \item $aa, bb, ab, ba \in B$
                \item Si $w \in B$, entonces $ww \in B$
                \item Nada pertenece a $B$ a menos que se obtenga del caso base
                y la regla recursiva. 
            \end{itemize}

            ¿Cuál o cuáles de las siguientes expresiones son 
            \textbf{verdaderas}?
        }
        \begin{checkboxes}
            \choice $B$ es el conjunto de todas las cadenas de a's y b's que 
            tienen la misma cantidad de a's que de b's. 

            \choice $B$ es el conjunto de todas las cadenas de a's y b's que 
            resulta de concatenar una cadena consigo misma. 

            \choice $B$ es el conjunto de todas las cadenas de a's y b's que 
            son palíndromas. 

            \choice $B$ es el conjunto de todas las cadenas de a's y b's cuya 
            longitud es par. 
            
            \choice Ninguna de las anteriores. % Correcta
        \end{checkboxes}

        % Question 11
        \question
        {
            ¿Cuál de las siguientes definiciones recursivas construyen al 
            conjunto, digamos $A$, de las personas que son descendientes 
            de Shakira?
        }
        \begin{checkboxes}
            \choice Definición:
            \begin{itemize}
                \item Si $x$ es hijo de Shakira, entonces $x \in A$.
                \item Si $y \in A$ y $x$ es hijo de $y$, entonces $x \in A$.
                \item Ninguna otra persona está en $A$, excepto las definidas 
                por el caso base y la regla recursiva. 
            \end{itemize} % Correcta

            \choice Definición:
            \begin{itemize}
                \item Los hijos de Shakira pertenecen a $A$. 
                \item Si una persona $x$ es hijo de la persona $y$, tal que 
                $x \in A$, entonces $y \in A$.
                \item Ninguna otra persona está en $A$, excepto las definidas 
                por el caso base y la regla recursiva. 
            \end{itemize} 

            \choice Definición:
            \begin{itemize}
                \item Shakira está en $A$
                \item Si $x \in A$ entonces $y \in A$
                \item Si $y \in A$ y $z$ es hijo de $y$, entonces $z \in A$.
                \item Ninguna otra persona está en $A$, excepto las definidas 
                por el caso base y la regla recursiva. 
            \end{itemize}

            \choice Definición:
            \begin{itemize}
                \item Los hijos de Shakira pertenecen a $A$. 
                \item Si una persona $x$ es hijo de la persona $y$, tal que 
                $y \in A$, entonces $x \in A$.
                \item Ninguna otra persona está en $A$, excepto las definidas 
                por el caso base y la regla recursiva. 
            \end{itemize} % Correcta
            
            \choice Ninguna de las anteriores. 
        \end{checkboxes}

        \newpage
        % Question 12
        \question
        {
            Sea $A$ el conjunto de todas las cadenas de ceros y unos. Definimos 
            un conjunto $B$ de manera recursiva como sigue:
            \begin{itemize}
                \item $0,1 \in A$
                \item Si $w \in A$, entonces $w0, w1 \in A$
                \item Si $w \in A$, entonces $0w, 1w \in A$
                \item Nada pertenece a $A$ a menos que se obtenga del caso base
                y la regla recursiva. 
            \end{itemize}

            ¿Cuál o cuáles de las siguientes expresiones son 
            \textbf{verdaderas}?
        }
        \begin{checkboxes}
            \choice El componente \textsc{conexión} de la regla recursiva está
            mal escrito, pues hay dos reglas recursivas. 

            \choice $010101 \in B$ 

            \choice $101100111000111100001111100000111\ldots \in B$

            \choice Una de las reglas recursivas se puede omitir sin alterar
            la definición del conjunto. 
            
            \choice Ninguna de las anteriores. % Correcta
        \end{checkboxes}

        % Question 13
        \question{¿Cuál o cuáles de las siguientes expresiones son 
        \textbf{verdaderas}?}
        \begin{checkboxes}
            \choice Si una definición recursiva tiene más de una regla 
            recursiva, entonces está mal definida. 

            \choice Si omitimos escribir la regla de extremo, entonces 
            cualquier elemento podría pertener al conjunto que queremos 
            construir. % Correcta

            \choice El componente \textsc{conexión} hace referencia a 
            que la regla recursiva sea una oración condicional. 

            \choice No es posible definir recursivamente el conjunto de todos 
            los números enteros que no son divisibles entre $3$.
            
            \choice Ninguna de las anteriores. 
        \end{checkboxes}

        % Question 14
        \question
        {
            Sea $A$ es el conjunto de todas las cadenas de a's y b's. Definimos 
            un conjunto $B$ de manera recursiva como sigue:
            \begin{itemize}
                \item $aa, bb \in B$
                \item Si $w \in B$, entonces $aaw,wbb \in B$
                \item Nada pertenece a $B$ a menos que se obtenga del caso base
                y la regla recursiva. 
            \end{itemize}

            ¿Cuál o cuáles de las siguientes expresiones son 
            \textbf{verdaderas}?
        }
        \begin{checkboxes}
            \choice $B$ es el conjunto de todas las cadenas de a's y b's que 
            empiezan con $aa$.

            \choice $B$ es el conjunto de todas las cadenas de a's y b's que 
            tienen al menos una $b$. 

            \choice $B$ es el conjunto de todas las cadenas de a's y b's que 
            tienen la misma cantidad de a's que de b's. 

            \choice $B$ es el conjunto de todas las cadenas de a's y b's que 
            terminan con $bb$.
            
            \choice Ninguna de las anteriores. % Correcta
        \end{checkboxes}

        \newpage
        % Question 15
        \question
        {
            Sea $A$ el conjunto de todas las cadenas de a's y b's. Además, sean 
            $B$ el conjunto de todas las cadenas de a's y $C$ el conjunto de 
            todas las cadenas de b's. Definimos un conjunto $D$ de manera 
            recursiva como sigue:
            \begin{itemize}
                \item $a,b \in D$
                \item Si $u \in B$ y $v \in C$, entonces $uv \in D$
                \item Nada pertenece a $D$ a menos que se obtenga del caso base
                y la regla recursiva. 
            \end{itemize}

            ¿Cuál o cuáles de las siguientes expresiones son 
            \textbf{verdaderas}?
        }
        \begin{checkboxes}
            \choice $baba \in D$

            \choice $D$ es el conjunto de todas las cadenas de a's y b's que no 
            inician con una $b$. 

            \choice $D$ es el conjunto de todas las cadenas de a's y b's que 
            terminan en al menos una $b$. 

            \choice La definición no incluye ninguna regla realmente recursiva. 
            % Correcta
            
            \choice Ninguna de las anteriores. 
        \end{checkboxes}

        % Question 16
        \question
        {
            Definimos un conjunto $A$ de manera recursiva como sigue:
            \begin{itemize}
                \item La lista vacía pertenece a $A$
                \item Si $n \in \mathbb{N}$ y $l \in A$, entonces $(n:l) \in A$.
                \item Nada pertenece a $A$ a menos que se obtenga del caso base
                y la regla recursiva. 
            \end{itemize}

            ¿Cuál o cuáles de las siguientes expresiones son 
            \textbf{verdaderas}?
        }
        \begin{checkboxes}
            \choice $A$ es el conjunto de todas las listas cuyos elementos son 
            números enteros. 

            \choice La definición no incluye ninguna regla realmente recursiva.

            \choice $(5: (9: (8: []))) \in A$ % Correcta

            \choice $(12: (9: (3: 2))) \in A$
            
            \choice Ninguna de las anteriores. 
        \end{checkboxes}

        % Question 17
        \question
        {
            Sea $A$ el conjunto de todas las cadenas de a's y b's. Definimos un 
            conjunto $B$ de manera recursiva como sigue:
            \begin{itemize}
                \item $ab, bab \in B$
                \item Si $w \in B$, entonces $awb \in B$
                \item Nada pertenece a $B$ a menos que se obtenga del caso base
                y la regla recursiva. 
            \end{itemize}

            ¿Cuál o cuáles de las siguientes expresiones son 
            \textbf{verdaderas}?
        }
        \begin{checkboxes}
            \choice Todas las cadenas que pertenecen a $B$ terminan con al menos 
            una $b$. % Correcta

            \choice $B$ contiene seis cadenas cuya longitud es igual o menor a
            $7$. % Correcta

            \choice $B$ es el conjunto de todas las cadenas de a's y b's cuya 
            longitud es impar. 

            \choice La definición no incluye ninguna regla realmente recursiva.
            
            \choice Ninguna de las anteriores. 
        \end{checkboxes}

        \newpage
        % Question 18
        \question
        {
            Definimos un conjunto $A$ de manera recursiva como sigue:
            \begin{itemize}
                \item $0 \in A$
                \item Si $n,m \in A$, entonces $n+m \in A$
                \item Nada pertenece a $A$ a menos que se obtenga del caso base
                y la regla recursiva. 
            \end{itemize}

            ¿Cuál o cuáles de las siguientes expresiones son 
            \textbf{verdaderas}?
        }
        \begin{checkboxes}
            \choice $A$ es un conjunto unitario. % Correcta

            \choice La definición recursiva está incompleta. 

            \choice La regla recursiva construye los mismos elementos que el 
            caso base. % Correcta

            \choice Sea $f(n,m) = n+m$, tal que $n,m \in \mathbb{N}$. Entonces 
            $A = f$.
            
            \choice Ninguna de las anteriores. 
        \end{checkboxes}

        % Question 19
        \question
        {
            Definimos un conjunto $A$ de manera recursiva como sigue:
            \begin{itemize}
                \item Si $u \in \mathbb{Z}$ entonces $u \in A$
                \item Si $a \in A$, entonces $a \in A$
                \item Nada pertenece a $A$ a menos que se obtenga del caso base
                y la regla recursiva. 
            \end{itemize}

            ¿Cuál o cuáles de las siguientes expresiones son 
            \textbf{verdaderas}?
        }
        \begin{checkboxes}
            \choice Una de las reglas recursivas se puede omitir sin alterar
            la definición del conjunto. % Correcta

            \choice $A \subseteq \mathbb{N}$

            \choice $(A - \mathbb{N}) \cap \{0, -1\} = \varnothing$

            \choice La definición no incluye ninguna regla realmente recursiva. 
            
            \choice Ninguna de las anteriores. 
        \end{checkboxes}

        % Question 20
        \question
        {
            Sea $A$ el conjunto de todas las cadenas de a's y b's. Definimos un 
            conjunto $B$ de manera recursiva como sigue:
            \begin{itemize}
                \item $bb \in B$
                \item Si $w \in B$ entonces $aw, bw, wa, wb \in B$
                \item Nada pertenece a $B$ a menos que se obtenga del caso base
                y la regla recursiva. 
            \end{itemize}

            ¿Cuál o cuáles de las siguientes expresiones son 
            \textbf{verdaderas}?
        }
        \begin{checkboxes}
            \choice $B$ es el conjunto de todas las cadenas de a's y b's que 
            inician y terminan en $b$.

            \choice $A \subseteq B$

            \choice Le falta un componente a la definición recursiva. 

            \choice El componente \textsc{conexión} de la regla recursiva está
            mal escrito. 
            
            \choice Ninguna de las anteriores. % Correcta
        \end{checkboxes}

        \newpage
        % Question 21
        \question
        {
            Definimos un conjunto $A$ de manera recursiva como sigue:
            \begin{itemize}
                \item $3 \in A$
                \item Si $n \in A$ entonces $3n \in A$
                \item Nada pertenece a $A$ a menos que se obtenga del caso base
                y la regla recursiva. 
            \end{itemize}

            ¿Cuál o cuáles de las siguientes expresiones son 
            \textbf{verdaderas}?
        }
        \begin{checkboxes}
            \choice $A \subseteq \{x \; | \; x \text{ es un múltiplo de } 3\}$
            % Correcta

            \choice $A \subset \{y \; | \; y \text{ es un número primo}\}$

            \choice $A = \{x \; | \; x \text{ es un múltiplo de } 3\}$

            \choice $A$ es un conjunto finito. 

            \choice Ninguna de las anteriores. 
        \end{checkboxes}

        % Question 22
        \question
        {
            ¿Cuál de las siguientes definiciones recursivas construyen al 
            conjunto $A = \{2n \; | \; n \in \mathbb{N}-\{0\}\}$?
        }
        \begin{checkboxes}
            \choice Definición:
            \begin{itemize}
                \item $2,4 \in A$
                \item Si $n \in A$ entonces $x+4 \in A$
                \item Estos y solo estos son elementos de $A$.
            \end{itemize} 

            \choice Definición:
            \begin{itemize}
                \item $0 \in A$
                \item Si $n \in A$, entonces $n+2 \in A$
                \item Estos y solo estos son elementos de $A$.
            \end{itemize}

            \choice Definición:
            \begin{itemize}
                \item $2 \in A$
                \item Si $n,m \in A$, entonces $n+m \in A$
                \item Estos y solo estos son elementos de $A$.
            \end{itemize} % Correcta

            \choice Definición:
            \begin{itemize}
                \item $2 \in A$
                \item Si $n\in A$, entonces $n+2 \in A$
                \item Estos y solo estos son elementos de $A$.
            \end{itemize} % Correcta
            
            \choice Ninguna de las anteriores. 
        \end{checkboxes}
    \end{questions}
\end{document}