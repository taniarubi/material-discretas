\documentclass[12pt, a4paper]{exam}

% Soporte para cambiar la fecha que sale en el examen
\usepackage{advdate}
% Soporte para los acentos.
\usepackage[utf8]{inputenc} 
\usepackage[T1]{fontenc}    
% Idioma español.
\usepackage[spanish,mexico,es-tabla]{babel}
\usepackage{graphicx}
\usepackage{tikz}
\usepackage{amsmath,amssymb,amsthm}

% Cambiamos los márgenes del documento. 
\usepackage[top=1.5cm,left=1.5cm,right=1.5cm]{geometry}

% Pie de página
\cfoot{Página \thepage\ de \numpages}

%%%%%%%%%%%%%%%%%%%%%%%%%%%%%%%%%%%%%%%%%%%%%%%%%%%%%%%%%%%%%%%%%%%%%%%%%%%%%%
\renewcommand{\thechoice}{\alph{choice}}

\makeatletter
\renewenvironment{checkboxes}%
   {\setcounter{choice}{0}\list{\checkbox@char}%
      {%
        \settowidth{\leftmargin}{W.\hskip\labelsep\hskip 2.5em}%
        \def\choice{%
          \if@correctchoice
            \color@endgroup \endgroup
          \fi
          \stepcounter{choice}
          \item[\checked@char]
          \do@choice@pageinfo
        } % choice
        \def\CorrectChoice{%
          \if@correctchoice
            \color@endgroup \endgroup
          \fi
          \ifprintanswers
            % We can't say \choice here, because that would
            % insert an \endgroup.
            % 2016/05/10: We say \color@begingroup in addition to
            % \begingroup in case \CorrectChoiceEmphasis involves color
            % and the text exactly fills the line (which would
            % otherwise create a blank line after this choice):
            % 2016/05/11: We leave hmode if we're in it,
            % i.e., if there's no blank line preceding this
            % \CorrectChoice command.  (Without this, the
            % \special created by a \color{whatever} command that might
            % be inserted by \CorrectChoice@Emphasis would be appended 
            % to the previous \choice, which could cause an extra
            % (blank) line to be inserted before this \CorrectChoice.)
            % Since \par and \endgraf seem to cancel \@totalleftmargin
            % (for reasons I don't understand), we'll do the following:
            % Motivated by  the def of \leavevmode, 
            %      \def\leavevmode{\unhbox\voidb@x}
            % we will now leave hmode (if we're in hmode):
            \ifhmode \unskip\unskip\unvbox\voidb@x \fi
            \begingroup \color@begingroup \@correctchoicetrue
            \CorrectChoice@Emphasis
            \stepcounter{choice}
            \item[\checked@char]
          \else
            \stepcounter{choice}
            \item[\checked@char]
          \fi
          \do@choice@pageinfo
        } % CorrectChoice
        \let\correctchoice\CorrectChoice
        \labelwidth\leftmargin\advance\labelwidth-\labelsep
        \topsep=0pt
        \partopsep=0pt
        \checkboxeshook
      }%
   }%
   {\if@correctchoice \color@endgroup \endgroup \fi \endlist}
 \makeatother

% Make checkbox character a circle with the letter
\checkboxchar{\tikz[baseline={([yshift=-.8ex]current bounding box.center)}]\node[shape=circle,minimum size=4mm,draw] at (0,0) {\thechoice};}
% Make checked box character bold WITH surd
%\checkedchar{\tikz[baseline={([yshift=-.8ex]current bounding box.center)}]\node[shape=circle,minimum size=8mm,draw] at (0,0) {} node at (0,0) {\thechoice\llap{$\surd$}};}
% Make checked box character bold
\checkedchar{\tikz[baseline={([yshift=-.8ex]current bounding box.center)}]\node[shape=circle,minimum size=4mm,draw] at (0,0) {} node at (0,0) {\thechoice};}
\printanswers
%%%%%%%%%%%%%%%%%%%%%%%%%%%%%%%%%%%%%%%%%%%%%%%%%%%%%%%%%%%%%%%%%%%%%%%%%%%%%%

\begin{document}
    %%%%%%%%%%%%%%%%%%%%%%%%%%%%%%%%%%%%%%%%%%%%%%%%%%%%%%%%%%%%%%%%%%%%%%%%%%%%%%%
    %%%%%%%%%%%%%%%%%%%%%%%%%%%%%%%% ENCABEZADO %%%%%%%%%%%%%%%%%%%%%%%%%%%%%%%%%%%
    \centering
    \hrule \hrule \hrule 
    \vspace{5mm}
    \begin{minipage}[c]{0.8\textwidth}
        \begin{center}
            {\large\textbf{Mission 01, Start!} \par
            \large \textbf{Estructuras Discretas} \par
            \large \textbf{Semestre 2023-1} \par
            \large \textbf{\today}	\par}
        \end{center}
    \end{minipage}

    \vspace{0.2in}
    \noindent
    \textbf{Tania Michelle Rubí Rojas}
    \vspace{2mm}
    \hrule \hrule \hrule 
    %%%%%%%%%%%%%%%%%%%%%%%%%%%%%%%%%%%%%%%%%%%%%%%%%%%%%%%%%%%%%%%%%%%%%%%%%%%%%%%
    %%%%%%%%%%%%%%%%%%%%%%%%%%%%%%%%%%%%%%%%%%%%%%%%%%%%%%%%%%%%%%%%%%%%%%%%%%%%%%%

    \vspace{5mm}
    \noindent
    Nombre y número de cuenta: \hrulefill\

    \vspace{5mm}
    \noindent
    \textbf{Notación para el examen:} $\subset \; \text{subconjunto propio} 
    \quad \quad \subseteq \; \text{subconjunto} \quad \quad - \; 
    \text{diferencia de conjuntos}$

    \begin{questions}
        % Question 01
        \question
        {
            Sean los conjuntos
            \begin{align*}
                S &= \{\bigstar, \{\bigstar\}, \{\{\bigstar\}\}\} \\ 
                T &= \{\varnothing, \{\bigstar\}, \{\bigstar, \{\bigstar\}\}\} \\ 
                U &= \{\bigstar\}
            \end{align*}
            
            tal que $\mathcal{U} = \{\varnothing, \bigstar, \{\bigstar\}, 
            \{\{\bigstar\}\}, \{\bigstar, \{\bigstar\}\}\}$. ¿Cuáles de las 
            siguientes expresiones son \textbf{verdaderas}?
        }
        
        \begin{checkboxes}
            \choice $\varnothing \cap T = \{\varnothing\}$
            
            \choice $S^c \cap T = T - \{U\}$ % Correcta 

            \choice $\{\varnothing\} \cap T = \varnothing$

            \choice $S \cap U = \bigstar$

            \choice Ninguna de las anteriores. 
        \end{checkboxes}
        
        % Question 02
        \question{¿Cuáles de las siguientes expresiones son \textbf{verdaderas}?}
        \begin{checkboxes}
            \choice El conjunto potencia del conjunto $A$ es un conjunto que 
            está formado por todos los subconjuntos propios de $A$.

            \choice $\{a,b,c\} \supseteq \{a,2,b,3,c,4\}$

            \choice El elemento $0$ pertenece a $\varnothing$.

            \choice $\varnothing = \{\varnothing\}$
            
            \choice Ninguna de las anteriores. % Correcta
        \end{checkboxes}

        % Question 03
        \question{¿Cuáles de las siguientes expresiones son \textbf{verdaderas}?}
        \begin{checkboxes}
            \choice $\mathcal{P}(\{1\}) = \{\{\{1\}\}, \varnothing\}$

            \choice $|\{\bigstar, \{\bigstar, \bigstar\}\}| = 3$

            \choice $\{x \in \mathbb{N} \; | \; 0 < x \leq 4\} = \{1,4,3,2\}$ 
            % Correcta

            \choice $\{x \; | \; x \in \text{ es el nombre de un Rey Mago}\} = 
            \{\text{Melchor, Gaspar y Baltazar}\}$
            
            \choice Ninguna de las anteriores. 
        \end{checkboxes}

        % Question 04
        \question
        {
            Sean los conjuntos
            \begin{align*}
                R &= \{a,c, \pi, \bullet, m, n\} \\ 
                S &= \{\{a\}, c, m, n\} 
            \end{align*}
            
            ¿Cuáles de las siguientes expresiones son \textbf{verdaderas}?
        }

        \begin{checkboxes}
            \choice $S \subseteq R$

            \choice $a \in R$ % Correcta

            \choice $a \in S$

            \choice $\{a\} \subseteq S$
            
            \choice Ninguna de las anteriores. 
        \end{checkboxes}

        % Question 05
        \question{¿Cuáles de las siguientes expresiones son \textbf{verdaderas}?}
        \begin{checkboxes}
            \choice $\{x \; | \; x \text{ es un número natural mayor a $1945$}\} 
            = \{1945, 1946, 1947, 1948, \ldots\}$

            \choice Si $A$ es un conjunto cualquiera, entonces $A \cap \varnothing = 
            \varnothing$ % Correcta

            \choice $\{2,2,2,3,3,4,5,5,5,4\} \neq \{5,3,4,2\}$

            \choice $\{x \; | \; x \text{ es un planeta de nuestro sistema 
            solar}\} = \{\text{Mercurio, Venus, \ldots, Urano, Neptuno}\}$ % Correcta
            
            \choice Ninguna de las anteriores. 
        \end{checkboxes}

        % Question 06
        \question
        {
            Sean los conjuntos 
            \begin{align*}
                A =& \{x \; | \; x \; \text{es una vocal}\} \\ 
                B =& \{x \; | \; x \; \text{es una letra de la palara alegría}\} \\ 
                C =& \{x \; | \; x \; \text{es una consonante}\}
            \end{align*}

            tal que $\mathcal{U} = \{y \; | \; y \; \text{es una letra del 
            abecedario}\}$. ¿Cuáles de las siguientes expresiones son 
            \textbf{verdaderas}?
        }
        
        \begin{checkboxes}
            \choice $((A-B)^c) \cap C = C$ % Correcta

            \choice $(C \cap B) \cup (A - \{o,u\}) = B$ % Correcta

            \choice $A \cap B \cap C \neq \varnothing$

            \choice $A \cup C = \mathcal{U}$ % Correcta
            
            \choice Ninguna de las anteriores. 
        \end{checkboxes}

        % Question 07
        \question{¿Cuáles de las siguientes expresiones son \textbf{verdaderas}?}
        \begin{checkboxes}
            \choice $\{x \in \mathbb{Z} \; | \; |x| < 4 \} = \{0,1,2,3\}$

            \choice $\{x \; | \; x \text{ es un estado de la República 
            Mexicana\}} \neq \{\text{CDMX, Monterrey, \ldots, Oaxaca}\}$
            % Correcta

            \choice $\{12,2,6,24,8,3,1,4\} = \{x \; | \; x \text{ divide a } 
            24\}$ % Correcta

            \choice $\{1,3,5,7,9,11,\ldots\} = \{x \in \mathbb{Z}\; | \; 
            x \text{ es un número impar}\}$
            
            \choice Ninguna de las anteriores. 
        \end{checkboxes}

        % Question 08
        \question
        {
            Sean los conjuntos
            \begin{align*}
                S &= \{x \in \mathbb{N} \; | \; 1 < x < 50\} \\ 
                T &= \{x \in \mathbb{Z} \; | \; |x| \geq 25\} 
            \end{align*}

            ¿Cuáles de las siguientes expresiones son \textbf{verdaderas}?
        }
        \begin{checkboxes}
            \choice $\{0,1,2\} \subseteq S$

            \choice $T \subseteq S$

            \choice $-40 \in T$ % Correcta 

            \choice $S \subset T$
            
            \choice Ninguna de las anteriores. 
        \end{checkboxes}

        % Question 09
        \question{¿Cuáles de las siguientes expresiones son \textbf{verdaderas}?}
        \begin{checkboxes}
            \choice $\{x \in \mathbb{N} \; | \; x \text{ es un número par y } 
            2 < x < 11\} \neq \{10,4,8,6\}$

            \choice La cardinalidad del conjunto $\{\varnothing\}$ es $0$. 

            \choice $\{0,1,10,11,100,101,110,111,1000,\ldots\} \neq 
            \{x \; | \; x \text{ es un número natural en su forma binaria}\}$

            \choice $\{x \; | \; x \text{ es el nombre del último presidente de
            México}\} =  \text{Andrés Manuel López Obrador}$
            
            \choice Ninguna de las anteriores. % Correcta
        \end{checkboxes}

        % Question 10
        \question{¿Cuáles de las siguientes expresiones son \textbf{verdaderas}?}
        \begin{checkboxes}
            \choice Si $S$ es un conjunto cualquiera, entonces $S \cap S \neq S$

            \choice $\varnothing \cap \{\varnothing\} = \varnothing$ % Correcta

            \choice $|\{\varnothing, \{\varnothing, \{\varnothing\}\}, 
            \{\varnothing, \{\varnothing, \{\varnothing\}\}\}\}| = 3$ % Correcta

            \choice Si $S = \{0,1\}$, entonces $\mathcal{P}(S) = \{\varnothing, 
            \{0\}, \{1\}, \{0,1\}\}$. % Correcta
            
            \choice Ninguna de las anteriores. 
        \end{checkboxes}

        % Question 11
        \question{¿Cuáles de las siguientes expresiones son \textbf{verdaderas}?}
        \begin{checkboxes}
            \choice Sean $A$ y $S$ dos conjuntos cualesquiera tal que $A \subseteq S$. 
            Entonces $A \cup S = S$ % Correcta

            \choice Bajo ninguna circunstancia puede existir un conjunto de 
            subconjuntos.

            \choice Un conjunto cuya longitud es infinita no puede ser representado 
            por extensión.

            \choice Si $S$ es un conjunto cualquiera, entonces $S \cup S \neq S$
            
            \choice Ninguna de las anteriores. 
        \end{checkboxes}

        % Question 12
        \question{¿Cuáles de las siguientes expresiones son \textbf{verdaderas}?}
        \begin{checkboxes}
            \choice Para cualquier conjunto $S$, $\mathcal{P}(S)$ siempre tendrá 
            al menos a los elementos $\varnothing$ y $S$ ya que $\varnothing 
            \subseteq S$ y $S \subseteq S$ siempre se cumplen. % Correcta

            \choice Sean $A$ y $B$ dos conjuntos cualesquiera. Se cumple que 
            $A = B \Leftrightarrow A \subset B$ y $B \subset A$.

            \choice Sea $S$ un conjunto cualquiera. Si $|S| = n$, entonces 
            $S$ tiene $n$ subconjuntos propios.

            \choice Sean $A$ y $B$ dos conjuntos cualesquiera. Se cumple que 
            $A \cap B \subseteq A$ y $A \cap B \subseteq B$. % Correcta
            
            \choice Ninguna de las anteriores. 
        \end{checkboxes}

        % Question 13
        \question
        {
            Sean los conjuntos
            \begin{align*}
                R &= \{a,c, \pi, \bullet, m, n\} \\ 
                S &= \{\{a\}, c, m, n\} \\ 
                T &= \{a,c, \pi\}
            \end{align*}
            
            ¿Cuáles de las siguientes expresiones son \textbf{verdaderas}?
        }
        \begin{checkboxes}
            \choice $T \subseteq R$ % Correcta

            \choice $S \subseteq \{m,a,n,c\}$

            \choice $T \not \in R$ % Correcta

            \choice $\{a\} \in S$ % Correcta
            
            \choice Ninguna de las anteriores. 
        \end{checkboxes}

        % Question 14
        \question
        {
            Sean los conjuntos
            \begin{align*}
                S &= \{p,q,r,s\} \\ 
                T &= \{r,t,v\} \\ 
                U &= \{p,s,t,u\}
            \end{align*}
            
            ¿Cuáles de las siguientes expresiones son \textbf{verdaderas}?
        }
        \begin{checkboxes}
            \choice $T-U = \{r,u\}$

            \choice $\mathcal{P}(S \cap U) = \{(S - (T \cup \{q,s\}))\}$

            \choice $S \cap T \cap U = \{r\}$

            \choice $U-T = \{p,s,u\}$ % Correcta
            
            \choice Ninguna de las anteriores. 
        \end{checkboxes}

        % Question 15
        \question{Sea $S = \{\bigstar, \bullet, a,1\}$. Sean $S_1$ el 
        conjunto de todos los subconjuntos de $S$ que no tienen como elemento a 
        $1$ y $S_2$ el conjunto de todos los subconjuntos de $S$ que tienen como 
        elemento a $1$. ¿Cuáles de las siguientes expresiones son 
        \textbf{verdaderas}?}
        \begin{checkboxes}
            \choice $S_1 = \{\varnothing, \{\bigstar\}, \{\bullet\}, \{a\}, \{\bigstar, 
            \bullet\}, \{\bigstar, a\}, \{\bullet, a\}, \{\bigstar, \bullet, 
            a\}\}$ % Correcta

            \choice $|S_1| = |S_2|$ % Correcta

            \choice $|S_1 \cup S_2| = 15$

            \choice $S_1$ y $S_2$ son disjuntos. % Correcta
            
            \choice Ninguna de las anteriores. 
        \end{checkboxes}

        % Question 16
        \question{¿Cuáles de las siguientes expresiones son \textbf{verdaderas}?}
        \begin{checkboxes}
            \choice $\mathcal{P}(\varnothing) = \{\varnothing\}$ % Correcta

            \choice Para todo conjunto $S$ con $n$ elementos, $\mathcal{P}(S)$ 
            siempre tendrá $n^2$ elementos.

            \choice Bajo ninguna circunstancia puede existir un conjunto de 
            conjuntos

            \choice Un conjunto cuya longitud es infinita no puede ser representado 
            por comprensión. 
            
            \choice Ninguna de las anteriores. 
        \end{checkboxes}

        % Question 17 
        \question
        {
            Sean los conjuntos
            \begin{align*}
                R &= \{a,c, \pi, \bullet, m, n\} \\ 
                S &= \{a,c, \pi\} \\ 
                T &= \{\{a,c,\pi\}, a\} 
            \end{align*}
            
            ¿Cuáles de las siguientes expresiones son \textbf{verdaderas}?
        }
        \begin{checkboxes}
            \choice $a \subseteq T$

            \choice $S \subseteq R$ % Correcta

            \choice $\{a\} \subseteq S$ % Correcta

            \choice $\{\{a\}\} \subseteq T$
            
            \choice Ninguna de las anteriores. 
        \end{checkboxes}

        % Question 18
        \question
        {
            Sean los conjuntos
            \begin{align*}
                S &= \{\{a\}, c, m, n\} \\ 
                T &= \{a,c, \pi\} \\
                U &= \{\{a,c,\pi\}, a\}
            \end{align*}
            
            ¿Cuáles de las siguientes expresiones son \textbf{verdaderas}?
        }
        \begin{checkboxes}
            \choice $T \subseteq U$

            \choice $T \in U$ % Correcta

            \choice $\varnothing \subseteq S$ % Correcta

            \choice $T \subset S$
            
            \choice Ninguna de las anteriores. 
        \end{checkboxes}

        % Question 19
        \question{Sea $S = \{\bigstar, \bullet, a,1\}$. Sean $S_1$ el 
        conjunto de todos los subconjuntos de $S$ que no tienen como elemento a 
        $1$ y $S_2$ el conjunto de todos los subconjuntos de $S$ que tienen como 
        elemento a $1$. ¿Cuáles de las siguientes expresiones son 
        \textbf{verdaderas}?}
        \begin{checkboxes}
            \choice $S_1 - S_2 = \{\bigstar, \bullet, a\}$

            \choice $S_1 \cup S_2 = \mathcal{P}(S)$ % Correcta

            \choice $S_2 = \{\{1\}, \{\bigstar, 1\}, \{\bullet, 1\}, \{a,1\}, \{\bigstar, 
            \bullet, 1\}, \{\bigstar, a, 1\}, \{\bullet, a, 1\}, \{\bigstar, 
            \bullet, a,1\}\}$ % Correcta

            \choice $|S_2 - S_1| = 1$
            
            \choice Ninguna de las anteriores. 
        \end{checkboxes}

        % Question 20
        \question{¿Cuáles de las siguientes expresiones son \textbf{verdaderas}?}
        \begin{checkboxes}
            \choice Si $S = \{\{\varnothing\}, \{\{\varnothing\}\}\}$, 
            entonces $\{\{\{\varnothing\}\}\} \subset S$. % Correcta

            \choice $\varnothing \subset \varnothing$

            \choice Si $S = \{\{\varnothing\}, \{\{\varnothing\}\}\}$, 
            entonces $\{\varnothing\} \not \in S$

            \choice $\varnothing \in \{\varnothing\}$ % Correcta
            
            \choice Ninguna de las anteriores. 
        \end{checkboxes}

        % Question 21
        \question
        {
            Sean los conjuntos
            \begin{align*}
                S &= \{p,q,r,s\} \\ 
                T &= \{r,t,v\} \\ 
                U &= \{p,s,t,u\}
            \end{align*}
            
            tal que $\mathcal{U} = \{p,q,r,s,t,u,v,w\}$. ¿Cuáles de las 
            siguientes expresiones son \textbf{verdaderas}?
        }
        \begin{checkboxes}
            \choice $(S \cup T)^c = \{u,v,w\}$

            \choice $(S \cup T) \cap U^c = \{r,s,v\}$ 

            \choice $S^c \cup T^c = S$

            \choice $(S-U)^c = \{p,s,u,v,w\}$
            
            \choice Ninguna de las anteriores. % Correcta
        \end{checkboxes}

        % Question 22
        \question
        {
            Sean los conjuntos
            \begin{align*}
                S &= \{\bigstar, \{\bigstar\}, \{\{\bigstar\}\}\} \\ 
                T &= \{\varnothing, \{\bigstar\}, \{\bigstar, \{\bigstar\}\}\} \\ 
                U &= \{\bigstar\}
            \end{align*}
            
            tal que $\mathcal{U} = \{\varnothing, \bigstar, \{\bigstar\}, 
            \{\{\bigstar\}\}, \{\bigstar, \{\bigstar\}\}\}$. ¿Cuáles de las siguientes 
            expresiones son \textbf{verdaderas}?
        }
        \begin{checkboxes}
            \choice $T \cap U^c = (S \cup T \cup U)^c$ 

            \choice $(T \cup U) \cap S = \{\{\{\bigstar\}\},\{\bigstar\}\}$

            \choice $\varnothing \cup T = \{\varnothing\}$

            \choice $S \cup T = \varnothing^c$ % Correcta
            
            \choice Ninguna de las anteriores.
        \end{checkboxes}
    \end{questions}
\end{document}