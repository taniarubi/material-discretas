\documentclass[oneside]{style}

\title{Versión 02}
\principal{Examen 11}
\author{Tania Michelle Rubí Rojas}
\semester{Semestre 2023-1}

\begin{document}
\maketitle

\vspace{5mm}
\noindent
Nombre y número de cuenta: \hrulefill\

\vspace*{5mm}
Para cada uno de los siguientes ejercicios, \textbf{justifica ampliamente} tu 
respuesta:

\begin{questions}[label=\protect\circled{\bfseries\arabic*}]

    % Ejercicio 01
    \question
    {
        El micromundo de figuras tiene los siguientes predicados:
        \begin{tasks}(3)
            \task $T(x): x$ es un triángulo. 
            \task $C(x): x$ es un cuadrado.
            \task $S(x): x$ es un círculo.
            \task $P(x): x$ es pequeño.
            \task $M(x): x$ es mediano.
            \task $G(x): x$ es grande. 
            \task $Su(x,y): x$ está al sur de $y$.
            \task $N(x,y): x$ está al norte de $y$.
            \task $E(x,y): x$ está al este de $y$.
            \task $O(x,y): x$ está al oeste de $y$.
            \task $Co(x,y): x$ está en la misma columna que $y$.
            \task $R(x,y): x$ está en el mismo renglón que $y$.
        \end{tasks}

        Para cada uno de los siguientes enunciados, da un 
        micromundo no vacío donde el enunciado sea verdadero y 
        uno donde sea falso. 
        \begin{itemize}
            \item $\neg \forall x (C(x) \rightarrow G(x)) \land \exists 
            z (P(z) \land \neg \exists y (T(y) \land O(y,z)))$

            \item $\forall x (S(x) \rightarrow \neg G(x)) \land 
            \forall x (C(x) \rightarrow \exists y (Co(x,y) \land 
            T(x) \land P(x)))$

            \item $\neg (\forall x \exists y (R(x,y) \rightarrow 
            (M(x) \lor G(y))))$

            \item $\exists x (C(x) \land \forall y (N(y,x) \rightarrow 
            P(y) \lor S(y))) \land \forall w (C(w) \rightarrow \exists 
            y G(y) \land E(y,w))$
        \end{itemize}
    }    
    
    % Ejercicio 02
    \question
    {
        Para cada una de las siguientes fórmulas:
        \begin{itemize}
            \item $\forall x (L(x) \rightarrow R(f(x,a), x) \land 
            C(f(x,a), a))$
            \item $\neg \exists x \exists y (T(x) \land T(y) \land Z(x,y)) 
            \land \forall z (T(z) \land G(z) \rightarrow \exists w (C(w) 
            \land P(w) \land E(w,z)))$
            \item $C(f(x,y),z) \land \exists y C(f(y,x),z) \rightarrow 
            \forall x (L(x) \land L(y) \land I(x,y))$
            \item $\forall x (S(x) \land P(x) \rightarrow \forall y 
            (T(y) \rightarrow C(x,y)))$
        \end{itemize}

        \textbf{realiza} lo siguiente:
        \begin{itemize}
            \item \textbf{Indica} con diferentes colores el alcance de cada uno 
            de los cuantificadores  
            \item \textbf{Subraya} las variables ligadas y libres con diferentes 
            colores. 
            \item \textbf{Indica} cuáles fórmulas son enunciados y explica
            por qué. 
        \end{itemize}

        \textbf{Extra:} ¿Una constante es una variable libre? 
    }

    % Ejercicio 03
    \question
    {
        Considera los siguientes predicados:
        \begin{itemize}
            \item $A(x,y):$ el alumno $x$ ha aprobado la materia $y$. 
            \item $R(x):$ $x$ es un alumno regular. 
            \item $F(x)$: $x$ es un alumno feliz. 
        \end{itemize}

        y la constante $ed: $ Estructuras Discretas, cuyo universo del 
        discurso es el conjunto de alumnos de la Facultad de Ciencias unido 
        con el conjunto de todas las materias que se imparten en dicha facultad.
        
        Dado lo anterior, \textbf{determina} el valor de verdad de las 
        siguientes fórmulas:
        \begin{tasks}(3)
            \task $\forall x (F(x) \rightarrow \exists y A(x,y))$
            \task $\exists x (F(x) \land \forall y (A(x,ed) \rightarrow R(x)))$
            \task $\forall x (A(x,ed) \rightarrow F(x))$
            \task $\forall x \neg \forall y (A(x,y) \rightarrow R(x))$
            \task $\exists x(R(x) \land \exists y (\neg A(x,y) \land F(x)))$
            \task $\exists x (F(x) \land \exists y (R(y) \land A(x,y)))$
        \end{tasks}
    }

    % Ejercicio 04
    \question
    {
        \textbf{Determina} si es posible capturar el sentido del siguiente 
        argumento usando únicamente lógica proposicional. En caso de que sea 
        posible, da su respectiva traducción a lógica proposicional:
        \begin{verbatim}
        Nubecita es más alta que Cuchis. Canadá está al norte de EE.UU. Las 
        personas que son más altas que Cuchis viven al norte de los EE.UU. 
        Por lo tanto, Nubecita vive en Canadá. 
        \end{verbatim}
    }

\end{questions}
\end{document}
