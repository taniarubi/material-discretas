\documentclass[12pt, a4paper]{exam}

% Soporte para cambiar la fecha que sale en el examen
\usepackage{advdate}
% Soporte para escribir en varias columnas
\usepackage{multicol}
% Soporte para los acentos.
\usepackage[utf8]{inputenc} 
\usepackage[T1]{fontenc}    
% Idioma español.
\usepackage[spanish,mexico,es-tabla]{babel}
\usepackage{graphicx}
\usepackage{tikz}
\usepackage{amsmath,amssymb,amsthm}
\usepackage{verbatim} % comentarios
\usepackage{tasks}
\usepackage{color}
% Cambiamos los márgenes del documento. 
\usepackage[top=1.5cm,left=1.5cm,right=1.5cm]{geometry}

% Pie de página
\cfoot{Página \thepage\ de \numpages}

%%%%%%%%%%%%%%%%%%%%%%%%%%%%%%%%%%%%%%%%%%%%%%%%%%%%%%%%%%%%%%%%%%%%%%%%%%%%%%
\renewcommand{\thechoice}{\alph{choice}}

\makeatletter
\renewenvironment{checkboxes}%
   {\setcounter{choice}{0}\list{\checkbox@char}%
      {%
        \settowidth{\leftmargin}{W.\hskip\labelsep\hskip 2.5em}%
        \def\choice{%
          \if@correctchoice
            \color@endgroup \endgroup
          \fi
          \stepcounter{choice}
          \item[\checked@char]
          \do@choice@pageinfo
        } % choice
        \def\CorrectChoice{%
          \if@correctchoice
            \color@endgroup \endgroup
          \fi
          \ifprintanswers
            % We can't say \choice here, because that would
            % insert an \endgroup.
            % 2016/05/10: We say \color@begingroup in addition to
            % \begingroup in case \CorrectChoiceEmphasis involves color
            % and the text exactly fills the line (which would
            % otherwise create a blank line after this choice):
            % 2016/05/11: We leave hmode if we're in it,
            % i.e., if there's no blank line preceding this
            % \CorrectChoice command.  (Without this, the
            % \special created by a \color{whatever} command that might
            % be inserted by \CorrectChoice@Emphasis would be appended 
            % to the previous \choice, which could cause an extra
            % (blank) line to be inserted before this \CorrectChoice.)
            % Since \par and \endgraf seem to cancel \@totalleftmargin
            % (for reasons I don't understand), we'll do the following:
            % Motivated by  the def of \leavevmode, 
            %      \def\leavevmode{\unhbox\voidb@x}
            % we will now leave hmode (if we're in hmode):
            \ifhmode \unskip\unskip\unvbox\voidb@x \fi
            \begingroup \color@begingroup \@correctchoicetrue
            \CorrectChoice@Emphasis
            \stepcounter{choice}
            \item[\checked@char]
          \else
            \stepcounter{choice}
            \item[\checked@char]
          \fi
          \do@choice@pageinfo
        } % CorrectChoice
        \let\correctchoice\CorrectChoice
        \labelwidth\leftmargin\advance\labelwidth-\labelsep
        \topsep=0pt
        \partopsep=0pt
        \checkboxeshook
      }%
   }%
   {\if@correctchoice \color@endgroup \endgroup \fi \endlist}
 \makeatother

% Make checkbox character a circle with the letter
\checkboxchar{\tikz[baseline={([yshift=-.8ex]current bounding box.center)}]\node[shape=circle,minimum size=4mm,draw] at (0,0) {\thechoice};}
% Make checked box character bold WITH surd
%\checkedchar{\tikz[baseline={([yshift=-.8ex]current bounding box.center)}]\node[shape=circle,minimum size=8mm,draw] at (0,0) {} node at (0,0) {\thechoice\llap{$\surd$}};}
% Make checked box character bold
\checkedchar{\tikz[baseline={([yshift=-.8ex]current bounding box.center)}]\node[shape=circle,minimum size=4mm,draw] at (0,0) {} node at (0,0) {\thechoice};}
\printanswers
%%%%%%%%%%%%%%%%%%%%%%%%%%%%%%%%%%%%%%%%%%%%%%%%%%%%%%%%%%%%%%%%%%%%%%%%%%%%%%

\begin{document}
    %%%%%%%%%%%%%%%%%%%%%%%%%%%%%%%%%%%%%%%%%%%%%%%%%%%%%%%%%%%%%%%%%%%%%%%%%%%%%%%
    %%%%%%%%%%%%%%%%%%%%%%%%%%%%%%%% ENCABEZADO %%%%%%%%%%%%%%%%%%%%%%%%%%%%%%%%%%%
    \centering
    \hrule \hrule \hrule 
    \vspace{5mm}
    \begin{minipage}[c]{0.8\textwidth}
        \begin{center}
            {\large\textbf{Mission 11, Start!} \par
            \large \textbf{Estructuras Discretas} \par
            \large \textbf{Semestre 2023-1} \par
            \large \textbf{\today}	\par}
        \end{center}
    \end{minipage}

    \vspace{0.2in}
    \noindent
    \textbf{Tania Michelle Rubí Rojas}
    \vspace{2mm}
    \hrule \hrule \hrule 
    %%%%%%%%%%%%%%%%%%%%%%%%%%%%%%%%%%%%%%%%%%%%%%%%%%%%%%%%%%%%%%%%%%%%%%%%%%%%%%%
    %%%%%%%%%%%%%%%%%%%%%%%%%%%%%%%%%%%%%%%%%%%%%%%%%%%%%%%%%%%%%%%%%%%%%%%%%%%%%%%

    \vspace{5mm}
    \noindent
    Nombre y número de cuenta: \hrulefill\

    \begin{questions}
        % Question 01
        \question
        {
            ¿Para cuál o cuáles de los siguientes argumentos es posible 
            capturar su sentido usando solo lógica proposicional?
        }
        \begin{checkboxes}
            \choice Alberto es más alto que Luis. Canadá está al norte de 
            EE.UU. Las personas que son más altas que Luis viven al norte 
            de los EE.UU. Por lo tanto, Alberto vive en Canadá. 

            \choice Escuché el concierto de Bad Bunny y el concierto de Ricky 
            Martin. Tengo los tímpanos inflamados. Por lo tanto, escuché el 
            concierto de Bad Bunny y tengo los tímpanos inflamados, o bien, 
            escuché el concierto de Ricky Martin y no tengo los tímpanos 
            inflamados. % Correcta 

            \choice Es ilegal que cualquier persona tenga más de tres perritos 
            y tres gatitos en su propiedad en el pueblito ¨Sin Nombre¨. Yo 
            vivo en el pueblito ¨Sin Nombre¨ y tengo cuatro perritos pero 
            ningún gato. Por lo tanto, cumplo con alguna ley del pueblito 
            ¨Sin Nombre¨.

            \choice Una computadora puede correr como ¨Flash¨, pero antes 
            tiene que tener 32 megabytes de almacenamiento. Si la computadora 
            puede correr como ¨Flash¨, entonces el sonido al reproducir la 
            canción “Culpable o no” será impresionante. Por lo tanto, si 
            la computadora tiene 32 megabytes de almacenamiento, entonces 
            el sonido al reproducir la canción “Culpable o no” será 
            impresionante. % Correcta 

            \choice Ninguna de las anteriores.
        \end{checkboxes}

        % Question 02
        \question
        {
            Sea el conjunto de los alumnos de la Facultad de Ciencias unido 
            con el conjunto de todas las materias que se imparten en la 
            Facultad de Ciencias el universo de discurso. Si consideramos los 
            siguientes predicados y constantes:
            \begin{tasks}(2)
                \task $A(x,y):$ el alumno $x$ ha aprobado la materia $y$
                \task $R(x): x$ es un alumno regular 
                \task $F(x): x$ es un alumno feliz 
                \task la constante $ed$ para Estructuras Discretas 
            \end{tasks}

            ¿Cuál o cuáles de las siguientes expresiones son 
            \textbf{verdaderas}?
        }
        \begin{checkboxes}
            \choice $\forall x (A(x,ed) \rightarrow F(x))$ es 
            verdadero. 

            \choice $\exists x (F(x) \land \forall y (A(x,ed) \rightarrow R(x)))$
            es falso. 

            \choice $\exists x (F(x) \land \exists y (R(y) \land A(x,y)))$ es 
            verdadero. 

            \choice $\exists x (R(x) \land \exists y (\neg A(x,y) \land F(x)))$ es 
            verdadero. % Correcta 

            \choice Ninguna de las anteriores.
        \end{checkboxes}

        \newpage
        % Question 03
        \question{¿Cuál o cuáles de las siguientes fórmulas son enunciados?}
        \begin{checkboxes}
            \choice $\forall x \forall y ((P(x) \land P(y) \land E(x,y) \land 
            E(y,x)) \rightarrow I(x,y))$ % Correcta 

            \choice $\forall x (P(x) \rightarrow M(x)) \land \exists y 
            (P(y) \lor M(y) \lor E(x,y))$

            \choice $\forall x (M(x) \land D(0,x))$ % Correcta

            \choice $\exists y (M(x) \land \neg P(x))$

            \choice Ninguna de las anteriores.
        \end{checkboxes}

        % Question 04
        \question
        {
            Sea el conjunto de los programadores el universo de discurso. Si 
            consideramos los siguientes predicados:
            \begin{tasks}(2)
                \task $P(x): x$ es un programador profesional 
                \task $J(x): x$ programa en Java 
            \end{tasks}

            ¿Cuál o cuáles de las siguientes expresiones son 
            \textbf{verdaderas}?
        }
        \begin{checkboxes}
            \choice $\forall x (P(x) \rightarrow J(x))$ es falso. % Correcta 

            \choice $\forall y (J(y) \rightarrow P(y))$ es verdadero. 

            \choice $\exists x (P(x) \lor J(x))$ es falso. 

            \choice $\exists x (P(x) \land J(x))$ es verdadero. % Correcta 

            \choice Ninguna de las anteriores.
        \end{checkboxes}

        % Question 05
        \question
        {
            Sea el conjunto de todas las personas el universo de discurso. Si 
            consideramos los siguientes predicados:
            \begin{tasks}(2)
                \task $S(x): x$ es una persona soltera 
                \task $A(x,y): y$ tiene más pretendientes que $x$ 
            \end{tasks}

            ¿Cuál o cuáles de las siguientes expresiones son 
            \textbf{verdaderas}?
        }
        \begin{checkboxes}
            \choice $\exists x \forall y (A(x,y) \rightarrow S(x))$ es 
            verdadero. % Correcta

            \choice $\exists x \exists y (S(x) \land S(y) \land A(x,y) 
            \land x \neq y)$ es verdadero. % Correcta

            \choice $\exists x \exists y (A(x,y) \rightarrow S(x))$ es 
            verdadero. % Correcta 

            \choice $\forall x \forall y (S(x) \rightarrow A(x,y))$ es 
            verdadero.  

            \choice Ninguna de las anteriores.
        \end{checkboxes}
        }

        % Question 06
        \question
        {
            ¿Para cuál o cuáles de los siguientes argumentos es posible 
            capturar su sentido usando solo lógica proposicional?
        }
        \begin{checkboxes}
            \choice Cuando había un juego de pelota, transportarse era 
            complicado. Si llegaban a tiempo, entonces no era complicado 
            transportarse. Llegaron a tiempo. Por lo tanto, no había un 
            juego de pelota. % Correcta 

            \choice Todos los perritos que conozco son adorables. El perro 
            del vecino no es adorable. Conozco al perro del vecino. Por lo 
            tanto, hay un perro cuyo pelaje es de color blanco.

            \choice Si Alejandro trabaja muy duro, entonces Tania o Luis se 
            divertirán. Si Tania se divierte, entonces Alejandro no trabajará 
            duro. Si Adriana se divierte, entonces Luis no. Por lo tanto, si 
            Alejandro trabaja duro, Adriana no se divertirá. % Correcta

            \choice Él está sentado entre Rafael y Tatiana. Sergio y Lucía 
            juegan billar contra Carlos y Óscar. Rafael es un experto en 
            matemáticas. Por lo tanto, si Rafael y Carlos juegan billar contra 
            él y Óscar, entonces el equipo de Rafael ganará.

            \choice Ninguna de las anteriores.
        \end{checkboxes}

        \newpage
        % Question 07
        \question{¿Cuál o cuáles de las siguientes expresiones son 
        \textbf{verdaderas}?}
        \begin{checkboxes}
            \choice Es posible definir recursivamente el conjunto de todas 
            las expresiones de Lógica de Primer Orden, pero dicha definición 
            es tan complicada que queda fuera del alcance de este curso. 

            \choice $P(x,y,z)$ es un predicado, pero no un enunciado. 
            % Correcta

            \choice Si $\forall x P(x)$ es falso y el universo de discurso 
            es diferente del vacío, entonces es posible encontrar 
            una interpretación que haga que $\exists x \neg P(x)$ sea verdadero. 
            % Correcta 

            \choice Si $a,b,c$ son constantes dentro de un universo de discurso 
            diferente del vacío, entonces $P(a,b,c)$ es un enunciado. % Correcta

            \choice Ninguna de las anteriores.
        \end{checkboxes}

        % Question 08
        \question
        {
            Sea $\mathbb{Z}$ el universo de discurso. Si consideramos los 
            siguientes predicados:
            \begin{tasks}(3)
                \task $I(x): x$ es impar 
                \task $L(x): x < 10$ 
                \task $G(x): x > 9$ 
            \end{tasks}

            ¿Cuál o cuáles de las siguientes expresiones son 
            \textbf{verdaderas}?
        }
        \begin{checkboxes}
            \choice $\exists x (I(x) \land L(x))$ es falso. 

            \choice $\forall x (I(x) \rightarrow G(x))$ es verdadero. 

            \choice $\forall x (L(x) \land G(x))$ es falso. % Correcta

            \choice $\exists x (L(x) \land G(x))$ es falso. % Correcta 

            \choice Ninguna de las anteriores.
        \end{checkboxes}

        % Question 09
        \question
        {
            Si $\forall x \exists y P(x,y)$ es verdadero y el universo de 
            discurso es diferente del vacío, ¿cuál o cuáles de las 
            siguientes expresiones son \textbf{verdaderas}?
        }
        \begin{checkboxes}
            \choice $\forall x \forall y P(x,y)$ es verdadero.

            \choice $\exists x \exists y P(x,y)$ es verdadero. % Correcta 

            \choice $\exists x \exists y P(y,x)$ es falso. 

            \choice $\exists y \forall x P(x,y)$ es verdadero. % Correcta

            \choice Ninguna de las anteriores.
        \end{checkboxes}
        
        % Question 10
        \question
        {
            Sea el conjunto de todas las personas el universo de discurso. 
            Si consideramos los siguientes predicados:
            \begin{tasks}(2)
                \task $M(x,y): x$ es madre de $y$ 
                \task $F(x): x$ es una mujer 
                \task $H(x): x$ es un hombre  
            \end{tasks}

            ¿Cuál o cuáles de las siguientes expresiones son 
            \textbf{verdaderas}?
        }
        \begin{checkboxes}
            \choice $\forall x \exists y F(y,x)$ es falso. % Correcta

            \choice $\exists x \exists y (M(x,y) \land F(y) \land x \neq y)$ 
            es verdadero. % Correcta 

            \choice $\forall x \forall y (M(x,y) \rightarrow H(x))$ es 
            verdadero. 

            \choice $\exists x \exists y (M(x,y) \land F(y))$ es falso. 

            \choice Ninguna de las anteriores.
        \end{checkboxes}

        % Question 11
        \question{¿Cuál o cuáles de las siguientes expresiones son 
        \textbf{verdaderas}?}
        \begin{checkboxes}
            \choice La fórmula
            \begin{equation*}
                \forall x \exists y (P(x, g(w,y)) \land R(a,v,f(w))) 
                \rightarrow \neg Q(a,z) \lor \forall z T(z,y,x,a)
            \end{equation*}

            tiene como variables ligadas a $\{x,y,z\}$ y tiene como variables 
            libres a $\{w,a,v\}$.

            \choice La fórmula
            \begin{equation*}
                \forall w T(w,x,g(y)) \rightarrow \neg \exists z R(x,f(w,y))
            \end{equation*}

            tiene como variables ligadas a $\{w,z\}$ y tiene como variables 
            libres a $\{x, g(y), f(w,y)\}$. 

            \choice La fórmula
            \begin{equation*}
                C(f(x,y), z) \land \exists y C(f(y,x), z) \rightarrow 
                \forall x (L(x) \land L(y) \land I(x,y))
            \end{equation*}

            tiene como variables ligadas a $\{x,y\}$ y tiene como variables 
            libres a $\{x,y,z\}$. % Correcta  

            \choice La fórmula
            \begin{equation*}
                \forall x (L(x) \rightarrow R(f(x,y), x) \land C(f(x,y), y))
            \end{equation*}

            tiene como variables ligadas a $\{x\}$ y tiene como variables 
            libres a $\{x,y\}$.

            \choice Ninguna de las anteriores.
        \end{checkboxes}

        % Question 12
        \question
        {
            Si $\exists x \exists y P(x,y)$ es verdadero y el universo de 
            discurso es diferente del vacío, ¿cuál o cuáles de las 
            siguientes expresiones son \textbf{verdaderas}?
        }
        \begin{checkboxes}
            \choice $\forall x \exists y P(x,y)$ es verdadero. 

            \choice $\exists y \exists x P(x,y)$ es falso. 

            \choice $\exists x \exists y P(y,x)$ es verdadero. % Correcta 

            \choice $\forall x \forall y P(x,y)$ es verdadero.

            \choice Ninguna de las anteriores.
        \end{checkboxes}

        % Question 13
        \question{¿Cuál o cuáles de las siguientes fórmulas son enunciados?}
        \begin{checkboxes}
            \choice $\forall x (P(x) \Leftrightarrow Q(x) \land \exists x 
            R(x)) \land S(x)$

            \choice $(\forall x (P(x) \land \exists y Q(y))) \lor (\forall z 
            P(z) \rightarrow Q(z))$

            \choice $\forall x (P(x) \land R(x)) \rightarrow \forall y (P(y)
            \land Q(y))$ % Correcta 

            \choice $\forall x (P(x) \rightarrow \forall y (P(y) \land L(y) 
            \rightarrow R(x,y)))$ % Correcta 

            \choice Ninguna de las anteriores.
        \end{checkboxes}

        % Question 14
        \question
        {
            Si $\forall x \forall y P(x,y)$ es verdadero y el universo de 
            discurso es diferente del vacío, ¿cuál o cuáles de las 
            siguientes expresiones son \textbf{verdaderas}?
        }
        \begin{checkboxes}
            \choice $\forall y \forall x P(x,y)$ es falso.  

            \choice $\exists x \forall y P(x,y)$ es verdadero. % Correcta

            \choice $\forall x \forall y P(y,x)$ es verdadera. % Correcta 

            \choice $\forall x \exists y P(y,x)$ es falso. 

            \choice Ninguna de las anteriores.
        \end{checkboxes}

        % Question 15
        \question{¿Cuál o cuáles de las siguientes expresiones son 
        \textbf{verdaderas}?}
        \begin{checkboxes}
            \choice Es falso que todas las ocurrencias de la letra $u$ en 
            \textit{¨Matemáticas Discretas¨} estén en minúsculas. 

            \choice Si el universo de discurso es el conjunto de todas las 
            flores y tenemos el predicado
            \begin{equation*}
                P(x): x \text{ es amarillo}
            \end{equation*}
            
            entonces $\forall x P(x)$ es falso. % Correcta

            \choice Si Nubecita dice \textit{¨Todos los unos que aparecen 
            en el número 8502 están a la izquierda de todos los ceros de 
            dicho número¨}, podemos afirmar que dicha oración es verdadera. 
            % Correcta 

            \choice Si el universo de discurso es $\mathbb{Z}$ y tenemos los 
            predicados 
            \begin{align*}
                P(x)&: \; x \text{ es par} \\ 
                I(x)&: \; x \text{ es impar}
            \end{align*}

            entonces $\neg \exists x (\neg P(x) \land \neg I(x))$ es verdadero.
            % Correcta 

            \choice Ninguna de las anteriores.
        \end{checkboxes}

        % Question 16
        \question{¿Cuál o cuáles de las siguientes fórmulas son enunciados?}
        \begin{checkboxes}
            \choice $\forall x P(x) \rightarrow \exists y Q(y)$ % Correcta

            \choice  $\forall y \forall x Q(x) \rightarrow P(y)$

            \choice $\exists x \exists y (A(x,y) \land B(y,z) 
            \rightarrow A(a,z))$

            \choice $\exists x (P(f(x)) \land \exists y (Q(x, 
            g(x,y)) \land R(x,x,y)))$ % Correcta 

            \choice Ninguna de las anteriores.
        \end{checkboxes}

        % Question 17
        \question{¿Cuál o cuáles de las siguientes expresiones son 
        \textbf{verdaderas}?}
        \begin{checkboxes}
            \choice Cuando dos variables están designadas por el mismo símbolo 
            (misma letra) entonces son la misma variable si están bajo el 
            alcance del mismo cuantificador o si las dos son libres. % Correcta 

            \choice Si el universo de discurso es $\mathbb{Z}$ y tenemos los 
            predicados 
            \begin{align*}
                A(x) &: \; x > 0 \\ 
                B(x,y) &: \; x > y \\ 
                C(y) &: y \leq 0
            \end{align*}

            entonces $\exists x (A(x) \land \forall y (B(x,y) \rightarrow 
            C(y)))$ es verdadero. % Correcta 

            \choice Dada la siguiente fórmula
            \begin{equation*}
                \forall x \exists y (S(x,y) \land L(y,a))
            \end{equation*}

            entonces el alcance del cuantificador $\exists y$ es $(S(x,y) 
            \land L(y,a))$, mientras que el alcance del cuantificador 
            $\forall x$ es $\exists y (S(x,y) \land L(y,a))$. % Correcta 

            \choice Si el universo de discurso es diferente del vacío, entonces 
            es posible encontrar una interpretación en la que $\forall x P(x)$
            sea falsa y $\exists x P(x)$ sea verdadera. % Correcta  

            \choice Ninguna de las anteriores.
        \end{checkboxes}

        \newpage
        % Question 18
        \question{¿Cuál o cuáles de las siguientes expresiones son 
        \textbf{verdaderas}?}
        \begin{checkboxes}
            \choice Si el universo de discurso es diferente del vacío, entonces 
            es posible encontrar una interpretación en la que $\forall x P(x)$
            sea verdadera y $\exists x P(x)$ sea falsa. 

            \choice Dada la siguiente fórmula
            \begin{equation*}
                \exists x (D(x) \land \forall y \neg R(y) \lor C(x,y))
            \end{equation*}

            entonces el alcance del cuantificador $\exists x$ es 
            $(D(x) \land \forall y \neg R(y) \lor C(x,y))$ y el alcance del 
            cuantificador $\forall y$ es $(\neg R(y) \lor C(x,y))$

            \choice Cuando dos variables están designadas por el mismo símbolo 
            (misma letra) entonces son variables diferentes si están bajo el 
            alcance de cuantificadores distintos, o si una es libre y la otra 
            no. % Correcta 

            \choice Cuando todas las variables que aparecen en una fórmula son 
            ligadas, entonces la fórmula es un enunciado. % Correcta 

            \choice Ninguna de las anteriores.
        \end{checkboxes}

        % Question 19
        \question
        {
            Sea $\mathbb{Z}$ el universo de discurso. ¿Cuál o cuáles de las 
            siguientes expresiones son \textbf{verdaderas}?
        }
        \begin{checkboxes}
            \choice $\forall x \exists y (x + y = x)$ es verdadero. % Correcta 

            \choice $\forall x (x^2 > 0)$ es falso. % Correcta 

            \choice $\forall x (x < 0 \rightarrow \exists y (y > 0 \land 
            x + y = 0))$ es falso. 

            \choice $\exists x \exists y (x^2 = y)$ es falso. 

            \choice Ninguna de las anteriores.
        \end{checkboxes}

        % Question 20
        \question
        {
            Sea el conjunto de todas las personas el universo de discurso. 
            Si consideramos el siguiente predicado: 
            \begin{equation*}
                A(x,y): x \text{ ama a } y 
            \end{equation*}

            ¿Cuál o cuáles de las siguientes expresiones son 
            \textbf{verdaderas}?
        }
        \begin{checkboxes}
            \choice $\exists x \forall y \neg A(x,y)$ es falso. 

            \choice $\forall x \forall y A(x,y)$ es verdadero. 

            \choice $\exists x \exists y A(x,y)$ es falso. 

            \choice $\forall x \exists y \neg A(x,y)$ es falso. 

            \choice Ninguna de las anteriores. % Correcta 
        \end{checkboxes}
    \end{questions}
\end{document}
