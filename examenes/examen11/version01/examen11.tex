\documentclass[oneside]{style}

\title{Versión 01}
\principal{Examen 11}
\author{Tania Michelle Rubí Rojas}
\semester{Semestre 2023-1}

\begin{document}
\maketitle

\vspace{5mm}
\noindent
Nombre y número de cuenta: \hrulefill\

\vspace*{5mm}
Para cada uno de los siguientes ejercicios, \textbf{justifica ampliamente} tu 
respuesta:

\begin{questions}[label=\protect\circled{\bfseries\arabic*}]

    % Ejercicio 01
    \question
    {
        Considera los siguientes predicados:
        \begin{itemize}
            \item $P(x): x$ es un número par
            \item $M(x,y): x$ es menor a $y$
            \item $D(x,y):$ la división de $x$ entre $y$ está dentro del 
            conjunto. 
        \end{itemize}

        Para cada uno de los siguientes enunciados, evalúa su valor de 
        verdad con respecto a los universos del discurso $\mathbb{N}$, 
        $\mathbb{R}$ y $\mathbb{Z}$. Además, \textbf{da} una interpretación 
        donde la fórmula sea verdadera y una donde sea falsa (si es que 
        existe):

        \begin{itemize}
            \item $\forall x \forall y (M(x,y) \rightarrow \exists z
            (M(x,z) \land M(z,y))$

            \item $\forall x (P(x) \rightarrow M(0,x))$
            
            \item $\forall x \forall y (x \neq 0 \land y \neq 0 
            \rightarrow (D(x,y) \land D(y,x)))$

            \item La negación del inciso anterior. 
        \end{itemize}
    }    
    
    % Ejercicio 02
    \question
    {
        Supongamos que nuestro universo del discurso es el conjunto de todas 
        las personas y todos los perritos. Considera los siguientes predicados:
        \begin{itemize}
            \item $H(x): x$ es una persona
            \item $P(y): y$ es un perrito 
            \item $A(x,y): x$ ama a $y$
        \end{itemize}
    }

    Para cada uno de los siguientes enunciados, \textbf{determina} su 
    valor de verdad:
    \begin{itemize}
        \item $\neg (\exists x (H(x) \land \exists y (A(x,y) \land P(y))))$
        \item $\forall x (H(x) \rightarrow \exists y (P(y) \land 
        \neg A(x,y)))$
        \item $\exists x (H(x) \land \forall y (A(x,y) \rightarrow P(y)))$
        \item $\neg (\forall x (H(x) \rightarrow \forall y (P(y) \rightarrow 
        A(x,y))))$
    \end{itemize}

    % Ejercicio 03
    \question
    {
        Para cada una de las siguientes fórmulas:
        \begin{itemize}
            \item $\forall x (A(x) \lor B(x)) \land (A(x) \land (B(x)))$
            \item $\forall x (P(x) \rightarrow \exists Q(x,y))$
            \item $\forall x \forall y ((P(x) \land P(y) \land E(x,y) \land 
            E(y,x)) \rightarrow I(x,y))$
            \item $\forall y \forall x Q(x) \rightarrow P(y)$        
            \item $\forall x \forall y (P(x,y) \rightarrow \exists x P(y,x))$
        \end{itemize}

        \textbf{realiza} lo siguiente:
        \begin{itemize}
            \item \textbf{Indica} con diferentes colores el alcance de cada uno 
            de los cuantificadores  
            \item \textbf{Subraya} las variables ligadas y libres con diferentes 
            colores. 
            \item \textbf{Indica} cuáles fórmulas son enunciados y explica
            por qué. 
        \end{itemize}
    }

     % Ejercicio 04
     \question
     {
        \textbf{Determina} si es posible capturar el sentido del siguiente 
        argumento usando únicamente lógica proposicional. En caso de que sea 
        posible, da su respectiva traducción a lógica proposicional:
        \begin{verbatim}
        Es ilegal que cualquier persona tenga más de 3 perros y 3 gatos en su 
        propiedad en el pueblito “Sin Nombre”. Yo vivo en el pueblito “Sin 
        Nombre” y tengo 4 perritos pero ningún gato. Por lo tanto, cumplo con 
        alguna ley del pueblito “Sin Nombre”.
        \end{verbatim}
     }
\end{questions}
\end{document}
