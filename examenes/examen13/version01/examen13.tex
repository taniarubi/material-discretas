\documentclass[oneside]{style}

\title{Versión 01}
\principal{Examen 13}
\author{Tania Michelle Rubí Rojas}
\semester{Semestre 2023-1}

\begin{document}
\maketitle

\vspace{2.5mm}
\noindent
Nombre y número de cuenta: \hrulefill\

\vspace{5mm}
\noindent
\textbf{Indicaciones especiales:}
{\small
\begin{multicols}{2}
\begin{itemize}
  \item No se pueden utilizar resultados que resuelvan directamente los 
  ejercicios. 

  \item Para cada ejercicio, si así lo requiere, se debe indicar claramente
  sobre cuál variable se está realizando la inducción. 

  \item Para cada ejercicio, si así lo requiere, se debe indicar claramente 
  cuál es el caso base, la hipótesis de inducción y el paso inductivo; 
  además de indicar cuál es la conclusión obtenida de la demostración. 

  \item Se debe justificar cada uno de los pasos que se realicen. 

  \item La letra debe ser lo más clara posible. En caso de que sea ilegible, 
  la calificación automática será de cero. 
\end{itemize}
\end{multicols}
}

\begin{questions}[label=\protect\circled{\bfseries\arabic*}]
    % Ejercicio 01
    \question
    {
        En una granja con mucho folklore se discute acerca del siguiente 
        razonamiento: 
        \begin{tcolorbox}
            El día que nace un becerro, cualquiera lo puede cargar con 
            facilidad. Y los becerros no crecen demasiado en un día, entonces 
            si puedes cargar a un becerro un día, lo puedes cargar también al 
            día siguiente. Siguiendo con este razonamiento, entonces también 
            debería serte posible cargar al becerro el día siguiente y el 
            siguiente y así sucesivamente. Pero después de un año, el becerro 
            se va a convertir en una vaca adulta de 1000kg, algo que claramente 
            ya no puedes cargar.
        \end{tcolorbox}

        \textbf{Demuestra}, si es posible, que el argumento es correcto 
        usando \textbf{inducción}. En caso contrario, \textbf{justifica 
        ampliamente} en donde está el error en el razonamiento inductivo. 
    }

    % Ejercicio 02
    \question
    {
        Sea \texttt{spar(n)} la función definida como 
        \begin{equation*}
            \texttt{spar(n)} = 2 + 4 + 6 + \cdots + 2n
        \end{equation*}

        \textbf{Realiza} lo siguiente:
        \begin{itemize}
            \item \textbf{Define} una función recursiva llamada 
            $f(n)$ para la función \texttt{spar(n)}.

            \item \textbf{Demuestra} usando \textbf{inducción 
            matemática} que $f(n) = n(n+1)$
        \end{itemize}
    }

    % Ejercicio 03
    \question
    {
        \textbf{Demuestra} usando \textbf{inducción matemática} que para cada 
        $n \geq 3 \in \mathbb{N}$ se cumple que
        \begin{center}
            \tcbox[tcbox raise base]{$\bigl( 1 + \frac{1}{n}^n \bigr) < n $}
        \end{center}
    }

    % Ejercicio 04
    \question
    {
        \textbf{Demuestra} usando \textbf{inducción fuerte} que todo número 
        entero positivo puede escribirse como una suma de potencias de $2$ que 
        sean distintas. 
    }
\end{questions}
\end{document}
