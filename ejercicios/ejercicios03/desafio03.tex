\documentclass[oneside]{style}

\title{Desafío 03}
\principal{Funciones}
\author{Tania Michelle Rubí Rojas}
\semester{Semestre 2023-1}

\begin{document}
\maketitle

Para cada uno de los siguientes ejercicios, \textbf{justifica ampliamente} tu 
respuesta:

\begin{questions}[label=\protect\circled{\bfseries\arabic*}]
    % Ejercicio 01
    \question{\textbf{Indica} si los siguientes enunciados son verdaderos o 
    falsos.}
    \begin{itemize}
        \item Si dos elementos en el dominio de una función son iguales, 
        entonces sus imagénes en el codominio son iguales.  

        \item Dos funciones definidas de manera diferente nunca pueden ser 
        iguales. 
    \end{itemize}

    % Ejercicio 02
    \question
    {
        Sean el conjunto $A = \{1,2,3,4,5\}$ y la función 
        $f: \mathcal{P}(A) \rightarrow \mathbb{Z}$ definida de la siguiente 
        manera: 
        \begin{equation*}
            f(X)=\begin{cases}
                      0 \quad &\text{si} \, $X$ \; \text{tiene un número par 
                      de elementos}\\
                      1 \quad &\text{si} \, $X$ \; \text{tiene un número impar 
                      de elementos }\\
                 \end{cases}
        \end{equation*}

        \textbf{Determina} lo siguiente:
        \begin{tasks}(4)
            \task $f(\{4,1,3\})$
            \task $f(\{2,3\})$
            \task $f(\varnothing)$
            \task $f(\{5,2,3,4\})$
        \end{tasks}
    }

    % Ejercicio 03
    \question
    {
        Sea la función $f: \mathbb{Z}^+ \rightarrow \mathbb{Z}^+$ que 
        está definida de la siguiente manera:
        \begin{equation*}
            f(n) = \text{la suma de los divisores positivos de } n
        \end{equation*}

        \textbf{Determina} lo siguiente:
        \begin{tasks}(4)
            \task $f(1)$
            \task $f(5)$
            \task $f(15)$
            \task $f(21)$
        \end{tasks}
    }

    % Ejercicio 04
    \question{\textbf{Indica} si los siguientes enunciados son verdaderos o 
    falsos.}
    \begin{itemize}
        \item Una función puede tener la misma entrada para más de una 
        salida. 

        \item El conjunto de todas las cadenas de ceros y unos es contable.
    \end{itemize}

    % Ejercicio 05
    \question{Sea $P$ el conjunto de todas las personas. \textbf{Da} dos 
    ejemplos de funciones de $P$ a $P$ que sean inyectivas, pero no 
    subrayectivas.}

    % Ejercicio 06
    \question{\textbf{Indica} si los siguientes enunciados son verdaderos o 
    falsos.}
    \begin{itemize}
        \item Sean $X$ y $Y$ conjuntos. Si $f: X \rightarrow Y$ es cualquier 
        función, entonces $I_Y \circ f = f$, donde $I_Y$ es la función 
        identidad en $Y$. 

        \item Si $f:X \rightarrow Y$ es una función inyectiva y sobreyectiva 
        con la función inversa $f^{-1}: Y \rightarrow X$, entonces 
        $f \circ f^{-1} = I_Y$, donde $I_Y$ es la función identidad en $Y$. 
    \end{itemize}

    % Ejercicio 07
    \question
    {
        Para las funciones definidas en los ejercicios $2$ y $3$, 
        \textbf{responde} lo siguiente:
        \begin{itemize}
            \item ¿Cuál es el dominio de $f$?
            \item ¿Cuál es la imagen de $f$?
            \item ¿Cuál es su regla de correspondencia?
            \item ¿Quién es $f^{-1}$?
            \item ¿Quién es $f \circ f$?
        \end{itemize}
    }

    % Ejercicio 08
    \question
    {
        Sean $X = \{a,b,c\}$ y $Y = \{u\}$ conjuntos. \textbf{Resuelve} lo
        siguiente:
        \begin{itemize}
            \item \textbf{Encuentra} todas las funciones de $X$ a $Y$.
            \item ¿Cuántas de esas funciones son inyectivas? ¿Cuántas son 
            suprayectivas?
        \end{itemize}
    }

    % Ejercicio 09
    \question{\textbf{Indica} si los siguientes enunciados son verdaderos o 
    falsos.}
    \begin{itemize}
        \item Sean $A, B$ y $C$ conjuntos. Si $f: A \rightarrow B$ y 
        $g:B \rightarrow C$ son funciones y $g \circ f$ es inyectiva, 
        entonces $g$ es inyectiva.

        \item Sean $A, B$ y $C$ conjuntos. Si $f: A \rightarrow B$ y 
        $g:B \rightarrow C$ son funciones y $g \circ f$ es inyectiva, 
        entonces $f$ es inyectiva.
    \end{itemize}

    % Ejercicio 10
    \question{\textbf{Indica} si los siguientes enunciados son verdaderos o 
    falsos.}
    \begin{itemize}
        \item Sean $A, B$ y $C$ conjuntos. Si $f: A \rightarrow B$ y 
        $g:B \rightarrow C$ son funciones y $g \circ f$ es suprayectiva, 
        entonces $g$ es suprayectiva.

        \item Sean $A, B$ y $C$ conjuntos. Si $f: A \rightarrow B$ y 
        $g:B \rightarrow C$ son funciones y $g \circ f$ es suprayectiva, 
        entonces $f$ es suprayectiva.
    \end{itemize}

    % Ejercicio 11
    \question
    {
        Sea $S$ el conjunto de todos los subconjuntos finitos de enteros 
        positivos. Sea la función $f: \mathbb{Z}^+ \rightarrow S$ que está 
        definida de la siguiente manera:
        \begin{equation*}
            f(n) = \text{el conjunto de divisores positivos de } n
        \end{equation*}

        \textbf{Determina} lo siguiente:
        \begin{tasks}(4)
            \task $f(21)$
            \task $f(17)$
            \task $f(15)$
            \task $f(1)$
        \end{tasks}
    }

    % Ejercicio 12
    \question{Sea $P$ el conjunto de todas las personas. \textbf{Da} dos 
    ejemplos de funciones de $P$ a $P$ que sean suprayectivas, pero no 
    inyectivas.}

    % Ejercicio 13
    \question{\textbf{Indica} si los siguientes enunciados son verdaderos o 
    falsos.}
    \begin{itemize}
        \item Cualquier conjunto infinito contiene un subconjunto 
        infinito contable. 

        \item La unión de dos conjuntos infinitos contables es infinito 
        contable. 
    \end{itemize}

    % Ejercicio 14
    \question{\textbf{Indica} si los siguientes enunciados son verdaderos o 
    falsos.}
    \begin{itemize}
        \item Sean $A, B, C$ y $D$ conjuntos. Si $f: A \rightarrow B$,  
        $g:B \rightarrow C$ y $h: C \rightarrow D$ son funciones, 
        entonces $h \circ (g \circ f) = (h \circ g) \circ f$.  

        \item Sea $A$ un conjunto. Sean las funciones $f: A \rightarrow A$, 
        $g: A \rightarrow A$ y $h: A \rightarrow$. Si $h$ es inyectiva y 
        $h \circ f = h \circ g$, entonces $f = g$. 
    \end{itemize}

    % Ejercicio 15
    \question{¿Cuál es la diferencia entre una relación binaria y una función?}

    % Ejercicio 16
    \question
    {
        ¿Cuál de las siguientes relaciones son funciones?
        \begin{align*}
            R &\subseteq \mathbb{R} \times \mathbb{R} \quad \quad \text{donde }
            (x,y) \in R \leftrightarrow x = y^2 \\ 
            S &\subseteq \mathbb{Z} \times \mathbb{Z} \quad \quad \text{donde }
            xRy \leftrightarrow x+y \; \text{es par}
        \end{align*}

        En caso de que alguna lo sea,
        \begin{itemize}
            \item ¿Cuál es su dominio?
            \item ¿Cuál es su codominio?
            \item ¿Cuál es su regla de correspondencia?
            \item ¿Cuál es su función inversa?
            \item ¿Cuál es el resultado de componer la función consigo misma?
        \end{itemize}
    }

    \newpage
    % Ejercicio 17
    \question
    {
        ¿Cuál de las siguientes relaciones son funciones?
        \begin{align*}
            T &\subseteq \{1,2,3\} \times \{\varnothing, a, b\} \quad \quad 
            \text{donde } T = \{(1, \varnothing), (2, \varnothing), (3,a), 
            (1,b)\} \\ 
            U &\subseteq \{1,2,3\} \times \{\varnothing, a, b\} \quad \quad 
            \text{donde } U = \{(1,a), (2,b), (3,\varnothing), (1,a)\}
        \end{align*}

        En caso de que alguna lo sea,
        \begin{itemize}
            \item ¿Cuál es su dominio?
            \item ¿Cuál es su codominio?
            \item ¿Cuál es su regla de correspondencia?
            \item ¿Cuál es su función inversa?
            \item ¿Cuál es el resultado de componer la función consigo misma?
        \end{itemize}
    }

    % Ejercicio 18
    \question
    {
        ¿Cuál de las siguientes relaciones son funciones?
        \begin{align*}
            R &\subseteq \{1,2,3\} \times \{\varnothing, a, b\} \quad \quad 
            \text{donde } R = \{(2,b), (3,\varnothing)\} \\ 
            A &= \{1,2,3\} \quad \quad \quad \quad \quad \quad \quad 
            \text{donde } S = \{(x,y) \in A^2 \; | \; x+1=y\}
        \end{align*}

        En caso de que alguna lo sea,
        \begin{itemize}
            \item ¿Cuál es su dominio?
            \item ¿Cuál es su codominio?
            \item ¿Cuál es su regla de correspondencia?
            \item ¿Cuál es su función inversa?
            \item ¿Cuál es el resultado de componer la función consigo misma?
        \end{itemize}
    }

    % Ejercicio 19
    \question
    {
        Sean $A,B$ y $C$ conjuntos no vacíos. Sean $f: A \rightarrow B$ y 
        $g: B \rightarrow C$ funciones. ¿Cuáles de las siguientes expresiones 
        son \textbf{verdaderas}?
        \begin{itemize}
            \item Si $g \circ f$ es inyectiva, entonces $f$ es inyectiva.
            \item Si $g \circ f$ es suprayectiva, entonces $f$ es suprayectiva.
        \end{itemize}
    }

    % Ejercicio 20
    \question
    {
        Sean $a,b \in \mathbb{R}$ con $b \neq 0$. Definimos la función
        $f: \mathbb{R} - \{0\} \rightarrow \mathbb{R} - \{a\}$ como sigue:
        \begin{equation*}
            f(x) = a + \frac{b}{x}
        \end{equation*}

        \textbf{Demuestra} que $f$ es una función biyectiva. 
    }

\end{questions}
\end{document}
