\documentclass[oneside]{style}

\title{Desafío 13}
\principal{Inducción Matemática}
\author{Tania Michelle Rubí Rojas}
\semester{Semestre 2023-1}

\begin{document}
\maketitle

Para cada uno de los siguientes ejercicios, \textbf{justifica ampliamente} tu 
respuesta:

\begin{questions}[label=\protect\circled{\bfseries\arabic*}]
    % Ejercicio 01
    \question
    {
        \textbf{Demuestra} usando \textbf{inducción} que para toda 
        $n \in \mathbb{Z}^+$ se cumple que
        \begin{center}
            \tcbox[tcbox raise base]{$\displaystyle\sum_{i=1} ^{n} i^3 
            = \left( \frac{n(n+1)}{2} \right)^2$}
        \end{center}
    }

    % Ejercicio 02
    \question
    {
        \textbf{Demuestra} usando \textbf{inducción} que para toda 
        $n \in \mathbb{N}$ se cumple que
        \begin{center}
            \tcbox[tcbox raise base]{$4 \vert (3^{2n} + 7)$}
        \end{center}
    }

    % Ejercicio 03
    \question
    {
        \textbf{Demuestra} usando \textbf{inducción} que para toda 
        $n \geq 4 \in \mathbb{N}$ se cumple que
        \begin{center}
            \tcbox[tcbox raise base]{$2^n \leq n! \leq n^n$}
        \end{center}
    }

    % Ejercicio 04
    \question
    {
        \textbf{Demuestra} usando \textbf{inducción fuerte} que para toda 
        $n \in \mathbb{Z}^+$ se cumple que
        \begin{center}
            \tcbox[tcbox raise base]{$F_n \leq \bigl( \frac{12}{7} \bigr)^n$}
        \end{center}

        donde $F_n$ es el $n$-ésimo número de la serie de Fibonacci. 
    }

    % Ejercicio 05
    \question
    {
        \textbf{Demuestra} usando \textbf{inducción} que para toda 
        $n \in \mathbb{Z}^+$ se cumple que
        \begin{center}
            \tcbox[tcbox raise base]{$(11)^{n+2} + (12)^{2n+1}$ es 
            divisible entre $133$}
        \end{center}
    }

    % Ejercicio 06
    \question
    {
        \textcolor{purple}{\textbf{Nubecita}} define la secuencia $a_1, a_2, 
        a_3, \ldots$ como sigue:
        \begin{align*}
            a_1 &= 1 \\ 
            a_2 &= 2 \\ 
            a_n &= 2a_{n-1} - a_{n-2} 
            && \text{para toda } n \geq 3
        \end{align*}

        \textbf{Demuestra} usando \textbf{inducción fuerte} que para toda 
        $n \in \mathbb{N}$ se cumple que 
        \begin{center}
            \tcbox[tcbox raise base]{$a_n = n$}
        \end{center} 
    }

    \newpage
    % Ejercicio 07
    \question
    {
        \textbf{Demuestra} usando \textbf{inducción} que para toda 
        $n \in \mathbb{Z}^+$ se cumple que
        \begin{center}
            \tcbox[tcbox raise base]{$\displaystyle\sum_{i=1} ^{n} i^2 
            = \frac{n(n+1)(2n+1)}{6}$}
        \end{center}
    }

    % Ejercicio 08
    \question
    {
        \textbf{Demuestra} usando \textbf{inducción} que para toda 
        $n,m \in \mathbb{N}$ se cumple que
        \begin{center}
            \tcbox[tcbox raise base]{$x^{m+n} = x^m x^n$}
        \end{center}
    }

    % Ejercicio 09
    \question
    {
        \textbf{Demuestra} usando \textbf{inducción} que para toda 
        $n \in \mathbb{N}$ se cumple que
        \begin{center}
            \tcbox[tcbox raise base]{$\displaystyle\sum_{i=0} ^{n} 9 \cdot 
            10^i = 10^{n+1} - 1$}
        \end{center}
    }

    % Ejercicio 10
    \question
    {
        \textbf{Demuestra} usando \textbf{inducción} que para toda 
        $n \in \mathbb{N}$ y $x \neq 1$ se cumple que
        \begin{center}
            \tcbox[tcbox raise base]{$\displaystyle\sum_{i=0} ^{n} x^i 
            = \frac{x^{n+1}-1}{x-1}$}
        \end{center}
    }

    % Ejercicio 11
    \question
    {
        \textbf{Demuestra} usando \textbf{inducción} que para toda 
        $n \in \mathbb{N}$ tal que $n > 6$ se cumple que 
        \begin{center}
            \tcbox[tcbox raise base]{$3^n < n!$}
        \end{center}
    }

    % Ejercicio 12
    \question
    {
        \textbf{Demuestra} usando \textbf{inducción} que para toda 
        $n \in \mathbb{Z}^+$ se cumple que
        \begin{center}
            \tcbox[tcbox raise base]{$\displaystyle\sum_{i=1} ^{n} i(i!) 
            = (n+1)! - 1$}
        \end{center}
    }

    % Ejercicio 13
    \question
    {
        \textbf{Demuestra} usando \textbf{inducción} que para toda 
        $n \in \mathbb{N}$ se cumple que
        \begin{center}
            \tcbox[tcbox raise base]{$n$ es un número par o impar}
        \end{center}
    }

    % Ejercicio 14
    \question
    {
        \textbf{Demuestra} usando \textbf{inducción} que para toda 
        $n \in \mathbb{N}$ se cumple que
        \begin{center}
            \tcbox[tcbox raise base]{$n^3 + (n+1)^3 + (n+2)^3$ es 
            divisible entre $9$}
        \end{center}
    }

    \newpage
    % Ejercicio 15
    \question
    {
        \textcolor{purple}{\textbf{Nubecita}} realiza un experimento con 
        diferentes computadoras y obtiene como resultado las siguientes 
        ecuaciones:
        \begin{align*}
            c_0 &= 3 \\ 
            c_1 &= 7 \\ 
            c_{n+2} &= 3c_{n+1} - 2c_n 
            && n \in \mathbb{N}
        \end{align*}

        \textbf{Demuestra} usando \textbf{inducción fuerte} que para toda 
        $n \in \mathbb{N}$ se cumple que 
        \begin{center}
            \tcbox[tcbox raise base]{$c_n = 2^{n+2} - 1$}
        \end{center} 
    }

    % Ejercicio 16
    \question
    {
        \textbf{Demuestra} usando \textbf{inducción} que para toda 
        $n \in \mathbb{Z^+}$ se cumple que
        \begin{center}
            \tcbox[tcbox raise base]{$2^{2n}$ es múltiplo de $3$} 
        \end{center}
    }

    % Ejercicio 17
    \question
    {
        \textbf{Demuestra} usando \textbf{inducción} que para toda 
        $n \in \mathbb{N}$ se cumple que
        \begin{center}
            \tcbox[tcbox raise base]{$5 \vert (n^5 - n)$}
        \end{center}
    }

    % Ejercicio 18
    \question
    {
        \textbf{Demuestra} usando \textbf{inducción} que para toda 
        $n \in \mathbb{Z}^+$ se cumple que
        \begin{center}
            \tcbox[tcbox raise base]{$2^n \leq 2^{n+1} - 2^{n-1} - 1$}
        \end{center}
    }

    % Ejercicio 19
    \question
    {
        \textbf{Demuestra} usando \textbf{inducción fuerte} que para toda 
        $n \in \mathbb{Z}^+$ se cumple que 
        \begin{center}
            \tcbox[tcbox raise base]{$12 \vert (n^4 - n^2)$}
        \end{center} 
    }

    % Ejercicio 20
    \question
    {
        \textcolor{purple}{\textbf{Nubecita}} define la siguiente secuencia 
        como sigue:
        \begin{align*}
            d_0 &= 2 \\ 
            d_1 &= 5 \\ 
            d_{n} &= 5d_{n-1} - 6d_{n-2} 
            && n > 1 \in \mathbb{N}
        \end{align*}

        \textbf{Demuestra} usando \textbf{inducción fuerte} que para toda 
        $n \in \mathbb{N}$ se cumple que 
        \begin{center}
            \tcbox[tcbox raise base]{$d_n = 2^n + 3^n$}
        \end{center}    
    }

    % Ejercicio 22
    \question
    {
        \textbf{Demuestra} usando \textbf{inducción fuerte} que si $n$ es un 
        número entero mayor qye $1$, entonces $n$ es un primo o $n$ se puede 
        escribir como el producto de primos. 
    }

    \newpage
    % Ejercicio 22
    \question
    {
        \textcolor{purple}{\textbf{Nubecita}} define la siguiente secuencia 
        como sigue:
        \begin{align*}
            d_0 &= 1 \\ 
            d_1 &= 4 \\ 
            d_{n} &= 4(d_{n-1} - d_{n-2}) 
            && n > 1 \in \mathbb{N}
        \end{align*}

        \textbf{Demuestra} usando \textbf{inducción fuerte} que para toda 
        $n \in \mathbb{N}$ se cumple que 
        \begin{center}
            \tcbox[tcbox raise base]{$d_n = 2^n (n+1)$}
        \end{center}   
    }

    % Ejercicio 23
    \question
    {
        \textbf{Demuestra} usando \textbf{inducción fuerte} que cualquier entero 
        $n \geq 12$ puede ser escrito de la forma $n = 4a + 5b$ con $a,b \in 
        \mathbb{Z}^+$.
    }

    % Ejercicio 24
    \question
    {
        Considera la siguiente función definida recursivamente:
        \begin{align*}
            f(0) &= 1 \\ 
            f(n) &= f(n-1) + 3
        \end{align*}

        \textbf{Demuestra} usando \textbf{inducción} que $f(n) = 3n + 1$. 
    }

    % Ejercicio 25
    \question
    {
        \textbf{Demuestra} usando \textbf{inducción fuerte} que para toda 
        $n \geq 3 \in \mathbb{N}$ se cumple que
        \begin{center}
            \tcbox[tcbox raise base]{$F_n > \bigl( \frac{1 + \sqrt{5}}{2} 
            \bigr)^{n-2}$}
        \end{center}

        donde $F_n$ es el $n$-ésimo número de la serie de Fibonacci. 
    }

    % Ejercicio 26
    \question
    {
        \textbf{Demuestra} usando \textbf{inducción} que para toda 
        $n \in \mathbb{Z}^+$ se cumple que
        \begin{center}
            \tcbox[tcbox raise base]{$\displaystyle\sum_{i=1} ^{n} (-1)^i i^2 
            = \frac{(-1)^n n (n+1)}{2}$}
        \end{center}
    }

    % Ejercicio 27
    \question
    {
        \textbf{Demuestra} usando \textbf{inducción} que para toda 
        $n \in \mathbb{Z}^+$ se cumple que
        \begin{center}
            \tcbox[tcbox raise base]{$\displaystyle\sum_{i=1} ^{n} 
            \frac{1}{i(i+1)} = \frac{n}{n+1}$}
        \end{center}
    }

    % Ejercicio 28
    \question
    {
        \textcolor{purple}{\textbf{Nubecita}} define la siguiente secuencia 
        como sigue:
        \begin{align*}
            a_1 &= 1 \\ 
            a_2 &= 8 \\ 
            a_{n} &= a_{n-1} + 2a_{n-2} 
            && n \geq 3 \in \mathbb{N}
        \end{align*}

        \textbf{Demuestra} usando \textbf{inducción fuerte} que para toda 
        $n \in \mathbb{N}$ se cumple que 
        \begin{center}
            \tcbox[tcbox raise base]{$a_n = 3 \cdot 2^{n-1} + 2(-1)^n$}
        \end{center}  
    }

    % Ejercicio 29
    \question
    {
        \textbf{Demuestra} usando \textbf{inducción} que un conjunto de $n$ 
        elementos tiene $2^n$ subconjuntos. 
    }

    % Ejercicio 30
    \question
    {
        \textbf{Demuestra} usando \textbf{inducción} que un conjunto de $n$ 
        elementos tiene $\frac{n(n-1)}{2}$ subconjuntos con $2$ elementos. 
    }

    % Ejercicio 31
    \question
    {
        ¿Cuáles son las diferencias entre la inducción matemática convencional 
        y la inducción matemática fuerte?
    }

    % Ejercicio 32
    \question
    {
        \textbf{Demuestra} usando \textbf{inducción} que para toda 
        $n \geq 2 \in \mathbb{N}$ se cumple que
        \begin{center}
            \tcbox[tcbox raise base]{$n! + k $ es divisible por $k$}
        \end{center}
    }

    % Ejercicio 33
    \question
    {
        \textbf{Demuestra} usando \textbf{inducción} que para cualquier 
        entero impar positivo se cumple que 
        \begin{center}
            \tcbox[tcbox raise base]{$(-2)^0 + (-2)^1 + (-2)^2 + \cdots + 
            (-2)^n = \frac{1-2^{n+1}}{3}$}
        \end{center}
    }

    % Ejercicio 34
    \question
    {
        \textbf{Demuestra} usando \textbf{inducción} que para cualquier 
        entero $n \geq 2$ se cumple que 
        \begin{center}
            \tcbox[tcbox raise base]{$\frac{1}{2} + \frac{2}{3} + \dots + 
            \frac{n}{n+1} < \frac{n^2}{n+1}$}
        \end{center}
    }

    % Ejercicio 35
    \question
    {
        \textbf{Demuestra} usando \textbf{inducción} que para cualquier entero 
        $n \geq 2$ se cumple que 
        \begin{center}
            \tcbox[tcbox raise base]{$\sqrt{n} < \frac{1}{\sqrt{1}} 
            + \frac{1}{\sqrt{2}} + \cdots + \frac{1}{\sqrt{n}}$}
        \end{center}
    }

    % Ejercicio 36
    \question
    {
        Sea $\{s_i\}_{i \in \mathbb{N}}$ la sucesión definida por 
        \begin{align*}
            s_1 &= \frac{9}{10} \\
            s_2 &= \frac{10}{11} \\  
            s_{k} &= s_{k-1} \cdot s_{k-2} 
        \end{align*}

        para todo entero $k \geq 3$. \textbf{Demuestra} usando 
        \textbf{inducción fuerte} que para cualquier entero positivo se cumple 
        que 
        \begin{center}
            \tcbox[tcbox raise base]{$0 < s_n \leq 1$}
        \end{center}
    }

    % Ejercicio 37
    \question
    {
        \textbf{Demuestra} usando \textbf{inducción} que para cualquier entero 
        $n \geq 3$ se cumple que 
        \begin{center}
            \tcbox[tcbox raise base]{$\bigl(1 + \frac{1}{n}\bigr)^n < n$}
        \end{center}
    }

    % Ejercicio 38
    \question
    {
        Un juego consiste en dos jugadores y dos pilas de monedas, cada una con 
        el mismo número $n \geq 1$ de monedas. Los jugadores se turnan, y cada 
        turno permite a un jugador eliminar cualquier número de monedas de uno 
        de los montones. El juego continúa hasta que se hayan eliminado todas 
        las monedas de ambos montones. El ganador es el jugador que retira 
        la(s) última(s) moneda(s). Utiliza \textbf{inducción fuerte} para 
        \textbf{demostrar} que el jugador que va en segundo lugar siempre va 
        a ganar. 
    }

    % Ejercicio 39
    \question
    {
        Utiliza \textbf{inducción fuerte} para \textbf{demostrar} que todo 
        entero positivo puede ser escrito como la suma de potencias de dos 
        (es decir, todo entero positivo tiene una representación binaria). 
    }

    % Ejercicio 40
    \question
    {
        Utiliza \textbf{inducción fuerte} para \textbf{demostrar} que 
        \begin{center}
            \tcbox[tcbox raise base]{$\sqrt{2}$ es irracional}
        \end{center}
    }

    % Ejercicio 41
    \question
    {
        Un rompecabezas se arma uniendo sucesivamente piezas que encajan en 
        bloques. Se realiza un movimiento cada vez que se agrega una pieza a 
        un bloque, o cuando se unen dos bloques. Utiliza \textbf{inducción 
        fuerte} para \textbf{demostrar} que sin importar la secuencia de 
        movimientos, se requieren exactamente $n-1$ movimientos para armar 
        un rompecabezas que tiene $n$ piezas.
    }

    % Ejercicio 42
    \question
    {
        \textbf{Demuestra} usando \textbf{inducción} que cualquier entero 
        $n \geq 18$ puede ser escrito de la forma $n = 4a + 7b$ con $a,b \in 
        \mathbb{Z}^+$.
    }

    % Ejercicio 43
    \question
    {
        \textbf{Realiza} lo siguiente:
        \begin{itemize}
            \item \textbf{Determina} qué números $n \in \mathbb{Z^+}$ pueden 
            ser escritos de la forma $n = 4a + 11b$, donde $a,b \in 
            \mathbb{Z}^+$.

            \item \textbf{Demuestra} tu afirmación usando \textbf{inducción}.
        \end{itemize}
    }

    % Ejercicio 44
    \question
    {
        \textbf{Demuestra} usando \textbf{inducción} que si $n \in \mathbb{Z}^+$
        es par, entonces $n^2$ también lo es. 
    }

\end{questions}
\end{document}
