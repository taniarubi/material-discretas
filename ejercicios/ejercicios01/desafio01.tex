\documentclass[oneside]{style}

\title{Desafío 1}
\principal{Conjuntos}
\author{Tania Michelle Rubí Rojas}
\semester{Semestre 2023-1}

\begin{document}
\maketitle

Para cada uno de los siguientes ejercicios, \textbf{justifica ampliamente} tu 
respuesta:

\begin{questions}[label=\protect\circled{\bfseries\arabic*}]
    % Ejercicio 01
    \question Sea $S = \{2,5,17,27\}$. ¿Cuáles de las siguientes expresiones 
    son verdaderas?
    \begin{tasks}(4)
        \task $2+5 \in S$
        \task $\varnothing \in S$
        \task $S \in S$
    \end{tasks}

    % Ejercicio 02
    \question \textbf{Da} la definición por extensión de los siguientes 
    conjuntos:
    \begin{itemize}
        \item $\{x \; | \; x \text{ es uno de los tres últimos presidentes 
        de México}\}$
        
        \item $\{x \; | \; x \text{ es uno de los países de Latinoamérica}\}$
        
        \item $\{x \; | \; x \text{ es la primer letra de las palabras cristal, 
        nubecita o luciernaga}\}$
    \end{itemize}

    % Ejercicio 03
    \question{\textbf{Encuentra} todos los subconjuntos del conjunto 
    $\{\{\star, \bullet\}, \varnothing\}$}.

    % Ejercicio 04
    \question Dados los siguientes conjuntos: 
    \begin{equation*}
        A = \{2,5,7\} \quad \quad \quad 
        B = \{1,2,4,7,8\} \quad \quad \quad
        C = \{7,8\}
    \end{equation*}

    ¿Cuáles de las siguientes expresiones son \textbf{verdaderas}?
    \begin{tasks}(4)
        \task $5 \subseteq A$
        \task $C \subseteq B$
        \task $\varnothing \in A$
    \end{tasks}

    % Ejercicio 05
    \question \textbf{Responde} las siguientes preguntas:
    \begin{itemize}
        \item ¿Cuál es la cardinalidad de $\varnothing$?

        \item Si un conjunto $X$ tiene $n$ elementos, ¿cuántos subconjuntos 
        propios tiene $X$?

        \item ¿Cuál es la cardinalidad de $\{\{a\}, \{a,b\}, \{a,c\}, a, b\}$?
    \end{itemize}

    % Ejercicio 06
    \question ¿Cuáles de las siguientes expresiones son \textbf{verdaderas}?
    \begin{tasks}(4)
        \task $\{1\} \subseteq \{1\}$
        \task $1 \in \{\{1\}, 2\}$
        \task $\{2\} \subseteq \{1, \{2\}, \{3\}\}$
    \end{tasks}

    % Ejercicio 07
    \question{Si $S = \{\{a\}, \{b\}\}$ y $T = \{\{a\}, b\}$. ¿Es cierto que 
    $S = T$?}

    % Ejercicio 08
    \question \textbf{Da} la definición por comprensión de los siguientes 
    conjuntos:
    \begin{itemize}
        \item $\{1,3,5,7,9,11, \ldots\}$
        \item $\{\text{Melchor, Gaspar, Baltazar}\}$
        \item $\{2,a,3,b,4,c\}$
    \end{itemize}

    % Ejercicio 09
    \question{Si $S = \{\circ, 8, \star\}$.}
    \begin{itemize}
        \item ¿Quién es $S \cap \varnothing$?
        \item ¿Quién es $S \cap S$?
    \end{itemize}

    % Ejercicio 10
    \question{Si $S \subset T$ y $T \subset S$, ¿sucede que $S = T$?}

    % Ejercicio 11
    \question{\textbf{Responde} las siguientes preguntas:}
    \begin{itemize}
        \item ¿Cuál es la cardinalidad de $\{\varnothing\}$?
        \item Si un conjunto $X$ tiene $n$ elementos, ¿cuántos subconjuntos
        tiene $X$?
        \item Si $A$ es un conjunto cualquiera, ¿siempre sucede que
        $\varnothing \subset A$? En caso negativo, ¿cuándo sí sucede?
    \end{itemize}

    \newpage
    % Ejercicio 12
    \question{¿Cuáles de las siguientes expresiones son \textbf{verdaderas}?}
    \begin{tasks}(3)
        \task $1 \subseteq \{1\}$
        \task $\{2\} \in \{1,2\}$
        \task $1 \in \{1, \{2\}\}$
    \end{tasks}

    % Ejercicio 13
    \question{Si $S = \{A, b, 4, \text{Carlos}, 5\}$.}
    \begin{itemize}
        \item ¿Quién es $S \cup \varnothing$?
        \item ¿Quién es $S \cup S$?
    \end{itemize}

    % Ejercicio 14
    \question{\textbf{Da} la definición por comprensión de los siguientes 
    conjuntos:}
    \begin{itemize}
        \item $\{\text{m,n,o,p}\}$ 
        \item $\{0,3,6,9,12\}$
        \item $\{\ldots, -3, -2, -1, 1, 2, 3, \ldots\}$
    \end{itemize}

    % Ejercicio 15
    \question{¿Cuántos diferentes conjuntos hay descritos a continuación? 
    ¿Cuáles son?}
    \begin{tasks}(2)
        \task $\varnothing$

        \task $\{2,3,4\}$

        \task $\{x \; | \; x \in \mathbb{N} \text{ y } 2 \leq x \leq 4\}$

        \task $\{e, o\}$ 
        
        \task $\{3,4,2\}$
        
        \task $\{2, a, 3, b, 4, c\}$
    \end{tasks}
    \begin{itemize}
        \item[g)] $\{x \; | \; x \text{ es la primer y última vocal de la frase 
        <<Te extraño>>}\}$
    \end{itemize}

    % Ejercicio 16
    \question{¿Cuál es la cardinalidad de cada uno de los siguientes 
    conjuntos? \textbf{Colorea} de diferentes colores cada uno de los 
    diferentes elementos de los conjuntos.}
    \begin{tasks}(2)
        \task $S = \{\star, \{\star, \{\star\}\}\}$
        \task $S = \{\{\bullet, \{\{\bullet\}\}\}\}$
    \end{tasks}

    % Ejercicio 17
    \question{¿Cuáles de las siguientes expresiones son \textbf{verdaderas}?}
    \begin{tasks}(3)
        \task $\{\varnothing\} = \varnothing$
        \task $\{\varnothing\} = \{0\}$
        \task $\varnothing \in \{\varnothing\}$
    \end{tasks}

    % Ejercicio 18
    \question{\textbf{Responde} las siguientes preguntas:}
    \begin{itemize}
        \item Si $A$ es un conjunto cualquiera, ¿siempre sucede que 
        $\varnothing \in A$? En caso negativo, ¿cuándo sí sucede?

        \item ¿Se cumple que $\{\varnothing\} = \{\{\varnothing\}\}$?
    \end{itemize}

    % Ejercicio 19
    \question{\textbf{Encuentra} $\mathcal{P}(\{\varnothing\})$} y 
    $\mathcal{P}(\{1,2,3,4\})$. ¿Cuántos elementos tienen los conjuntos 
    esperados?

    % Ejercicio 20
    \question{\textbf{Determina} si el elemento $\bigstar$ pertenece a los 
    siguientes conjuntos:}
    \begin{tasks}(3)
        \task $\{\bigstar, \{\bigstar\}\}$
        \task $\{\{\{\bigstar\}\}\}$
        \task $\{\{\bigstar\}, \{\bigstar, \{\bigstar\}\}\}$
    \end{tasks}

    % Ejercicio 21
    \question{Sean $S = \{n \in \mathbb{N} \; | \; n \text{ es par}\}$ y
    $T = \{0,1,2,3,4,5,6,7,8,9\}$. \textbf{Indica} quiénes son $S \cup T, 
    S \cap T, S-T, T-S$}.

    % Ejercicio 22
    \question{\textbf{Define} por extensión un conjunto $S$ de $6$ conjuntos 
    de tal modo que:}
    \begin{itemize}
        \item Cada elemento de $S$ sea distinto a todos los demás. 

        \item Por lo menos $4$ de los elementos de $S$ tengan como elemento 
        a algún otro elemento del conjunto. 

        \item Por lo menos $4$ de los elementos de $S$ tengan como subconjunto 
        a algún otro elemento del conjunto.
    \end{itemize}

    % Ejercicio 23
    \question{\textbf{Encuentra} $\mathcal{P}(\mathcal{P}(\{a,b\}))$ y 
    $\mathcal{P}(\{\varnothing, \{\varnothing\}, \{\varnothing, 
    \{\varnothing\}\}\})$}. ¿Cuántos elementos tienen los conjuntos 
    esperados?

    \newpage
    % Ejercicio 24
    \question{Dentro del Universo de los personajes de la serie <<Los Simpson>>, 
    \textbf{define} por comprensión cuatro conjuntos (digamos $S, T, U, V$)
    tal que cumplan lo siguiente:}
    \begin{tasks}(2)
        \task Todos deben ser distintos entre sí.
        \task $S \cap T \neq \varnothing$
        \task $(S \cap T) \cap U = \varnothing$
        \task $V \cap S \neq \varnothing$
        \task $V \cap T \neq \varnothing$
        \task $V \cap U \neq \varnothing$
        \task $S \not \subset V$
        \task $T \not \subset V$
        \task $U \not \subset V$
    \end{tasks}

    % Ejercicio 25
    \question{Sea $\mathcal{U} = \{1,2,3\} \cup 2^{\{1,2,3\}}$.}
    \begin{itemize}
        \item \textbf{Describe} todos los elementos de este Universo. 
        \item \textbf{Describe} por extensión $5$ conjuntos de $3$ elementos
        o más en $\mathcal{U}$. 
        \item ¿Es cierto que $\varnothing \in \mathcal{U}$?
    \end{itemize}

    % Ejercicio 26
    \question{Sean $S$ y $T$ dos conjuntos cualesquiera. Para cada una de las 
    siguientes expresiones, \textbf{encuentra} condiciones generales para $S$ y 
    $T$ de tal forma que las expresiones sean \textbf{verdaderas}:}
    \begin{tasks}(3)
        \task $S \cup T = S$
        \task $S \cup T \subseteq S \cap T$
        \task $T-S = \varnothing$
    \end{tasks}

    % Ejercicio 27
    \question{¿Cuál es la cardinalidad de cada uno de los siguientes 
    conjuntos? \textbf{Colorea} de diferentes colores cada uno de los 
    diferentes elementos de los conjuntos.}
    \begin{tasks}(2)
        \task $\{\bullet, \{\varnothing\}, \varnothing\}$
        \task $\{\varnothing, \{\varnothing, \{\varnothing\}\}, 
        \{\varnothing, \{\varnothing, \{\varnothing\}\}\}\}$
    \end{tasks}

    % Ejercicio 28
    \question{Considera el Universo como el conjunto de todas las palabras del 
    diccionario. Dados los siguientes conjuntos:}
    \begin{align*}
        S &= \{x \; | \; x \text{ es una palabra que aparece antes que 
        <<dinosaurio>> en el diccionario}\} \\
        T &= \{x \; | \; x \text{ es una palabra que aparece después que 
        <<celeste>> en el diccionario}\} \\ 
        U &= \{x \; | \; x \text{ es una paabra de más de cuatro letras}\}
    \end{align*}

    ¿Cuáles de las siguientes expresiones son \textbf{verdaderas}? 
    \begin{itemize}
        \item $T \subseteq U$
        \item $S \cup T = \{x \; | \; x \text{ es una palabra del diccionario}\}$
        \item casa $\in T \cap U^c$
        \item bamboo $\in S-T$
    \end{itemize}

    % Ejercicio 29
    \question{Sean $S$ y $T$ dos conjuntos cualesquiera. Para cada una de las 
    siguientes expresiones, \textbf{encuentra} condiciones generales para $S$ y 
    $T$ de tal forma que las expresiones sean \textbf{verdaderas}:}
    \begin{tasks}(3)
        \task $S \cap T = S$
        \task $S \cup \varnothing = \varnothing$
        \task $S \cup T = S-T$
    \end{tasks}

    % Ejercicio 30
    \question
    {
        ¿Cuál es el conjunto potencia de $\{1, \{2\}, \varnothing, 
        \{\varnothing\}\}$?
    }
\end{questions}
\end{document}
