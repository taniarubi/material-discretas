\documentclass[oneside]{style}

\title{Desafío 04}
\principal{Definición Recursiva de Conjuntos}
\author{Tania Michelle Rubí Rojas}
\semester{Semestre 2023-1}

\begin{document}
\maketitle

Para cada uno de los siguientes ejercicios, \textbf{justifica ampliamente} tu 
respuesta:

\begin{questions}[label=\protect\circled{\bfseries\arabic*}]
    % Ejercicio 01
    \question
    {
        Para el siguiente conjunto 
        \begin{equation*}
            S = \{n^2 \; | \; n \in \mathbb{N}\}
        \end{equation*}

        \textbf{realiza} lo siguiente.
        \begin{itemize}
            \item \textbf{Da} la definición por extensión del conjunto $S$.
            \item \textbf{Define} a $S$ recursivamente.
            \item ¿Tu definición es correcta? \textbf{Justifica} tu respuesta 
            mostrando cada uno de los componentes de tu definición y cada uno 
            de los componentes de las reglas que definiste. 
            \item Usando la definición que escribiste en el inciso anterior, 
            muestra que $16 \in S$ y $5 \not \in S$.  
        \end{itemize}
    }

    % Ejercicio 02
    \question
    {
        Para el siguiente conjunto 
        \begin{equation*}
            S = \{2^n - 2 \; | \; n \in \mathbb{N} \text{ y } n > 0\}
        \end{equation*}

        \textbf{realiza} lo siguiente.
        \begin{itemize}
            \item \textbf{Da} la definición por extensión del conjunto $S$.
            \item \textbf{Define} a $S$ recursivamente.
            \item ¿Tu definición es correcta? \textbf{Justifica} tu respuesta 
            mostrando cada uno de los componentes de tu definición y cada uno 
            de los componentes de las reglas que definiste. 
            \item Usando la definición que escribiste en el inciso anterior, 
            muestra que $30 \in S$ y $2 \in S$
        \end{itemize}
    }

    % Ejercicio 03
    \question
    {
        Para el siguiente conjunto 
        \begin{equation*}
            S = \{a^n b c^n \; | \; n \in \mathbb{N}\}
        \end{equation*}

        \textbf{realiza} lo siguiente.
        \begin{itemize}
            \item \textbf{Da} la definición por extensión del conjunto $S$.
            \item \textbf{Define} a $S$ recursivamente.
            \item ¿Tu definición es correcta? \textbf{Justifica} tu respuesta 
            mostrando cada uno de los componentes de tu definición y cada uno 
            de los componentes de las reglas que definiste. 
            \item Usando la definición que escribiste en el inciso anterior, 
            muestra que $aabcc \in S$ y $aabbcc \not \in S$
        \end{itemize}
    }

    % Ejercicio 04
    \question
    {
        Sea $S$ el conjunto de los números naturales. Para dicho conjunto, 
        \textbf{realiza} lo siguiente:
        \begin{itemize}
            \item \textbf{Da} la definición por comprensión del conjunto $S$.
            \item \textbf{Define} a $S$ recursivamente.
            \item ¿Tu definición es correcta? \textbf{Justifica} tu respuesta 
            mostrando cada uno de los componentes de tu definición y cada uno 
            de los componentes de las reglas que definiste. 
            \item Usando la definición que escribiste en el inciso anterior, 
            muestra que $5 \in S$ y $2 \in S$. 
        \end{itemize}
    }

    % Ejercicio 05
    \question{¿Todo conjunto puede ser definido de manera recursiva?}

    % Ejercicio 06
    \question{¿Cuáles son las características de una definición recursiva?}

    % Ejercicio 07
    \question
    {
        Sea $A \subseteq \mathbb{Z}$ un conjunto cuya definición recursiva es 
        la siguiente:
        \begin{itemize}
            \item[i)] $1 \in A$
            \item[ii)] Si $a \in A$ entonces $a \in A$
            \item[iii)] Estos y sólo estos son elementos de $A$.
        \end{itemize}   

        \textbf{Resuelve} lo siguiente:
        \begin{itemize}
            \item ¿Cuál es el pegamento, el antecedente y el consecuente de
            cada una de las reglas de la definición?

            \item \textbf{Escribe} por extensión y comprensión el conjunto $A$. 
        \end{itemize}
    }

    % Ejercicio 08
    \question
    {
        Sea $S$ el conjunto de los números negativos que son pares. Para dicho 
        conjunto, \textbf{realiza} lo siguiente:
        \begin{itemize}
            \item \textbf{Da} la definición por comprensión del conjunto $S$.
            \item \textbf{Define} a $S$ recursivamente.
            \item ¿Tu definición es correcta? \textbf{Justifica} tu respuesta 
            mostrando cada uno de los componentes de tu definición y cada uno 
            de los componentes de las reglas que definiste. 
            \item Usando la definición que escribiste en el inciso anterior, 
            muestra que $-10 \in S$ y $0 \not \in S$ 
        \end{itemize}
    }

    % Ejercicio 09
    \question
    {
        Sea $S$ el conjunto de los números enteros que son impares. Para 
        dicho conjunto, \textbf{realiza} lo siguiente:
        \begin{itemize}
            \item \textbf{Da} la definición por comprensión del conjunto $S$.
            \item \textbf{Define} a $S$ recursivamente.
            \item ¿Tu definición es correcta? \textbf{Justifica} tu respuesta 
            mostrando cada uno de los componentes de tu definición y cada uno 
            de los componentes de las reglas que definiste. 
            \item Usando la definición que escribiste en el inciso anterior, 
            muestra que $-3 \in S$ y $7 \in S$. 
        \end{itemize}
    }

    % Ejercicio 10
    \question
    {
        Sea $S$ el conjunto de todas las cadenas de ceros y unos. Para dicho 
        conjunto, \textbf{realiza} lo siguiente:
        \begin{itemize}
            \item \textbf{Da} la definición por comprensión del conjunto $S$.
            \item \textbf{Define} a $S$ recursivamente.
            \item ¿Tu definición es correcta? \textbf{Justifica} tu respuesta 
            mostrando cada uno de los componentes de tu definición y cada uno 
            de los componentes de las reglas que definiste. 
            \item Usando la definición que escribiste en el inciso anterior, 
            muestra que $0101 \in S$ y $031 \not \in S$ 
        \end{itemize}
    }

    % Ejercicio 11
    \question
    {
        Para el siguiente conjunto 
        \begin{equation*}
            S = \{0^n 1^n \; | \; n \in \mathbb{N} \text{ y } n > 0\}
        \end{equation*}

        \textbf{realiza} lo siguiente.
        \begin{itemize}
            \item \textbf{Da} la definición por extensión del conjunto $S$.
            \item \textbf{Define} a $S$ recursivamente.
            \item ¿Tu definición es correcta? \textbf{Justifica} tu respuesta 
            mostrando cada uno de los componentes de tu definición y cada uno 
            de los componentes de las reglas que definiste. 
            \item Usando la definición que escribiste en el inciso anterior, 
            muestra que $0011 \in S$ y $00001111 \in S$
        \end{itemize}
    }

    % Ejercicio 12
    \question
    {
        Sea $S$ el conjunto de todas las cadenas de ceros y unos que terminan 
        con un cero. Para dicho conjunto, \textbf{realiza} lo siguiente:
        \begin{itemize}
            \item \textbf{Da} la definición por extensión del conjunto $S$.
            \item \textbf{Define} a $S$ recursivamente.
            \item ¿Tu definición es correcta? \textbf{Justifica} tu respuesta 
            mostrando cada uno de los componentes de tu definición y cada uno 
            de los componentes de las reglas que definiste. 
            \item Usando la definición que escribiste en el inciso anterior, 
            muestra que $111010 \in S$ y $1 \not \in S$ 
        \end{itemize}
    }

    % Ejercicio 13
    \question{¿Una definición recursiva puede tener más de una regla 
    recursiva? \textbf{Justifica} tu respuesta con un ejemplo. }

    % Ejercicio 14
    \question
    {
        Sea $F$ un conjunto cuya definición recursiva es la siguiente:
        \begin{itemize}
            \item[i)] $1 \in F$
            \item[ii)] Si $n \in F$ entonces $n^2 + n + 3 \in F$
            \item[iii)] Estos y sólo estos son elementos de $F$.
        \end{itemize}   

        ¿Cuáles de los siguientes elementos pertenecen a $F$?
        \begin{tasks}(3)
            \task $52$
            \task $33$
            \task $106$
          \end{tasks}    
    }

    % Ejercicio 15
    \question
    {
        Sea $S$ un conjunto de números cuya definición recursiva es 
        la siguiente:
        \begin{itemize}
            \item[i)] $2 \in S$
            \item[ii)] Si $n \in S$ entonces $n+3 \in S$ y $2*n \in S$
            (donde $+, *$ denotan, respectivamente, la suma y 
            multiplicación habitual de números naturales).
            \item[iii)] Estos y sólo estos son elementos de $S$.
        \end{itemize}   

        \textbf{Resuelve} lo siguiente:
        \begin{itemize}
            \item \textbf{Da} la definición por comprensión del conjunto 
            $S$.
            \item ¿Los elementos $6,7,19$ pertenecen a $S$?
            \item ¿La definición es correcta? \textbf{Justifica} tu respuesta 
            mostrando cada uno de los componentes de la definición y cada uno 
            de los componentes de las reglas definidas.  
        \end{itemize}
    }

    % Ejercicio 16
    \question
    {
        Sea $S$ un conjunto de cadenas cuya definición recursiva es 
        la siguiente:
        \begin{itemize}
            \item[i)] $a \in S$
            \item[ii)] $b \in S$  
            \item[ii)] Si $x \in S$ entonces $xb \in S$.
            \item[iii)] Estos y sólo estos son elementos de $S$.
        \end{itemize}   

        \textbf{Resuelve} lo siguiente:
        \begin{itemize}
            \item \textbf{Da} la definición por comprensión del conjunto 
            $S$.
            \item ¿Los elementos $a, aba, bbbbb$ pertenecen a $S$?
            \item ¿La definición es correcta? \textbf{Justifica} tu respuesta 
            mostrando cada uno de los componentes de la definición y cada uno 
            de los componentes de las reglas definidas.  
        \end{itemize}
    }

    % Ejercicio 17
    \question
    {
        Sea $S$ un conjunto de cadenas cuya definición recursiva es 
        la siguiente:
        \begin{itemize}
            \item[i)] $pqq \in S$
            \item[ii)] Si $x,y \in S$ entonces $pxqq,qqxp,xy \in S$
            \item[iii)] Estos y sólo estos son elementos de $S$.
        \end{itemize}   

        \textbf{Resuelve} lo siguiente:
        \begin{itemize}
            \item \textbf{Da} la definición por comprensión del conjunto 
            $S$.
            \item ¿Los elementos $ppqqqq,qqpp,qqqqqqpppp$ pertenecen a $S$?
            \item ¿La definición es correcta? \textbf{Justifica} tu respuesta 
            mostrando cada uno de los componentes de la definición y cada uno 
            de los componentes de las reglas definidas.  
        \end{itemize}
    }

    % Ejercicio 18
    \question
    {
        Sea $M$ el conjunto de números naturales que son impares cuya 
        definición recursiva es la siguiente:
        \begin{itemize}
            \item[i)] $1 \in M$
            \item[ii)] Si $n \in M$ entonces $n+2 \in M$
            \item[ii)] Este es el conjunto $M$  
        \end{itemize}

        ¿La definición recursiva de $M$ es correcta, válida y finita?
    }

    \newpage
    % Ejercicio 19
    \question
    {
        ¿Cuáles de las siguientes afirmaciones son \textbf{verdaderas}?
        \begin{itemize}
            \item Una definición recursiva es válida si y sólo si 
            construye todos los elementos del conjunto en cuestión. 

            \item La regla de extremo garantiza que las reglas base 
            y las reglas recursivas son las únicas maneras de obtener 
            o construir elementos del conjunto. 
        \end{itemize}
    }

    % Ejercicio 20
    \question{¿Una definición recursiva puede tener más de un caso base?}
\end{questions}
\end{document}
