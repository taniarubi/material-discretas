\documentclass[oneside]{style}

\title{Desafío 14}
\principal{Inducción Estructural}
\author{Tania Michelle Rubí Rojas}
\instructor{Odin Miguel Escorza Soria}
\semester{Semestre 2023-1}

\begin{document}
\maketitle

Para cada uno de los siguientes ejercicios, \textbf{justifica ampliamente} tu 
respuesta:

\begin{questions}[label=\protect\circled{\bfseries\arabic*}]
    % Ejercicio 01
    \question
    {
        Una cadena de caracteres $w$ es palindroma si es de la forma 
        $w = uu^R$, donde $u^R$ es la cadena $u$ escrita de atrás hacia 
        delante. 

        \begin{itemize}
            \item \textbf{Define recursivamente} el conjunto de las cadenas 
            palíndromas. 

            \item \textbf{Demuestra}, usando \textbf{inducción estructural}, 
            que todas las cadenas palíndromas definidas anteriormente tiene 
            un número par de símbolos.  
        \end{itemize}
    }

    % Ejercicio 02
    \question
    {
        La función \texttt{agrega} se define como sigue:
        \begin{equation*}
            \texttt{agrega}(e, [a_1, \ldots, a_n]) = 
            [a_1, \ldots, a_n, e]
        \end{equation*}

        Por otro lado, la función \texttt{concat} se define como sigue:
        \begin{equation*}
            \texttt{concat}([a_1, \ldots, a_n], [b_1, \ldots, b_n]) = 
            [a_1, \ldots, a_n, b_1, \ldots, b_n]
        \end{equation*}

        \textbf{Realiza} lo siguiente:
        \begin{itemize}
            \item \textbf{Define recursivamente} las funciones 
            \texttt{agrega} y \texttt{concat}. 

            \item \textbf{Demuestra}, usando \textbf{inducción estructural} y 
            tus funciones recursivas, que 
            \begin{center}
                \tcbox[tcbox raise base]{\texttt{agrega(e, concat($l_1, l_2$)) 
                = concat($l_1$, agrega(e, $l_2$))}}
            \end{center}
        \end{itemize}
    }

    % Ejercicio 03
    \question
    {
        El conjunto de paréntesis bien balanceados es el siguiente:
        \begin{equation*}
            P = \{(), (()), ((())), (())(()), \ldots\}
        \end{equation*}

        \textbf{Realiza} lo siguiente:
        \begin{itemize}
            \item \textbf{Define recursivamente} el conjunto de paréntesis bien 
            balanceados. 

            \item \textbf{Demuestra}, usando \textbf{inducción estructural} y 
            tu función recursiva, que todos los elementos del conjunto $P$ 
            contienen un número igual de paréntesis izquierdos que derechos. 
        \end{itemize}
    }

    % Ejercicio 04
    \question
    {
        La función \texttt{cuenta} se define como sigue:
        \begin{equation*}
            \texttt{cuenta}(s) = \text{ el número de unos que tiene } s
        \end{equation*}

        donde $s$ es una cadena binaria. \textbf{Realiza} lo siguiente:
        \begin{itemize}
            \item \textbf{Define recursivamente} la función \texttt{cuenta}.
            \item \textbf{Demuestra}, usando \textbf{inducción estructural} y 
            tu función recursiva, que 
            \begin{center}
                \tcbox[tcbox raise base]{\texttt{cuenta(st) = cuenta(s) + 
                cuenta(t)}}
            \end{center}
        \end{itemize}
    }

    \newpage
    % Ejercicio 05
    \question
    {
        La función \texttt{reversa} se define como sigue:
        \begin{center}
            \texttt{reversa}$([a_1, \ldots, a_n]) = [a_n, \ldots, a_1]$ 
        \end{center}

        Por otro lado, la función \texttt{fusion} se define como sigue: 
        \begin{center}
            \texttt{fusion}$([a_1, \ldots, a_n], [b_1, \ldots, b_n]) 
            = [b_1, \ldots, b_n, a_1, \ldots, a_n]$ 
        \end{center}

        \begin{itemize}
            \item \textbf{Define recursivamente} las funciones \texttt{reversa} 
            y \texttt{fusion}.

            \item \textbf{Demuestra}, usando \textbf{inducción estructural} y 
            tus funciones recursivas, que:  
            \begin{center}
                \tcbox[tcbox raise base]{\texttt{reversa(fusion($l_1, l_2$)) 
                = fusion(reversa($l_1$), reversa($l_2$))}}
            \end{center}
        \end{itemize}
    }

    % Ejercicio 06
    \question
    {
        La función \texttt{reversa} se define como sigue:
        \begin{center}
            \texttt{reversa}$([a_1, \ldots, a_n]) = [a_n, \ldots, a_1]$ 
        \end{center}

        Por otro lado, la función \texttt{misterio} está definida 
        recursivamente como sigue: 
        \begin{align*}
            \text{misterio}([], l_2) &= l_2 \\ 
            \text{misterio}((x:xs), l_2) &= \text{misterio}(xs, (x:l_2))
        \end{align*}

        \begin{itemize}
            \item \textbf{Define recursivamente} la función \texttt{reversa}.
            
            \item ¿Qué hace la función \texttt{misterio}?
            
            \item \textbf{Ejecuta} la función \texttt{misterio} con dos 
            listas no triviales. 

            \item \textbf{Demuestra}, usando \textbf{inducción estructural} y 
            las funciones recursivas anteriores, que:  
            \begin{center}
                \tcbox[tcbox raise base]{\texttt{reversa($l$) = misterio($l$, 
                [])))}}
            \end{center}
        \end{itemize}
    }

    % Ejercicio 07
    \question
    {
        Sea $A$ una fórmula de $\mathcal{LPROP}$ cuyos únicos conectivos 
        lógicos son $\land, \lor, \neg$.

        \begin{itemize}
            \item \textbf{Define recursivamente} la función \texttt{swap}, la cual 
            hace los siguientes intercambios en la fórmula $A$: 
            \begin{center}
                $\land$ por $\lor$ \quad \quad $\lor$ por $\land$ \quad \quad 
                $p$ por $\neg p$
            \end{center}

            donde $p$ es una variable proposicional. 

            \item \textbf{Demuestra}, usando \textbf{inducción estructural} y 
            tu función recursiva, que 
            \begin{center}
                \tcbox[tcbox raise base]{\texttt{$\neg A \equiv$ swap(A)}}
            \end{center}
        \end{itemize}
    }

    % Ejercicio 08
    \question
    {
        El conjunto de cadenas $S$ está definido recursivamente como sigue: 
        \begin{itemize}
            \item [i.] $1 \in S$
            \item [ii.] Si $s \in S$, entonces $0s, 1s \in S$.
            \item [iii.] No hay nada en $S$ que no sean objetos definidos con 
            las reglas anteriores.  
        \end{itemize}

        Utiliza \textbf{inducción estructural} para \textbf{demostrar} que 
        cada cadena en $S$ termina en $1$.
    }

    % Ejercicio 09
    \question
    {
        El conjunto de cadenas $T$ está definido recursivamente como sigue:
        \begin{itemize}
            \item [i.] $a \in T$
            \item [ii.] Si $t \in T$, entonces $ta, tb \in T$
            \item [iii.] No hay nada en $T$ que no sean objetos definidos con 
            las reglas anteriores.  
        \end{itemize}

        Utiliza \textbf{inducción estructural} para \textbf{demostrar} que 
        cada cadena en $T$ comienza con una $a$
    }

    % Ejercicio 10
    \question
    {
        El conjunto de cadenas $S$ está definido recursivamente como sigue: 
        \begin{itemize}
            \item [i.] $\epsilon \in S$
            \item [ii.] Si $s \in S$, entonces $bs, sb, saa, aas \in S$.
            \item [iii.] No hay nada en $S$ que no sean objetos definidos con 
            las reglas anteriores.  
        \end{itemize}

        Utiliza \textbf{inducción estructural} para \textbf{demostrar} que 
        cada cadena en $S$ tiene un número par de $a's$.
    }

    % Ejercicio 11
    \question
    {
        El conjunto de cadenas $T$ está definido recursivamente como sigue:
        \begin{itemize}
            \item [i.] $1,2,3,4,5,6,7,8,9 \in T$
            \item [ii.] Si $t, u \in T$, entonces $t0, tu \in T$
            \item [iii.] No hay nada en $T$ que no sean objetos definidos con 
            las reglas anteriores.  
        \end{itemize}

        Utiliza \textbf{inducción estructural} para \textbf{demostrar} que 
        ninguna cadena en $T$ representa un entero con un cero principal. 
    }

    % Ejercicio 12
    \question
    {
        El conjunto de cadenas $S$ está definido recursivamente como sigue: 
        \begin{itemize}
            \item [i.] $1,3,5,7,9 \in S$
            \item [ii.] Si $s,t \in S$, entonces $st, 2s, 4s, 6s, 8s \in S$.
            \item [iii.] No hay nada en $S$ que no sean objetos definidos con 
            las reglas anteriores.  
        \end{itemize}

        Utiliza \textbf{inducción estructural} para \textbf{demostrar} que 
        cadena en $S$ representa un entero impar. 
    }

    % Ejercicio 13
    \question
    {
        El conjunto de cadenas $T$ está definido recursivamente como sigue:
        \begin{itemize}
            \item [i.] $\epsilon \in T$
            \item [ii.] Si $x \in T$, entonces $1x0, 0x1 \in T$
            \item [iii.] No hay nada en $T$ que no sean objetos definidos con 
            las reglas anteriores.  
        \end{itemize}

        Utiliza \textbf{inducción estructural} para \textbf{demostrar} que
        cadena en $T$ contiene un número igual de ceros y unos. 
    }

    % Ejercicio 14
    \question
    {
        Un conjunto de árboles binarios está definido recursivamente 
        como sigue:
        \begin{itemize}
            \item [i.] Si $a \in A$, entonces \texttt{hoja(a)} es un árbol.
            \item [ii.] Si $t_1, t_2$ son árboles, entonces \texttt{mk($t_1,t_2$)} 
            es un árbol. 
            \item [iii.] Éstos y sólo éstos son árboles. 
        \end{itemize}

        \textbf{Resuelve} lo siguiente: 

        \begin{itemize}
            \item \textbf{Define recursivamente} las funciones \texttt{nh} y 
            \texttt{nni} que regresen el número de hojas y el número de nodos 
            internos de este tipo de árboles, respectivamente.   

            \item \textbf{Demuestra}, usando \textbf{inducción estructural} y 
            las funciones recursivas anteriores, que:   
            \begin{center}
                \tcbox[tcbox raise base]{\texttt{nh(t) = nni(t) + 1}}
            \end{center}
        \end{itemize}
    }

    % Ejercicio 15
    \question
    {
        Sea $\mathcal{T}$ un árbol binario. Utiliza \textbf{inducción 
        estructural} para \textbf{demostrar} que si la altura de $\mathcal{T}$ 
        es $n$, entonces el mínimo número de nodos en $\mathcal{T}$ es de $n$. 
    }

    % Ejercicio 16
    \question
    {
        Sea $\mathcal{T}$ un árbol binario. Utiliza \textbf{inducción 
        estructural} para \textbf{demostrar} que si la altura de $\mathcal{T}$ 
        es $n$, entonces el máximo número de nodos internos es de $2^{n-1}-1$.
    }

    % Ejercicio 17
    \question
    {
        ¿Cuál o cuáles son las similitudes y diferencias entre la inducción 
        matemática y la inducción estructural?
    }

    % Ejercicio 18
    \question
    {
        ¿Es posible realizar una demostración por inducción dentro de otra 
        demostración por inducción? En caso afirmativo, \textbf{muestra} un
        ejemplo. En caso contrario, \textbf{explica} por qué.  
    }

    % Ejercicio 19
    \question
    {
        La función \texttt{sp} está definida como sigue:
        \begin{center}
            \texttt{sp(k, $[a_1, \ldots, a_n]$) $= a_1^k + \ldots + a_n^k$}
        \end{center}

        Por otro lado, la función reversa está definida como sigue:
        \begin{center}
            \texttt{reversa($[a_1, \ldots, a_n]$) $= [a_n, \ldots, a_1]$}
        \end{center}

        \begin{itemize}
            \item \textbf{Define recursivamente} las funciones \texttt{sp} y 
            \texttt{reversa}. 

            \item \textbf{Demuestra}, usando \textbf{inducción estructural} y 
            las funciones recursivas anteriores, que:  
            \begin{center}
                \tcbox[tcbox raise base]{\texttt{sp($l$) = sp(reversa($l$))}}
            \end{center}
        \end{itemize}
    }

    % Ejercicio 20
    \question
    {
        Sea $A$ una fórmula en $\mathcal{LPROP}$. 
        \begin{itemize}
            \item \textbf{Define recursivamente} las funciones \texttt{nc} y 
            \texttt{np}, las cuales regresan el número de conectivos lógicos en 
            $A$ y el número de variables proposicionales en $A$, respectivamente. 
            \item \textbf{Demuestra}, usando \textbf{inducción estructural} y 
            las funciones recursivas anteriores, que: 
            \begin{center}
                \tcbox[tcbox raise base]{\texttt{nc} $\leq$ \texttt{np}}
            \end{center}
        \end{itemize}
    }
\end{questions}
\end{document}
