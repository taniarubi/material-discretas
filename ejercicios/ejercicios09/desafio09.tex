\documentclass[oneside]{style}

\title{Desafío 09}
\principal{Lógica Proposicional: La venganza}
\author{Tania Michelle Rubí Rojas}
\semester{Semestre 2023-1}

\begin{document}
\maketitle

Para cada uno de los siguientes ejercicios, \textbf{justifica ampliamente} tu 
respuesta:

\begin{questions}[label=\protect\circled{\bfseries\arabic*}]
    % Ejercicio 01
    \question
    {
        \textbf{Realiza} lo siguiente:
        \begin{itemize}
            \item \textbf{Traduce} la siguiente oración al lenguaje de la 
            lógica proposicional. 

            \begin{quote}
                \centering
                Mañana es enero sólo si hoy es víspera de año nuevo. 
            \end{quote}

            \item \textbf{Escribe} la \textbf{negación} de la oración anterior 
            en lenguaje español y \textbf{tradúcelo} al lenguaje de la lógica 
            proposicional.

            \item \textbf{Determina} si las dos traducciones anteriores son 
            una tautología, una contradicción o una contingencia. 
        \end{itemize}
    }

    % Ejercicio 02
    \question
    {

        \textbf{Realiza} lo siguiente:
        \begin{itemize}
            \item Para el siguiente argumento lógico, marca con color 
            \textcolor{orange}{\textbf{naranja}} la(s) premisa(s) y marca con 
            color \textcolor{blue}{\textbf{azul}} la conclusión.

            \begin{quote}
                \centering
                Si todos los números enteros son racionales, entonces 
                el número $1$ es racional. Todos los números enteros son 
                racionales. Por lo tanto, el número $1$ es racional.
            \end{quote}
    
            \item \textbf{Traduce} el argumento anterior al lenguaje de la lógica 
            proposicional.
    
            \item Utilizando la traducción que construiste en el inciso anterior, 
            \textbf{determina} si el argumento lógico es correcto usando el 
            concepto de \textbf{consecuencia lógica}.
        \end{itemize}
    }

    % Ejercicio 03
    \question
    {
        \textbf{Determina}, usando \textbf{interpretaciones}, si los siguientes 
        conjuntos de fórmulas son satisfacibles. En caso afirmativo, 
        \textbf{muestra} un modelo que los satisfaga.
        \begin{itemize}
            \item $\Gamma = \{p \rightarrow q, \neg q \lor r, p \land \neg r\}$
            \item $\Gamma = \{(p \lor q) \rightarrow r, \neg((\neg p \land 
            \neg q) \lor r)\}$
        \end{itemize}
    }

    % Ejercicio 04
    \question
    {
        \textbf{Realiza} lo siguiente:
        \begin{itemize}
            \item \textbf{Traduce} la siguiente oración al lenguaje de la 
            lógica proposicional. 

            \begin{quote}
                \centering
                Si Tommy es el padre de Amanda, entonces Charly es su tío y 
                Susana es su tía. 
            \end{quote}

            \item \textbf{Escribe} la \textbf{negación} de la oración anterior 
            en lenguaje español y \textbf{tradúcelo} al lenguaje de la lógica 
            proposicional.

            \item \textbf{Determina} si las dos traducciones anteriores son 
            una tautología, una contradicción o una contingencia. 
        \end{itemize}
    }

    % Ejercicio 05
    \question
    {

        \textbf{Realiza} lo siguiente:
        \begin{itemize}
            \item Para el siguiente argumento lógico, marca con color 
            \textcolor{orange}{\textbf{naranja}} la(s) premisa(s) y marca con 
            color \textcolor{blue}{\textbf{azul}} la conclusión.

            \begin{quote}
                \centering
                Este número es par o este número es impar. Este número no es 
                par. Por lo tanto, este número es impar.
            \end{quote}
    
            \item \textbf{Traduce} el argumento anterior al lenguaje de la lógica 
            proposicional.
    
            \item Utilizando la traducción que construiste en el inciso anterior, 
            \textbf{determina} si el argumento lógico es correcto usando el 
            concepto de \textbf{consecuencia lógica}.
        \end{itemize}
    }

    % Ejercicio 06
    \question
    {
        \textbf{Determina}, usando \textbf{interpretaciones}, si los siguientes 
        conjuntos de fórmulas son satisfacibles. En caso afirmativo, 
        \textbf{muestra} un modelo que los satisfaga.
        \begin{itemize}
            \item $\Gamma = \{p \leftrightarrow q, q \leftrightarrow s, p, 
            \neg s\}$
            \item $\Gamma = \{(p \land q) \rightarrow r, \neg r, \neg p\}$
        \end{itemize}
    }

    % Ejercicio 07
    \question
    {
        \textbf{Realiza} lo siguiente:
        \begin{itemize}
            \item \textbf{Traduce} la siguiente oración al lenguaje de la 
            lógica proposicional. 

            \begin{quote}
                \centering
                Si $n$ es divisible entre $6$, entonces $n$ es divisible 
                entre $2$ y $3$. 
            \end{quote}

            \item \textbf{Escribe} la \textbf{negación} de la oración anterior 
            en lenguaje español y \textbf{tradúcelo} al lenguaje de la lógica 
            proposicional.

            \item \textbf{Determina} si las dos traducciones anteriores son 
            una tautología, una contradicción o una contingencia. 
        \end{itemize}
    }

    % Ejercicio 08
    \question
    {
        \textbf{Determina} si son \textbf{verdaderas} las siguientes 
        equivalencias lógicas:
        \begin{itemize}
            \item $(p \lor q) \rightarrow r \equiv (p \rightarrow r) \land 
            (q \rightarrow r)$

            \item $(p \land q) \rightarrow r \equiv (p \rightarrow r) \lor 
            (q \rightarrow r)$
        \end{itemize}
    }

    % Ejercicio 09
    \question
    {

        \textbf{Realiza} lo siguiente:
        \begin{itemize}
            \item Para el siguiente argumento lógico, marca con color 
            \textcolor{orange}{\textbf{naranja}} la(s) premisa(s) y marca con 
            color \textcolor{blue}{\textbf{azul}} la conclusión.

            \begin{quote}
                \centering
                Si todos los programas de computadora contienen errores, 
                entonces este programa contiene un error. Este programa no 
                contiene un error. Por lo tanto, no pasa que todos los 
                programas de computadora tengan errores.
            \end{quote}
    
            \item \textbf{Traduce} el argumento anterior al lenguaje de la lógica 
            proposicional.
    
            \item Utilizando la traducción que construiste en el inciso anterior, 
            \textbf{determina} si el argumento lógico es correcto usando el 
            \textbf{método de refutación}.
        \end{itemize}
    }

    % Ejercicio 10
    \question
    {
        \textbf{Determina}, usando \textbf{interpretaciones}, si las 
        siguientes fórmulas son satisfacibles, tautologías o 
        contradicciones. 
        \begin{itemize}
            \item $p \lor (\neg p \land q) \rightarrow p \lor q$
            \item $(\neg p \lor q) \rightarrow ((p \land r) \leftrightarrow
            ((s \land t) \rightarrow (u \lor p)))$
        \end{itemize}
    }

    % Ejercicio 11
    \question
    {
        \textbf{Determina}, usando \textbf{interpretaciones}, si los siguientes 
        conjuntos de fórmulas son satisfacibles. En caso afirmativo, 
        \textbf{muestra} un modelo que los satisfaga.
        \begin{itemize}
            \item $\Gamma = \{(p \lor q) \rightarrow r, p, r \rightarrow t, 
            \neg (t \lor q)\}$
            \item $\Gamma = \{p \rightarrow q, p \lor r \land s, q \rightarrow 
            t\}$
        \end{itemize}
    }

    % Ejercicio 12
    \question
    {
        \textbf{Determina} si los siguientes argumentos son correctos o no. En 
        caso de no serlo, \textbf{da} una interpretación que haga verdaderas a 
        las premisas y falsa a la conclusión. 
        \begin{itemize}
            \item $\{p \rightarrow q, q \lor r, \neg (r \land s)\} \models 
            (p \rightarrow q) \rightarrow (q \lor \neg s)$

            \item $\{p \lor q, \neg (p \land r), \neg q\} \models r \rightarrow
            s$
        \end{itemize}
    }

    % Ejercicio 13
    \question
    {
        \textbf{Determina} si son \textbf{verdaderas} las siguientes 
        equivalencias lógicas:
        \begin{itemize}
            \item $(\neg p \rightarrow (q \land \neg q)) \equiv p$

            \item $p \lor (p \land q) \equiv p$
        \end{itemize}
    }

    \newpage
    % Ejercicio 14
    \question
    {
        \textbf{Realiza} lo siguiente:
        \begin{itemize}
            \item Para el siguiente argumento lógico, marca con color 
            \textcolor{orange}{\textbf{naranja}} la(s) premisa(s) y marca con 
            color \textcolor{blue}{\textbf{azul}} la conclusión.

            \begin{quote}
                \centering
                César estudia la licenciatura en Ciencia de Datos o César 
                estudia la licenciatura en Economía. Si Oleg estudia la 
                licenciatura en Ciencia de Datos, entonces César cursa la 
                materia de Bases de Datos. Por lo tanto, César estudia la 
                licenciatura en Economía o César no requiere cursar la materia 
                de Bases de Datos.
            \end{quote}
    
            \item \textbf{Traduce} el argumento anterior al lenguaje de la lógica 
            proposicional.
    
            \item Utilizando la traducción que construiste en el inciso anterior, 
            \textbf{determina} si el argumento lógico es correcto usando el 
            \textbf{método de refutación}.
        \end{itemize}
    }

    % Ejercicio 15
    \question
    {
        \textbf{Realiza} lo siguiente:
        \begin{itemize}
            \item Para cada una de las siguientes fórmulas, \textbf{elimina} 
            los paréntesis superfluos si es que hay.
            \begin{itemize}
                \item[i)] $((p \rightarrow r) \leftrightarrow (q \rightarrow r))$
                \item[ii)] $(((\neg p) \lor q) \rightarrow r)$
            \end{itemize}
            
            \item \textbf{Determina} si cada expresión se trata de una 
            tautología, una contradicción o una contingencia.
        \end{itemize}
    }

    % Ejercicio 16
    \question
    {
        \textbf{Realiza} lo siguiente:
        \begin{itemize}
            \item \textbf{Traduce} la siguiente oración al lenguaje de la 
            lógica proposicional. 

            \begin{quote}
                \centering
                Hoy no es lunes, pero tal vez mañana sea octubre.
            \end{quote}

            \item \textbf{Escribe} la \textbf{negación} de la oración anterior 
            en lenguaje español y \textbf{tradúcelo} al lenguaje de la lógica 
            proposicional.

            \item \textbf{Determina} si las dos traducciones anteriores son 
            una tautología, una contradicción o una contingencia. 
        \end{itemize}
    }

    % Ejercicio 17
    \question
    {
        \textbf{Determina} si son \textbf{verdaderas} las siguientes 
        equivalencias lógicas:
        \begin{itemize}
            \item $p \lor (q \land r) \equiv (p \lor q) \land (p \lor r)$

            \item $\neg (p \rightarrow q) \equiv p \land \neg q$
        \end{itemize}
    }

    % Ejercicio 18
    \question
    {
        \textbf{Realiza} lo siguiente:
        \begin{itemize}
            \item Para el siguiente argumento lógico, marca con color 
            \textcolor{orange}{\textbf{naranja}} la(s) premisa(s) y marca con 
            color \textcolor{blue}{\textbf{azul}} la conclusión.

            \begin{quote}
                \centering
                Si Sebastián no está en el equipo A, entonces Tommy está en el 
                equipo B. Si Tommy no está en el equipo B, entonces Sebastián 
                está en el equipo A. Por lo tanto, Sebastián no está en el 
                equipo A o Tommy no está en el equipo B
            \end{quote}
    
            \item \textbf{Traduce} el argumento anterior al lenguaje de la lógica 
            proposicional.
    
            \item Utilizando la traducción que construiste en el inciso anterior, 
            \textbf{determina} si el argumento lógico es correcto usando el 
            \textbf{método de refutación}.
        \end{itemize}
    }

    % Ejercicio 19
    \question
    {
        \textbf{Realiza} lo siguiente:
        \begin{itemize}
            \item Para cada una de las siguientes fórmulas, \textbf{elimina} 
            los paréntesis superfluos si es que hay.
            \begin{itemize}
                \item[i)] $((p \land (\neg q)) \lor p)$
                \item[ii)] $(((\neg p) \lor q) \lor q) \lor (p \land (\neg q))$
            \end{itemize}
            
            \item \textbf{Determina} si cada expresión se trata de una 
            tautología, una contradicción o una contingencia.
        \end{itemize}
    }

    \newpage
    % Ejercicio 20
    \question
    {
        \textbf{Realiza} lo siguiente:
        \begin{itemize}
            \item Para el siguiente argumento lógico, marca con color 
            \textcolor{orange}{\textbf{naranja}} la(s) premisa(s) y marca con 
            color \textcolor{blue}{\textbf{azul}} la conclusión.

            \begin{quote}
                \centering
                Si Ana es buena nadadora, entonces ella es buena corredora. 
                Si Ana es buena corredora, entonces ella es una buena ciclista. 
                Por lo tanto, si Ana es buena nadadora entonces ella es una 
                buena ciclista.
            \end{quote}
    
            \item \textbf{Traduce} el argumento anterior al lenguaje de la lógica 
            proposicional.
    
            \item Utilizando la traducción que construiste en el inciso anterior, 
            \textbf{determina} si el argumento lógico es correcto usando el 
            concepto de \textbf{consecuencia lógica}.
        \end{itemize}
    }

    % Ejercicio 21
    \question
    {
        \textbf{Realiza} lo siguiente:
        \begin{itemize}
            \item Para cada una de las siguientes fórmulas, \textbf{agrega} 
            los paréntesis necesarios de acuerdo a su precedencia y 
            asociatividad. 
            \begin{itemize}
                \item[i)] $p \land \neg q \land r \leftrightarrow p \land r 
                \land \neg q$
                \item[ii)] $\neg (\neg p) \leftrightarrow$
            \end{itemize}
            
            \item \textbf{Determina} si cada expresión se trata de una 
            tautología, una contradicción o una contingencia.
        \end{itemize}
    }

    % Ejercicio 22
    \question
    {
        \textbf{Realiza} lo siguiente:
        \begin{itemize}
            \item \textbf{Traduce} la siguiente oración al lenguaje de la 
            lógica proposicional. 

            \begin{quote}
                \centering
                El servicio es excelente cuando la comida es buena. 
            \end{quote}

            \item \textbf{Escribe} la \textbf{negación} de la oración anterior 
            en lenguaje español y \textbf{tradúcelo} al lenguaje de la lógica 
            proposicional.

            \item \textbf{Determina} si las dos traducciones anteriores son 
            una tautología, una contradicción o una contingencia. 
        \end{itemize}
    }

    % Ejercicio 23
    \question
    {
        \textbf{Define} una función recursiva que reciba una fórmula 
        proposicional y una lista de tuplas cuya primer componente sea una 
        variable proposicional y cuya segunda componente sea un elemento del 
        conjunto ${\texttt{true, false}}$. Esta función nos debe regresar el 
        valor de verdad de la expresión en  considerando que los valores de 
        verdad de sus variables proposicionales son los que se especifican 
        en la lista. 
        \begin{itemize}
            \item Debes \textbf{definir} la firma de la función y 
            \textbf{describirla}.

            \item \textbf{Explica} por qué tu función está bien definida. 

            \item \textbf{Ejecuta} tu función con las expresiones con dos 
            ejemplos no triviales. 
        \end{itemize}
    }

    % Ejercicio 24
    \question
    {
        \textbf{Determina}, usando \textbf{interpretaciones}, si los siguientes 
        conjuntos de fórmulas son satisfacibles. En caso afirmativo, 
        \textbf{muestra} un modelo que los satisfaga.
        \begin{itemize}
            \item $\Gamma = \{\neg q \land r \lor p \lor q, p \land r\}$
            \item $\Gamma = \{p \land \neg q, \neg (q \lor \neg p), q \land p
            \lor q \lor \neg p\}$
        \end{itemize}
    }

    % Ejercicio 25
    \question
    {
        \textbf{Determina} si los siguientes argumentos son correctos o no. En 
        caso de no serlo, \textbf{da} una interpretación que haga verdaderas a 
        las premisas y falsa a la conclusión. 
        \begin{itemize}
            \item $\{p \lor q, p \rightarrow r, q \rightarrow r\} \models r$

            \item $\{r \land s \rightarrow t, \neg t\} \models t \rightarrow q$
        \end{itemize}
    }

    % Ejercicio 26
    \question
    {
        \textbf{Realiza} lo siguiente:
        \begin{itemize}
            \item \textbf{Traduce} la siguiente oración al lenguaje de la 
            lógica proposicional. 

            \begin{quote}
                \centering
                Las flores florecerán sólo si llueve.  
            \end{quote}

            \item \textbf{Escribe} la \textbf{negación} de la oración anterior 
            en lenguaje español y \textbf{tradúcelo} al lenguaje de la lógica 
            proposicional.

            \item \textbf{Determina} si las dos traducciones anteriores son 
            una tautología, una contradicción o una contingencia. 
        \end{itemize}
    }

    % Ejercicio 27
    \question
    {
        \textbf{Determina} si son \textbf{verdaderas} las siguientes 
        equivalencias lógicas:
        \begin{itemize}
            \item $q \rightarrow p \equiv \neg p \rightarrow \neg q$

            \item $\neg (p \lor (\neg p \land q)) \equiv \neg (p \lor q)$
        \end{itemize}
    }

    % Ejercicio 28
    \question
    {
        \textbf{Determina} si los siguientes argumentos son correctos o no. En 
        caso de no serlo, \textbf{da} una interpretación que haga verdaderas a 
        las premisas y falsa a la conclusión. 
        \begin{itemize}
            \item $\{\neg q \rightarrow \neg r, \neg r \rightarrow \neg p, 
            \neg p \rightarrow \neg q\} \models q \leftrightarrow r$

            \item $\{p, \neg q\} \models \neg (p \rightarrow q)$
        \end{itemize}
    }

    % Ejercicio 29
    \question
    {
        \textbf{Determina}, usando \textbf{interpretaciones}, si los siguientes 
        conjuntos de fórmulas son satisfacibles. En caso afirmativo, 
        \textbf{muestra} un modelo que los satisfaga.
        \begin{itemize}
            \item $\Gamma = \{q \lor r \lor s, \neg (q \lor r), \neg (r \lor s), 
            \neg (s \lor q)\}$
            \item $\Gamma = \{\neg (p \land q) \land \neg (p \land r), q \lor r, 
            \neg (p \lor \neg r)\}$
        \end{itemize}
    }

    % Ejercicio 30
    \question
    {
        \textbf{Realiza} lo siguiente:
        \begin{itemize}
            \item \textbf{Traduce} la siguiente oración al lenguaje de la 
            lógica proposicional. 

            \begin{quote}
                \centering
                Mi práctica de ICC es complicada y estoy triste. 
            \end{quote}

            \item \textbf{Escribe} la \textbf{negación} de la oración anterior 
            en lenguaje español y \textbf{tradúcelo} al lenguaje de la lógica 
            proposicional.

            \item \textbf{Determina} si las dos traducciones anteriores son 
            una tautología, una contradicción o una contingencia. 
        \end{itemize}
    }
\end{questions}
\end{document}
