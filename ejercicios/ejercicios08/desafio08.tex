\documentclass[oneside]{style}

\title{Desafío 08}
\principal{Lógica Proposicional}
\author{Tania Michelle Rubí Rojas}
\semester{Semestre 2023-1}

\begin{document}
\maketitle

Para cada uno de los siguientes ejercicios, \textbf{justifica ampliamente} tu 
respuesta:

\begin{questions}[label=\protect\circled{\bfseries\arabic*}]
    % Ejercicio 01
    \question
    {
        Sean $p$ la proposición "He comprado un billete de lotería" y $q$ la 
        proposición "Gané un premio". ¿Cómo se escriben las siguientes 
        expresiones en oraciones naturales en Español?
        \begin{tasks}(4)
            \task $\neg p$
            \task $\neg p \lor (p \land q)$
            \task $\neg (\neg p) \lor \neg q$
            \task $\neg \lor (p \land q)$
        \end{tasks}
    }

    % Ejercicio 02
    \question
    {
        ¿Cuáles de las siguientes son fórmulas proposicionales bien formadas?
        \begin{tasks}(4)
            \task $\lor pq$
            \task $(\neg (p \rightarrow (q \land p)))$
            \task $p \neg r$
            \task $(p \land \neg q) \lor (q \rightarrow r)$
        \end{tasks}
    }

    % Ejercicio 03
    \question
    {
        \textbf{Define} una función recursiva que dada una expresión de la 
        lógica proposicional, nos diga si se repite alguna variable 
        proposicional en dicha expresión.
        \begin{itemize}
            \item Debes \textbf{definir} la firma de la función, 
            \textbf{describirla} y \textbf{explicarla}. 

            \item \textbf{Explica} por qué tu función recursiva está bien 
            escrita. 

            \item \textbf{Ejecuta} tu función con las expresiones 
            \begin{itemize}
                \item[i)] $p \land q \lor \neg r \rightarrow s$
                \item[ii)] $p \leftrightarrow q \land p \lor \neg q$
            \end{itemize}
        \end{itemize}
    }

    % Ejercicio 04
    \question
    {
        \textbf{Construye} los árboles de sintaxis para las siguientes 
        expresiones, además de \textbf{indicar} cuál es el conectivo principal 
        de cada expresión. 
        \begin{tasks}(3)
            \task $5 + 6 \cdot (5 \cdot 9 / 8 \cdot 3)$
            \task $5 + 6 \cdot (7/ 8) + (5 / 0)$
            \task $2 + 3 + 4 \cdot 8 / (4 - 5)$
        \end{tasks}
    }

    % Ejercicio 05
    \question
    {
        Tomando en cuenta las fórmulas proposicionales y los árboles de 
        sintaxis, \textbf{responde} lo siguiente:
        \begin{itemize}
            \item Si $e_1$ es una subfórmula de $e_2$, ¿se cumple que 
            el árbol de sintaxis asociado a $e_1$ es un subárbol del árbol 
            de sintaxis asociado a $e_2$?

            \item ¿Cuántos nodos tiene el árbol de sintaxis de una fórmula 
            proposicional $e$?

            \item ¿Cuáles son los elementos que siempre están en las hojas 
            del árbol de sintaxis de una fórmula proposicional?
        \end{itemize}
    }

    % Ejercicio 06
    \question
    {
        Sean $\mathcal{A}_{\mathcal{LPROP}}$ el conjunto de todos los árboles 
        de sintaxis de la lógica proposicional y $\mathcal{LPROP}$ el 
        conjunto de todas las expresiones que son fórmulas proposicionales. 
        \textbf{Define} una función recursiva 
        \begin{center}
            \texttt{st}$: \mathcal{A}_{\mathcal{LPROP}} \rightarrow
            \mathcal{LPROP}$
        \end{center}

        que regrese la fórmula proposicional que le corresponde al árbol de 
        sintaxis que le pasamos como entrada.  
    }

    % Ejercicio 07
    \question
    {
        \textbf{Investiga} la notación BFN de la lógica proposicional 
        (tranquis, está en el libro de Favio bb). 
        \begin{itemize}
            \item \textbf{Da} la definición recursiva asociada a esa 
            gramática. 

            \item \textbf{Explica} por qué tu definición recursiva está bien 
            escrita. 

            \item \textbf{Construye} las siguientes expresiones utilizando tu 
            definición recursiva:
            \begin{itemize}
                \item[i)] $p \land q \lor s \rightarrow s$
                \item[ii)] $a \rightarrow \neg a \leftrightarrow b \land c$ 
            \end{itemize}
        \end{itemize}
    }

    \newpage
    % Ejercicio 08
    \question
    {
        Usando las reglas de precedencia y asociatividad de operaciones, 
        \textbf{elimina} los paréntesis superfluos de las siguientes 
        expresiones:
        \begin{tasks}(2)
            \task $(5 + 9) \cdot (8 \cdot (9 / 5) + 3)$
            \task $(((4 + (98 / 10) + -6) / 9)/(8 / (8 + 5)))$
        \end{tasks}
    }

    % Ejercicio 09
    \question
    {
        Para las siguientes gramáticas:
        \begin{equation*}
            \begin{aligned}[c]
                E &::= () \\
                E &::= (E) \\
                E &::= EE
            \end{aligned}
            \qquad \qquad \qquad \qquad
            \begin{aligned}[c]
                E &::= ( \\
                E &::= ) \\
                E &::= (EE \\ 
                E &::= EE) 
            \end{aligned}
        \end{equation*}

        \textbf{Escribe} su definición recursiva equivalente. ¿Sucede que las 
        definiciones recursivas equivalentes generan el mismo conjunto? 
    }

    % Ejercicio 10
    \question
    {
        \textbf{Construye} los árboles de sintaxis para las siguientes 
        expresiones, además de \textbf{indicar} cuál es el conectivo principal 
        de cada expresión. 
        \begin{tasks}(3)
            \task $1 + 1 + 1 + 1 + 1$
            \task $5 + 6 \cdot 8 \cdot 7 / 7 \cdot 9 / 7$
            \task $2 - 5 \cdot 9 \cdot (8 / 8 / -9)-/$
        \end{tasks}
    }

    % Ejercicio 11
    \question
    {
        Usando las reglas de precedencia y asociatividad de operaciones, 
        \textbf{elimina} los paréntesis superfluos de las siguientes 
        expresiones:
        \begin{tasks}(2)
            \task $((\neg p \rightarrow (r \land (q \leftrightarrow t)))) 
            \leftrightarrow (((p \lor r) \leftrightarrow \neg (s \lor p)))$
            \task $(\neg (p \rightarrow \neg q) \land (t \lor (s \land p) \lor r))$
        \end{tasks}
    }

    % Ejercicio 12
    \question
    {
        Para cada una de las siguientes expresiones, \textbf{responde}:
        \begin{center}
            ¿Es una proposición? En caso afirmativo, \textbf{indica} si 
            es una proposición simple o compuesta. 
        \end{center}

        Expresiones:
        \begin{itemize}
            \item Lee esto con cuidado.
            \item Es triste cuando todo el mundo sabe quién eres, pero nadie 
            te conoce. 
            \item Washington, D.C., es la capital de los Estados Unidos de 
            América. 
            \item Por un momento olvidamos todo lo que es difícil y nos 
            permitimos sentir lo que queríamos. 
            \item $x + 1 = 2$
        \end{itemize}
    }

    % Ejercicio 13
    \question
    {
        Para cada una de las siguientes expresiones
        \begin{align*}
            &(p \rightarrow (\neg q)) \\ 
            &p \rightarrow q \lor \neg p \rightarrow r \\ 
            &((p \lor (\neg r)) \land (q \lor (\neg s)))
        \end{align*}

        \textbf{Responde} lo siguiente:
        \begin{itemize}
            \item ¿Hay paréntesis superfluos en la expresión? En caso de que 
            sea cierto, \textbf{elimínalos} y \textbf{muestra} cómo queda la 
            expresión final. 

            \item ¿Cuántos posibles árboles de sintaxis se pueden crear para 
            dicha expresión? \textbf{Dibuja} cada uno de ellos. 
        \end{itemize}
    }

    % Ejercicio 14
    \question
    {
        Usando las reglas de precedencia y asociatividad de operaciones, 
        \textbf{elimina} los paréntesis superfluos de las siguientes 
        expresiones:
        \begin{tasks}(2)
            \task $(((p \lor q) \rightarrow (p \land q) \rightarrow e) 
            \leftrightarrow r)$
            \task $(((\neg p) \rightarrow p) \leftrightarrow ((\neg q) \land 
            \neg((p \lor q) \land (p \land p))) \leftrightarrow p)$
        \end{tasks}
    }

    \newpage
    % Ejercicio 15
    \question
    {
        Para cada una de las siguientes expresiones, \textbf{dibuja} su árbol 
        de sintaxis. 
        \begin{tasks}(2)
            \task $p \lor q \rightarrow r \rightarrow s \leftrightarrow t$
            \task $((\neg p \lor q) \rightarrow ((p \land r) \leftrightarrow 
            ((s \land t) \rightarrow (u \lor p))))$
        \end{tasks}
    }

    % Ejercicio 16
    \question
    {
        \textbf{Escribe} los paréntesis en las siguientes expresiones de 
        acuerdo a su precedencia y asociatividad de operadores.
        \begin{itemize}
            \item $p \leftrightarrow q \rightarrow s \leftrightarrow p \land q
            \leftrightarrow s \lor t \land u \land v \lor r \rightarrow s 
            \leftrightarrow r$

            \item $a < b < c < d \rightarrow b > c > d \rightarrow c < d 
            \leftrightarrow d$

            \item $p \rightarrow q \rightarrow r \rightarrow p \land q \lor s
            \rightarrow t \leftrightarrow u$
        \end{itemize}
    }

    % Ejercicio 17
    \question
    {
        Para cada una de las siguientes expresiones, \textbf{dibuja} su árbol 
        de sintaxis. 
        \begin{tasks}(2)
            \task $(((((p \land q) \lor r) \land s) \lor t) \land u)$
            \task $((((p \rightarrow q) \rightarrow r) \rightarrow s) 
            \rightarrow t)$
        \end{tasks}
    }

    % Ejercicio 18
    \question
    {
        Para cada una de las siguientes expresiones, \textbf{responde}:
        \begin{center}
            ¿Es una proposición? En caso afirmativo, \textbf{indica} si 
            es una proposición simple o compuesta. 
        \end{center}

        \begin{itemize}
            \item Para todo entero positivo $n$, existe un número primo mayor
            que $n$.  
            \item Después de que algo realmente malo sucede, la siguiente cosa 
            peor es que la gente se sienta mal acerca de ello. 
            \item El grito de Dolores, en $1810$, sentó las bases para la 
            independencia de México. 
            \item  Así que tal vez, cuando podemos decir cosas, cuando podemos 
            escribir las palabras, cuando podemos expresar cómo se siente, no 
            estamos tan indefensos. 
            \item Para pasar el examen es necesario que los alumnos estudien, 
            hagan la tarea y asistan a clase. 
        \end{itemize}
    }

    % Ejercicio 19
    \question
    {
        \textbf{Escribe} los paréntesis en las siguientes expresiones de 
        acuerdo a su precedencia y asociatividad de operadores.
        \begin{itemize}
            \item $2/8/5 \cdot a \cdot 6 \cdot 9 - 8 + b**4 \cdot c - 5 
            + c + d/d$

            \item $a \cdot b - a \cdot c \leftrightarrow a > 0 \land 
            b >c$

            \item $a < b \land b < c \rightarrow a < c$
        \end{itemize}
    }

    % Ejercicio 20
    \question
    {
        Para cada una de las siguientes expresiones, \textbf{responde}:
        \begin{center}
            ¿Es una proposición? En caso afirmativo, \textbf{indica} si 
            es una proposición simple o compuesta. 
        \end{center}

        \begin{itemize}
            \item ¿Por qué no bailamos? 
            \item Cada uno de nosotros somos raros de diferentes maneras, pero 
            en conjunto, eso es en realidad normal. 
            \item $a^3 + 3a^2b + 3ab^2 +a^3$
            \item Un amigo es alguien que te da libertad total de ser tú mismo, 
            especialmente para sentir o para no sentir. 
            \item Una vez que le temes a algo, te pueden dar miedo muchas cosas. 
        \end{itemize}
    }

    % Ejercicio 21
    \question
    {
        Para cada una de las siguientes expresiones, \textbf{responde}:
        \begin{center}
            ¿Es una proposición? En caso afirmativo, \textbf{indica} si 
            es una proposición simple o compuesta. 
        \end{center}

        \begin{itemize}
            \item No me digas qué hacer. 
            \item ¿Tenemos examen?
            \item Mi práctica de ICC es complicada y estoy triste. 
            \item Este perrito es cariñoso y da besitos. 
            \item Si la melancolía es tu condena, líbrate de cadenas y ponte 
            a vivir. 
        \end{itemize}
    }

    \newpage
    % Ejercicio 22
    \question
    {
        Para cada una de las siguientes expresiones, \textbf{responde}:
        \begin{center}
            ¿Es una proposición? En caso afirmativo, \textbf{indica} si 
            es una proposición simple o compuesta. 
        \end{center}

        \begin{itemize}
            \item Aprende a soñar y podrás tener el Sol. 
            \item Suaves son esas palabras, aunque sean falsas promesas. 
            \item Lo único que tengo de tí es este triste lamento. 
            \item Si no te oigo, escúchame. 
            \item Si no hay palabras, quédate. 
        \end{itemize}
    }

    % Ejercicio 23
    \question
    {
        Para cada una de las siguientes expresiones, \textbf{responde}:
        \begin{center}
            ¿Es una proposición? En caso afirmativo, \textbf{indica} si 
            es una proposición simple o compuesta. 
        \end{center}

        \begin{itemize}
            \item Si no te he entendido, ayúdame. 
            \item Soy un kiwi. 
            \item Tienes la mirada más bonita que he visto en mi vida. 
            \item Pasaste, te ví y pense: "Él no es para mí"
            \item Ya es muy tarde. 
        \end{itemize}
    }

    % Ejercicio 24
    \question
    {
        Para cada una de las siguientes expresiones, \textbf{responde}:
        \begin{center}
            ¿Es una proposición? En caso afirmativo, \textbf{indica} si 
            es una proposición simple o compuesta. 
        \end{center}

        \begin{itemize}
            \item Sólo ven y ayúdame con una duda. 
            \item $2$ es el único número primo que es par. 
            \item Él es el mejor estudiante de la clase. 
            \item Kakarotto, ¡eres el número $1$!
            \item John será un excelente programador solo si él aprueba su 
            curso de matemáticas discretas. 
        \end{itemize}
    }

    % Ejercicio 25
    \question
    {
        Para cada una de las siguientes expresiones, \textbf{responde}:
        \begin{center}
            ¿Es una proposición? En caso afirmativo, \textbf{indica} si 
            es una proposición simple o compuesta. 
        \end{center}

        \begin{itemize}
            \item Escribiré tu historia cuando ya no estés. 
            \item Hay que buscar la solución y desafiar las reglas con el 
            corazón. 
            \item Ya es muy tarde para la cordura, para decirte que no. 
            \item Cuando andes perdido, sin rumbo y te quieras rendir; 
            para y piensa en mí. 
            \item Si los precios suben, Mr. Satán no podrá comprar 
            chocolates para Majin Buu. 
        \end{itemize}
    }

    % Ejercicio 26
    \question
    {
        Para cada una de las siguientes expresiones, \textbf{responde}:
        \begin{center}
            ¿Es una proposición? En caso afirmativo, \textbf{indica} si 
            es una proposición simple o compuesta. 
        \end{center}

        \begin{itemize}
            \item Ir a estudiar o ir a entrenar es condición suficiente 
            para que Gohan pueda ir a dormir. 
            \item O bien Freezer es el mejor guerrero o bien es el peor 
            guerrero, pero no ambos. 
            \item Si goku gana o pierde, estará cansad y tendrá mucha hambre. 
            \item Los árboles son estructuras de datos que pueden parecer 
            difíciles en un inicio, pero estudiando y programándolos es 
            más sencillo comprenderlos. 
            \item Voy a pedir un Uber para que no me lleve al médico.
        \end{itemize}
    }

    \newpage
    % Ejercicio 27
    \question
    {
        Para cada una de las siguientes expresiones, \textbf{responde}:
        \begin{center}
            ¿Es una proposición? En caso afirmativo, \textbf{indica} si 
            es una proposición simple o compuesta. 
        \end{center}

        \begin{itemize}
            \item Voy a pedir un Uber para que no me lleve al médico. 
            \item $2 + 4 < 8$
            \item No hace calor, pero hay Sol. 
            \item Te extraño. 
            \item Los perritos son amor. 
        \end{itemize}
    }

    % Ejercicio 28
    \question
    {
        Para cada una de las siguientes expresiones, \textbf{responde}:
        \begin{center}
            ¿Es una proposición? En caso afirmativo, \textbf{indica} si 
            es una proposición simple o compuesta. 
        \end{center}

        \begin{itemize}
            \item No quiero nada de tí. 
            \item Me tropecé con tu forma perfecta de besar. 
            \item Isa tiene conocimientos de Machine Learning y Seguridad 
            Informática. 
            \item Por favor, por un minuto ponte en mi lugar. 
            \item Leonardo es fiel fanático del América. 
        \end{itemize}
    }

    % Ejercicio 29
    \question
    {
        Para cada una de las siguientes expresiones, \textbf{responde}:
        \begin{center}
            ¿Es una proposición? En caso afirmativo, \textbf{indica} si 
            es una proposición simple o compuesta. 
        \end{center}

        \begin{itemize}
            \item Dame una razón para no desconfiar de todas tus promesas. 
            \item El pastel es delicioso, pero no tiene suficiente chocolate. 
            \item Las cortinas están rotas sólo si mi gato las rompió. 
            \item Este crimen yo no lo cometí. 
            \item Si el programa es eficiente, entonces se ejecuta eficientemente.
        \end{itemize}
    }

    % Ejercicio 30
    \question
    {
        Para cada una de las siguientes expresiones, \textbf{responde}:
        \begin{center}
            ¿Es una proposición? En caso afirmativo, \textbf{indica} si 
            es una proposición simple o compuesta. 
        \end{center}

        \begin{itemize}
            \item Te admiro mucho. 
            \item Todos los políticos son corruptos. 
            \item Ahora estoy aquí buscándola y no está. 
            \item Te acercaste, me invitaste, me negué. Pero la curiosidad 
            me ganó y acepté. 
            \item No todas las personas que nacieron en India viven en India. 
        \end{itemize}
    }
\end{questions}
\end{document}
