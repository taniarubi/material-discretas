\documentclass[oneside]{style}

\title{Desafío 07}
\principal{Relaciones de Orden y de Equivalencia}
\author{Tania Michelle Rubí Rojas}
\semester{Semestre 2023-1}

\begin{document}
\maketitle

Para cada uno de los siguientes ejercicios, \textbf{justifica ampliamente} tu 
respuesta:

\begin{questions}[label=\protect\circled{\bfseries\arabic*}]
    % Ejercicio 01
    \question
    {
        Sea $A = \{a,b,c\}$.
        \begin{itemize}
            \item \textbf{Describe} todas las relaciones de orden parcial sobre 
            $A$.
            \item \textbf{Describe} todas las relaciones de orden parcial sobre 
            $A$ para las que $a$ es un elemento máximo. 
            \item \textbf{Describe} todas las relaciones de orden parcial sobre 
            $A$ para las que $b$ es un elemento maximal.
        \end{itemize} 
    }

    % Ejercicio 02
    \question
    {
        Sea $\mathcal{E}$ el conjunto de todas los estados de México que tengan 
        aeropuertos. Definimos la relación $\sim$ sobre $\mathcal{E}$ como sigue:
        \begin{equation*}
            x \sim y \Leftrightarrow \text{es posible tomar un avión desde el 
            estado } x \text{ y llegar al estado } y \text{ usando vuelo 
            directo}
        \end{equation*}

        Nota: Se considera que existe un vuelo entre dos estados si 
        ambos estados tienen un aeropuerto. 

        \textbf{Realiza} lo siguiente: 
        \begin{itemize}
            \item \textbf{Describe} la cerradura transitiva de $\sim$.

            \item ¿La cerradura transitiva de $\sim$ es un orden parcial? 
            ¿Por qué?

            \item En caso de que $\sim$ sea de orden parcial, \textbf{responde} 
            lo siguiente:
            \begin{itemize}
                \item ¿Cuáles son los elementos maximales y minimales de 
                $(\mathcal{E}, \sim)$?
                \item ¿Cuál es el elemento mínimo y máximo de $(\mathcal{E}, 
                \sim)$?
            \end{itemize}

            \item Sea $\sim \sim$ la relación sobre el conjunto 
            \begin{equation*}
                M = \{\text{Baja California Sur, CDMX, Nuevo León, Yucatán}\} 
            \end{equation*}

            tal que 
            \begin{align*}
                \sim \sim = \{(\text{CDMX, Nuevo León}), (\text{Nuevo León, 
                Baja California Sur}), \\ (\text{CDMX, Yucatán}),(\text{CDMX,
                CDMX}),(\text{Nuevo León, Nuevo León}), \\ (\text{Baja 
                California Sur, Baja California Sur}), (\text{Yucatán, 
                Yucatán})\} 
            \end{align*}

            Si $\sim \sim$ es de orden parcial, \textbf{dibuja} su diagrama de 
            Hasse. En caso contrario, \textbf{explica} por qué no es de orden 
            parcial. 
        \end{itemize}
    }

    % Ejercicio 03
    \question
    {
        \textbf{Realiza} lo siguiente:
        \begin{itemize}
            \item \textbf{Encuentra} todas las posibles particiones del conjunto
            $\{a,b,c,d\}$, donde $a,b,c$ y $d$ son todos distintos entre sí.

            \item Para cada una de las particiones del inciso anterior, 
            \textbf{da} la relación de equivalencia asociada. 
        \end{itemize}
    }

    % Ejercicio 04
    \question
    {
        \textbf{Determina} si las siguientes afirmaciones son verdaderas o 
        falsas:
        \begin{itemize}
            \item Si $R$ y $S$ son relaciones de equivalencia definidas sobre 
            un conjunto no vacío $A$, entonces $R \cup S$ es una relación de 
            equivalencia sobre $A$.

            \item Si $R$ y $S$ son relaciones de equivalencia definidas sobre 
            un conjunto no vacío $A$ tal que $R \cup S$ es una relación de 
            equivalencia, entonces tanto $R$ como $S$ son relaciones de 
            equivalencia. 
        \end{itemize}
    }

    % Ejercicio 05
    \question
    {
        \textbf{Determina} si las siguientes afirmaciones son verdaderas o 
        falsas:
        \begin{itemize}
            \item Si $R$ y $S$ son relaciones de equivalencia definidas sobre 
            un conjunto no vacío $A$, entonces $R \cap S$ es una relación de 
            equivalencia sobre $A$. 

            \item Si $R$ y $S$ son relaciones de equivalencia definidas sobre 
            un conjunto no vacío $A$ tal que $R \cap S$ es una relación de 
            equivalencia sobre $A$, entonces tanto $R$ como $S$ son relaciones 
            de equivalencia sobre $A$.  
        \end{itemize}
    }

    % Ejercicio 06
    \question
    {
        \textbf{Determina} si las siguientes relaciones son de equivalencia. 
        En caso de que lo sean, \textbf{determina} las clases de equivalencia 
        y \textbf{da} la partición inducida por dichas relaciones. En caso de 
        que alguna no lo sea, \textbf{justifica} por qué no. 
        \begin{itemize}
            \item Sea $X = \{a,b,c\}$. Definimos la relación $\sim$ sobre 
            $\mathcal{P}(X)$ como 
            \begin{equation*}
                A \sim B \Leftrightarrow \text{ el número de elementos en } A 
                \text{ es igual al número de elementos en } B
            \end{equation*}

            \item Sea $X$ un conjunto no vacío y $Y \subseteq X$. Definimos la 
            relación $\sim$ sobre $\mathcal{P}(X)$ como sigue: 
            \begin{equation*}
                A \sim B \Leftrightarrow A \cap Y = B \cap Y
            \end{equation*}
        \end{itemize}
    }

    % Ejercicio 07
    \question
    {
        \textbf{Determina} si las siguientes relaciones son de equivalencia. 
        En caso de que lo sean, \textbf{determina} las clases de equivalencia 
        y \textbf{da} la partición inducida por dichas relaciones. En caso de 
        que alguna no lo sea, \textbf{justifica} por qué no.
        \begin{itemize}
            \item Sea $C$ el conjunto de todas las cadenas de a's y de b's 
            cuya longitud es $4$. Definimos la relación $\sim$ sobre $C$ como 
            sigue:
            \begin{equation*}
                s \sim t \Leftrightarrow s \text{ tiene los dos mismos primeros 
                caracteres que } t
            \end{equation*}

            \item Sea $X = \{1,2,3\}$. Definimos la relación $\sim$ sobre 
            $\mathcal{P}(X)$ como sigue:
            \begin{equation*}
                A \sim B \Leftrightarrow \text{el número de elementos en } A 
                \text{ es menor que el número de elementos en } B
            \end{equation*}
        \end{itemize}
    }

    % Ejercicio 08
    \question
    {
        \textbf{Determina} si las siguientes relaciones son de equivalencia. 
        En caso de que lo sean, \textbf{determina} las clases de equivalencia 
        y \textbf{da} la partición inducida por dichas relaciones. En caso de 
        que alguna no lo sea, \textbf{justifica} por qué no.
        \begin{itemize}
            \item Sea $S$ el conjunto de todas las cadenas de ceros, unos y 
            dos, cuya longitud es $2$. Definimos la relación $\sim$ sobre $S$
            como sigue:
            \begin{equation*}
                s \sim t \Leftrightarrow \text{ si la suma de los caracteres 
                de } s \text{ es igual a la suma de los caracteres en } t
            \end{equation*} 

            \item Sea $X$ un conjunto no vacío. Definimos la relación $\sim$ 
            sobre $\mathcal{P}(X)$ como sigue:
            \begin{equation*}
                A \sim B \Leftrightarrow A \subseteq B
            \end{equation*}
        \end{itemize}
    }

    % Ejercicio 09
    \question
    {
        \textbf{Determina} si las siguientes relaciones son de equivalencia. 
        En caso de que lo sean, \textbf{determina} las clases de equivalencia 
        y \textbf{da} la partición inducida por dichas relaciones. En caso de 
        que alguna no lo sea, \textbf{justifica} por qué no.
        \begin{itemize}
            \item Sea $X$ un conjunto no vacío. Definimos la relación $\sim$
            sobre $\mathcal{P}(X)$ como sigue:
            \begin{equation*}
                A \sim B \Leftrightarrow A \neq B
            \end{equation*}

            \item Definimos la relación $\sim$ sobre $\mathbb{R}$ como 
            sigue:
            \begin{equation*}
                x \sim y \Leftrightarrow x^2 - x = y^2 - y
            \end{equation*}
        \end{itemize}
    }

    % Ejercicio 10
    \question
    {
        Sea $X = \{1,2,3,4\}$. Para cada una de las siguientes particiones de 
        $X$, \textbf{determina} la relación de equivalencia inducida por ésta, 
        \textbf{mostrando} todos sus elementos.
        \begin{itemize}
            \item $\{\{1,2\}, \{3,4\}\}$
            \item $\{\{1\}, \{2\}, \{3,4\}\}$
            \item $\{\{1\}, \{2\}, \{3\}, \{4\}\}$
        \end{itemize}
    }

    % Ejercicio 11
    \question
    {
        \textbf{Determina} si las siguientes relaciones son de equivalencia. 
        En caso de que lo sean, \textbf{determina} las clases de equivalencia 
        y \textbf{da} la partición inducida por dichas relaciones. En caso de 
        que alguna no lo sea, \textbf{justifica} por qué no.
        \begin{itemize}
            \item Sea $P$ el conjunto de todas las personas. Definimos la 
            relación $R$ sobre $P$ como sigue:
            \begin{equation*}
                R = \{(a,b) \; | \; a \text{ es más alto que } b \}
            \end{equation*}

            \item Sea $P$ el conjunto de todas las personas. Definimos la
            relación $S$ sobre $P$ como sigue:
            \begin{equation*}
                S = \{(a,b) \; | \; a \text{ y } b \text{ en algún momento han 
                vivido en el mismo país}\}
            \end{equation*}
        \end{itemize}
    }

    % Ejercicio 12
    \question
    {
        \textbf{Determina} si las siguientes relaciones son de equivalencia. 
        En caso de que lo sean, \textbf{determina} las clases de equivalencia 
        y \textbf{da} la partición inducida por dichas relaciones. En caso de 
        que alguna no lo sea, \textbf{justifica} por qué no.
        \begin{itemize}
            \item Sea $P$ el conjunto de todas las personas. Definimos la 
            relación $R$ sobre $P$ como sigue:
            \begin{equation*}
                R = \{(a,b) \; | \; a \text{ y } b \text{ están enamorados}\}
            \end{equation*}

            \item Sea $P$ el conjunto de todas las personas. Definimos la
            relación $S$ sobre $P$ como sigue:
            \begin{equation*}
                S = \{(a,b) \; | \; b \text{ tiene más mascotas que } a\}
            \end{equation*}
        \end{itemize}
    }

    % Ejercicio 13
    \question
    {
        \textbf{Determina} si las siguientes relaciones son de equivalencia. 
        En caso de que lo sean, \textbf{determina} las clases de equivalencia 
        y \textbf{da} la partición inducida por dichas relaciones. En caso de 
        que alguna no lo sea, \textbf{justifica} por qué no.
        \begin{itemize}
            \item Sea $P$ el conjunto de todas las personas. Definimos la 
            relación $R$ sobre $P$ como sigue:
            \begin{equation*}
                R = \{(a,b) \; | \; a \text{ y } b \text{ tienen la misma 
                altura}\}
            \end{equation*}

            \item Sea $P$ el conjunto de todas las personas. Definimos la
            relación $S$ sobre $P$ como sigue:
            \begin{equation*}
                S = \{(a,b) \; | \; a \text{ y } b \text{ tienen el mismo color 
                de cabello}\}
            \end{equation*}
        \end{itemize}
    }

    % Ejercicio 14
    \question
    {
        Sea $X = \{1,2,3,4\}$. Para cada una de las siguientes particiones de 
        $X$, \textbf{determina} la relación de equivalencia inducida por ésta, 
        \textbf{mostrando} todos sus elementos.
        \begin{itemize}
            \item $\{\{1,2,3\}, \{4\}\}$
            \item $\{\{1,2,3,4\}\}$
            \item $\{\{1\}, \{2,4\}, \{3\}\}$
        \end{itemize}
    }

    % Ejercicio 15
    \question
    {
        \textbf{Determina} si las siguientes relaciones son de equivalencia. 
        En caso de que lo sean, \textbf{determina} las clases de equivalencia 
        y \textbf{da} la partición inducida por dichas relaciones. En caso de 
        que alguna no lo sea, \textbf{justifica} por qué no.
        \begin{itemize}
            \item Definimos la relación $R$ sobre $\mathbb{Z}$ como sigue:
            \begin{equation*}
                xRy \Leftrightarrow x+y \text{ es par}
            \end{equation*}

            \item Definimos la relación $R$ sobre $\mathbb{Z}$ como sigue:
            \begin{equation*}
                xRy \Leftrightarrow 7 \text{ divide a } n-m
            \end{equation*}
        \end{itemize}
    }

    \newpage
    % Ejercicio 16
    \question
    {
        Definimos a $\mathcal{I}$ como el conjunto de instrucciones que deben 
        realizarse para preparar cereal con leche.
        \begin{itemize}
            \item[I.] Tomamos un plato hondo para poner la mezcla.
            \item[II.] Tomamos la caja del cereal.
            \item[III.] Servimos la porción de cereal deseada en el plato.
            \item[IV.] Tomamos el cartón de leche
            \item[V.] Servimos la porción de leche deseada en el plato. 
        \end{itemize}
        
        Algunas instrucciones deben realizarse antes que otras. Por ejemplo, 
        la instrucción I debe realizarse antes que la instrucción III. Por 
        otro lado, otras instrucciones se pueden realizar en cualquier orden, 
        como es el caso de las instrucciones I y II. 
        
        Dicho esto, definimos la relación $\sim$ sobre  como sigue:
        \begin{equation*}
            i \sim j \Leftrightarrow i=j \text{ o la instrucción } i 
            \text{ debe realizarse antes que la instrucción } j
        \end{equation*}

        \textbf{Realiza} lo siguiente:
        \begin{itemize}
            \item \textbf{Describe} la relación $\sim$, \textbf{mostrando} 
            todos sus elementos.

            \item \textbf{Representa} gráficamente la relación $\sim$ usando 
            gráficas dirigidas.

            \item ¿$\sim$ es reflexiva? 
            
            \item ¿$\sim$ es antirreflexiva? 

            \item ¿$\sim$ es simétrica? 

            \item ¿$\sim$ es asimétrica? 

            \item ¿$\sim$ es antisimétrica? 

            \item ¿$\sim$ es transitiva? 

            \item ¿$\sim$ es un orden parcial? 

            \item \textbf{Dibuja} su diagrama de Hasse sobre el conjunto 
            \{(I, IV), (I,I), (III,V)\}
        \end{itemize}
    }

    % Ejercicio 17
    \question
    {
        Sea $\mathcal{H}$ el conjunto de todas las personas que han vivido. 
        Definimos la relación $\mathcal{H}$$\mathcal{H}$ sobre $\mathcal{H}$ como sigue:
        \begin{equation*}
            p \sim q \Leftrightarrow p \text{ es ancestro de } q \text{ o } 
            p=q
        \end{equation*}

        \textbf{Realiza} lo siguiente:
        \begin{itemize}
            \item ¿$\mathcal{H}$ es reflexiva?
            \item ¿$\mathcal{H}$ es antirreflexiva?
            \item ¿$\mathcal{H}$ es simétrica?
            \item ¿$\mathcal{H}$ es asimétrica?
            \item ¿$\mathcal{H}$ es antisimétrica?
            \item ¿$\mathcal{H}$ es transitiva?
            \item ¿$\mathcal{H}$ es un orden parcial?
            \item \textbf{Determina} los elementos minimales y maximales de 
            $(\mathcal{H}, \sim)$
            \item \textbf{Determina} las cotas superiores e inferiores de 
            $(\mathcal{H}, \sim)$
            \item \textbf{Determina} el elemento supremo e ínfimo de 
            $(\mathcal{H}, \sim)$
        \end{itemize}
    }

    % Ejercicio 18
    \question
    {
        Sea $(\mathcal{P}(1,2,3), \subseteq)$ un conjunto parcialmente ordenado. 
        \textbf{Encuentra} dos elementos en dicho conjunto que no sean 
        comparables. 
    }


    % Ejercicio 19
    \question
    {
        Sea $S$ el conjunto de todas las cadenas de a's y b's. Definimos la 
        relación binaria $R$ sobre $S$ como sigue:
        \begin{equation*}
            sRt \Leftrightarrow \text{ la longitud de } s \text{ es menor o 
            igual a la longitud de } t
        \end{equation*}

        \textbf{Determina} si $R$ es un orden parcial. 
    }

    % Ejercicio 20
    \question
    {
        Definimos la relación binaria $R$ sobre $\mathbb{R} \times \mathbb{R}$ 
        como sigue:
        \begin{equation*}
            (a,b)R(c,d) \Leftrightarrow (a < c) \lor (a = c \land b \leq d)    
        \end{equation*}

        \textbf{Determina} si $R$ es un orden parcial. 
    }

    % Ejercicio 21 
    \question
    {
        Definimos la relación binaria $R$ sobre $\mathbb{Z}$ como sigue: 
        \begin{equation*}
            xRy \Leftrightarrow x+y \text{ es par}
        \end{equation*}

        \textbf{Determina} si $R$ es un orden parcial. 
    }

    % Ejercicio 22 
    \question
    {
        Sea $A = \{a,b,c,d\}$. Definimos la relación $R$ sobre $A$ como sigue:
        \begin{equation*}
            R = \{(a,a), (b,b), (c,c), (d,d), (c,a), (a,d), (c,d), (b,c), (b,d),
            (b,a)\}
        \end{equation*}

        \textbf{Determina} si $R$ es un orden total. 
    }

    % Ejercicio 23 
    \question
    {
        Sea $A = \{1,2,3,6\}$. Definimos la relación binaria $R$ sobre $A$ como 
        sigue:
        \begin{equation*}
            xRy \Leftrightarrow \frac{x}{y} \text{ es impar}
        \end{equation*}

        ¿Es $R$ un orden parcial? En caso afirmativo, ¿es un orden total?
    }

    % Ejercicio 24
    \question
    {
        Definimos la relación binaria $R$ sobre $\mathbb{Z}$ como sigue:
        \begin{equation*}
            xRy \Leftrightarrow x-y \text{ es par}
        \end{equation*}

        ¿Es $R$ un orden parcial? En caso afirmativo, ¿es un orden total?
    }

    % Ejercicio 25
    \question
    {
        Definimos la relación binaria $R$ sobre $Z^+$ como sigue:
        \begin{equation*}
            xRy \Leftrightarrow x = y^k \quad \quad \text{con }
            k \in \{0,1,2,\ldots\}
        \end{equation*}

        ¿Es $R$ un orden parcial? En caso afirmativo, ¿es un orden total?
    }

    % Ejercicio 26
    \question
    {
        Sea $P$ el conjunto de todas las personas. Definimos la relación 
        binaria $R$ sobre $P$ como  sigue:
        \begin{equation*}
            xRy \Leftrightarrow \frac{x}{y} \text{ es impar}
        \end{equation*}

        ¿Es $R$ un orden parcial? En caso afirmativo, ¿es un orden total?
    }

    % Ejercicio 27
    \question
    {
        Sean $R$ y $S$ dos relaciones de orden parcial sobre un conjunto $A$.
        Definimos una relación $T$ sobre el conjunto $A$ como sigue:
        \begin{equation*}
            xTy \Leftrightarrow xRy \land xSy 
        \end{equation*} 

        \textbf{Determina} si $T$ es un orden parcial sobre $T$. 
    }

    % Ejercicio 28
    \question
    {
        Sea $A = \{0,1\}$. \textbf{Describe} todas las relaciones de orden 
        parcial sobre $A$. 
    }

    % Ejercicio 29
    \question 
    {
        Sea $A = \{0,1,2\}$. \textbf{Realiza} lo siguiente:
        \begin{itemize}
            \item \textbf{Describe} todas las relaciones de orden parcial sobre 
            $A$ donde $0$ es un elemento máximo. 

            \item \textbf{Describe} todas las relaciones de orden parcial sobre 
            $A$ donde $0$ es un elemento mínimo. 
        \end{itemize}
    }

    % Ejercicio 30
    \question
    {
        Supongamos que $R$ es una relación reflexiva, simétrica, transitiva y 
        antisimétrica sobre un conjunto $A$. ¿Qué podemos concluir acerca de la 
        relación $R$?
    }

    % Ejercicio 31
    \question
    {
        Sea $R$ una relación de equivalencia sobre un conjunto $A$. Si $R$ tiene 
        sólo una clase de equivalencia, ¿cómo es la relación $R$?
    }

    % Ejercicio 32
    \question
    {
        Sea $R$ una relación de equivalencia sobre un conjunto $A$. Si 
        $|X| = |R|$, ¿cómo es la relación $R$?
    }

    % Ejercicio 33
    \question
    {
        Sea $A = \{1,2,3,4,5,6\}$. \textbf{Proporciona} una relación de 
        equivalencia $R$ sobre $A$ que tenga exactamente cuatro clases de 
        equivalencia. 
    }

    % Ejercicio 34
    \question
    {
        ¿Cuántas relaciones de equivalencia hay en el conjunto 
        $\{1,2,3\}$? \textbf{Escribe} cada una de ellas.  
    }

    % Ejercicio 35
    \question
    {
        Sean $X$ y $Y$ conjuntos distintos del vacío. Sea $f: X 
        \rightarrow Y$ una función. Definimos el conjunto $S$ como sigue:
        \begin{equation*}
            S = \{f^{-1}(\{y\}) \; | \; y \in Y\}
        \end{equation*}

        ¿Es $S$ una partición de $X$? En caso afirmativo, \textbf{describe} 
        una relación de equivalencia que genere esta partición.
    }

    % Ejercicio 36
    \question
    {
        Sea $R$ una relación de equivalencia sobre un conjunto $A$. 
        \textbf{Define} una función $f$ de $A$ al conjunto de clases de 
        equivalencia de $A$ mediante la regla de correspondencia 
        \begin{equation*}
            f(x) = [x]
        \end{equation*}

        ¿Cuándo sucede que $f(x) = f(y)$?
    }

    % Ejercicio 37
    \question
    {
        Sea $R$ una relación sobre un conjunto $A$. Sea \texttt{refl}, 
        \texttt{sim} y \texttt{trans} la cerradura reflexiva, simétrica y 
        transitiva de $R$, respectivamente. \textbf{Muestra} que 
        \texttt{trans(\texttt{sim(\texttt{refl$(R)$})})} es una relación de 
        equivalencia que contiene a $R$. 
    }

    % Ejercicio 38
    \question
    {
        Sea $R$ una relación de equivalencia sobre un conjunto $A$. ¿Cuál o 
        cuáles de las siguientes expresiones son \textbf{verdaderas}?
        \begin{itemize}
            \item Si $(a,b) \not \in R$, entonces $[a] = [b]$.
            \item Si $[a] = [b]$, entonces $[a] \cap [b] = \varnothing$.
        \end{itemize}
    }

    % Ejercicio 39
    \question
    {
        Sea $R$ una relación de equivalencia sobre un conjunto no vacío $A$ y 
        sean $a,b \in A$. \textbf{Demostrar} que 
        \begin{equation*}
            [a] = [b] \Leftrightarrow aRb
        \end{equation*}       
    }

    % Ejercicio 40
    \question
    {
        Sea $A$ un conjunto tal que $|A| = 10$ y sea $R$ una relación de 
        equivalencia sobre $A$. Sean $x,y,z \in A$ tal que $|[x]| = 3, 
        |[y]| = 5$ y $|[z]| = 1$. ¿Cuántas clases de equivalencia tiene 
        $R$?
    }

    % Ejercicio 41
    \question
    {
        ¿Cuáles de las siguientes relaciones son de equivalencia?
        \begin{itemize}
            \item Definimos la relación $R$ sobre $\mathbb{Z}$ como sigue:
            \begin{equation*}
                aRb \Leftrightarrow a^2 - b^2 \leq 7
            \end{equation*}

            \item Definimos la relación $S$ sobre $\mathbb{Z}$ como 
            sigue:
            \begin{equation*}
                aSb \Leftrightarrow a + b \equiv 0 \pmod{5}
            \end{equation*}
        \end{itemize}
    }

    % Question 42
    \question
    {
        ¿Cuáles de las siguientes relaciones son de equivalencia?
        \begin{itemize}
            \item Definimos la relación $R$ sobre $\mathbb{Z}$ como 
            sigue:
            \begin{equation*}
                aRb \Leftrightarrow 2a + 5b \equiv 0 \pmod{7}
            \end{equation*}

            \item Definimos la relación $S$ sobre $\mathbb{Z}$ como 
            sigue:
            \begin{equation*}
                aSb \Leftrightarrow a^2 + b^ = 0
            \end{equation*}
        \end{itemize}
    }

    % Question 43
    \question
    {
        Sea $R$ una relación simétrica y transitiva sobre un conjunto $A$. 
        \textbf{Demuestra} que si para cada $x \in A$ existe una $y \in A$ 
        tal que $(x,y) \in R$, entonces $R$ es una relación de equivalencia. 
    }

    % Question 44
    \question
    {
        Sea $R$ una relación reflexiva y transitiva sobre un conjunto $A$. Sea
        $S$ una relación binaria sobre el conjunto $A$ tal que $(x,y) \in S$ si 
        y sólo si $(x,y), (y,x) \in R$. \textbf{Demuestra} que $S$ es una 
        relación de equivalencia. 
    }

    % Question 45
    \question
    {
        Para cada uno de los siguientes diagramas de Hasse, \textbf{realiza} 
        lo siguiente:
        \begin{itemize}
            \item \textbf{Describe} todos los pares ordenados del orden 
            parcial. 
            \item \textbf{Describe} los elementos minimales y maximales, 
            además del máximo y del mínimo (si es que existen). 
        \end{itemize}

        \begin{multicols}{3}
            \begin{itemize}
                \item[a)]  

                \begin{tikzpicture}
                    \matrix (A) [matrix of nodes, row sep=0.5cm, 
                                 column sep=0.5cm]
                    { 
                        $e$ & & $d$ \\  
                        $b$ & & $c$ \\
                        & $a$ & \\
                    };
                    \draw (A-1-1)--(A-2-1);
                    \draw (A-2-1)--(A-1-3);
                    \draw (A-2-1)--(A-3-2);
                    \draw (A-1-3)--(A-2-3);
                    \draw (A-3-2)--(A-2-3);
                \end{tikzpicture}
    
                \item[b)]

                \begin{tikzpicture}
                    \matrix (A) [matrix of nodes, row sep=0.5cm, 
                                 column sep=0.5cm]
                    { 
                        & & $g$ & \\ 
                        & $e$ & & $f$ \\  
                        & $d$ & & \\
                        $a$ & & $b$ & $c$ \\
                    };
                    \draw (A-1-3)--(A-2-2);
                    \draw (A-1-3)--(A-2-4);
                    \draw (A-2-2)--(A-3-2);
                    \draw (A-2-4)--(A-3-2);
                    \draw (A-2-4)--(A-4-4);
                    \draw (A-3-2)--(A-4-1);
                    \draw (A-3-2)--(A-4-3);
                \end{tikzpicture}
                
                \item[c)]

                \begin{tikzpicture}
                    \matrix (A) [matrix of nodes, row sep=0.5cm, 
                                 column sep=0.5cm]
                    { 
                        $g$ & \\
                        $f$ & \\  
                        $d$ & $e$ \\
                        $b$ & $c$ \\
                        & $a$ \\ 
                    };
                    \draw (A-1-1)--(A-2-1);
                    \draw (A-2-1)--(A-3-1);
                    \draw (A-2-1)--(A-3-2);
                    \draw (A-3-1)--(A-4-1);
                    \draw (A-3-2)--(A-4-1);
                    \draw (A-3-2)--(A-4-2);
                    \draw (A-4-1)--(A-5-2);
                    \draw (A-4-2)--(A-5-2);
                \end{tikzpicture}
            \end{itemize}
        \end{multicols}
    }

    % Question 46
    \question
    {
        Sea $\mathcal{E}$ la colección de todos los subconjuntos finitos de 
        $\mathbb{N}$ que tienen un número par de elementos. Para el conjunto
        parcialmente ordenado $(\mathcal{E}, \subseteq)$ se consideran los 
        elementos $A = \{1,2\}$ y $B = \{1,3\}$. 
        \begin{itemize}
            \item \textbf{Encuentra} cuatro cotas superiores para $\{A,B\}$.
            \item ¿Tiene $\{A,B\}$ supremo en $(\mathcal{E}, \subseteq)$?
        \end{itemize}       
    }

    % Ejercicio 47
    \question
    {
        Sea $\mathcal{C}$ la colección de todos los subconjuntos finitos de 
        $\mathbb{N}$. ¿Tiene $(\mathcal{C}, \subseteq)$ algún elemento 
        maximal o minimal?
    }

    % Ejercicio 48
    \question
    {
        Para cada uno de los siguientes diagramas de Hasse, \textbf{realiza} 
        lo siguiente:
        \begin{itemize}
            \item \textbf{Encuentra} las cotas superiores e inferiores del
            subconjunto $B$ en $A$. 
            parcial. 
            \item \textbf{Encuentra} el supremo y el ínfimo del conjunto $B$.
        \end{itemize}

        \begin{multicols}{3}
            \begin{itemize}
                \item[a)] $B = \{c,d,e\}$
                
                \begin{tikzpicture}
                    \matrix (A) [matrix of nodes, row sep=0.5cm, 
                                 column sep=0.5cm]
                    { 
                        $a$ & & $b$ & \\ 
                        & \textcolor{red}{$c$} & \\  
                        \textcolor{red}{$d$} & & \textcolor{red}{$e$} \\
                        & $f$ & & $g$ \\
                    };
                    \draw (A-1-1)--(A-2-2);
                    \draw (A-1-3)--(A-2-2);
                    \draw (A-2-2)--(A-3-1);
                    \draw (A-2-2)--(A-3-3);
                    \draw (A-3-1)--(A-4-2);
                    \draw (A-3-3)--(A-4-2);
                    \draw (A-3-3)--(A-4-4);
                \end{tikzpicture}

                \columnbreak
    
                \item[b)] $B = \{4,5,6\}$

                \begin{tikzpicture}
                    \matrix (A) [matrix of nodes, row sep=0.5cm, 
                                 column sep=0.5cm]
                    { 
                        $1$ & & $2$ & \\ 
                        & $3$ & \\  
                        \textcolor{red}{$4$} & & \textcolor{red}{$5$} \\
                        & \textcolor{red}{$6$} & & $7$ \\
                        & & $8$ & \\ 
                    };
                    \draw (A-1-1)--(A-2-2);
                    \draw (A-1-3)--(A-2-2);
                    \draw (A-2-2)--(A-3-1);
                    \draw (A-2-2)--(A-3-3);
                    \draw (A-3-1)--(A-4-2);
                    \draw (A-3-3)--(A-4-2);
                    \draw (A-3-3)--(A-4-4);
                    \draw (A-4-2)--(A-5-3);
                    \draw (A-4-4)--(A-5-3);
                \end{tikzpicture}
                
                \columnbreak

                \item[c)] $B = \{2,3,4\}$

                \begin{tikzpicture}
                    \matrix (A) [matrix of nodes, row sep=0.5cm, 
                                 column sep=0.5cm]
                    { 
                        & $1$ & \\
                        & \textcolor{red}{$2$} & \\  
                        \textcolor{red}{$3$} & & \textcolor{red}{$4$} \\
                        $5$ & & $6$ \\
                    };
                    \draw (A-1-2)--(A-2-2);
                    \draw (A-2-2)--(A-3-1);
                    \draw (A-2-2)--(A-3-3);
                    \draw (A-3-1)--(A-4-1);
                    \draw (A-3-1)--(A-4-3);
                    \draw (A-3-3)--(A-4-1);
                    \draw (A-3-3)--(A-4-3);
                \end{tikzpicture}
            \end{itemize}
        \end{multicols}
    }
\end{questions}
\end{document}
