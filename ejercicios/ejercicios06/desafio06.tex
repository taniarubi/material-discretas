\documentclass[oneside]{style}

\title{Desafío 06}
\principal{Relaciones 02: La venganza}
\author{Tania Michelle Rubí Rojas}
\semester{Semestre 2023-1}

\begin{document}
\maketitle

Para cada uno de los siguientes ejercicios, \textbf{justifica ampliamente} tu 
respuesta:

\begin{questions}[label=\protect\circled{\bfseries\arabic*}]
    % Ejercicio 01
    \question
    {
        Sea $A$ un conjunto y $R = \varnothing \subset A^2$ la 
        relación vacía. 

        \textbf{Responde}:
        \begin{itemize}
            \item ¿$R$ es reflexiva sobre $\varnothing$?
            \item ¿$R$ es reflexiva sobre cualquier conjunto $A \neq 
            \varnothing$? 
            \item ¿$R$ es antirreflexiva?
            \item ¿$R$ es simétrica?
            \item ¿$R$ es antisimétrica?
            \item ¿$R$ es asimétrica?
            \item ¿$R$ es transitiva?
        \end{itemize}
    }

    % Ejercicio 02
    \question
    {
        Para cada uno de los siguientes incisos, \textbf{proporciona} un ejemplo 
        de una relación $R$ definida sobre un conjunto $A$ que cumpla lo 
        siguiente y \textbf{justifica} tu respuesta, y en caso de que no sea 
        posible, \textbf{justifica} por qué no lo es. 
        \begin{itemize}
            \item $R$ es transitiva, pero no es reflexiva sobre $A$ ni simétrica.
            \item $R$ no es reflexiva ni antirreflexiva.
            \item $R$ es simétrica y transitiva, pero no reflexiva sobre $A$.
            \item $R$ es reflexiva y antisimétrica.
        \end{itemize}
    }

    % Ejercicio 03
    \question
    {
        Sea el conjunto $A = \{1,2,3,4,5\}$. Definimos la relación $R$ como 
        sigue: 
        \begin{equation*}
            R = \{(1,1), (1,2), (2,1), (2,2), (3,3), (3,4), (4,3), (4,4), (5,5)\}
        \end{equation*}

        \textbf{Realiza} lo siguiente:
        \begin{itemize}
            \item \textbf{Representa gráficamente} la relación $R$ usando 
            gráficas dirigidas.
            \item \textbf{Determina} si la relación $R$ es reflexiva sobre $A$, 
            antirreflexiva sobre $A$, simétrica, antisimétrica y/o transitiva.
        \end{itemize}
    }

    % Ejercicio 04
    \question
    {
        Sea $A = \{0,1,2,3\}$. Definimos las relaciones $R, S$ y $T$ sobre $A$
        de la siguiente manera:
        \begin{align*}
            R &= \{(0,1), (0,2), (1,1), (1,3), (2,2), (3,0)\} \\ 
            S &= \{(0,0), (0,3), (1,0), (1,2), (2,0), (3,2)\} \\ 
            T &= \{(0,2), (1,0), (2,3), (3,1)\}
        \end{align*}

        \textbf{Determina} la cerradura transitiva de las relaciones $R, S$ y 
        $T$.
    }

    % Ejercicio 05
    \question
    {
        Sea el conjunto $A = \{1,2,3,4,5,6,7,8,9\}$, cuyo Universo del Discurso 
        es $\mathcal{U} = \{1,2,3,4,5,6,7,8,9\}$. Definimos las relaciones $R$ y 
        $S$ sobre $\mathcal{P}(A)$ como sigue:
        \begin{align*}
            R &= \{(X, Y) \; | \; X \cap Y = \varnothing\} \\ 
            S &= \{(X, Y) \; | \; X - Y^c\}
        \end{align*}
        
        \textbf{Determina} si $R$ y $S$ son reflexivas, antirreflexivas, 
        simétricas, antisimétricas y transitivas. 
    }

    \newpage
    % Ejercicio 06
    \question
    {
        Sean $R$ y $S$ relaciones sobre el conjunto $X$. \textbf{Analiza} las 
        siguientes afirmaciones. Si alguna es verdadera, \textbf{explica} 
        ampliamente por qué. En caso contrario, \textbf{proporciona} un ejemplo 
        donde no se cumpla. 
        \begin{itemize}
            \item Si $R$ y $S$ son antisimétricas, entonces $R \cap S$ es 
            antisimétrica.
            \item Si $R$ es antisimétrica, entonces $R^{-1}$ es antisimétrica. 
            \item Si $R$ y $S$ son simétricas, entonces $S \circ R$ es simétrica.
        \end{itemize}
    }

    % Ejercicio 07
    \question
    {
        Sean $A = \{1,2,3\}$. Definimos la relación $R$ sobre $A$, 
        respectivamente, como sigue: 
        \begin{equation*}
            R = \{(1,1), (1,2), (1,3), (3,1), (2,3)\}
        \end{equation*}

        \textbf{Realiza} lo siguiente:
        \begin{itemize}
            \item \textbf{Determina} la cerradura reflexiva de $R$.
            \item \textbf{Determina} la cerradura simétrica de $R$.
            \item \textbf{Determina} la cerradura transitiva de $R$.
        \end{itemize}
    }

    % Ejercicio 08
    \question
    {
        Para cada una de las siguientes relaciones, \textbf{determina} si son 
        reflexivas, antirreflexivas, simétricas, antisimétricas, asimétricas y 
        transitivas. 
        \begin{itemize}
            \item Sea $\mathcal{T}$ el conjunto de todos los árboles binarios 
            cuyos nodos están etiquetados con elementos en $\mathbb{N}$. 
            Definimos la relación $R$ como sigue: 
            \begin{equation*}
                ARB \Leftrightarrow A \text{ tiene la misma altura que } B
            \end{equation*} 

            \item Sea $\mathcal{T}$ el conjunto de todos los árboles binarios 
            cuyos nodos están etiquetados con elementos en $\mathbb{N}$. 
            Definimos la relación $R$ como sigue: 
            \begin{equation*}
                ARB \Leftrightarrow \text{el número de nodos de } A \text{ es 
                mayor que el número de nodos de } B
            \end{equation*} 
        \end{itemize}
    }

    % Ejercicio 09
    \question
    {
        Para cada una de las siguientes relaciones, \textbf{determina} si son 
        reflexivas, antirreflexivas, simétricas, antisimétricas, asimétricas y 
        transitivas. 
        \begin{itemize}
            \item Sea $\mathcal{T}$ el conjunto de todos los árboles binarios 
            cuyos nodos están etiquetados con elementos en $\mathbb{N}$. 
            Definimos la relación $R$ como sigue: 
            \begin{equation*}
                ARB \Leftrightarrow A \text{ tiene el mismo número de hojas que } 
                B
            \end{equation*} 

            \item Sea $\mathcal{L}_{A}$ el conjunto de todas las listas cuyos
            elementos se encuentran en el conjunto $A$. Definimos la relación 
            $R$ como sigue: 
            \begin{equation*}
                (l_1, l_2) \in R \Leftrightarrow \text{la cabeza de la lista } 
                l_1 \text{ es diferente a la cabeza de la lista } l_2
            \end{equation*} 
        \end{itemize}
    }

    % Ejercicio 10
    \question
    {
        Para cada una de las siguientes relaciones, \textbf{determina} si son 
        reflexivas, antirreflexivas, simétricas, antisimétricas, asimétricas y 
        transitivas. 
        \begin{itemize}
            \item Sea $\mathcal{L}_{A}$ el conjunto de todas las listas cuyos
            elementos se encuentran en el conjunto $A$. Definimos la relación 
            $R$ como sigue: 
            \begin{equation*}
                (l_1, l_2) \in R \Leftrightarrow \text{la longitud de la lista } 
                l_1 \text{ es mayor o igual que la longitud de la lista } l_2
            \end{equation*} 

            \item Sea $\mathcal{L}_{A}$ el conjunto de todas las listas cuyos
            elementos se encuentran en el conjunto $A$. Definimos la relación 
            $R$ como sigue: 
            \begin{equation*}
                (l_1, l_2) \in R \Leftrightarrow \text{los elementos de la 
                lista } l_1 \text{ son iguales a los elementos de la lista } l_2
            \end{equation*} 
        \end{itemize}
    }

    \newpage
    % Ejercicio 11
    \question
    {
        Sean $R$ y $S$ relaciones sobre el conjunto $X$. \textbf{Analiza} las 
        siguientes afirmaciones. Si alguna es verdadera, \textbf{explica} 
        ampliamente por qué. En caso contrario, \textbf{proporciona} un ejemplo 
        donde no se cumpla. 
        \begin{itemize}
            \item Si $R$ y $S$ son reflexivas, entonces $R \cup S$ es reflexiva.
            \item Si $R$ es transitiva, entonces $R^{-1}$ es transitiva.
            \item Si $R$ y $S$ son transitivas, entonces $S \circ R$ es 
            transitiva. 
        \end{itemize}
    }

    % Ejercicio 12
    \question
    {
        \textbf{Analiza} las siguientes afirmaciones. Si alguna es verdadera, 
        \textbf{explica} ampliamente por qué. En caso contrario, 
        \textbf{proporciona} un ejemplo donde no se cumpla. 
        \begin{itemize}
            \item Todas las relaciones que son asimétricas también son 
            antirreflexivas.
            \item Todas las relaciones que son antisimétricas y transitivas 
            también son asimétricas. 
            \item Todas las relaciones que no son transitivas son antirreflexivas.
        \end{itemize}
    }

    % Ejercicio 13
    \question
    {
        \textbf{Analiza} las siguientes afirmaciones. Si alguna es verdadera, 
        \textbf{explica} ampliamente por qué. En caso contrario, 
        \textbf{proporciona} un ejemplo donde no se cumpla. 
        \begin{itemize}
            \item Existe una relación $R$ sobre un conjunto $X$ que es 
            reflexiva, pero no transitiva.  
            \item Existe una relación $R$ sobre un conjunto $X$ que no es 
            reflexiva pero sí es asimétrica. 
            \item Existe una relación $R$ sobre un conjunto $X$ que es 
            reflexiva y transitiva, pero no es asimétrica. 
            \item Existe una relación $R$ sobre un conjunto $X$ que no es 
            simétrica, pero sí es transitiva y antirreflexiva. 
        \end{itemize}
    }

    % Ejercicio 14
    \question
    {
        Para cada una de las siguientes relaciones, \textbf{determina} si son 
        reflexivas, antirreflexivas, simétricas, antisimétricas, asimétricas y 
        transitivas. 
        \begin{itemize}
            \item Sea $P$ el conjunto de todas las personas vivas. Definimos la 
            relación $R$ como sigue: 
            \begin{equation*}
                (p,q) \in R \Leftrightarrow p \text{ y } q \text{ se conocen}
            \end{equation*} 

            \item Sea $P$ el conjunto de todas las personas vivas. Definimos la 
            relación $R$ como sigue: 
            \begin{equation*}
                pRq\Leftrightarrow p \text{ y } q \text{ hablan el mismo idioma}
            \end{equation*}
        \end{itemize}
    }

    % Ejercicio 15
    \question
    {
        Para cada una de las siguientes relaciones, \textbf{determina} si son 
        reflexivas, antirreflexivas, simétricas, antisimétricas, asimétricas y 
        transitivas. 
        \begin{itemize}
            \item Sea $P$ el conjunto de todas las personas vivas. Definimos la 
            relación $R$ como sigue: 
            \begin{equation*}
                pRq \Leftrightarrow p \text{ y } q \text{ tienen la misma edad}
            \end{equation*} 

            \item Sea $P$ el conjunto de todas las personas vivas. Definimos la 
            relación $R$ como sigue: 
            \begin{equation*}
                pRq\Leftrightarrow p \text{ y } q \text{ tienen un amigo en común}
            \end{equation*}
        \end{itemize}
    }

    % Ejercicio 16
    \question
    {
        Para cada una de las siguientes relaciones, \textbf{determina} si son 
        reflexivas, antirreflexivas, simétricas, antisimétricas, asimétricas y 
        transitivas.
        \begin{itemize}
            \item Definimos la relación $R$ sobre $\mathbb{Z}$ como sigue:
            \begin{equation*}
                R = \{(a,b) \; | \; a-b \text{ es un número entero positivo 
                impar}\}
            \end{equation*}

            \item Definimos la relación $S$ sobre $\mathbb{Z}$ como sigue:
            \begin{equation*}
                S = \{(a,b) \; | \; a = b^2\}
            \end{equation*}
        \end{itemize}
    }
\end{questions}
\end{document}
